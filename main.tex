\documentclass{report}

\usepackage[utf8]{inputenc}
\usepackage{eumat}
\usepackage[Conny]{fncychap}
\usepackage[bahasa]{babel}

% Rename Contents
\addto\captionsenglish{\renewcommand{\contentsname}{\vspace{-0.5cm} \textbf{Daftar Isi} \vspace{-2cm}}}

\begin{document}

% Cover Page
\begin{titlepage}
    \begin{center}
        \vspace*{1cm}
        
        \vspace{0.5cm}
        
        \LARGE
        Tugas Latex Aplikasi Komputer  
        
        \vspace{1cm}
        
        \includegraphics[width=0.5\textwidth]{image.png}

        \vspace{1cm}
        
        \textbf{Rasyid Shalahuddin}\\
        22305144016\\
        Matematika E 2022
        
        \vspace{2cm}
        
        \Large
        \textbf{PRODI MATEMATIKA}\\
        \textbf{DEPARTEMEN PENDIDIKAN MATEMATIKA}\\
        \textbf{FAKULTAS MATEMATIKA DAN ILMU PENGETAHUAN ALAM}
        \textbf{UNIVERSITAS NEGERI YOGYAKARTA}\\
        \textbf{2022}
        
    \end{center}
\end{titlepage}

\newpage
\tableofcontents{}

\chapter{KB Pekan 2 (Belajar Menggunakan Software EMT)}
\documentclass[a4paper,10pt]{article}
\usepackage{eumat}

\begin{document}
\begin{eulernotebook}
\eulerheading{Pendahuluan dan Pengenalan Cara Kerja EMT}
\begin{eulercomment}
Rasyid Shalahuddin\\
22305144016\\
Matematika E 2022

Selamat datang! Ini adalah pengantar pertama ke Euler Math Toolbox
(disingkat EMT atau Euler). EMT adalah sistem terintegrasi yang
merupakan perpaduan kernel numerik Euler dan program komputer aljabar
Maxima.

- Bagian numerik, GUI, dan komunikasi dengan Maxima telah dikembangkan
oleh R. Grothmann, seorang profesor matematika di Universitas
Eichstätt, Jerman. Banyak algoritma numerik dan pustaka software open
source yang digunakan di dalamnya.

- Maxima adalah program open source yang matang dan sangat kaya untuk
perhitungan simbolik dan aritmatika tak terbatas. Software ini
dikelola oleh sekelompok pengembang di internet.

- Beberapa program lain (LaTeX, Povray, Tiny C Compiler, Python) dapat
digunakan di Euler untuk memungkinkan perhitungan yang lebih cepat
maupun tampilan atau grafik yang lebih baik.

Yang sedang Anda baca (jika dibaca di EMT) ini adalah berkas notebook
di EMT. Notebook aslinya bawaan EMT (dalam bahasa Inggris) dapat
dibuka melalui menu File, kemudian pilih "Open Tutorias and Example",
lalu pilih file "00 First Steps.en". Perhatikan, file notebook EMT
memiliki ekstensi ".en". Melalui notebook ini Anda akan belajar
menggunakan software Euler untuk menyelesaikan berbagai masalah
matematika.
\end{eulercomment}
\begin{eulercomment}
Panduan ini ditulis dengan Euler dalam bentuk notebook Euler, yang berisi teks
(deskriptif), baris-baris perintah, tampilan hasil perintah (numerik, ekspresi
matematika, atau gambar/plot), dan gambar yang disisipkan dari file gambar.

Untuk menambah jendela EMT, Anda dapat menekan [F11]. EMT akan menampilkan
jendela grafik di layar desktop Anda. Tekan [F11] lagi untuk kembali ke tata
letak favorit Anda. Tata letak disimpan untuk sesi berikutnya.

Anda juga dapat menggunakan [Ctrl]+[G] untuk menyembunyikan jendela grafik.
Selanjutnya Anda dapat beralih antara grafik dan teks dengan tombol [TAB].

Seperti yang Anda baca, notebook ini berisi tulisan (teks) berwarna hijau, yang
dapat Anda edit dengan mengklik kanan teks atau tekan menu Edit -\textgreater{} Edit Comment
atau tekan [F5], dan juga baris perintah EMT yang ditandai dengan "\textgreater{}" dan
berwarna merah. Anda dapat menyisipkan baris perintah baru dengan cara menekan
tiga tombol bersamaan: [Shift]+[Ctrl]+[Enter].

\end{eulercomment}
\eulersubheading{Komentar (Teks Uraian)}
\begin{eulercomment}
Komentar atau teks penjelasan dapat berisi beberapa "markup" dengan sintaks
sebagai berikut.

\end{eulercomment}
\begin{eulerttcomment}
   - * Judul
   - ** Sub-Judul
   - latex: F (x) = \(\backslash\)int_a^x f (t) \(\backslash\), dt
   - mathjax: \(\backslash\)frac\{x^2-1\}\{x-1\} = x + 1
   - maxima: 'integrate(x^3,x) = integrate(x^3,x) + C
   - http://www.euler-math-toolbox.de
   - See: http://www.google.de | Google
   - image: hati.png
   - ---
\end{eulerttcomment}
\begin{eulercomment}

Hasil sintaks-sintaks di atas (tanpa diawali tanda strip) adalah sebagai berikut.

\begin{eulercomment}
\eulerheading{Judul}
\begin{eulercomment}
\end{eulercomment}
\eulersubheading{Sub-Judul}
\begin{eulercomment}
\end{eulercomment}
\begin{eulerformula}
\[
F(x) = \int_a^x f(t) \, dt
\]
\end{eulerformula}
\begin{eulerformula}
\[
\frac{x^2-1}{x-1} = x + 1
\]
\end{eulerformula}
\begin{eulercomment}
maxima: 'integrate(x\textasciicircum{}3,x) = integrate(x\textasciicircum{}3,x) + C\\
http://www.euler-math-toolbox.de\\
See: http://www.google.de \textbar{} Google\\
image: hati.png\\
\end{eulercomment}
\eulersubheading{}
\begin{eulercomment}
Gambar diambil dari folder images di tempat file notebook berada dan tidak dapat
dibaca dari Web. Untuk "See:", tautan (URL)web lokal dapat digunakan.

Paragraf terdiri atas satu baris panjang di editor. Pergantian baris akan memulai
baris baru. Paragraf harus dipisahkan dengan baris kosong.
\end{eulercomment}
\begin{eulerprompt}
>// baris perintah diawali dengan >, komentar (keterangan) diawali dengan //
\end{eulerprompt}
\eulerheading{Baris Perintah}
\begin{eulercomment}
Mari kita tunjukkan cara menggunakan EMT sebagai kalkulator yang sangat
canggih.

EMT berorientasi pada baris perintah. Anda dapat menuliskan satu atau lebih
perintah dalam satu baris perintah. Setiap perintah harus diakhiri dengan koma
atau titik koma.

- Titik koma menyembunyikan output (hasil) dari perintah.\\
- Sebuah koma mencetak hasilnya.\\
- Setelah perintah terakhir, koma diasumsikan secara otomatis (boleh tidak
ditulis).

Dalam contoh berikut, kita mendefinisikan variabel r yang diberi nilai 1,25.
Output dari definisi ini adalah nilai variabel. Tetapi karena tanda titik koma,
nilai ini tidak ditampilkan. Pada kedua perintah di belakangnya, hasil kedua
perhitungan tersebut ditampilkan.
\end{eulercomment}
\begin{eulerprompt}
>r=1.25; pi*r^2, 2*pi*r
\end{eulerprompt}
\begin{euleroutput}
  4.90873852123
  7.85398163397
\end{euleroutput}
\eulersubheading{Latihan untuk Anda}
\begin{eulercomment}
- Sisipkan beberapa baris perintah baru\\
- Tulis perintah-perintah baru untuk melakukan suatu perhitungan yang Anda
inginkan, boleh menggunakan variabel, boleh tanpa variabel.\\
\end{eulercomment}
\eulersubheading{}
\begin{eulercomment}
Beberapa catatan yang harus Anda perhatikan tentang penulisan sintaks perintah
EMT.

- Pastikan untuk menggunakan titik desimal, bukan koma desimal untuk bilangan!\\
- Gunakan * untuk perkalian dan \textasciicircum{} untuk eksponen (pangkat).\\
- Seperti biasa, * dan / bersifat lebih kuat daripada + atau -.\\
- \textasciicircum{} mengikat lebih kuat dari *, sehingga pi * r \textasciicircum{} 2 merupakan rumus luas
lingkaran.\\
- Jika perlu, Anda harus menambahkan tanda kurung, seperti pada 2 \textasciicircum{} (2 \textasciicircum{} 3).

Perintah r = 1.25 adalah menyimpan nilai ke variabel di EMT. Anda juga dapat
menulis r: = 1.25 jika mau. Anda dapat menggunakan spasi sesuka Anda.

Anda juga dapat mengakhiri baris perintah dengan komentar yang diawali dengan dua
garis miring (//).
\end{eulercomment}
\begin{eulerprompt}
>r := 1.25 // Komentar: Menggunakan  := sebagai ganti =
\end{eulerprompt}
\begin{euleroutput}
  1.25
\end{euleroutput}
\begin{eulercomment}
Argumen atau input untuk fungsi ditulis di dalam tanda kurung.
\end{eulercomment}
\begin{eulerprompt}
>sin(45°), cos(pi), log(sqrt(E))
\end{eulerprompt}
\begin{euleroutput}
  0.707106781187
  -1
  0.5
\end{euleroutput}
\begin{eulercomment}
Seperti yang Anda lihat, fungsi trigonometri bekerja dengan radian, dan derajat
dapat diubah dengan °. Jika keyboard Anda tidak memiliki karakter derajat tekan
[F7], atau gunakan fungsi deg() untuk mengonversi.

EMT menyediakan banyak sekali fungsi dan operator matematika.Hampir semua fungsi
matematika sudah tersedia di EMT. Anda dapat melihat daftar lengkap fungsi-fungsi
matematika di EMT pada berkas Referensi (klik menu Help -\textgreater{} Reference)

Untuk membuat rangkaian komputasi lebih mudah, Anda dapat merujuk ke hasil
sebelumnya dengan "\%". Cara ini sebaiknya hanya digunakan untuk merujuk hasil
perhitungan dalam baris perintah yang sama.
\end{eulercomment}
\begin{eulerprompt}
>(sqrt(5)+1)/2, %^2-%+1 // Memeriksa solusi x^2-x+1=0
\end{eulerprompt}
\begin{euleroutput}
  1.61803398875
  2
\end{euleroutput}
\eulersubheading{Latihan untuk Anda}
\begin{eulercomment}
- Buka berkas Reference dan baca fungsi-fungsi matematika yang tersedia di EMT.\\
- Sisipkan beberapa baris perintah baru.\\
- Lakukan contoh-contoh perhitungan menggunakan fungsi-fungsi matematika di EMT.\\
\end{eulercomment}
\eulersubheading{}
\begin{eulercomment}
\begin{eulercomment}
\eulerheading{Satuan}
\begin{eulercomment}
EMT dapat mengubah unit satuan menjadi sistem standar internasional (SI).
Tambahkan satuan di belakang angka untuk konversi sederhana.
\end{eulercomment}
\begin{eulerprompt}
>1miles  // 1 mil = 1609,344 m
\end{eulerprompt}
\begin{euleroutput}
  1609.344
\end{euleroutput}
\begin{eulercomment}
Beberapa satuan yang sudah dikenal di dalam EMT adalah sebagai
berikut. Semua unit diakhiri dengan tanda dolar (\textdollar{}), namun boleh tidak
perlu ditulis dengan mengaktifkan easyunits. 

kilometer\textdollar{}:=1000;\\
km\textdollar{}:=kilometer\textdollar{};\\
cm\textdollar{}:=0.01;\\
mm\textdollar{}:=0.001;\\
minute\textdollar{}:=60;\\
min\textdollar{}:=minute\textdollar{};\\
minutes\textdollar{}:=minute\textdollar{};\\
hour\textdollar{}:=60*minute\textdollar{};\\
h\textdollar{}:=hour\textdollar{};\\
hours\textdollar{}:=hour\textdollar{};\\
day\textdollar{}:=24*hour\textdollar{};\\
days\textdollar{}:=day\textdollar{};\\
d\textdollar{}:=day\textdollar{};\\
year\textdollar{}:=365.2425*day\textdollar{};\\
years\textdollar{}:=year\textdollar{};\\
y\textdollar{}:=year\textdollar{};\\
inch\textdollar{}:=0.0254;\\
in\textdollar{}:=inch\textdollar{};\\
feet\textdollar{}:=12*inch\textdollar{};\\
foot\textdollar{}:=feet\textdollar{};\\
ft\textdollar{}:=feet\textdollar{};\\
yard\textdollar{}:=3*feet\textdollar{};\\
yards\textdollar{}:=yard\textdollar{};\\
yd\textdollar{}:=yard\textdollar{};\\
mile\textdollar{}:=1760*yard\textdollar{};\\
miles\textdollar{}:=mile\textdollar{};\\
kg\textdollar{}:=1;\\
sec\textdollar{}:=1;\\
ha\textdollar{}:=10000;\\
Ar\textdollar{}:=100;\\
Tagwerk\textdollar{}:=3408;\\
Acre\textdollar{}:=4046.8564224;\\
pt\textdollar{}:=0.376mm;

Untuk konversi ke dan antar unit, EMT menggunakan operator khusus,
yakni -\textgreater{}.
\end{eulercomment}
\begin{eulerprompt}
>4km -> miles, 4inch -> " mm"
\end{eulerprompt}
\begin{euleroutput}
  2.48548476895
  101.6 mm
\end{euleroutput}
\eulerheading{Format Tampilan Nilai}
\begin{eulercomment}
Akurasi internal untuk nilai bilangan di EMT adalah standar IEEE,
sekitar 16 digit desimal. Aslinya, EMT tidak mencetak semua digit
suatu bilangan. Ini untuk menghemat tempat dan agar terlihat lebih
baik. Untuk mengatrtamilan satu bilangan, operator berikut dapat
digunakan.

\end{eulercomment}
\begin{eulerprompt}
>pi
\end{eulerprompt}
\begin{euleroutput}
  3.14159265359
\end{euleroutput}
\begin{eulerprompt}
>longest pi
\end{eulerprompt}
\begin{euleroutput}
        3.141592653589793 
\end{euleroutput}
\begin{eulerprompt}
>long pi
\end{eulerprompt}
\begin{euleroutput}
  3.14159265359
\end{euleroutput}
\begin{eulerprompt}
>short pi
\end{eulerprompt}
\begin{euleroutput}
  3.1416
\end{euleroutput}
\begin{eulerprompt}
>shortest pi
\end{eulerprompt}
\begin{euleroutput}
     3.1 
\end{euleroutput}
\begin{eulerprompt}
>fraction pi
\end{eulerprompt}
\begin{euleroutput}
  312689/99532
\end{euleroutput}
\begin{eulerprompt}
>short 1200*1.03^10, long E, longest pi
\end{eulerprompt}
\begin{euleroutput}
  1612.7
  2.71828182846
        3.141592653589793 
\end{euleroutput}
\begin{eulercomment}
Format aslinya untuk menampilkan nilai menggunakan sekitar 10 digit.
Format tampilan nilai dapat diatur secara global atau hanya untuk satu
nilai.

Anda dapat mengganti format tampilan bilangan untuk semua perintah
selanjutnya. Untuk mengembalikan ke format aslinya dapat digunakan
perintah "defformat" atau "reset".
\end{eulercomment}
\begin{eulerprompt}
>longestformat; pi, defformat; pi
\end{eulerprompt}
\begin{euleroutput}
  3.141592653589793
  3.14159265359
\end{euleroutput}
\begin{eulercomment}
Kernel numerik EMT bekerja dengan bilangan titik mengambang (floating point)
dalam presisi ganda IEEE (berbeda dengan bagian simbolik EMT). Hasil numerik
dapat ditampilkan dalam bentuk pecahan.
\end{eulercomment}
\begin{eulerprompt}
>1/7+1/4, fraction %
\end{eulerprompt}
\begin{euleroutput}
  0.392857142857
  11/28
\end{euleroutput}
\eulerheading{Perintah Multibaris}
\begin{eulercomment}
Perintah multi-baris membentang di beberapa baris yang terhubung
dengan "..." di setiap akhir baris, kecuali baris terakhir. Untuk
menghasilkan tanda pindah baris tersebut, gunakan tombol
[Ctrl]+[Enter]. Ini akan menyambung perintah ke baris berikutnya dan
menambahkan "..." di akhir baris sebelumnya. Untuk menggabungkan suatu
baris ke baris sebelumnya, gunakan [Ctrl]+[Backspace].

Contoh perintah multi-baris berikut dapat dijalankan setiap kali
kursor berada di salah satu barisnya. Ini juga menunjukkan bahwa ...
harus berada di akhir suatu baris meskipun baris tersebut memuat
komentar.
\end{eulercomment}
\begin{eulerprompt}
>a=4; b=15; c=2; // menyelesaikan a*x^2+b*x+c=0 secara manual ...
>D=sqrt(b^2/(a^2*4)-c/a); ...
>-b/(2*a) + D, ...
>-b/(2*a) - D
\end{eulerprompt}
\begin{euleroutput}
  -0.138444501319
  -3.61155549868
\end{euleroutput}
\eulerheading{Menampilkan Daftar Variabe}
\begin{eulercomment}
Untuk menampilkan semua variabel yang sudah pernah Anda definisikan
sebelumnya (dan dapat dilihat kembali nilainya), gunakan perintah
"listvar".
\end{eulercomment}
\begin{eulerprompt}
>listvar
\end{eulerprompt}
\begin{euleroutput}
  r                   1.25
  a                   4
  b                   15
  c                   2
  D                   1.73655549868123
\end{euleroutput}
\begin{eulercomment}
Perintah listvar hanya menampilkan variabel buatan pengguna.
Dimungkinkan untuk menampilkan variabel lain, dengan menambahkan
string  termuat di dalam nama variabel yang diinginkan.

Perlu Anda perhatikan, bahwa EMT membedakan huruf besar dan huruf
kecil. Jadi variabel "d" berbeda dengan variabel "D".

Contoh berikut ini menampilkan semua unit yang diakhiri dengan "m"
dengan mencari semua variabel yang berisi "m\textdollar{}".
\end{eulercomment}
\begin{eulerprompt}
>listvar m$
\end{eulerprompt}
\begin{euleroutput}
  km$                 1000
  cm$                 0.01
  mm$                 0.001
  nm$                 1853.24496
  gram$               0.001
  m$                  1
  hquantum$           6.62606957e-34
  atm$                101325
\end{euleroutput}
\begin{eulercomment}
Untuk menghapus variabel tanpa harus memulai ulang EMT gunakan
perintah "remvalue".
\end{eulercomment}
\begin{eulerprompt}
>remvalue a,b,c,D
>D
\end{eulerprompt}
\begin{euleroutput}
  Variable D not found!
  Error in:
  D ...
   ^
\end{euleroutput}
\eulerheading{Menampilkan Panduan}
\begin{eulercomment}
Untuk mendapatkan panduan tentang penggunaan perintah atau fungsi di EMT, buka
jendela panduan dengan menekan [F1] dan cari fungsinya. Anda juga dapat
mengklik dua kali pada fungsi yang tertulis di baris perintah atau di teks
untuk membuka jendela panduan.

Coba klik dua kali pada perintah "intrandom" berikut ini!
\end{eulercomment}
\begin{eulerprompt}
>intrandom(10,6)
\end{eulerprompt}
\begin{euleroutput}
  [4,  2,  6,  2,  4,  2,  3,  2,  2,  6]
\end{euleroutput}
\begin{eulercomment}
Di jendela panduan, Anda dapat mengklik kata apa saja untuk menemukan
referensi atau fungsi.

Misalnya, coba klik kata "random" di jendela panduan. Kata tersebut
boleh ada dalam teks atau di bagian "See:" pada panduan. Anda akan
menemukan penjelasan fungsi "random", untuk menghasilkan bilangan acak
berdistribusi uniform antara 0,0 dan 1,0. Dari panduan untuk "random"
Anda dapat menampilkan panduan untuk fungsi "normal", dll.
\end{eulercomment}
\begin{eulerprompt}
>random(10)
\end{eulerprompt}
\begin{euleroutput}
  [0.270906,  0.704419,  0.217693,  0.445363,  0.308411,  0.914541,  0.193585,
  0.463387,  0.095153,  0.595017]
\end{euleroutput}
\begin{eulerprompt}
>normal(10)
\end{eulerprompt}
\begin{euleroutput}
  [-0.495418,  1.6463,  -0.390056,  -1.98151,  3.44132,  0.308178,  -0.733427,
  -0.526167,  1.10018,  0.108453]
\end{euleroutput}
\eulerheading{Matriks dan Vektor}
\begin{eulercomment}
EMT merupakan suatu aplikasi matematika yang mengerti "bahasa matriks". Artinya,
EMT menggunakan vektor dan matriks untuk perhitungan-perhitungan tingkat lanjut.
Suatu vektor atau matriks dapat didefinisikan dengan tanda kurung siku.
Elemen-elemennya dituliskan di dalam tanda kurung siku, antar elemen dalam satu
baris dipisahkan oleh koma(,), antar baris dipisahkan oleh titik koma (;).

Vektor dan matriks dapat diberi nama seperti variabel biasa.
\end{eulercomment}
\begin{eulerprompt}
>v=[4,5,6,3,2,1]
\end{eulerprompt}
\begin{euleroutput}
  [4,  5,  6,  3,  2,  1]
\end{euleroutput}
\begin{eulerprompt}
>A=[1,2,3;4,5,6;7,8,9]
\end{eulerprompt}
\begin{euleroutput}
              1             2             3 
              4             5             6 
              7             8             9 
\end{euleroutput}
\begin{eulercomment}
Karena EMT mengerti bahasa matriks, EMT memiliki kemampuan yang sangat canggih
untuk melakukan perhitungan matematis untuk masalah-masalah aljabar linier,
statistika, dan optimisasi.

Vektor juga dapat didefinisikan dengan menggunakan rentang nilai dengan interval
tertentu menggunakan tanda titik dua (:),seperti contoh berikut ini.
\end{eulercomment}
\begin{eulerprompt}
>c=1:5
\end{eulerprompt}
\begin{euleroutput}
  [1,  2,  3,  4,  5]
\end{euleroutput}
\begin{eulerprompt}
>w=0:0.1:1
\end{eulerprompt}
\begin{euleroutput}
  [0,  0.1,  0.2,  0.3,  0.4,  0.5,  0.6,  0.7,  0.8,  0.9,  1]
\end{euleroutput}
\begin{eulerprompt}
>mean(w^2)
\end{eulerprompt}
\begin{euleroutput}
  0.35
\end{euleroutput}
\eulerheading{Bilangan Kompleks}
\begin{eulercomment}
EMT juga dapat menggunakan bilangan kompleks. Tersedia banyak fungsi untuk
bilangan kompleks di EMT. Bilangan imaginer

\end{eulercomment}
\begin{eulerformula}
\[
i = \sqrt{-1}
\]
\end{eulerformula}
\begin{eulercomment}
dituliskan dengan huruf I (huruf besar I), namun akan ditampilkan dengan huruf i
(i kecil).

\end{eulercomment}
\begin{eulerttcomment}
  re(x) : bagian riil pada bilangan kompleks x.
  im(x) : bagian imaginer pada bilangan kompleks x.
  complex(x) : mengubah bilangan riil x menjadi bilangan kompleks.
  conj(x) : Konjugat untuk bilangan bilangan komplkes x.
  arg(x) : argumen (sudut dalam radian) bilangan kompleks x.
  real(x) : mengubah x menjadi bilangan riil.
\end{eulerttcomment}
\begin{eulercomment}

Apabila bagian imaginer x terlalu besar, hasilnya akan menampilkan pesan
kesalahan.

\end{eulercomment}
\begin{eulerttcomment}
  >sqrt(-1) // Error!
  >sqrt(complex(-1))
\end{eulerttcomment}
\begin{eulerprompt}
>z=2+3*I, re(z), im(z), conj(z), arg(z), deg(arg(z)), deg(arctan(3/2))
\end{eulerprompt}
\begin{euleroutput}
  2+3i
  2
  3
  2-3i
  0.982793723247
  56.309932474
  56.309932474
\end{euleroutput}
\begin{eulerprompt}
>deg(arg(I)) // 90°
\end{eulerprompt}
\begin{euleroutput}
  90
\end{euleroutput}
\begin{eulerprompt}
>sqrt(-1)
\end{eulerprompt}
\begin{euleroutput}
  Floating point error!
  Error in sqrt
  Error in:
  sqrt(-1) ...
          ^
\end{euleroutput}
\begin{eulerprompt}
>sqrt(complex(-1))
\end{eulerprompt}
\begin{euleroutput}
  0+1i
\end{euleroutput}
\begin{eulercomment}
EMT selalu menganggap semua hasil perhitungan berupa bilangan riil dan tidak
akan secara otomatis mengubah ke bilangan kompleks.

Jadi akar kuadrat -1 akan menghasilkan kesalahan, tetapi akar kuadrat kompleks
didefinisikan untuk bidang koordinat dengan cara seperti biasa. Untuk mengubah
bilangan riil menjadi kompleks, Anda dapat menambahkan 0i atau menggunakan
fungsi "complex".
\end{eulercomment}
\begin{eulerprompt}
>complex(-1), sqrt(%)
\end{eulerprompt}
\begin{euleroutput}
  -1+0i 
  0+1i
\end{euleroutput}
\eulerheading{Matematika Simbolik}
\begin{eulercomment}
EMT dapat melakukan perhitungan matematika simbolis (eksak) dengan bantuan
software Maxima. Software Maxima otomatis sudah terpasang di komputer Anda ketika
Anda memasang EMT. Meskipun demikian, Anda dapat juga memasang software Maxima
tersendiri (yang terpisah dengan instalasi Maxima di EMT).

Pengguna Maxima yang sudah mahir harus memperhatikan bahwa terdapat sedikit
perbedaan dalam sintaks antara sintaks asli Maxima dan sintaks ekspresi simbolik
di EMT.

Untuk melakukan perhitungan matematika simbolis di EMT, awali perintah Maxima
dengan tanda "\&". Setiap ekspresi yang dimulai dengan "\&" adalah ekspresi
simbolis dan dikerjakan oleh Maxima.
\end{eulercomment}
\begin{eulerprompt}
>&(a+b)^2
\end{eulerprompt}
\begin{euleroutput}
  
                                             2
                                      (b + a)
  
\end{euleroutput}
\begin{eulerprompt}
>&expand((a+b)^2), &factor(x^2+5*x+6)
\end{eulerprompt}
\begin{euleroutput}
  
                                    2            2
                                   b  + 2 a b + a
  
  
                                   (x + 2) (x + 3)
  
\end{euleroutput}
\begin{eulerprompt}
>&solve(a*x^2+b*x+c,x) // rumus abc
\end{eulerprompt}
\begin{euleroutput}
  
                            2                          2
                   (- sqrt(b  - 4 a c)) - b      sqrt(b  - 4 a c) - b
              [x = ------------------------, x = --------------------]
                             2 a                         2 a
  
\end{euleroutput}
\begin{eulerprompt}
>&(a^2-b^2)/(a+b), &ratsimp(%) // ratsimp menyederhanakan bentuk pecahan
\end{eulerprompt}
\begin{euleroutput}
  
                                        2    2
                                       a  - b
                                       -------
                                        b + a
  
  
                                        a - b
  
\end{euleroutput}
\begin{eulerprompt}
>10! // nilai faktorial (modus EMT)
\end{eulerprompt}
\begin{euleroutput}
  3628800
\end{euleroutput}
\begin{eulerprompt}
>&10! //nilai faktorial (simbolik dengan Maxima)
\end{eulerprompt}
\begin{euleroutput}
  
                                       3628800
  
\end{euleroutput}
\begin{eulercomment}
Untuk menggunakan perintah Maxima secara langsung (seperti perintah pada layar
Maxima) awali perintahnya dengan tanda "::" pada baris perintah EMT. Sintaks
Maxima disesuaikan dengan sintaks EMT (disebut "modus kompatibilitas").
\end{eulercomment}
\begin{eulerprompt}
>factor(1000) // mencari semua faktor 1000 (EMT)
\end{eulerprompt}
\begin{euleroutput}
  [2,  2,  2,  5,  5,  5]
\end{euleroutput}
\begin{eulerprompt}
>:: factor(1000) // faktorisasi prima 1000 (dengan Maxima) 
\end{eulerprompt}
\begin{euleroutput}
  
                                         3  3
                                        2  5
  
\end{euleroutput}
\begin{eulerprompt}
>:: factor(20!)
\end{eulerprompt}
\begin{euleroutput}
  
                               18  8  4  2
                              2   3  5  7  11 13 17 19
  
\end{euleroutput}
\begin{eulercomment}
Jika Anda sudah mahir menggunakan Maxima, Anda dapat menggunakan sintaks asli
perintah Maxima dengan menggunakan tanda ":::" untuk mengawali setiap perintah
Maxima di EMT. Perhatikan, harus ada spasi antara ":::" dan perintahnya.
\end{eulercomment}
\begin{eulerprompt}
>::: binomial(5,2); // nilai C(5,2)
\end{eulerprompt}
\begin{euleroutput}
  
                                         10
  
\end{euleroutput}
\begin{eulerprompt}
>::: binomial(m,4); // C(m,4)=m!/(4!(m-4)!)
\end{eulerprompt}
\begin{euleroutput}
  
                              (m - 3) (m - 2) (m - 1) m
                              -------------------------
                                         24
  
\end{euleroutput}
\begin{eulerprompt}
>::: trigexpand(cos(x+y)); // rumus cos(x+y)=cos(x) cos(y)-sin(x)sin(y) 
\end{eulerprompt}
\begin{euleroutput}
  
                            cos(x) cos(y) - sin(x) sin(y)
  
\end{euleroutput}
\begin{eulerprompt}
>::: trigexpand(sin(x+y));
\end{eulerprompt}
\begin{euleroutput}
  
                            cos(x) sin(y) + sin(x) cos(y)
  
\end{euleroutput}
\begin{eulerprompt}
>::: trigsimp(((1-sin(x)^2)*cos(x))/cos(x)^2+tan(x)*sec(x)^2) //menyederhanakan fungsi trigonometri
\end{eulerprompt}
\begin{euleroutput}
  
                                              4
                                  sin(x) + cos (x)
                                  ----------------
                                         3
                                      cos (x)
  
\end{euleroutput}
\begin{eulercomment}
Untuk menyimpan ekspresi simbolik ke dalam suatu variabel digunakan tanda "\&=".
\end{eulercomment}
\begin{eulerprompt}
>p1 &= (x^3+1)/(x+1)
\end{eulerprompt}
\begin{euleroutput}
  
                                        3
                                       x  + 1
                                       ------
                                       x + 1
  
\end{euleroutput}
\begin{eulerprompt}
>&ratsimp(p1)
\end{eulerprompt}
\begin{euleroutput}
  
                                      2
                                     x  - x + 1
  
\end{euleroutput}
\begin{eulercomment}
Untuk mensubstitusikan suatu nilai ke dalam variabel dapat digunakan perintah
"with".
\end{eulercomment}
\begin{eulerprompt}
>&p1 with x=3 // (3^3+1)/(3+1)
\end{eulerprompt}
\begin{euleroutput}
  
                                          7
  
\end{euleroutput}
\begin{eulerprompt}
>&p1 with x=a+b, &ratsimp(%) //substitusi dengan variabel baru
\end{eulerprompt}
\begin{euleroutput}
  
                                           3
                                    (b + a)  + 1
                                    ------------
                                     b + a + 1
  
  
                             2                  2
                            b  + (2 a - 1) b + a  - a + 1
  
\end{euleroutput}
\begin{eulerprompt}
>&diff(p1,x) //turunan p1 terhadap x
\end{eulerprompt}
\begin{euleroutput}
  
                                     2      3
                                  3 x      x  + 1
                                  ----- - --------
                                  x + 1          2
                                          (x + 1)
  
\end{euleroutput}
\begin{eulerprompt}
>&integrate(p1,x) // integral p1 terhadap x
\end{eulerprompt}
\begin{euleroutput}
  
                                     3      2
                                  2 x  - 3 x  + 6 x
                                  -----------------
                                          6
  
\end{euleroutput}
\eulerheading{Tampilan Matematika Simbolik dengan LaTeX}
\begin{eulercomment}
Anda dapat menampilkan hasil perhitunagn simbolik secara lebih bagus
menggunakan LaTeX. Untuk melakukan hal ini, tambahkan tanda dolar (\textdollar{}) di depan
tanda \& pada setiap perintah Maxima.\\
Perhatikan, hal ini hanya dapat menghasilkan tampilan yang diinginkan apabila
komputer Anda sudah terpasang software LaTeX.
\end{eulercomment}
\begin{eulerprompt}
>$&(a+b)^2
>$&expand((a+b)^2), $&factor(x^2+5*x+6)
>$&solve(a*x^2+b*x+c,x) // rumus abc
>$&(a^2-b^2)/(a+b), $&ratsimp(%)
\end{eulerprompt}
\eulerheading{Selamat Belajar dan Berlatih!}
\begin{eulercomment}
Baik, itulah sekilas pengantar penggunaan software EMT. Masih banyak kemampuan
EMT yang akan Anda pelajari dan praktikkan.

Sebagai latihan untuk memperlancar penggunaan perintah-perintah EMT yang sudah
dijelaskan di atas, silakan Anda lakukan hal-hal sebagai berikut.

- Carilah soal-soal matematika dari buku-buku Matematika.\\
- Tambahkan beberapa baris perintah EMT pada notebook ini.\\
- Selesaikan soal-soal matematika tersebut dengan menggunakan EMT.\\
Pilih soal-soal yang sesuai dengan perintah-perintah yang sudah dijelaskan dan
dicontohkan di atas.
\end{eulercomment}
\end{eulernotebook}
\end{document}


\newpage
\chapter{KB Pekan 3: Menggunakan EMT untuk menyelesaikan masalah-masalah Aljabar}
\documentclass[a4paper,10pt]{article}
\usepackage{eumat}

\begin{document}
\begin{eulernotebook}
\eulerheading{EMT untuk Perhitungan Aljabar}
\begin{eulercomment}
Rasyid Shalahuddin\\
22305144016\\
Matematika E

Pada notebook ini Anda belajar menggunakan EMT untuk melakukan
berbagai perhitungan terkait dengan materi atau topik dalam Aljabar.
Kegiatan yang harus Anda lakukan adalah sebagai berikut:

- Membaca secara cermat dan teliti notebook ini;\\
- Menerjemahkan teks bahasa Inggris ke bahasa Indonesia;\\
- Mencoba contoh-contoh perhitungan (perintah EMT) dengan cara
meng-ENTER setiap perintah EMT yang ada (pindahkan kursor ke baris
perintah)\\
- Jika perlu Anda dapat memodifikasi perintah yang ada dan memberikan
keterangan/penjelasan tambahan terkait hasilnya.\\
- Menyisipkan baris-baris perintah baru untuk mengerjakan soal-soal
Aljabar dari file PDF yang saya berikan;\\
- Memberi catatan hasilnya.\\
- Jika perlu tuliskan soalnya pada teks notebook (menggunakan format
LaTeX).\\
- Gunakan tampilan hasil semua perhitungan yang eksak atau simbolik
dengan format LaTeX. (Seperti contoh-contoh pada notebook ini.)

\end{eulercomment}
\eulersubheading{Contoh pertama}
\begin{eulercomment}
Menyederhanakan bentuk aljabar:

\end{eulercomment}
\begin{eulerformula}
\[
6x^{-3}y^5\times -7x^2y^{-9}
\]
\end{eulerformula}
\begin{eulercomment}
\end{eulercomment}
\begin{eulerprompt}
>$&6*x^(-3)*y^5*-7*x^2*y^(-9)
\end{eulerprompt}
\begin{eulercomment}
Menjabarkan:

\end{eulercomment}
\begin{eulerformula}
\[
(6x^{-3}+y^5)(-7x^2-y^{-9})
\]
\end{eulerformula}
\begin{eulerprompt}
>$&showev('expand((6*x^(-3)+y^5)*(-7*x^2-y^(-9))))
\end{eulerprompt}
\begin{eulercomment}
\end{eulercomment}
\eulersubheading{The Command Line}
\begin{eulercomment}
A command line of Euler consists of one or several Euler commands followed by
a semicolon ";" or a comma ",". The semicolon prevents the printing of the
result. The comma after the last command can be omitted.

The following command line will only print the result of the expression, not
the assignments or the format commands.
\end{eulercomment}
\begin{eulerprompt}
>r:=2; h:=4; pi*r^2*h/3
\end{eulerprompt}
\begin{euleroutput}
  16.7551608191
\end{euleroutput}
\begin{eulercomment}
Commands must be separated with a blank. The following command line prints
its two results.
\end{eulercomment}
\begin{eulerprompt}
>pi*2*r*h, %+2*pi*r*h // Ingat tanda % menyatakan hasil perhitungan terakhir sebelumnya
\end{eulerprompt}
\begin{euleroutput}
  50.2654824574
  100.530964915
\end{euleroutput}
\begin{eulercomment}
Command lines are executed in the order the user presses return. So you get a
new value each time you execute the second line.
\end{eulercomment}
\begin{eulerprompt}
>x := 1;
>x := cos(x) // nilai cosinus (x dalam radian)
\end{eulerprompt}
\begin{euleroutput}
  0.540302305868
\end{euleroutput}
\begin{eulerprompt}
>x := cos(x)
\end{eulerprompt}
\begin{euleroutput}
  0.857553215846
\end{euleroutput}
\begin{eulercomment}
If two lines are connected with "..." both lines will always execute
simultaneously.
\end{eulercomment}
\begin{eulerprompt}
>x := 1.5; ...
>x := (x+2/x)/2, x := (x+2/x)/2, x := (x+2/x)/2, 
\end{eulerprompt}
\begin{euleroutput}
  1.41666666667
  1.41421568627
  1.41421356237
\end{euleroutput}
\begin{eulercomment}
This is also a good way to spread a long command over two or more lines. You
can press Ctrl+Return to split a line in two at the current cursor position,
or Ctlr+Back to join the lines.

To fold all multi-lines press Ctrl+L. Then the subsequent lines will only be
visible, if one of them has the focus. To fold a single multi-line start the
first line with "\%+ ".
\end{eulercomment}
\begin{eulerprompt}
>%+ x=4+5; ...
\end{eulerprompt}
\begin{eulercomment}
A line starting with \%\% will be completely invisible.
\end{eulercomment}
\begin{euleroutput}
  81
\end{euleroutput}
\begin{eulercomment}
Euler supports loops in command lines, as long as they fit into one single
line or a multi-line. In programs, this restrictions does not hold, of
course. For more information consult the following introduction.

\end{eulercomment}
\begin{eulerprompt}
>x=1; for i=1 to 5; x := (x+2/x)/2, end; // menghitung akar 2
\end{eulerprompt}
\begin{euleroutput}
  1.5
  1.41666666667
  1.41421568627
  1.41421356237
  1.41421356237
\end{euleroutput}
\begin{eulercomment}
It is okay to use a multi-line. Make sure the line ends with " ...".
\end{eulercomment}
\begin{eulerprompt}
>x := 1.5; // comments go here before the ...
>repeat xnew:=(x+2/x)/2; until xnew~=x; ...
>   x := xnew; ...
>end; ...
>x,
\end{eulerprompt}
\begin{euleroutput}
  1.41421356237
\end{euleroutput}
\begin{eulercomment}
Conditional structures do also work.
\end{eulercomment}
\begin{eulerprompt}
>if E^pi>pi^E; then "Thought so!", endif;
\end{eulerprompt}
\begin{euleroutput}
  Thought so!
\end{euleroutput}
\begin{eulercomment}
When you execute a command, the cursor can be at any position in the command
line. You can go back to a previous command or skip to the next command with
the arrow keys. Or you can click into the comment section above the command
to go to the command.

When you move the cursor along the line the opening and closing pairs of
brackets or parentheses will highlight. Also, watch the status line. After
the opening bracket of the sqrt() function, the status line will display a
help text for the function. Execute the command with the return key.
\end{eulercomment}
\begin{eulerprompt}
>sqrt(sin(10°)/cos(20°))
\end{eulerprompt}
\begin{euleroutput}
  0.429875017772
\end{euleroutput}
\begin{eulercomment}
To see help for the most recent command, open the help window with F1. There,
you can enter text to search for. On an empty line, the help for the help
window will be displayed. You can press escape to clear the line, or to close
the help window.

You can double click on any command to open the help for this command. Try
double clicking the exp command below in the command line.
\end{eulercomment}
\begin{eulerprompt}
>exp(log(2.5))
\end{eulerprompt}
\begin{euleroutput}
  2.5
\end{euleroutput}
\begin{eulercomment}
You can copy and paste in Euler too. Use Ctrl-C and Ctrl-V for this. To mark
a text, drag the mouse or use shift together with any cursor key. Moreover,
you can copy the highlighted brackets.
\end{eulercomment}
\begin{eulercomment}

\end{eulercomment}
\eulersubheading{Basic Syntax}
\begin{eulercomment}
Euler knows the usual mathematical functions. As you have seen above, trigonometric
functions work in radian or degree. To convert to degrees, append the degree symbol
(with the F7 key) to the value, or use the function rad(x). The square root function is
called sqrt in Euler. Of course, x\textasciicircum{}(1/2) is also possible.

To set variables, use either "=" or ":=". For the sake of clarity, this introduction
uses the latter form. Spaces do not matter. But a space between commands is expected.

Multiple commands in one line are separated with "," or ";". The semicolon suppresses
the output of the command. At the end of the command line a "," is assumed, if ";" is
missing.
\end{eulercomment}
\begin{eulerprompt}
>g:=9.81; t:=2.5; 1/2*g*t^2
\end{eulerprompt}
\begin{euleroutput}
  30.65625
\end{euleroutput}
\begin{eulercomment}
EMT uses a programming syntax for expressions. To enter

\end{eulercomment}
\begin{eulerformula}
\[
e^2 \cdot \left( \frac{1}{3+4 \log(0.6)}+\frac{1}{7} \right)
\]
\end{eulerformula}
\begin{eulercomment}
you have to set the correct brackets and use / for fractions. Watch the highlighted
brackets for assistance. Note that the Euler constant e is named E in EMT.
\end{eulercomment}
\begin{eulerprompt}
>E^2*(1/(3+4*log(0.6))+1/7)
\end{eulerprompt}
\begin{euleroutput}
  8.77908249441
\end{euleroutput}
\begin{eulercomment}
To compute a complicate expression like

\end{eulercomment}
\begin{eulerformula}
\[
\left(\frac{\frac17 + \frac18 + 2}{\frac13 + \frac12}\right)^2 \pi
\]
\end{eulerformula}
\begin{eulercomment}
you need to enter it in line form.
\end{eulercomment}
\begin{eulerprompt}
>((1/7 + 1/8 + 2) / (1/3 + 1/2))^2 * pi
\end{eulerprompt}
\begin{euleroutput}
  23.2671801626
\end{euleroutput}
\begin{eulercomment}
Carefully put brackets around sub-expressions that need to be computed first. 
EMT assists you by highlighting the expression that the closing bracket finishes. 
You will also have to enter the name "pi" for the Greek letter pi.

The result of this computation is a floating point number. It is by
default printed with about 12 digits accuracy.
In the following command line, we also learn how we can refer to the previous
result within the same line.
\end{eulercomment}
\begin{eulerprompt}
>1/3+1/7, fraction %
\end{eulerprompt}
\begin{euleroutput}
  0.47619047619
  10/21
\end{euleroutput}
\begin{eulercomment}
An Euler command can be an expression or a primitive command. An expression
is made of operators and functions. If necessary, it must contain brackets to
force the correct order of execution. In doubt, setting a bracket is a good
idea. Note that EMT shows opening and closing brackets while editing the
command line.
\end{eulercomment}
\begin{eulerprompt}
>(cos(pi/4)+1)^3*(sin(pi/4)+1)^2
\end{eulerprompt}
\begin{euleroutput}
  14.4978445072
\end{euleroutput}
\begin{eulercomment}
The numerical operators of Euler include

\end{eulercomment}
\begin{eulerttcomment}
 + unary or operator plus
 - unary or operator minus
 *, /
 . the matrix product
 a^b power for positive a or integer b (a**b works too)
 n! the factorial operator
\end{eulerttcomment}
\begin{eulercomment}

and many more.

Here are some of the functions you might need. There are many more.

\end{eulercomment}
\begin{eulerttcomment}
 sin,cos,tan,atan,asin,acos,rad,deg
 log,exp,log10,sqrt,logbase
 bin,logbin,logfac,mod,floor,ceil,round,abs,sign
 conj,re,im,arg,conj,real,complex
 beta,betai,gamma,complexgamma,ellrf,ellf,ellrd,elle
 bitand,bitor,bitxor,bitnot
\end{eulerttcomment}
\begin{eulercomment}

Some commands have aliases, e.g. ln for log.
\end{eulercomment}
\begin{eulerprompt}
>ln(E^2), arctan(tan(0.5))
\end{eulerprompt}
\begin{euleroutput}
  2
  0.5
\end{euleroutput}
\begin{eulerprompt}
>sin(30°)
\end{eulerprompt}
\begin{euleroutput}
  0.5
\end{euleroutput}
\begin{eulercomment}
Make sure to use parentheses (round brackets), whenever there is doubt about
the order of execution! The following is not the same as (2\textasciicircum{}3)\textasciicircum{}4, which is
the default for 2\textasciicircum{}3\textasciicircum{}4 in EMT (some numerical systems do it the other way).
\end{eulercomment}
\begin{eulerprompt}
>2^3^4, (2^3)^4, 2^(3^4)
\end{eulerprompt}
\begin{euleroutput}
  2.41785163923e+24
  4096
  2.41785163923e+24
\end{euleroutput}
\eulersubheading{Real Numbers}
\begin{eulercomment}
The primary data type in Euler is the real number. Reals are
represented in IEEE format with about 16 decimal digits of accuracy.
\end{eulercomment}
\begin{eulerprompt}
>longest 1/3
\end{eulerprompt}
\begin{euleroutput}
       0.3333333333333333 
\end{euleroutput}
\begin{eulercomment}
The internal dual representation takes 8 bytes.
\end{eulercomment}
\begin{eulerprompt}
>printdual(1/3)
\end{eulerprompt}
\begin{euleroutput}
  1.0101010101010101010101010101010101010101010101010101*2^-2
\end{euleroutput}
\begin{eulerprompt}
>printhex(1/3)
\end{eulerprompt}
\begin{euleroutput}
  5.5555555555554*16^-1
\end{euleroutput}
\eulersubheading{Strings}
\begin{eulercomment}
A string in Euler is defined with "...".
\end{eulercomment}
\begin{eulerprompt}
>"A string can contain anything."
\end{eulerprompt}
\begin{euleroutput}
  A string can contain anything.
\end{euleroutput}
\begin{eulercomment}
Strings can be concatenated with \textbar{} or with +. This also works with numbers,
which are converted to strings in that case.
\end{eulercomment}
\begin{eulerprompt}
>"The area of the circle with radius " + 2 + " cm is " + pi*4 + " cm^2."
\end{eulerprompt}
\begin{euleroutput}
  The area of the circle with radius 2 cm is 12.5663706144 cm^2.
\end{euleroutput}
\begin{eulercomment}
The print function does also convert a number to a string. It can take a
number of digits and a number of places (0 for dense output), and optimally a
unit.
\end{eulercomment}
\begin{eulerprompt}
>"Golden Ratio : " + print((1+sqrt(5))/2,5,0)
\end{eulerprompt}
\begin{euleroutput}
  Golden Ratio : 1.61803
\end{euleroutput}
\begin{eulercomment}
There is a special string none, which does not print. It is returned by some
functions, when the result does not matter. (It is returned automatically, if
the function does not have a return statement.)
\end{eulercomment}
\begin{eulerprompt}
>none
\end{eulerprompt}
\begin{eulercomment}
To convert a string to a number simply evaluate it. This works for
expressions too (see below).
\end{eulercomment}
\begin{eulerprompt}
>"1234.5"()
\end{eulerprompt}
\begin{euleroutput}
  1234.5
\end{euleroutput}
\begin{eulercomment}
To define a string vector, use the vector [...] notation.
\end{eulercomment}
\begin{eulerprompt}
>v:=["affe","charlie","bravo"]
\end{eulerprompt}
\begin{euleroutput}
  affe
  charlie
  bravo
\end{euleroutput}
\begin{eulercomment}
The empty string vector is denoted by [none]. String vectors can be
concatenated.
\end{eulercomment}
\begin{eulerprompt}
>w:=[none]; w|v|v
\end{eulerprompt}
\begin{euleroutput}
  affe
  charlie
  bravo
  affe
  charlie
  bravo
\end{euleroutput}
\begin{eulercomment}
Strings can contain Unicode characters. Internally, these strings contain
UTF-8 code. To generate such a string, use u"..." and one of the HTML
entities.

Unicode strings can be concatenated like other strings.
\end{eulercomment}
\begin{eulerprompt}
>u"&alpha; = " + 45 + u"&deg;" // pdfLaTeX mungkin gagal menampilkan secara benar
\end{eulerprompt}
\begin{euleroutput}
  α = 45°
\end{euleroutput}
\begin{eulercomment}
I
\end{eulercomment}
\begin{eulercomment}
In comments, the same entities like α, β etc. can be used. This may be
a quick alternative to Latex. (More details on comments below).
\end{eulercomment}
\begin{eulercomment}
There are some functions to create or analyze unicode strings. The function
strtochar() will recognize Unicode strings, and translate them correctly.
\end{eulercomment}
\begin{eulerprompt}
>v=strtochar(u"&Auml; is a German letter")
\end{eulerprompt}
\begin{euleroutput}
  [196,  32,  105,  115,  32,  97,  32,  71,  101,  114,  109,  97,  110,
  32,  108,  101,  116,  116,  101,  114]
\end{euleroutput}
\begin{eulercomment}
The result is a vector of Unicode numbers. The converse function is
chartoutf().
\end{eulercomment}
\begin{eulerprompt}
>v[1]=strtochar(u"&Uuml;")[1]; chartoutf(v)
\end{eulerprompt}
\begin{euleroutput}
  Ü is a German letter
\end{euleroutput}
\begin{eulercomment}
The function utf() can translate a string with entities in a variable into a
Unicode string.
\end{eulercomment}
\begin{eulerprompt}
>s="We have &alpha;=&beta;."; utf(s) // pdfLaTeX mungkin gagal menampilkan secara benar
\end{eulerprompt}
\begin{euleroutput}
  We have α=β.
\end{euleroutput}
\begin{eulercomment}
It is also possible to use numerical entities.
\end{eulercomment}
\begin{eulerprompt}
>u"&#196;hnliches"
\end{eulerprompt}
\begin{euleroutput}
  Ähnliches
\end{euleroutput}
\eulersubheading{Boolean Values}
\begin{eulercomment}
Boolean values are represented with 1=true or 0=false in Euler.
Strings can be compared, just like numbers.
\end{eulercomment}
\begin{eulerprompt}
>2<1, "apel"<"banana"
\end{eulerprompt}
\begin{euleroutput}
  0
  1
\end{euleroutput}
\begin{eulercomment}
"and" is the operator "\&\&" and "or" is the operator "\textbar{}\textbar{}", as in the C
language. (The words "and" and "or" can only be used in conditions for "if".)
\end{eulercomment}
\begin{eulerprompt}
>2<E && E<3
\end{eulerprompt}
\begin{euleroutput}
  1
\end{euleroutput}
\begin{eulercomment}
Boolean operators obey the rules of the matrix language.
\end{eulercomment}
\begin{eulerprompt}
>(1:10)>5, nonzeros(%)
\end{eulerprompt}
\begin{euleroutput}
  [0,  0,  0,  0,  0,  1,  1,  1,  1,  1]
  [6,  7,  8,  9,  10]
\end{euleroutput}
\begin{eulercomment}
You can use the function nonzeros() to extract specific elements form a
vector. In the example, we use the conditional isprime(n).
\end{eulercomment}
\begin{eulerprompt}
>N=2|3:2:99 // N berisi elemen 2 dan bilangan2 ganjil dari 3 s.d. 99
\end{eulerprompt}
\begin{euleroutput}
  [2,  3,  5,  7,  9,  11,  13,  15,  17,  19,  21,  23,  25,  27,  29,
  31,  33,  35,  37,  39,  41,  43,  45,  47,  49,  51,  53,  55,  57,
  59,  61,  63,  65,  67,  69,  71,  73,  75,  77,  79,  81,  83,  85,
  87,  89,  91,  93,  95,  97,  99]
\end{euleroutput}
\begin{eulerprompt}
>N[nonzeros(isprime(N))] //pilih anggota2 N yang prima
\end{eulerprompt}
\begin{euleroutput}
  [2,  3,  5,  7,  11,  13,  17,  19,  23,  29,  31,  37,  41,  43,  47,
  53,  59,  61,  67,  71,  73,  79,  83,  89,  97]
\end{euleroutput}
\eulersubheading{Output Formats}
\begin{eulercomment}
The default output format of EMT prints 12 digits. To make sure that
we see the default, we reset the format.
\end{eulercomment}
\begin{eulerprompt}
>defformat; pi
\end{eulerprompt}
\begin{euleroutput}
  3.14159265359
\end{euleroutput}
\begin{eulercomment}
Internally, EMT uses the IEEE standard for double numbers with about 16
decimal digits. To see the full number of digits, use the command
"longestformat", or we use the operator "longest" to display the result in
the longest format.
\end{eulercomment}
\begin{eulerprompt}
>longest pi
\end{eulerprompt}
\begin{euleroutput}
        3.141592653589793 
\end{euleroutput}
\begin{eulercomment}
Here is the internal hexadecimal representation of a double number.
\end{eulercomment}
\begin{eulerprompt}
>printhex(pi)
\end{eulerprompt}
\begin{euleroutput}
  3.243F6A8885A30*16^0
\end{euleroutput}
\begin{eulercomment}
The output format can be changed permanently with a format command.
\end{eulercomment}
\begin{eulerprompt}
>format(12,5); 1/3, pi, sin(1)
\end{eulerprompt}
\begin{euleroutput}
      0.33333 
      3.14159 
      0.84147 
\end{euleroutput}
\begin{eulercomment}
The default is format(12).
\end{eulercomment}
\begin{eulerprompt}
>format(12); 1/3
\end{eulerprompt}
\begin{euleroutput}
  0.333333333333
\end{euleroutput}
\begin{eulercomment}
Functions like "shortestformat", "shortformat", "longformat" work for vectors
in the following way.
\end{eulercomment}
\begin{eulerprompt}
>shortestformat; random(3,8)
\end{eulerprompt}
\begin{euleroutput}
    0.66    0.2   0.89   0.28   0.53   0.31   0.44    0.3 
    0.28   0.88   0.27    0.7   0.22   0.45   0.31   0.91 
    0.19   0.46  0.095    0.6   0.43   0.73   0.47   0.32 
\end{euleroutput}
\begin{eulercomment}
The default format for scalars is format(12). But this can be changed.
\end{eulercomment}
\begin{eulerprompt}
>setscalarformat(5); pi
\end{eulerprompt}
\begin{euleroutput}
  3.1416
\end{euleroutput}
\begin{eulercomment}
The function "longestformat" set the scalar format too.
\end{eulercomment}
\begin{eulerprompt}
>longestformat; pi
\end{eulerprompt}
\begin{euleroutput}
  3.141592653589793
\end{euleroutput}
\begin{eulercomment}
For reference, here is a list of the most important output formats.

\end{eulercomment}
\begin{eulerttcomment}
 shortestformat shortformat longformat, longestformat
 format(length,digits) goodformat(length)
 fracformat(length)
 defformat
\end{eulerttcomment}
\begin{eulercomment}

The internal accuracy of EMT is about 16 decimal places, which is the IEEE
standard. Numbers are stored in this internal format.

But the output format of EMT can be set in a flexible way.
\end{eulercomment}
\begin{eulerprompt}
>longestformat; pi,
\end{eulerprompt}
\begin{euleroutput}
  3.141592653589793
\end{euleroutput}
\begin{eulerprompt}
>format(10,5); pi
\end{eulerprompt}
\begin{euleroutput}
    3.14159 
\end{euleroutput}
\begin{eulercomment}
The default is defformat().
\end{eulercomment}
\begin{eulerprompt}
>defformat; // default
\end{eulerprompt}
\begin{eulercomment}
There are short operators which print only one value. The operator "longest"
will print all valid digits of a number.
\end{eulercomment}
\begin{eulerprompt}
>longest pi^2/2
\end{eulerprompt}
\begin{euleroutput}
        4.934802200544679 
\end{euleroutput}
\begin{eulercomment}
There is also a short operator for printing a result in fractional format. We
have already used it above.
\end{eulercomment}
\begin{eulerprompt}
>fraction 1+1/2+1/3+1/4
\end{eulerprompt}
\begin{euleroutput}
  25/12
\end{euleroutput}
\begin{eulercomment}
Since the internal format uses a binary way to store numbers, the value 0.1
will not be represented exactly. The error adds up a bit, as you see in the
following computation.
\end{eulercomment}
\begin{eulerprompt}
>longest 0.1+0.1+0.1+0.1+0.1+0.1+0.1+0.1+0.1+0.1-1
\end{eulerprompt}
\begin{euleroutput}
   -1.110223024625157e-16 
\end{euleroutput}
\begin{eulercomment}
But with the default "longformat" you will not notice this. For convenience,
the output of very small numbers is 0.
\end{eulercomment}
\begin{eulerprompt}
>0.1+0.1+0.1+0.1+0.1+0.1+0.1+0.1+0.1+0.1-1
\end{eulerprompt}
\begin{euleroutput}
  0
\end{euleroutput}
\eulerheading{Expressions}
\begin{eulercomment}
Strings or names can be used to store mathematical expressions, which can be evaluated
by EMT. For this, use parentheses after the expression. If you intend to use a string
as an expression, use the convention to name it "fx" or "fxy" etc. Expressions take
precedence over functions.

Global variables can be used in the evaluation.
\end{eulercomment}
\begin{eulerprompt}
>r:=2; fx:="pi*r^2"; longest fx()
\end{eulerprompt}
\begin{euleroutput}
        12.56637061435917 
\end{euleroutput}
\begin{eulercomment}
Parameters are assigned to x, y, and z in that order. Additional parameters
can be added using assigned parameters.
\end{eulercomment}
\begin{eulerprompt}
>fx:="a*sin(x)^2"; fx(5,a=-1)
\end{eulerprompt}
\begin{euleroutput}
  -0.919535764538
\end{euleroutput}
\begin{eulercomment}
Note that expression will always use global variables, even if there is a
variable in a function with the same name. (Otherwise the evaluation of
expressions in functions could have very confusing results for the user that
called the function.)
\end{eulercomment}
\begin{eulerprompt}
>at:=4; function f(expr,x,at) := expr(x); ...
>f("at*x^2",3,5) // computes 4*3^2 not 5*3^2
\end{eulerprompt}
\begin{euleroutput}
  36
\end{euleroutput}
\begin{eulercomment}
If you want to use another value for "at" than the global value you need to
add "at=value".
\end{eulercomment}
\begin{eulerprompt}
>at:=4; function f(expr,x,a) := expr(x,at=a); ...
>f("at*x^2",3,5)
\end{eulerprompt}
\begin{euleroutput}
  45
\end{euleroutput}
\begin{eulercomment}
For reference, we remark that call collections (discussed elsewhere) can
contain expressions. So we can make the above example as follows.
\end{eulercomment}
\begin{eulerprompt}
>at:=4; function f(expr,x) := expr(x); ...
>f(\{\{"at*x^2",at=5\}\},3)
\end{eulerprompt}
\begin{euleroutput}
  45
\end{euleroutput}
\begin{eulercomment}
Expressions in x are often used just like functions.\\
Note that defining a function with the same name like a global symbolic
expression deletes this variable to avoid confusion between symbolic
expressions and functions.
\end{eulercomment}
\begin{eulerprompt}
>f &= 5*x;
>function f(x) := 6*x;
>f(2)
\end{eulerprompt}
\begin{euleroutput}
  12
\end{euleroutput}
\begin{eulercomment}
By way of convention, symbolic or numerical expressions should be named fx,
fxy etc. This naming scheme should not be used for functions.
\end{eulercomment}
\begin{eulerprompt}
>fx &= diff(x^x,x); $&fx
\end{eulerprompt}
\begin{eulercomment}
A special form of an expression allows any variable as an unnamed parameter
to the evaluation of the expression, not just "x", "y" etc. For this, start
the expression with "@(variables) ...".
\end{eulercomment}
\begin{eulerprompt}
>"@(a,b) a^2+b^2", %(4,5)
\end{eulerprompt}
\begin{euleroutput}
  @(a,b) a^2+b^2
  41
\end{euleroutput}
\begin{eulercomment}
This allows to manipulate expressions in other variables for functions of EMT
which need an expression in "x".

The most elementary way to define a simple function is to store its formula
in a symbolic or numerical expression. If the main variable is x, the
expression can be evaluated just like a function.

As you see in the following example, global variables are visible during the
evaluation.
\end{eulercomment}
\begin{eulerprompt}
>fx &= x^3-a*x;  ...
>a=1.2; fx(0.5)
\end{eulerprompt}
\begin{euleroutput}
  -0.475
\end{euleroutput}
\begin{eulercomment}
All other variables in the expression can be specified in the evaluation
using an assigned parameter.
\end{eulercomment}
\begin{eulerprompt}
>fx(0.5,a=1.1)
\end{eulerprompt}
\begin{euleroutput}
  -0.425
\end{euleroutput}
\begin{eulercomment}
An expression needs not be symbolic. This is necessary, if the expression
contains functions, which are only known in the numerical kernel, not in
Maxima.

\begin{eulercomment}
\eulerheading{Symbolic Mathematics}
\begin{eulercomment}
EMT does symbolic math with the help of Maxima. For details, start with the
following tutorial, or browse the reference for Maxima. Experts in Maxima
should note that there are differences in the syntax between the original
syntax of Maxima and the default syntax of symbolic expressions in EMT.

Symbolic math is integrated seamlessly into Euler with \&. Any expression
starting with \& is a symbolic expression. It is evaluated and printed by
Maxima.

First of all, Maxima has an "infinite" arithmetic which can handle
very large numbers.
\end{eulercomment}
\begin{eulerprompt}
>$&44!
\end{eulerprompt}
\begin{eulercomment}
This way, you can compute large results exactly. Let us compute

\end{eulercomment}
\begin{eulerformula}
\[
C(44,10) = \frac{44!}{34! \cdot 10!}
\]
\end{eulerformula}
\begin{eulerprompt}
>$& 44!/(34!*10!) // nilai C(44,10)
\end{eulerprompt}
\begin{eulercomment}
Of course, Maxima has a more efficient function for this (as does the
numerical part of EMT).
\end{eulercomment}
\begin{eulerprompt}
>$binomial(44,10) //menghitung C(44,10) menggunakan fungsi binomial()
\end{eulerprompt}
\begin{eulercomment}
To learn more about a specific function double click on it. E.g., try double clicking
on "\&binomial" in the previous command line. This opens the documentation of Maxima as
provided by the authors of that program.

You will learn that the following works too.

\end{eulercomment}
\begin{eulerformula}
\[
C(x,3)=\frac{x!}{(x-3)!3!}=\frac{(x-2)(x-1)x}{6}
\]
\end{eulerformula}
\begin{eulerprompt}
>$binomial(x,3) // C(x,3)
\end{eulerprompt}
\begin{eulercomment}
If you want to replace x with any specific value use "with".
\end{eulercomment}
\begin{eulerprompt}
>$&binomial(x,3) with x=10 // substitusi x=10 ke C(x,3)
\end{eulerprompt}
\begin{eulercomment}
That way you can use a solution of an equation in another equation.

Symbolic expressions are printed by Maxima in 2D form. The reason for this is a special
symbolic flag in the string.

As you will have seen in previous and following examples, if you have LaTeX installed,
you can print a symbolic expression with Latex. If not, the following command will
issue an error message.

To print a symbolic expression with LaTeX, use \textdollar{} infront of \& (or you may ommit \&)
before the command. Do not run the Maxima commands with \textdollar{}, if you don't have LaTeX
installed.
\end{eulercomment}
\begin{eulerprompt}
>$(3+x)/(x^2+1)
\end{eulerprompt}
\begin{eulercomment}
Symbolic expressions are parsed by Euler. If you need a complex syntax in one
expression, you can enclose the expression in "...". To use more than a
simple expression is possible, but strongly discouraged.
\end{eulercomment}
\begin{eulerprompt}
>&"v := 5; v^2"
\end{eulerprompt}
\begin{euleroutput}
  
                                    25
  
\end{euleroutput}
\begin{eulercomment}
For completeness, we remark that symbolic expressions can be used in
programs, but need to be enclosed in quotes. Moreover, it is much more
effective to call Maxima at compile time if possible.
\end{eulercomment}
\begin{eulerprompt}
>$&expand((1+x)^4), $&factor(diff(%,x)) // diff: turunan, factor: faktor
\end{eulerprompt}
\begin{eulercomment}
Again, \% refers to the previous result.

To make things easier we save the solution to a symbolic variable.
Symbolic variables are defined with "\&=".
\end{eulercomment}
\begin{eulerprompt}
>fx &= (x+1)/(x^4+1); $&fx
\end{eulerprompt}
\begin{eulercomment}
Symbolic expressions can be used in other symbolic expressions.
\end{eulercomment}
\begin{eulerprompt}
>$&factor(diff(fx,x))
\end{eulerprompt}
\begin{eulercomment}
A direct input of Maxima commands is available too. Start the command line
with "::". The syntax of Maxima is adapted to the syntax of EMT (called the
"compatibility mode").
\end{eulercomment}
\begin{eulerprompt}
>&factor(20!)
\end{eulerprompt}
\begin{euleroutput}
  
                           2432902008176640000
  
\end{euleroutput}
\begin{eulerprompt}
>::: factor(10!)
\end{eulerprompt}
\begin{euleroutput}
  
                                 8  4  2
                                2  3  5  7
  
\end{euleroutput}
\begin{eulerprompt}
>:: factor(20!)
\end{eulerprompt}
\begin{euleroutput}
  
                          18  8  4  2
                         2   3  5  7  11 13 17 19
  
\end{euleroutput}
\begin{eulercomment}
If you are an expert in Maxima, you may wish to use the original syntax of
Maxima. You can do this with ":::".
\end{eulercomment}
\begin{eulerprompt}
>::: av:g$ av^2;
\end{eulerprompt}
\begin{euleroutput}
  
                                     2
                                    g
  
\end{euleroutput}
\begin{eulerprompt}
>fx &= x^3*exp(x), $fx
\end{eulerprompt}
\begin{euleroutput}
  
                                   3  x
                                  x  E
  
\end{euleroutput}
\begin{eulercomment}
Such variables can be used in other symbolic expressions. Note, that in the
following command the right hand side of \&= is evaluated before the
assignment to Fx.
\end{eulercomment}
\begin{eulerprompt}
>&(fx with x=5), $%, &float(%)
\end{eulerprompt}
\begin{euleroutput}
  
                                       5
                                  125 E
  
  
                            18551.64488782208
  
\end{euleroutput}
\begin{eulerprompt}
>fx(5)
\end{eulerprompt}
\begin{euleroutput}
  18551.6448878
\end{euleroutput}
\begin{eulercomment}
For the evaluation of an expression with specific values of the variables,
you can use the "with" operator.

The following command line also demonstrates that Maxima can evaluate an
expression numerically with float().
\end{eulercomment}
\begin{eulerprompt}
>&(fx with x=10)-(fx with x=5), &float(%)
\end{eulerprompt}
\begin{euleroutput}
  
                                  10        5
                            1000 E   - 125 E
  
  
                           2.20079141499189e+7
  
\end{euleroutput}
\begin{eulerprompt}
>$factor(diff(fx,x,2))
\end{eulerprompt}
\begin{eulercomment}
To get the Latex code for an expression, you can use the tex command.
\end{eulercomment}
\begin{eulerprompt}
>tex(fx)
\end{eulerprompt}
\begin{euleroutput}
  x^3\(\backslash\),e^\{x\}
\end{euleroutput}
\begin{eulercomment}
Symbolic expressions can be evaluated just like numerical expressions.
\end{eulercomment}
\begin{eulerprompt}
>fx(0.5)
\end{eulerprompt}
\begin{euleroutput}
  0.206090158838
\end{euleroutput}
\begin{eulercomment}
In symbolic expressions, this does not work, since Maxima does not support
it. Instead, use the "with" syntax (a nicer form of the at(...) command of
Maxima).
\end{eulercomment}
\begin{eulerprompt}
>$&fx with x=1/2
\end{eulerprompt}
\begin{eulercomment}
The assignment can also be symbolic.
\end{eulercomment}
\begin{eulerprompt}
>$&fx with x=1+t
\end{eulerprompt}
\begin{eulercomment}
The command solve solves symbolic expressions for a variable in Maxima. The
result is a vector of solutions.
\end{eulercomment}
\begin{eulerprompt}
>$&solve(x^2+x=4,x)
\end{eulerprompt}
\begin{eulercomment}
Compare with the numerical "solve" command in Euler, which needs a start
value, and optionally a target value.
\end{eulercomment}
\begin{eulerprompt}
>solve("x^2+x",1,y=4)
\end{eulerprompt}
\begin{euleroutput}
  1.56155281281
\end{euleroutput}
\begin{eulercomment}
The numerical values of the symbolic solution can be computed by evaluation
of the symbolic result. Euler will read over the assignments x= etc. If you
do not need the numerical results for further computations you can also let
Maxima find the numerical values.
\end{eulercomment}
\begin{eulerprompt}
>sol &= solve(x^2+2*x=4,x); $&sol, sol(), $&float(sol)
\end{eulerprompt}
\begin{euleroutput}
  [-3.23607,  1.23607]
\end{euleroutput}
\begin{eulercomment}
To get a specific symbolic solution, one can use "with" and an index.
\end{eulercomment}
\begin{eulerprompt}
>$&solve(x^2+x=1,x), x2 &= x with %[2]; $&x2
\end{eulerprompt}
\begin{eulercomment}
To solve a system of equations, use a vector of equations. The result is a
vector of solutions.
\end{eulercomment}
\begin{eulerprompt}
>sol &= solve([x+y=3,x^2+y^2=5],[x,y]); $&sol, $&x*y with sol[1]
\end{eulerprompt}
\begin{eulercomment}
Symbolic expressions can have flags, which indicate a special treatment in
Maxima. Some flags can be used as commands too, others can't. Flags are
appended with "\textbar{}" (a nicer form of "ev(...,flags)")
\end{eulercomment}
\begin{eulerprompt}
>$& diff((x^3-1)/(x+1),x) //turunan bentuk pecahan
>$& diff((x^3-1)/(x+1),x) | ratsimp //menyederhanakan pecahan
>$&factor(%)
\end{eulerprompt}
\eulerheading{Functions}
\begin{eulercomment}
In EMT, functions are programs defined with the command "function". It can be a
one-line function or multiline function.\\
A one-line function can be numerical or symbolic. A numerical one-line function is
defined by ":=".
\end{eulercomment}
\begin{eulerprompt}
>function f(x) := x*sqrt(x^2+1)
\end{eulerprompt}
\begin{eulercomment}
For an overview, we show all possible definitions for one-line functions. A
function can be evaluated just like any built-in Euler function.
\end{eulercomment}
\begin{eulerprompt}
>f(2)
\end{eulerprompt}
\begin{euleroutput}
  4.472135955
\end{euleroutput}
\begin{eulercomment}
This function will work for vectors too, obeying the matrix language of
Euler, since the expressions used in the function are vectorized.
\end{eulercomment}
\begin{eulerprompt}
>f(0:0.1:1)
\end{eulerprompt}
\begin{euleroutput}
  [0,  0.100499,  0.203961,  0.313209,  0.430813,  0.559017,  0.699714,
  0.854459,  1.0245,  1.21083,  1.41421]
\end{euleroutput}
\begin{eulercomment}
Functions can be plotted. Instead of expressions, we need only provide the
function name.

In contrast to symbolic or numerical expressions, the function name must be
provided in a string.
\end{eulercomment}
\begin{eulerprompt}
>solve("f",1,y=1)
\end{eulerprompt}
\begin{euleroutput}
  0.786151377757
\end{euleroutput}
\begin{eulercomment}
By default, if you need to overwrite a built-in function, you must add the
keyword "overwrite". Overwriting built-in functions is dangerous and can
cause problems for other functions depending on them.

You can still call the built-in function as "\_...", if it is function in the
Euler core.
\end{eulercomment}
\begin{eulerprompt}
>function overwrite sin (x) := _sin(x°) // redine sine in degrees
>sin(45)
\end{eulerprompt}
\begin{euleroutput}
  0.707106781187
\end{euleroutput}
\begin{eulercomment}
We better remove this redefinition of sin.
\end{eulercomment}
\begin{eulerprompt}
>forget sin; sin(pi/4)
\end{eulerprompt}
\begin{euleroutput}
  0.707106781187
\end{euleroutput}
\eulersubheading{Default Parameters}
\begin{eulercomment}
Numerical function can have default parameters.
\end{eulercomment}
\begin{eulerprompt}
>function f(x,a=1) := a*x^2
\end{eulerprompt}
\begin{eulercomment}
Omitting this parameter uses the default value.
\end{eulercomment}
\begin{eulerprompt}
>f(4)
\end{eulerprompt}
\begin{euleroutput}
  16
\end{euleroutput}
\begin{eulercomment}
Setting it overwrites the default value.
\end{eulercomment}
\begin{eulerprompt}
>f(4,5)
\end{eulerprompt}
\begin{euleroutput}
  80
\end{euleroutput}
\begin{eulercomment}
An assigned parameter overwrite it too. This is used by many Euler functions
like plot2d, plot3d.
\end{eulercomment}
\begin{eulerprompt}
>f(4,a=1)
\end{eulerprompt}
\begin{euleroutput}
  16
\end{euleroutput}
\begin{eulercomment}
If a variable is not a parameter, it must be global. One-line functions can
see global variables.
\end{eulercomment}
\begin{eulerprompt}
>function f(x) := a*x^2
>a=6; f(2)
\end{eulerprompt}
\begin{euleroutput}
  24
\end{euleroutput}
\begin{eulercomment}
But an assigned parameter overrides the global value.

If the argument is not in the list of pre-defined parameters, it must be
declared with ":="!
\end{eulercomment}
\begin{eulerprompt}
>f(2,a:=5)
\end{eulerprompt}
\begin{euleroutput}
  20
\end{euleroutput}
\begin{eulercomment}
Symbolic functions are defined with "\&=". They are defined in Euler and
Maxima, and work in both worlds. The defining expression is run through
Maxima before the definition.
\end{eulercomment}
\begin{eulerprompt}
>function g(x) &= x^3-x*exp(-x); $&g(x)
\end{eulerprompt}
\begin{eulercomment}
Symbolic functions can be used in symbolic expressions.
\end{eulercomment}
\begin{eulerprompt}
>$&diff(g(x),x), $&% with x=4/3
\end{eulerprompt}
\begin{eulercomment}
They can also be used in numerical expressions. Of course, this will only
work if EMT can interpret everything inside the function.
\end{eulercomment}
\begin{eulerprompt}
>g(5+g(1))
\end{eulerprompt}
\begin{euleroutput}
  178.635099908
\end{euleroutput}
\begin{eulercomment}
They can be used to define other symbolic functions or expressions.
\end{eulercomment}
\begin{eulerprompt}
>function G(x) &= factor(integrate(g(x),x)); $&G(c) // integrate: mengintegralkan
>solve(&g(x),0.5)
\end{eulerprompt}
\begin{euleroutput}
  0.703467422498
\end{euleroutput}
\begin{eulercomment}
The following works too, since Euler uses the symbolic expression in the
function g, if it does not find a symbolic variable g, and if there is a
symbolic function g.
\end{eulercomment}
\begin{eulerprompt}
>solve(&g,0.5)
\end{eulerprompt}
\begin{euleroutput}
  0.703467422498
\end{euleroutput}
\begin{eulerprompt}
>function P(x,n) &= (2*x-1)^n; $&P(x,n)
>function Q(x,n) &= (x+2)^n; $&Q(x,n)
>$&P(x,4), $&expand(%)
>P(3,4)
\end{eulerprompt}
\begin{euleroutput}
  625
\end{euleroutput}
\begin{eulerprompt}
>$&P(x,4)+ Q(x,3), $&expand(%)
>$&P(x,4)-Q(x,3), $&expand(%), $&factor(%)
>$&P(x,4)*Q(x,3), $&expand(%), $&factor(%)
>$&P(x,4)/Q(x,1), $&expand(%), $&factor(%)
>function f(x) &= x^3-x; $&f(x)
\end{eulerprompt}
\begin{eulercomment}
With \&= the function is symbolic, and can be used in other symbolic
expressions.
\end{eulercomment}
\begin{eulerprompt}
>$&integrate(f(x),x)
\end{eulerprompt}
\begin{eulercomment}
With := the function is numerical. A good example is a definite integral like

\end{eulercomment}
\begin{eulerformula}
\[
f(x) = \int_1^x t^t \, dt,
\]
\end{eulerformula}
\begin{eulercomment}
which can not be evaluated symbolically.

If we redefine the function with the keyword "map" it can be used for vectors
x. Internally, the function is called for all values of x once, and the
results are stored in a vector.
\end{eulercomment}
\begin{eulerprompt}
>function map f(x) := integrate("x^x",1,x)
>f(0:0.5:2)
\end{eulerprompt}
\begin{euleroutput}
  [-0.783431,  -0.410816,  0,  0.676863,  2.05045]
\end{euleroutput}
\begin{eulercomment}
Functions can have default values for parameters.
\end{eulercomment}
\begin{eulerprompt}
>function mylog (x,base=10) := ln(x)/ln(base);
\end{eulerprompt}
\begin{eulercomment}
Now the function can be called with or without a parameter "base".
\end{eulercomment}
\begin{eulerprompt}
>mylog(100), mylog(2^6.7,2)
\end{eulerprompt}
\begin{euleroutput}
  2
  6.7
\end{euleroutput}
\begin{eulercomment}
Moreover, it is possible to use assigned parameters.
\end{eulercomment}
\begin{eulerprompt}
>mylog(E^2,base=E)
\end{eulerprompt}
\begin{euleroutput}
  2
\end{euleroutput}
\begin{eulercomment}
Often, we want to use functions for vectors at one place, and for individual
elements at other places. This is possible with vector parameters.
\end{eulercomment}
\begin{eulerprompt}
>function f([a,b]) &= a^2+b^2-a*b+b; $&f(a,b), $&f(x,y)
\end{eulerprompt}
\begin{eulercomment}
Such a symbolic function can be used for symbolic variables.

But the function can also be used for a numerical vector.
\end{eulercomment}
\begin{eulerprompt}
>v=[3,4]; f(v)
\end{eulerprompt}
\begin{euleroutput}
  17
\end{euleroutput}
\begin{eulercomment}
There are also purely symbolic functions, which cannot be used numerically.
\end{eulercomment}
\begin{eulerprompt}
>function lapl(expr,x,y) &&= diff(expr,x,2)+diff(expr,y,2)//turunan parsial kedua
\end{eulerprompt}
\begin{euleroutput}
  
                   diff(expr, y, 2) + diff(expr, x, 2)
  
\end{euleroutput}
\begin{eulerprompt}
>$&realpart((x+I*y)^4), $&lapl(%,x,y)
\end{eulerprompt}
\begin{eulercomment}
But of course, they can be used in symbolic expressions or in the definition
of symbolic functions.
\end{eulercomment}
\begin{eulerprompt}
>function f(x,y) &= factor(lapl((x+y^2)^5,x,y)); $&f(x,y)
\end{eulerprompt}
\begin{eulercomment}
To summarize

- \&= defines symbolic functions,\\
- := defines numerical functions,\\
- \&\&= defines purely symbolic functions.

\begin{eulercomment}
\eulerheading{Solving Expressions}
\begin{eulercomment}
Expressions can be solved numerically and symbolically.

To solve a simple expression of one variable, we can use the solve()
function. It needs a start value to start the search. Internally,
solve() uses the secant method.
\end{eulercomment}
\begin{eulerprompt}
>solve("x^2-2",1)
\end{eulerprompt}
\begin{euleroutput}
  1.41421356237
\end{euleroutput}
\begin{eulercomment}
This works for symbolic expression too. Take the following function.
\end{eulercomment}
\begin{eulerprompt}
>$&solve(x^2=2,x)
>$&solve(x^2-2,x)
>$&solve(a*x^2+b*x+c=0,x)
>$&solve([a*x+b*y=c,d*x+e*y=f],[x,y])
>px &= 4*x^8+x^7-x^4-x; $&px
\end{eulerprompt}
\begin{eulercomment}
Now we search the point, where the polynomial is 2. In solve(), the default
target value y=0 can be changed with an assigned variable.\\
We use y=2 and check by evaluating the polynomial at the previous result.
\end{eulercomment}
\begin{eulerprompt}
>solve(px,1,y=2), px(%)
\end{eulerprompt}
\begin{euleroutput}
  0.966715594851
  2
\end{euleroutput}
\begin{eulercomment}
Solving a symbolic expression in symbolic form returns a list of solutions.
We use the symbolic solver solve() provided by Maxima.
\end{eulercomment}
\begin{eulerprompt}
>sol &= solve(x^2-x-1,x); $&sol
\end{eulerprompt}
\begin{eulercomment}
The easiest way to get the numerical values is to evaluate the solution
numerically just like an expression.
\end{eulercomment}
\begin{eulerprompt}
>longest sol()
\end{eulerprompt}
\begin{euleroutput}
      -0.6180339887498949       1.618033988749895 
\end{euleroutput}
\begin{eulercomment}
To use the solutions symbolically in other expressions, the easiest way is
"with".
\end{eulercomment}
\begin{eulerprompt}
>$&x^2 with sol[1], $&expand(x^2-x-1 with sol[2])
\end{eulerprompt}
\begin{eulercomment}
Solving systems of equations symbolically can be done with vectors of
equations and the symbolic solver solve(). The answer is a list of lists of
equations.
\end{eulercomment}
\begin{eulerprompt}
>$&solve([x+y=2,x^3+2*y+x=4],[x,y])
\end{eulerprompt}
\begin{eulercomment}
The function f() can see global variables. But often we want to use
local parameters.

\end{eulercomment}
\begin{eulerformula}
\[
a^x-x^a = 0.1
\]
\end{eulerformula}
\begin{eulercomment}
with a=3.
\end{eulercomment}
\begin{eulerprompt}
>function f(x,a) := x^a-a^x;
\end{eulerprompt}
\begin{eulercomment}
One way to pass the additional parameter to f() is to use a list with the
function name and the parameters (the other way are semicolon parameters).
\end{eulercomment}
\begin{eulerprompt}
>solve(\{\{"f",3\}\},2,y=0.1)
\end{eulerprompt}
\begin{euleroutput}
  2.54116291558
\end{euleroutput}
\begin{eulercomment}
This does also work with expressions. But then, a named list element has to
be used. (More on lists in the tutorial about the syntax of EMT).
\end{eulercomment}
\begin{eulerprompt}
>solve(\{\{"x^a-a^x",a=3\}\},2,y=0.1)
\end{eulerprompt}
\begin{euleroutput}
  2.54116291558
\end{euleroutput}
\eulerheading{Menyelesaikan Pertidaksamaan}
\begin{eulercomment}
Untuk menyelesaikan pertidaksamaan, EMT tidak akan dapat melakukannya,
melainkan dengan bantuan Maxima, artinya secara eksak (simbolik).
Perintah Maxima yang digunakan adalah fourier\_elim(), yang harus
dipanggil dengan perintah "load(fourier\_elim)" terlebih dahulu.
\end{eulercomment}
\begin{eulerprompt}
>&load(fourier_elim)
\end{eulerprompt}
\begin{euleroutput}
  
          C:/Program Files/Euler x64/maxima/share/maxima/5.35.1/share/f\(\backslash\)
  ourier_elim/fourier_elim.lisp
  
\end{euleroutput}
\begin{eulerprompt}
>$&fourier_elim([x^2 - 1>0],[x]) // x^2-1 > 0
>$&fourier_elim([x^2 - 1<0],[x]) // x^2-1 < 0
>$&fourier_elim([x^2 - 1 # 0],[x]) // x^-1 <> 0
>$&fourier_elim([x # 6],[x])
>$&fourier_elim([x < 1, x > 1],[x]) // tidak memiliki penyelesaian
>$&fourier_elim([minf < x, x < inf],[x]) // solusinya R
>$&fourier_elim([x^3 - 1 > 0],[x])
>$&fourier_elim([cos(x) < 1/2],[x]) // ??? gagal
>$&fourier_elim([y-x < 5, x - y < 7, 10 < y],[x,y]) // sistem pertidaksamaan
>$&fourier_elim([y-x < 5, x - y < 7, 10 < y],[y,x])
>$&fourier_elim((x + y < 5) and (x - y >8),[x,y])
>$&fourier_elim(((x + y < 5) and x < 1) or  (x - y >8),[x,y])
>&fourier_elim([max(x,y) > 6, x # 8, abs(y-1) > 12],[x,y])
\end{eulerprompt}
\begin{euleroutput}
  
          [6 < x, x < 8, y < - 11] or [8 < x, y < - 11]
   or [x < 8, 13 < y] or [x = y, 13 < y] or [8 < x, x < y, 13 < y]
   or [y < x, 13 < y]
  
\end{euleroutput}
\begin{eulerprompt}
>$&fourier_elim([(x+6)/(x-9) <= 6],[x])
\end{eulerprompt}
\eulerheading{The Matrix Language}
\begin{eulercomment}
The documentation of the EMT core contains a detailed discussion on the
matrix language of Euler.

Vectors and matrices are entered with square brackets, elements separated by
commas, rows separated by semicolons.
\end{eulercomment}
\begin{eulerprompt}
>A=[1,2;3,4]
\end{eulerprompt}
\begin{euleroutput}
              1             2 
              3             4 
\end{euleroutput}
\begin{eulercomment}
The matrix product is denoted by a dot.
\end{eulercomment}
\begin{eulerprompt}
>b=[3;4]
\end{eulerprompt}
\begin{euleroutput}
              3 
              4 
\end{euleroutput}
\begin{eulerprompt}
>b' // transpose b
\end{eulerprompt}
\begin{euleroutput}
  [3,  4]
\end{euleroutput}
\begin{eulerprompt}
>inv(A) //inverse A
\end{eulerprompt}
\begin{euleroutput}
             -2             1 
            1.5          -0.5 
\end{euleroutput}
\begin{eulerprompt}
>A.b //perkalian matriks
\end{eulerprompt}
\begin{euleroutput}
             11 
             25 
\end{euleroutput}
\begin{eulerprompt}
>A.inv(A)
\end{eulerprompt}
\begin{euleroutput}
              1             0 
              0             1 
\end{euleroutput}
\begin{eulercomment}
The main point of a matrix language is that all functions and operators work
element for element.
\end{eulercomment}
\begin{eulerprompt}
>A.A
\end{eulerprompt}
\begin{euleroutput}
              7            10 
             15            22 
\end{euleroutput}
\begin{eulerprompt}
>A^2 //perpangkatan elemen2 A
\end{eulerprompt}
\begin{euleroutput}
              1             4 
              9            16 
\end{euleroutput}
\begin{eulerprompt}
>A.A.A
\end{eulerprompt}
\begin{euleroutput}
             37            54 
             81           118 
\end{euleroutput}
\begin{eulerprompt}
>power(A,3) //perpangkatan matriks
\end{eulerprompt}
\begin{euleroutput}
             37            54 
             81           118 
\end{euleroutput}
\begin{eulerprompt}
>A/A //pembagian elemen-elemen matriks yang seletak
\end{eulerprompt}
\begin{euleroutput}
              1             1 
              1             1 
\end{euleroutput}
\begin{eulerprompt}
>A/b //pembagian elemen2 A oleh elemen2 b kolom demi kolom (karena b vektor kolom)
\end{eulerprompt}
\begin{euleroutput}
       0.333333      0.666667 
           0.75             1 
\end{euleroutput}
\begin{eulerprompt}
>A\(\backslash\)b // hasilkali invers A dan b, A^(-1)b 
\end{eulerprompt}
\begin{euleroutput}
             -2 
            2.5 
\end{euleroutput}
\begin{eulerprompt}
>inv(A).b
\end{eulerprompt}
\begin{euleroutput}
             -2 
            2.5 
\end{euleroutput}
\begin{eulerprompt}
>A\(\backslash\)A   //A^(-1)A
\end{eulerprompt}
\begin{euleroutput}
              1             0 
              0             1 
\end{euleroutput}
\begin{eulerprompt}
>inv(A).A
\end{eulerprompt}
\begin{euleroutput}
              1             0 
              0             1 
\end{euleroutput}
\begin{eulerprompt}
>A*A //perkalin elemen-elemen matriks seletak
\end{eulerprompt}
\begin{euleroutput}
              1             4 
              9            16 
\end{euleroutput}
\begin{eulercomment}
This is not the matrix product, but a multiplication element by element. The
same works for vectors.
\end{eulercomment}
\begin{eulerprompt}
>b^2 // perpangkatan elemen-elemen matriks/vektor
\end{eulerprompt}
\begin{euleroutput}
              9 
             16 
\end{euleroutput}
\begin{eulercomment}
If one of the operands is a vector or a scalar it is expanded in the
natural way.
\end{eulercomment}
\begin{eulerprompt}
>2*A
\end{eulerprompt}
\begin{euleroutput}
              2             4 
              6             8 
\end{euleroutput}
\begin{eulercomment}
E.g., if the operand is a column vector its elements are applied to
all rows of A.
\end{eulercomment}
\begin{eulerprompt}
>[1,2]*A
\end{eulerprompt}
\begin{euleroutput}
              1             4 
              3             8 
\end{euleroutput}
\begin{eulercomment}
If it is a row vector it is applied to all columns of A.
\end{eulercomment}
\begin{eulerprompt}
>A*[2,3]
\end{eulerprompt}
\begin{euleroutput}
              2             6 
              6            12 
\end{euleroutput}
\begin{eulercomment}
One can imagine this multiplication as if the row vector v had been
duplicated to form a matrix of the same size as A.
\end{eulercomment}
\begin{eulerprompt}
>dup([1,2],2) // dup: menduplikasi/menggandakan vektor [1,2] sebanyak 2 kali (baris)
\end{eulerprompt}
\begin{euleroutput}
              1             2 
              1             2 
\end{euleroutput}
\begin{eulerprompt}
>A*dup([1,2],2) 
\end{eulerprompt}
\begin{euleroutput}
              1             4 
              3             8 
\end{euleroutput}
\begin{eulercomment}
This does also apply for two vectors where one is a row vector and the
other is a column vector. We compute i*j for i,j from 1 to 5. The trick is to multiply 1:5
with its transpose. The matrix language of Euler automatically
generates a table of values.
\end{eulercomment}
\begin{eulerprompt}
>(1:5)*(1:5)' // hasilkali elemen-elemen vektor baris dan vektor kolom
\end{eulerprompt}
\begin{euleroutput}
              1             2             3             4             5 
              2             4             6             8            10 
              3             6             9            12            15 
              4             8            12            16            20 
              5            10            15            20            25 
\end{euleroutput}
\begin{eulercomment}
Again, remember that this is not the matrix product!
\end{eulercomment}
\begin{eulerprompt}
>(1:5).(1:5)' // hasilkali vektor baris dan vektor kolom
\end{eulerprompt}
\begin{euleroutput}
  55
\end{euleroutput}
\begin{eulerprompt}
>sum((1:5)*(1:5)) // sama hasilnya
\end{eulerprompt}
\begin{euleroutput}
  55
\end{euleroutput}
\begin{eulercomment}
Even operators like \textless{} or == work in the same way.
\end{eulercomment}
\begin{eulerprompt}
>(1:10)<6 // menguji elemen-elemen yang kurang dari 6
\end{eulerprompt}
\begin{euleroutput}
  [1,  1,  1,  1,  1,  0,  0,  0,  0,  0]
\end{euleroutput}
\begin{eulercomment}
E.g., we can count the number of elements satisfying a certain
condition with the function sum().
\end{eulercomment}
\begin{eulerprompt}
>sum((1:10)<6) // banyak elemen yang kurang dari 6
\end{eulerprompt}
\begin{euleroutput}
  5
\end{euleroutput}
\begin{eulercomment}
Euler has comparison operators, like "==", which checks for equality.

We get a vector of 0 and 1, where 1 stands for true.
\end{eulercomment}
\begin{eulerprompt}
>t=(1:10)^2; t==25 //menguji elemen2 t yang sama dengan 25 (hanya ada 1)
\end{eulerprompt}
\begin{euleroutput}
  [0,  0,  0,  0,  1,  0,  0,  0,  0,  0]
\end{euleroutput}
\begin{eulercomment}
From such a vector, "nonzeros" selects the non-zero elements.

In this case, we get the indices of all elements greater than 50.
\end{eulercomment}
\begin{eulerprompt}
>nonzeros(t>50) //indeks elemen2 t yang lebih besar daripada 50
\end{eulerprompt}
\begin{euleroutput}
  [8,  9,  10]
\end{euleroutput}
\begin{eulercomment}
Of course, we can use this vector of indices to get the corresponding
values in t.
\end{eulercomment}
\begin{eulerprompt}
>t[nonzeros(t>50)] //elemen2 t yang lebih besar daripada 50
\end{eulerprompt}
\begin{euleroutput}
  [64,  81,  100]
\end{euleroutput}
\begin{eulercomment}
As an example, let us find all squares of the numbers 1 to 1000, which
are 5 modulo 11 and 3 modulo 13.
\end{eulercomment}
\begin{eulerprompt}
>t=1:1000; nonzeros(mod(t^2,11)==5 && mod(t^2,13)==3)
\end{eulerprompt}
\begin{euleroutput}
  [4,  48,  95,  139,  147,  191,  238,  282,  290,  334,  381,  425,
  433,  477,  524,  568,  576,  620,  667,  711,  719,  763,  810,  854,
  862,  906,  953,  997]
\end{euleroutput}
\begin{eulercomment}
EMT is not completely effective for integer computations. It uses
double precision floating point internally. However, it is often very
useful.

We can check for primality. Let us find out, how many squares plus 1
are primes.
\end{eulercomment}
\begin{eulerprompt}
>t=1:1000; length(nonzeros(isprime(t^2+1)))
\end{eulerprompt}
\begin{euleroutput}
  112
\end{euleroutput}
\begin{eulercomment}
The function nonzeros() works only for vectors. For matrices, there is
mnonzeros().
\end{eulercomment}
\begin{eulerprompt}
>seed(2); A=random(3,4)
\end{eulerprompt}
\begin{euleroutput}
       0.765761      0.401188      0.406347      0.267829 
        0.13673      0.390567      0.495975      0.952814 
       0.548138      0.006085      0.444255      0.539246 
\end{euleroutput}
\begin{eulercomment}
It returns the indices of the elements, which are not zeros.
\end{eulercomment}
\begin{eulerprompt}
>k=mnonzeros(A<0.4) //indeks elemen2 A yang kurang dari 0,4
\end{eulerprompt}
\begin{euleroutput}
              1             4 
              2             1 
              2             2 
              3             2 
\end{euleroutput}
\begin{eulercomment}
These indices can be used to set the elements to some value.
\end{eulercomment}
\begin{eulerprompt}
>mset(A,k,0) //mengganti elemen2 suatu matriks pada indeks tertentu
\end{eulerprompt}
\begin{euleroutput}
       0.765761      0.401188      0.406347             0 
              0             0      0.495975      0.952814 
       0.548138             0      0.444255      0.539246 
\end{euleroutput}
\begin{eulercomment}
The function mset() can also set the elements at the indices to the
entries of some other matrix.
\end{eulercomment}
\begin{eulerprompt}
>mset(A,k,-random(size(A)))
\end{eulerprompt}
\begin{euleroutput}
       0.765761      0.401188      0.406347     -0.126917 
      -0.122404     -0.691673      0.495975      0.952814 
       0.548138     -0.483902      0.444255      0.539246 
\end{euleroutput}
\begin{eulercomment}
And it is possible to get the elements in a vector.
\end{eulercomment}
\begin{eulerprompt}
>mget(A,k)
\end{eulerprompt}
\begin{euleroutput}
  [0.267829,  0.13673,  0.390567,  0.006085]
\end{euleroutput}
\begin{eulercomment}
Another useful function is extrema, which returns the minimal and
maximal values in each row of the matrix and their positions.
\end{eulercomment}
\begin{eulerprompt}
>ex=extrema(A)
\end{eulerprompt}
\begin{euleroutput}
       0.267829             4      0.765761             1 
        0.13673             1      0.952814             4 
       0.006085             2      0.548138             1 
\end{euleroutput}
\begin{eulercomment}
We can use this to extract the maximal values in each row.
\end{eulercomment}
\begin{eulerprompt}
>ex[,3]'
\end{eulerprompt}
\begin{euleroutput}
  [0.765761,  0.952814,  0.548138]
\end{euleroutput}
\begin{eulercomment}
This, of course, is the same as the function max().
\end{eulercomment}
\begin{eulerprompt}
>max(A)'
\end{eulerprompt}
\begin{euleroutput}
  [0.765761,  0.952814,  0.548138]
\end{euleroutput}
\begin{eulercomment}
But with mget(), we can extract the indices and use this information
to extract the elements at the same positions from another matrix.
\end{eulercomment}
\begin{eulerprompt}
>j=(1:rows(A))'|ex[,4], mget(-A,j)
\end{eulerprompt}
\begin{euleroutput}
              1             1 
              2             4 
              3             1 
  [-0.765761,  -0.952814,  -0.548138]
\end{euleroutput}
\begin{eulercomment}
\begin{eulercomment}
\eulerheading{Other Matrix Functions (Building Matrix)}
\begin{eulercomment}
To build a matrix, we can stack one matrix on top of another. If both
do not have the same number of columns, the shorter one will be filled
with 0.
\end{eulercomment}
\begin{eulerprompt}
>v=1:3; v_v
\end{eulerprompt}
\begin{euleroutput}
              1             2             3 
              1             2             3 
\end{euleroutput}
\begin{eulercomment}
Likewise, we can attach a matrix to another side by side, if both have
the same number of rows.
\end{eulercomment}
\begin{eulerprompt}
>A=random(3,4); A|v'
\end{eulerprompt}
\begin{euleroutput}
       0.032444     0.0534171      0.595713      0.564454             1 
        0.83916      0.175552      0.396988       0.83514             2 
      0.0257573      0.658585      0.629832      0.770895             3 
\end{euleroutput}
\begin{eulercomment}
If they do not have the same number of rows the shorter matrix is
filled with 0.

There is an exception to this rule. A real number attached to a matrix
will be used as a column filled with that real number.
\end{eulercomment}
\begin{eulerprompt}
>A|1
\end{eulerprompt}
\begin{euleroutput}
       0.032444     0.0534171      0.595713      0.564454             1 
        0.83916      0.175552      0.396988       0.83514             1 
      0.0257573      0.658585      0.629832      0.770895             1 
\end{euleroutput}
\begin{eulercomment}
It is possible to make a matrix of row and column vectors.
\end{eulercomment}
\begin{eulerprompt}
>[v;v]
\end{eulerprompt}
\begin{euleroutput}
              1             2             3 
              1             2             3 
\end{euleroutput}
\begin{eulerprompt}
>[v',v']
\end{eulerprompt}
\begin{euleroutput}
              1             1 
              2             2 
              3             3 
\end{euleroutput}
\begin{eulercomment}
The main purpose of this is to interpret a vector of expressions for
column vectors.
\end{eulercomment}
\begin{eulerprompt}
>"[x,x^2]"(v')
\end{eulerprompt}
\begin{euleroutput}
              1             1 
              2             4 
              3             9 
\end{euleroutput}
\begin{eulercomment}
To get the size of A, we can use the following functions.
\end{eulercomment}
\begin{eulerprompt}
>C=zeros(2,4); rows(C), cols(C), size(C), length(C)
\end{eulerprompt}
\begin{euleroutput}
  2
  4
  [2,  4]
  4
\end{euleroutput}
\begin{eulercomment}
For vectors, there is length().
\end{eulercomment}
\begin{eulerprompt}
>length(2:10)
\end{eulerprompt}
\begin{euleroutput}
  9
\end{euleroutput}
\begin{eulercomment}
There are many other functions, which generate matrices.
\end{eulercomment}
\begin{eulerprompt}
>ones(2,2)
\end{eulerprompt}
\begin{euleroutput}
              1             1 
              1             1 
\end{euleroutput}
\begin{eulercomment}
This can also be used with one parameter. To get a vector with another
number than 1, use the following.
\end{eulercomment}
\begin{eulerprompt}
>ones(5)*6
\end{eulerprompt}
\begin{euleroutput}
  [6,  6,  6,  6,  6]
\end{euleroutput}
\begin{eulercomment}
Also a matrix of random numbers can be generated with random (uniform
distribution) or normal (Gauß distribution).
\end{eulercomment}
\begin{eulerprompt}
>random(2,2)
\end{eulerprompt}
\begin{euleroutput}
        0.66566      0.831835 
          0.977      0.544258 
\end{euleroutput}
\begin{eulercomment}
Here is another useful function, which restructures the elements of a
matrix into another matrix.
\end{eulercomment}
\begin{eulerprompt}
>redim(1:9,3,3) // menyusun elemen2 1, 2, 3, ..., 9 ke bentuk matriks 3x3
\end{eulerprompt}
\begin{euleroutput}
              1             2             3 
              4             5             6 
              7             8             9 
\end{euleroutput}
\begin{eulercomment}
With the following function, we can use this and the dup function to
write a rep() function, which repeats a vector n times.
\end{eulercomment}
\begin{eulerprompt}
>function rep(v,n) := redim(dup(v,n),1,n*cols(v))
\end{eulerprompt}
\begin{eulercomment}
Let us test.
\end{eulercomment}
\begin{eulerprompt}
>rep(1:3,5)
\end{eulerprompt}
\begin{euleroutput}
  [1,  2,  3,  1,  2,  3,  1,  2,  3,  1,  2,  3,  1,  2,  3]
\end{euleroutput}
\begin{eulercomment}
The function multdup() duplicates elements of a vector.
\end{eulercomment}
\begin{eulerprompt}
>multdup(1:3,5), multdup(1:3,[2,3,2])
\end{eulerprompt}
\begin{euleroutput}
  [1,  1,  1,  1,  1,  2,  2,  2,  2,  2,  3,  3,  3,  3,  3]
  [1,  1,  2,  2,  2,  3,  3]
\end{euleroutput}
\begin{eulercomment}
The functions flipx() and flipy() revert the order of the rows or
columns of a matrix. I.e., the function flipx() flips horizontally.
\end{eulercomment}
\begin{eulerprompt}
>flipx(1:5) //membalik elemen2 vektor baris
\end{eulerprompt}
\begin{euleroutput}
  [5,  4,  3,  2,  1]
\end{euleroutput}
\begin{eulercomment}
For rotations, Euler has rotleft() and rotright().
\end{eulercomment}
\begin{eulerprompt}
>rotleft(1:5) // memutar elemen2 vektor baris
\end{eulerprompt}
\begin{euleroutput}
  [2,  3,  4,  5,  1]
\end{euleroutput}
\begin{eulercomment}
A special function is drop(v,i), which removes the elements with the
indices in i from the vector v.
\end{eulercomment}
\begin{eulerprompt}
>drop(10:20,3)
\end{eulerprompt}
\begin{euleroutput}
  [10,  11,  13,  14,  15,  16,  17,  18,  19,  20]
\end{euleroutput}
\begin{eulercomment}
Note that the vector i in drop(v,i) refers to indices of elements in
v, not the values of the elements. If you want to remove elements, you
need to find the elements first. The function indexof(v,x) can be used
to find elements x in a sorted vector v.
\end{eulercomment}
\begin{eulerprompt}
>v=primes(50), i=indexof(v,10:20), drop(v,i)
\end{eulerprompt}
\begin{euleroutput}
  [2,  3,  5,  7,  11,  13,  17,  19,  23,  29,  31,  37,  41,  43,  47]
  [0,  5,  0,  6,  0,  0,  0,  7,  0,  8,  0]
  [2,  3,  5,  7,  23,  29,  31,  37,  41,  43,  47]
\end{euleroutput}
\begin{eulercomment}
As you see, it does not harm to include indices out of range (like 0),
double indices, or unsorted indices.
\end{eulercomment}
\begin{eulerprompt}
>drop(1:10,shuffle([0,0,5,5,7,12,12]))
\end{eulerprompt}
\begin{euleroutput}
  [1,  2,  3,  4,  6,  8,  9,  10]
\end{euleroutput}
\begin{eulercomment}
There are some special functions to set diagonals or to generate a
diagonal matrix.

We start with the identity matrix.
\end{eulercomment}
\begin{eulerprompt}
>A=id(5) // matriks identitas 5x5
\end{eulerprompt}
\begin{euleroutput}
              1             0             0             0             0 
              0             1             0             0             0 
              0             0             1             0             0 
              0             0             0             1             0 
              0             0             0             0             1 
\end{euleroutput}
\begin{eulercomment}
Then we set the lower diagonal (-1) to 1:4.
\end{eulercomment}
\begin{eulerprompt}
>setdiag(A,-1,1:4) //mengganti diagonal di bawah diagonal utama
\end{eulerprompt}
\begin{euleroutput}
              1             0             0             0             0 
              1             1             0             0             0 
              0             2             1             0             0 
              0             0             3             1             0 
              0             0             0             4             1 
\end{euleroutput}
\begin{eulercomment}
Note that we did not change the matrix A. We get a new matrix as
result of setdiag().

Here is a function, which returns a tri-diagonal matrix.
\end{eulercomment}
\begin{eulerprompt}
>function tridiag (n,a,b,c) := setdiag(setdiag(b*id(n),1,c),-1,a); ...
>tridiag(5,1,2,3)
\end{eulerprompt}
\begin{euleroutput}
              2             3             0             0             0 
              1             2             3             0             0 
              0             1             2             3             0 
              0             0             1             2             3 
              0             0             0             1             2 
\end{euleroutput}
\begin{eulercomment}
The diagonal of a matrix can also be extracted from the matrix. To
demonstrate this, we restructure the vector 1:9 to a 3x3 matrix.
\end{eulercomment}
\begin{eulerprompt}
>A=redim(1:9,3,3)
\end{eulerprompt}
\begin{euleroutput}
              1             2             3 
              4             5             6 
              7             8             9 
\end{euleroutput}
\begin{eulercomment}
Now we can extract the diagonal.
\end{eulercomment}
\begin{eulerprompt}
>d=getdiag(A,0)
\end{eulerprompt}
\begin{euleroutput}
  [1,  5,  9]
\end{euleroutput}
\begin{eulercomment}
E.g. We can divide the matrix by its diagonal. The matrix language
takes care that the column vector d is applied to the matrix row by
row.
\end{eulercomment}
\begin{eulerprompt}
>fraction A/d'
\end{eulerprompt}
\begin{euleroutput}
          1         2         3 
        4/5         1       6/5 
        7/9       8/9         1 
\end{euleroutput}
\eulerheading{Vectorization}
\begin{eulercomment}
Almost all functions in Euler work for matrix and vector input too,
whenever this makes sense.

E.g., the sqrt() function computes the square root of all elements of
the vector or matrix.
\end{eulercomment}
\begin{eulerprompt}
>sqrt(1:3)
\end{eulerprompt}
\begin{euleroutput}
  [1,  1.41421,  1.73205]
\end{euleroutput}
\begin{eulercomment}
So you can easily create a table of values. This is one way to plot a
function (the alternative uses an expression).
\end{eulercomment}
\begin{eulerprompt}
>x=1:0.01:5; y=log(x)/x^2; // terlalu panjang untuk ditampikan
\end{eulerprompt}
\begin{eulercomment}
With this and the colon operator a:delta:b, vectors of values of functions
can be generated easily.

In the following example, we generate a vector of values t[i] with spacing
0.1 from -1 to 1. Then we generate a vector of values of the function

\end{eulercomment}
\begin{eulerformula}
\[
s = t^3-t
\]
\end{eulerformula}
\begin{eulerprompt}
>t=-1:0.1:1; s=t^3-t
\end{eulerprompt}
\begin{euleroutput}
  [0,  0.171,  0.288,  0.357,  0.384,  0.375,  0.336,  0.273,  0.192,
  0.099,  0,  -0.099,  -0.192,  -0.273,  -0.336,  -0.375,  -0.384,
  -0.357,  -0.288,  -0.171,  0]
\end{euleroutput}
\begin{eulercomment}
EMT expands operators for scalars, vectors, and matrices in the obvious way.

E.g., a column vector times a row vector expands to matrix, if an an operator
is applied. In the following, v' is the transposed vector (a column vector).
\end{eulercomment}
\begin{eulerprompt}
>shortest (1:5)*(1:5)'
\end{eulerprompt}
\begin{euleroutput}
       1      2      3      4      5 
       2      4      6      8     10 
       3      6      9     12     15 
       4      8     12     16     20 
       5     10     15     20     25 
\end{euleroutput}
\begin{eulercomment}
Note, that this is quite different from the matrix product. The matrix
product is denoted with a dot "." in EMT.
\end{eulercomment}
\begin{eulerprompt}
>(1:5).(1:5)'
\end{eulerprompt}
\begin{euleroutput}
  55
\end{euleroutput}
\begin{eulercomment}
By default, row vectors are printed in a compact format.
\end{eulercomment}
\begin{eulerprompt}
>[1,2,3,4]
\end{eulerprompt}
\begin{euleroutput}
  [1,  2,  3,  4]
\end{euleroutput}
\begin{eulercomment}
For matrices the special operator . denotes matrix multiplication, and A'
denotes transposing. A 1x1 matrix can be used just like a real number.
\end{eulercomment}
\begin{eulerprompt}
>v:=[1,2]; v.v', %^2
\end{eulerprompt}
\begin{euleroutput}
  5
  25
\end{euleroutput}
\begin{eulercomment}
To transpose a matrix we use the apostrophe.
\end{eulercomment}
\begin{eulerprompt}
>v=1:4; v'
\end{eulerprompt}
\begin{euleroutput}
              1 
              2 
              3 
              4 
\end{euleroutput}
\begin{eulercomment}
So we can compute matrix A times vector b.
\end{eulercomment}
\begin{eulerprompt}
>A=[1,2,3,4;5,6,7,8]; A.v'
\end{eulerprompt}
\begin{euleroutput}
             30 
             70 
\end{euleroutput}
\begin{eulercomment}
Note that v is still a row vector. So v'.v is different from v.v'.
\end{eulercomment}
\begin{eulerprompt}
>v'.v
\end{eulerprompt}
\begin{euleroutput}
              1             2             3             4 
              2             4             6             8 
              3             6             9            12 
              4             8            12            16 
\end{euleroutput}
\begin{eulercomment}
v.v' computes the norm of v squared for row vectors v. The result is a
1x1 vector, which works just like a real number.
\end{eulercomment}
\begin{eulerprompt}
>v.v'
\end{eulerprompt}
\begin{euleroutput}
  30
\end{euleroutput}
\begin{eulercomment}
There is also the function norm (along with many other function of
Linear Algebra).
\end{eulercomment}
\begin{eulerprompt}
>norm(v)^2
\end{eulerprompt}
\begin{euleroutput}
  30
\end{euleroutput}
\begin{eulercomment}
Operators and functions obey the matrix language of Euler.

Here is a summary of the rules.

- A function applied to a vector or matrix is applied to each element.

- An operator operating on two matrices of same size is applied pairwise to
the elements of the matrices.

- If the two matrices have different dimensions, both are expanded in a
sensible way, so that they have the same size.

E.g., a scalar value times a vector multiplies the value with each element of
the vector. Or a matrix times a vector (with *, not .) expands the vector to
the size of the matrix by duplicating it.

The following is a simple case with the operator \textasciicircum{}.
\end{eulercomment}
\begin{eulerprompt}
>[1,2,3]^2
\end{eulerprompt}
\begin{euleroutput}
  [1,  4,  9]
\end{euleroutput}
\begin{eulercomment}
Here is a more complicated case. A row vector times a column vector expands
both by duplicating.
\end{eulercomment}
\begin{eulerprompt}
>v:=[1,2,3]; v*v'
\end{eulerprompt}
\begin{euleroutput}
              1             2             3 
              2             4             6 
              3             6             9 
\end{euleroutput}
\begin{eulercomment}
Note that the scalar product uses the matrix product, not the *!
\end{eulercomment}
\begin{eulerprompt}
>v.v'
\end{eulerprompt}
\begin{euleroutput}
  14
\end{euleroutput}
\begin{eulercomment}
There are numerous functions for matrices. We give a short list. You should to consult
the documentation for more information on these commands.

\end{eulercomment}
\begin{eulerttcomment}
  sum,prod computes the sum and products of the rows
  cumsum,cumprod does the same cumulatively
  computes the extremal values of each row
  extrema returns a vector with the extremal information
  diag(A,i) returns the i-th diagonal
  setdiag(A,i,v) sets the i-th diagonal
  id(n) the identity matrix
  det(A) the determinant
  charpoly(A) the characteristic polynomial
  eigenvalues(A) the eigenvalues
\end{eulerttcomment}
\begin{eulerprompt}
>v*v, sum(v*v), cumsum(v*v)
\end{eulerprompt}
\begin{euleroutput}
  [1,  4,  9]
  14
  [1,  5,  14]
\end{euleroutput}
\begin{eulercomment}
The : operator generates an equally spaces row vector, optionally with a step
size.
\end{eulercomment}
\begin{eulerprompt}
>1:4, 1:2:10
\end{eulerprompt}
\begin{euleroutput}
  [1,  2,  3,  4]
  [1,  3,  5,  7,  9]
\end{euleroutput}
\begin{eulercomment}
To concatenate matrices and vectors there are the operators "\textbar{}" and "\_".
\end{eulercomment}
\begin{eulerprompt}
>[1,2,3]|[4,5], [1,2,3]_1
\end{eulerprompt}
\begin{euleroutput}
  [1,  2,  3,  4,  5]
              1             2             3 
              1             1             1 
\end{euleroutput}
\begin{eulercomment}
The elements of a matrix are referred with "A[i,j]".
\end{eulercomment}
\begin{eulerprompt}
>A:=[1,2,3;4,5,6;7,8,9]; A[2,3]
\end{eulerprompt}
\begin{euleroutput}
  6
\end{euleroutput}
\begin{eulercomment}
For row or column vectors, v[i] is the i-th element of the vector. For
matrices, this returns the complete i-th row of the matrix.
\end{eulercomment}
\begin{eulerprompt}
>v:=[2,4,6,8]; v[3], A[3]
\end{eulerprompt}
\begin{euleroutput}
  6
  [7,  8,  9]
\end{euleroutput}
\begin{eulercomment}
The indices can also be row vectors of indices. : denotes all indices.
\end{eulercomment}
\begin{eulerprompt}
>v[1:2], A[:,2]
\end{eulerprompt}
\begin{euleroutput}
  [2,  4]
              2 
              5 
              8 
\end{euleroutput}
\begin{eulercomment}
A short form for : is omitting the index completely.
\end{eulercomment}
\begin{eulerprompt}
>A[,2:3]
\end{eulerprompt}
\begin{euleroutput}
              2             3 
              5             6 
              8             9 
\end{euleroutput}
\begin{eulercomment}
For purposes of vectorization, the elements of a matrix can be accessed as if
they were vectors.
\end{eulercomment}
\begin{eulerprompt}
>A\{4\}
\end{eulerprompt}
\begin{euleroutput}
  4
\end{euleroutput}
\begin{eulercomment}
A matrix can also be flattened, using the redim() function. This is
implemented in the function flatten().
\end{eulercomment}
\begin{eulerprompt}
>redim(A,1,prod(size(A))), flatten(A)
\end{eulerprompt}
\begin{euleroutput}
  [1,  2,  3,  4,  5,  6,  7,  8,  9]
  [1,  2,  3,  4,  5,  6,  7,  8,  9]
\end{euleroutput}
\begin{eulercomment}
To use matrices for tables, let us reset to the default format, and
compute a table of sine and cosine values. Note that angles are in
radians by default.
\end{eulercomment}
\begin{eulerprompt}
>defformat; w=0°:45°:360°; w=w'; deg(w)
\end{eulerprompt}
\begin{euleroutput}
              0 
             45 
             90 
            135 
            180 
            225 
            270 
            315 
            360 
\end{euleroutput}
\begin{eulercomment}
Now we append columns to a matrix.
\end{eulercomment}
\begin{eulerprompt}
>M = deg(w)|w|cos(w)|sin(w)
\end{eulerprompt}
\begin{euleroutput}
              0             0             1             0 
             45      0.785398      0.707107      0.707107 
             90        1.5708             0             1 
            135       2.35619     -0.707107      0.707107 
            180       3.14159            -1             0 
            225       3.92699     -0.707107     -0.707107 
            270       4.71239             0            -1 
            315       5.49779      0.707107     -0.707107 
            360       6.28319             1             0 
\end{euleroutput}
\begin{eulercomment}
Using the matrix language, we can generate several tables of several
functions at once.

In the following example, we compute t[j]\textasciicircum{}i for i from 1 to n. We get a matrix,
where each row is a table of t\textasciicircum{}i for one i. I.e., the matrix has the
elements
\end{eulercomment}
\begin{eulerformula}
\[
a_{i,j} = t_j^i, \quad 1 \le j \le 101, \quad 1 \le i \le n
\]
\end{eulerformula}
\begin{eulercomment}
A function which does not work for vector input should be "vectorized". This
can be achieved by the "map" keyword in the function definition. Then the
function will be evaluated for each element of a vector parameter.

The numerical integration integrate() works only for scalar interval bounds.
So we need to vectorize it.
\end{eulercomment}
\begin{eulerprompt}
>function map f(x) := integrate("x^x",1,x)
\end{eulerprompt}
\begin{eulercomment}
The "map" keyword vectorizes the function. The function will now work\\
for vectors of numbers.
\end{eulercomment}
\begin{eulerprompt}
>f([1:5])
\end{eulerprompt}
\begin{euleroutput}
  [0,  2.05045,  13.7251,  113.336,  1241.03]
\end{euleroutput}
\eulerheading{Sub-Matrices and Matrix-Elements}
\begin{eulercomment}
To access a matrix element, use the bracket notation.
\end{eulercomment}
\begin{eulerprompt}
>A=[1,2,3;4,5,6;7,8,9], A[2,2]
\end{eulerprompt}
\begin{euleroutput}
              1             2             3 
              4             5             6 
              7             8             9 
  5
\end{euleroutput}
\begin{eulercomment}
We can access a complete line of a matrix.
\end{eulercomment}
\begin{eulerprompt}
>A[2]
\end{eulerprompt}
\begin{euleroutput}
  [4,  5,  6]
\end{euleroutput}
\begin{eulercomment}
In case of row or column vectors, this returns an element of the
vector.
\end{eulercomment}
\begin{eulerprompt}
>v=1:3; v[2]
\end{eulerprompt}
\begin{euleroutput}
  2
\end{euleroutput}
\begin{eulercomment}
To make sure, you get the first row for a 1xn and a mxn matrix,
specify all columns using an empty second index.
\end{eulercomment}
\begin{eulerprompt}
>A[2,]
\end{eulerprompt}
\begin{euleroutput}
  [4,  5,  6]
\end{euleroutput}
\begin{eulercomment}
If the index is a vector of indices, Euler will return the
corresponding rows of the matrix.

Here we want the first and second row of A.
\end{eulercomment}
\begin{eulerprompt}
>A[[1,2]]
\end{eulerprompt}
\begin{euleroutput}
              1             2             3 
              4             5             6 
\end{euleroutput}
\begin{eulercomment}
We can even reorder A using vectors of indices. To be precise, we do
not change A here, but compute a reordered version of A.
\end{eulercomment}
\begin{eulerprompt}
>A[[3,2,1]]
\end{eulerprompt}
\begin{euleroutput}
              7             8             9 
              4             5             6 
              1             2             3 
\end{euleroutput}
\begin{eulercomment}
The index trick works with columns too.

This example selects all rows of A and the second and third column.
\end{eulercomment}
\begin{eulerprompt}
>A[1:3,2:3]
\end{eulerprompt}
\begin{euleroutput}
              2             3 
              5             6 
              8             9 
\end{euleroutput}
\begin{eulercomment}
For abbreviation ":" denotes all row or column indices.
\end{eulercomment}
\begin{eulerprompt}
>A[:,3]
\end{eulerprompt}
\begin{euleroutput}
              3 
              6 
              9 
\end{euleroutput}
\begin{eulercomment}
Alternatively, leave the first index empty.
\end{eulercomment}
\begin{eulerprompt}
>A[,2:3]
\end{eulerprompt}
\begin{euleroutput}
              2             3 
              5             6 
              8             9 
\end{euleroutput}
\begin{eulercomment}
We can also get the last line of A.
\end{eulercomment}
\begin{eulerprompt}
>A[-1]
\end{eulerprompt}
\begin{euleroutput}
  [7,  8,  9]
\end{euleroutput}
\begin{eulercomment}
Now let us change elements of A by assigning a submatrix of A to some
value. This does in fact change the stored matrix A.
\end{eulercomment}
\begin{eulerprompt}
>A[1,1]=4
\end{eulerprompt}
\begin{euleroutput}
              4             2             3 
              4             5             6 
              7             8             9 
\end{euleroutput}
\begin{eulercomment}
We can also assign a value to a row of A.
\end{eulercomment}
\begin{eulerprompt}
>A[1]=[-1,-1,-1]
\end{eulerprompt}
\begin{euleroutput}
             -1            -1            -1 
              4             5             6 
              7             8             9 
\end{euleroutput}
\begin{eulercomment}
We can even assign to a sub-matrix if it has the proper size.
\end{eulercomment}
\begin{eulerprompt}
>A[1:2,1:2]=[5,6;7,8]
\end{eulerprompt}
\begin{euleroutput}
              5             6            -1 
              7             8             6 
              7             8             9 
\end{euleroutput}
\begin{eulercomment}
Moreover, some shortcuts are allowed.
\end{eulercomment}
\begin{eulerprompt}
>A[1:2,1:2]=0
\end{eulerprompt}
\begin{euleroutput}
              0             0            -1 
              0             0             6 
              7             8             9 
\end{euleroutput}
\begin{eulercomment}
A warning: Indices out of bounds return empty matrices, or an error
message, depending on a system setting. The default is an error
message. Remember, however, that negative indices may be used to
access the elements of a matrix counting from the end.
\end{eulercomment}
\begin{eulerprompt}
>A[4]
\end{eulerprompt}
\begin{euleroutput}
  Row index 4 out of bounds!
  Error in:
  A[4] ...
      ^
\end{euleroutput}
\eulerheading{Sorting and Shuffling}
\begin{eulercomment}
The function sort() sorts a row vector.
\end{eulercomment}
\begin{eulerprompt}
>sort([5,6,4,8,1,9])
\end{eulerprompt}
\begin{euleroutput}
  [1,  4,  5,  6,  8,  9]
\end{euleroutput}
\begin{eulercomment}
It is often necessary to know the indices of the sorted vector in the
original vector. This can be used to reorder another vector in the
same way.

Let us shuffle a vector.
\end{eulercomment}
\begin{eulerprompt}
>v=shuffle(1:10)
\end{eulerprompt}
\begin{euleroutput}
  [4,  5,  10,  6,  8,  9,  1,  7,  2,  3]
\end{euleroutput}
\begin{eulercomment}
The indices contain the proper order of v.
\end{eulercomment}
\begin{eulerprompt}
>\{vs,ind\}=sort(v); v[ind]
\end{eulerprompt}
\begin{euleroutput}
  [1,  2,  3,  4,  5,  6,  7,  8,  9,  10]
\end{euleroutput}
\begin{eulercomment}
This works for string vectors too.
\end{eulercomment}
\begin{eulerprompt}
>s=["a","d","e","a","aa","e"]
\end{eulerprompt}
\begin{euleroutput}
  a
  d
  e
  a
  aa
  e
\end{euleroutput}
\begin{eulerprompt}
>\{ss,ind\}=sort(s); ss
\end{eulerprompt}
\begin{euleroutput}
  a
  a
  aa
  d
  e
  e
\end{euleroutput}
\begin{eulercomment}
As you see, the position of double entries is somewhat random.
\end{eulercomment}
\begin{eulerprompt}
>ind
\end{eulerprompt}
\begin{euleroutput}
  [4,  1,  5,  2,  6,  3]
\end{euleroutput}
\begin{eulercomment}
The function unique returns a sorted list of unique elements of a
vector.
\end{eulercomment}
\begin{eulerprompt}
>intrandom(1,10,10), unique(%)
\end{eulerprompt}
\begin{euleroutput}
  [4,  4,  9,  2,  6,  5,  10,  6,  5,  1]
  [1,  2,  4,  5,  6,  9,  10]
\end{euleroutput}
\begin{eulercomment}
This works for string vectors too.
\end{eulercomment}
\begin{eulerprompt}
>unique(s)
\end{eulerprompt}
\begin{euleroutput}
  a
  aa
  d
  e
\end{euleroutput}
\eulerheading{Linear Algebra}
\begin{eulercomment}
EMT has lots of functions to solve linear systems, sparse systems, or
regression problems.

For linear systems Ax=b, you can use the Gauss algorithm, the inverse matrix
or a linear fit. The operator A\textbackslash{}b uses a version of the Gauss algorithm.
\end{eulercomment}
\begin{eulerprompt}
>A=[1,2;3,4]; b=[5;6]; A\(\backslash\)b
\end{eulerprompt}
\begin{euleroutput}
             -4 
            4.5 
\end{euleroutput}
\begin{eulercomment}
For another example, we generate a 200x200 matrix and the sum of its rows.
Then we solve Ax=b using the inverse matrix. We measure the error as the
maximal deviation of all elements from 1, which of course is the correct
solution.
\end{eulercomment}
\begin{eulerprompt}
>A=normal(200,200); b=sum(A); longest totalmax(abs(inv(A).b-1))
\end{eulerprompt}
\begin{euleroutput}
    8.790745908981989e-13 
\end{euleroutput}
\begin{eulercomment}
If the system does not have a solution, a linear fit minimizes the norm of
the error Ax-b.
\end{eulercomment}
\begin{eulerprompt}
>A=[1,2,3;4,5,6;7,8,9]
\end{eulerprompt}
\begin{euleroutput}
              1             2             3 
              4             5             6 
              7             8             9 
\end{euleroutput}
\begin{eulercomment}
The determinant of this matrix is 0.
\end{eulercomment}
\begin{eulerprompt}
>det(A)
\end{eulerprompt}
\begin{euleroutput}
  0
\end{euleroutput}
\eulerheading{Symbolic Matrices}
\begin{eulercomment}
Maxima has symbolic matrices. Of course, Maxima can be used for such simple linear algebra problems.
We can define the matrix for Euler and Maxima with \&:=, and then use
it in symbolic expressions.
The usual [...] form to define matrices can be used in Euler to define symbolic
matrices.
\end{eulercomment}
\begin{eulerprompt}
>A &= [a,1,1;1,a,1;1,1,a]; $A
>$&det(A), $&factor(%)
>$&invert(A) with a=0
>A &= [1,a;b,2]; $A
\end{eulerprompt}
\begin{eulercomment}
Like all symbolic variables, these matrices can be used in other
symbolic expressions.
\end{eulercomment}
\begin{eulerprompt}
>$&det(A-x*ident(2)), $&solve(%,x)
\end{eulerprompt}
\begin{eulercomment}
The eigenvalues can also be computed automatically. The result is a
vector with two vectors of eigenvalues and multiplicities.
\end{eulercomment}
\begin{eulerprompt}
>$&eigenvalues([a,1;1,a])
\end{eulerprompt}
\begin{eulercomment}
To extract a specific eigenvector needs careful indexing.
\end{eulercomment}
\begin{eulerprompt}
>$&eigenvectors([a,1;1,a]), &%[2][1][1]
\end{eulerprompt}
\begin{euleroutput}
  
                                 [1, - 1]
  
\end{euleroutput}
\begin{eulercomment}
Symbolic matrices can be evaluated in Euler numerically just like
other symbolic expressions.
\end{eulercomment}
\begin{eulerprompt}
>A(a=4,b=5)
\end{eulerprompt}
\begin{euleroutput}
              1             4 
              5             2 
\end{euleroutput}
\begin{eulercomment}
In symbolic expressions, use with.
\end{eulercomment}
\begin{eulerprompt}
>$&A with [a=4,b=5]
\end{eulerprompt}
\begin{eulercomment}
Access to rows of symbolic matrices work just like with numerical
matrices.
\end{eulercomment}
\begin{eulerprompt}
>$&A[1]
\end{eulerprompt}
\begin{eulercomment}
A symbolic expression can contain an assignment. And that changes the
matrix A.
\end{eulercomment}
\begin{eulerprompt}
>&A[1,1]:=t+1; $&A
\end{eulerprompt}
\begin{eulercomment}
There are symbolic functions in Maxima to create vectors and matrices.
For this, refer to the documentation of Maxima or to the tutorial
about Maxima in EMT.
\end{eulercomment}
\begin{eulerprompt}
>v &= makelist(1/(i+j),i,1,3); $v
\end{eulerprompt}
\begin{eulerttcomment}
 
\end{eulerttcomment}
\begin{eulerprompt}
>B &:= [1,2;3,4]; $B, $&invert(B)
\end{eulerprompt}
\begin{eulercomment}
The result can be evaluated numerically in Euler. For more information
about Maxima, see the introduction to Maxima.
\end{eulercomment}
\begin{eulerprompt}
>$&invert(B)()
\end{eulerprompt}
\begin{euleroutput}
             -2             1 
            1.5          -0.5 
\end{euleroutput}
\begin{eulercomment}
Euler has also a powerful function xinv(), which makes a bigger effort
and gets more exact results.

Note, that with \&:= the matrix B has been defined as symbolic in
symbolic expressions and as numerical in numerical expressions. So we
can use it here.
\end{eulercomment}
\begin{eulerprompt}
>longest B.xinv(B)
\end{eulerprompt}
\begin{euleroutput}
                        1                       0 
                        0                       1 
\end{euleroutput}
\begin{eulercomment}
E.g. the eigenvalues of A can be computed numerically.
\end{eulercomment}
\begin{eulerprompt}
>A=[1,2,3;4,5,6;7,8,9]; real(eigenvalues(A))
\end{eulerprompt}
\begin{euleroutput}
  [16.1168,  -1.11684,  0]
\end{euleroutput}
\begin{eulercomment}
Or symbolically. See the tutorial about Maxima for details on this.
\end{eulercomment}
\begin{eulerprompt}
>$&eigenvalues(@A)
\end{eulerprompt}
\eulerheading{Numerical Values in symbolic Expressions}
\begin{eulercomment}
A symbolic expression is just a string containing an expression. If we
want to define a value both for symbolic expressions and for numerical
expressions, we must use "\&:=".
\end{eulercomment}
\begin{eulerprompt}
>A &:= [1,pi;4,5]
\end{eulerprompt}
\begin{euleroutput}
              1       3.14159 
              4             5 
\end{euleroutput}
\begin{eulercomment}
There is still a difference between the numerical and the symbolic
form. When transferring the matrix to the symbolic form, fractional
approximations for reals will be used.
\end{eulercomment}
\begin{eulerprompt}
>$&A
\end{eulerprompt}
\begin{eulercomment}
To avoid this, there is the function "mxmset(variable)".
\end{eulercomment}
\begin{eulerprompt}
>mxmset(A); $&A
\end{eulerprompt}
\begin{eulercomment}
Maxima can also compute with floating point numbers, and even with big
floating numbers with 32 digits. The evaluation is much slower,
however.
\end{eulercomment}
\begin{eulerprompt}
>$&bfloat(sqrt(2)), $&float(sqrt(2))
\end{eulerprompt}
\begin{eulercomment}
The precision of the big floating point numbers can be changed.
\end{eulercomment}
\begin{eulerprompt}
>&fpprec:=100; &bfloat(pi)
\end{eulerprompt}
\begin{euleroutput}
  
          3.14159265358979323846264338327950288419716939937510582097494\(\backslash\)
  4592307816406286208998628034825342117068b0
  
\end{euleroutput}
\begin{eulercomment}
A numerical variable can be used in any symbolic expressions using
"@var".

Note that this is only necessary, if the variable has been defined
with ":=" or "=" as a numerical variable.
\end{eulercomment}
\begin{eulerprompt}
>B:=[1,pi;3,4]; $&det(@B)
\end{eulerprompt}
\begin{eulercomment}
\begin{eulercomment}
\eulerheading{Demo - Interest Rates}
\begin{eulercomment}
Below, we use Euler Math Toolbox (EMT) for the calculation of interest rates.
We do that numerically and symbolically to show you how Euler can be used to
solve real life problems.

Assume you have a seed capital of 5000 (say in dollars).
\end{eulercomment}
\begin{eulerprompt}
>K=5000
\end{eulerprompt}
\begin{euleroutput}
  5000
\end{euleroutput}
\begin{eulercomment}
Now we assume an interest rate of 3\% per year. Let us add one simple rate and
compute the result.
\end{eulercomment}
\begin{eulerprompt}
>K*1.03
\end{eulerprompt}
\begin{euleroutput}
  5150
\end{euleroutput}
\begin{eulercomment}
Euler would understand the following syntax too.
\end{eulercomment}
\begin{eulerprompt}
>K+K*3%
\end{eulerprompt}
\begin{euleroutput}
  5150
\end{euleroutput}
\begin{eulercomment}
But it is easier to use the factor
\end{eulercomment}
\begin{eulerprompt}
>q=1+3%, K*q
\end{eulerprompt}
\begin{euleroutput}
  1.03
  5150
\end{euleroutput}
\begin{eulercomment}
For 10 years, we can simply multiply the factors and get the final value with
compound interest rates.
\end{eulercomment}
\begin{eulerprompt}
>K*q^10
\end{eulerprompt}
\begin{euleroutput}
  6719.58189672
\end{euleroutput}
\begin{eulercomment}
For our purposes, we can set the format to 2 digits after the decimal dot.
\end{eulercomment}
\begin{eulerprompt}
>format(12,2); K*q^10
\end{eulerprompt}
\begin{euleroutput}
      6719.58 
\end{euleroutput}
\begin{eulercomment}
Let us print that rounded to 2 digits in a complete sentence.
\end{eulercomment}
\begin{eulerprompt}
>"Starting from " + K + "$ you get " + round(K*q^10,2) + "$."
\end{eulerprompt}
\begin{euleroutput}
  Starting from 5000$ you get 6719.58$.
\end{euleroutput}
\begin{eulercomment}
What if we want to know the intermediate results from year 1 to year 9? For
this, Euler's matrix language is a big help. You do not have to write a loop,
but simply enter
\end{eulercomment}
\begin{eulerprompt}
>K*q^(0:10)
\end{eulerprompt}
\begin{euleroutput}
  Real 1 x 11 matrix
  
      5000.00     5150.00     5304.50     5463.64     ...
\end{euleroutput}
\begin{eulercomment}
How does this miracle work? First the expression 0:10 returns a vector of
integers.
\end{eulercomment}
\begin{eulerprompt}
>short 0:10
\end{eulerprompt}
\begin{euleroutput}
  [0,  1,  2,  3,  4,  5,  6,  7,  8,  9,  10]
\end{euleroutput}
\begin{eulercomment}
Then all operators and functions in Euler can be applied to vectors element
for element. So
\end{eulercomment}
\begin{eulerprompt}
>short q^(0:10)
\end{eulerprompt}
\begin{euleroutput}
  [1,  1.03,  1.0609,  1.0927,  1.1255,  1.1593,  1.1941,  1.2299,
  1.2668,  1.3048,  1.3439]
\end{euleroutput}
\begin{eulercomment}
is a vector of factors q\textasciicircum{}0 to q\textasciicircum{}10. This is multiplied by K, and we get the
vector of values.
\end{eulercomment}
\begin{eulerprompt}
>VK=K*q^(0:10);
\end{eulerprompt}
\begin{eulercomment}
Of course, the realistic way to compute these interest rates would be to
round to the nearest cent after each year. Let us add a function for this.
\end{eulercomment}
\begin{eulerprompt}
>function oneyear (K) := round(K*q,2)
\end{eulerprompt}
\begin{eulercomment}
Let us compare the two results, with and without rounding.
\end{eulercomment}
\begin{eulerprompt}
>longest oneyear(1234.57), longest 1234.57*q
\end{eulerprompt}
\begin{euleroutput}
                  1271.61 
                1271.6071 
\end{euleroutput}
\begin{eulercomment}
Now there is no simple formula for the n-th year, and we must loop over the
years. Euler provides many solutions for this.

The easiest way is the function iterate, which iterates a given function a
number of times.
\end{eulercomment}
\begin{eulerprompt}
>VKr=iterate("oneyear",5000,10)
\end{eulerprompt}
\begin{euleroutput}
  Real 1 x 11 matrix
  
      5000.00     5150.00     5304.50     5463.64     ...
\end{euleroutput}
\begin{eulercomment}
We can print that in a friendly way, using our format with fixed decimal
places.
\end{eulercomment}
\begin{eulerprompt}
>VKr'
\end{eulerprompt}
\begin{euleroutput}
      5000.00 
      5150.00 
      5304.50 
      5463.64 
      5627.55 
      5796.38 
      5970.27 
      6149.38 
      6333.86 
      6523.88 
      6719.60 
\end{euleroutput}
\begin{eulercomment}
To get a specific element of the vector, we use indices in square brackets.
\end{eulercomment}
\begin{eulerprompt}
>VKr[2], VKr[1:3]
\end{eulerprompt}
\begin{euleroutput}
      5150.00 
      5000.00     5150.00     5304.50 
\end{euleroutput}
\begin{eulercomment}
Surprisingly, we can also use a vector of indices. Remember that 1:3 produced
the vector [1,2,3].

Let us compare the last element of the rounded values with the full values.
\end{eulercomment}
\begin{eulerprompt}
>VKr[-1], VK[-1]
\end{eulerprompt}
\begin{euleroutput}
      6719.60 
      6719.58 
\end{euleroutput}
\begin{eulercomment}
The difference is very small.

\begin{eulercomment}
\eulerheading{Solving Equations}
\begin{eulercomment}
Now we take a more advanced function, which adds a certain rate of money each
year.
\end{eulercomment}
\begin{eulerprompt}
>function onepay (K) := K*q+R
\end{eulerprompt}
\begin{eulercomment}
We do not have to specify q or R for the definition of the function. Only if
we run the command, we have to define these values. We select R=200.
\end{eulercomment}
\begin{eulerprompt}
>R=200; iterate("onepay",5000,10)
\end{eulerprompt}
\begin{euleroutput}
  Real 1 x 11 matrix
  
      5000.00     5350.00     5710.50     6081.82     ...
\end{euleroutput}
\begin{eulercomment}
What if we remove the same amount each year?
\end{eulercomment}
\begin{eulerprompt}
>R=-200; iterate("onepay",5000,10)
\end{eulerprompt}
\begin{euleroutput}
  Real 1 x 11 matrix
  
      5000.00     4950.00     4898.50     4845.45     ...
\end{euleroutput}
\begin{eulercomment}
We see that the money decreases. Obviously, if we get only 150 of interest in
the first year, but remove 200, we lose money each year.

How can we determine the number of years the money will last? We would have
to write a loop for this. The easiest way is to iterate long enough.
\end{eulercomment}
\begin{eulerprompt}
>VKR=iterate("onepay",5000,50)
\end{eulerprompt}
\begin{euleroutput}
  Real 1 x 51 matrix
  
      5000.00     4950.00     4898.50     4845.45     ...
\end{euleroutput}
\begin{eulercomment}
Using the matrix language, we can determine the first negative value in the
following way.
\end{eulercomment}
\begin{eulerprompt}
>min(nonzeros(VKR<0))
\end{eulerprompt}
\begin{euleroutput}
        48.00 
\end{euleroutput}
\begin{eulercomment}
The reason for this is that nonzeros(VKR\textless{}0) returns a vector of indices i,
where VKR[i]\textless{}0, and min computes the minimal index.

Since vectors always start with index 1, the answer is 47 years.

The function iterate() has one more trick. It can take an end condition as an
argument. Then it will return the value and the number of iterations.
\end{eulercomment}
\begin{eulerprompt}
>\{x,n\}=iterate("onepay",5000,till="x<0"); x, n,
\end{eulerprompt}
\begin{euleroutput}
       -19.83 
        47.00 
\end{euleroutput}
\begin{eulercomment}
Let us try to answer a more ambiguous question. Assume we know that the value
is 0 after 50 years. What would be the interest rate?

This is a question, which can only be answered numerically. Below, we will
derive the necessary formulas. Then you will see that there is no easy
formula for the interest rate. But for now, we aim for a numerical solution.

The first step is to define a function which does the iteration n times. We
add all parameters to this function.
\end{eulercomment}
\begin{eulerprompt}
>function f(K,R,P,n) := iterate("x*(1+P/100)+R",K,n;P,R)[-1]
\end{eulerprompt}
\begin{eulercomment}
The iteration is just as above

\end{eulercomment}
\begin{eulerformula}
\[
x_{n+1} = x_n \cdot \left(1+ \frac{P}{100}\right) + R
\]
\end{eulerformula}
\begin{eulercomment}
But we do longer use the global value of R in our expression. Functions like
iterate() have a special trick in Euler. You can pass the values of variables
in the expression as semicolon parameters. In this case P and R.

Moreover, we are only interested in the last value. So we take the index
[-1].

Let us try a test.
\end{eulercomment}
\begin{eulerprompt}
>f(5000,-200,3,47)
\end{eulerprompt}
\begin{euleroutput}
       -19.83 
\end{euleroutput}
\begin{eulercomment}
Now we can solve our problem.
\end{eulercomment}
\begin{eulerprompt}
>solve("f(5000,-200,x,50)",3)
\end{eulerprompt}
\begin{euleroutput}
         3.15 
\end{euleroutput}
\begin{eulercomment}
The solve routine solves expression=0 for the variable x. The answer is 3.15\%
per year. We take the start value of 3\% for the algorithm. The solve()
function always needs a start value.

We can use the same function to solve the following question: How much can we
remove per year so that the seed capital is exhausted after 20 years assuming
an interest rate of 3\% per year.
\end{eulercomment}
\begin{eulerprompt}
>solve("f(5000,x,3,20)",-200)
\end{eulerprompt}
\begin{euleroutput}
      -336.08 
\end{euleroutput}
\begin{eulercomment}
Note that you cannot solve for the number of years, since our function
assumes n to be an integer value.

\end{eulercomment}
\eulersubheading{Symbolic Solutions to the Interest Rate Problem}
\begin{eulercomment}
We can use the symbolic part of Euler to study the problem. First we define
our function onepay() symbolically.
\end{eulercomment}
\begin{eulerprompt}
>function op(K) &= K*q+R; $&op(K)
\end{eulerprompt}
\begin{eulercomment}
We can now iterate this.
\end{eulercomment}
\begin{eulerprompt}
>$&op(op(op(op(K)))), $&expand(%)
\end{eulerprompt}
\begin{eulercomment}
We see a pattern. After n periods we have

\end{eulercomment}
\begin{eulerformula}
\[
K_n = q^n K + R (1+q+\ldots+q^{n-1}) = q^n K + \frac{q^n-1}{q-1} R
\]
\end{eulerformula}
\begin{eulercomment}
The formula is the formula for the geometric sum, which is known to Maxima.
\end{eulercomment}
\begin{eulerprompt}
>&sum(q^k,k,0,n-1); $& % = ev(%,simpsum)
\end{eulerprompt}
\begin{eulercomment}
This is a bit tricky. The sum is evaluated with the flag "simpsum" to reduce
it to the quotient.

Let us make a function for this.
\end{eulercomment}
\begin{eulerprompt}
>function fs(K,R,P,n) &= (1+P/100)^n*K + ((1+P/100)^n-1)/(P/100)*R; $&fs(K,R,P,n)
\end{eulerprompt}
\begin{eulercomment}
The function does the same as our function f before. But it is more
effective.
\end{eulercomment}
\begin{eulerprompt}
>longest f(5000,-200,3,47), longest fs(5000,-200,3,47)
\end{eulerprompt}
\begin{euleroutput}
       -19.82504734650985 
       -19.82504734652684 
\end{euleroutput}
\begin{eulercomment}
We can now use it to ask for the time n. When is our capital exhausted? Our
initial guess is 30 years.
\end{eulercomment}
\begin{eulerprompt}
>solve("fs(5000,-330,3,x)",30)
\end{eulerprompt}
\begin{euleroutput}
        20.51 
\end{euleroutput}
\begin{eulercomment}
This answer says that it will be negative after 21 years.

We can also use the symbolic side of Euler to compute formulas for the
payments.

Assume we get a loan of K, and pay n payments of R (starting after the first
year) leaving a residual debt of Kn (at the time of the last payment). The
formula for this is clearly
\end{eulercomment}
\begin{eulerprompt}
>equ &= fs(K,R,P,n)=Kn; $&equ
\end{eulerprompt}
\begin{eulercomment}
Usually this formula is given in terms of

\end{eulercomment}
\begin{eulerformula}
\[
i = \frac{P}{100}
\]
\end{eulerformula}
\begin{eulerprompt}
>equ &= (equ with P=100*i); $&equ
\end{eulerprompt}
\begin{eulercomment}
We can solve for the rate R symbolically.
\end{eulercomment}
\begin{eulerprompt}
>$&solve(equ,R)
\end{eulerprompt}
\begin{eulercomment}
As you can see from the formula, this function returns a floating point error
for i=0. Euler plots it nevertheless.

Of course, we have the following limit.
\end{eulercomment}
\begin{eulerprompt}
>$&limit(R(5000,0,x,10),x,0)
\end{eulerprompt}
\begin{eulercomment}
Clearly, without interest we have to pay back 10 rates of 500.

The equation can also be solved for n. It looks nicer, if we apply some
simplification to it.
\end{eulercomment}
\begin{eulerprompt}
>fn &= solve(equ,n) | ratsimp; $&fn
\end{eulerprompt}
\end{eulernotebook}
\end{document}


\newpage
\chapter{KB Pekan 4: Menggunakan EMT untuk mengambar grafik 2 dimensi (2D)}
\documentclass[a4paper,10pt]{article}
\usepackage{eumat}

\begin{document}
\begin{eulernotebook}
\eulerheading{EMT plot2D}
\begin{eulercomment}
Rasyid Salahuddin\\
22305144016\\
Matematika E



\begin{eulercomment}
\eulerheading{Sub Bab 1}
\begin{eulercomment}
Menggambar Grafik Fungsi Satu Variabel dalam Bentuk Ekspresi Langsung
Ekspresi tunggal

Di dalam program numerik EMT, ekspresi adalah string. Jika ditandai
sebagai simbolis, mereka akan mencetak melalui Maxima, jika tidak
melalui EMT. Ekspresi dalam string digunakan untuk membuat plot dan
banyak fungsi numerik. Untuk ini, variabel dalam ekspresi harus "x".

expresi dalam string
\end{eulercomment}
\begin{eulerprompt}
>expr := "x^5-x^2-3"
\end{eulerprompt}
\begin{euleroutput}
  x^5-x^2-3
\end{euleroutput}
\begin{eulercomment}
plot ekspresi
\end{eulercomment}
\begin{eulerprompt}
>plot2d(expr,-2,2) :
\end{eulerprompt}
\begin{eulercomment}
contoh 1
\end{eulercomment}
\begin{eulerprompt}
>expr := "sin (x-5)"
\end{eulerprompt}
\begin{euleroutput}
  sin (x-5)
\end{euleroutput}
\begin{eulerprompt}
>aspect (1) ; plot2d(expr,-2,2):
\end{eulerprompt}
\begin{eulercomment}
contoh 2 dan penggunaan grid
\end{eulercomment}
\begin{eulerprompt}
>aspect(1)plot2d("log(x) + 3",-0.1,2, grid=6):
\end{eulerprompt}
\begin{euleroutput}
  Commands must be separated by semicolon or comma!
  Found: plot2d("log(x) + 3",-0.1,2, grid=6): (character 112)
  You can disable this in the Options menu.
  Error in:
  aspect(1)plot2d("log(x) + 3",-0.1,2, grid=6): ...
           ^
\end{euleroutput}
\begin{eulercomment}
contoh 3 dan penggunaan parameter square (atau \textgreater{}square) untuk memilih
y-range secara otomatis 
\end{eulercomment}
\begin{eulerprompt}
>aspect(1,1) ; plot2d("x^4-2",-5,5, >square); insimg(15)
>aspect(2) ; plot2d("x^4-2", -5,5 ):
\end{eulerprompt}
\begin{eulercomment}
contoh 4 dan memberikan nama atau label pada garis sumbu
\end{eulercomment}
\begin{eulerprompt}
>plot2d("cos(x)", -4, 6, xl="x",yl="y") :
\end{eulerprompt}
\eulerheading{Sub Bab 2}
\begin{eulercomment}
Menggambar Grafik Fungsi Satu Variabel yang rumusnya Disimpan dalam
Variabel Ekspresi

\end{eulercomment}
\begin{eulerttcomment}
 ekspresi
\end{eulerttcomment}
\begin{eulerprompt}
>expr &= x^5-1
\end{eulerprompt}
\begin{euleroutput}
  
                                   5
                                  x  - 1
  
\end{euleroutput}
\begin{eulercomment}
plot dari ekspresi diatas 
\end{eulercomment}
\begin{eulerprompt}
>aspect(2); plot2d(expr,-1,1):
\end{eulerprompt}
\begin{eulercomment}
contoh 1
\end{eulercomment}
\begin{eulerprompt}
>expr := "x^10-x-5"
\end{eulerprompt}
\begin{euleroutput}
  x^10-x-5
\end{euleroutput}
\begin{eulerprompt}
>aspect(2) ; plot2d(expr,-1,1):
\end{eulerprompt}
\begin{eulercomment}
menggunakan variabel lokal
\end{eulercomment}
\begin{eulercomment}
Ekspresi dapat dievaluasi secara numerik. Variabel x,y,z ditetapkan
secara otomatis. Variabel lain dapat ditetapkan berdasarkan parameter
yang ditetapkan( variabel lokal ) atau melalui variabel global.
variabel global adalah variabel yang selalu bisa diakses kapan pun dan
di mana pun.
\end{eulercomment}
\begin{eulerprompt}
>expr &= a*x^5
\end{eulerprompt}
\begin{euleroutput}
  
                                      5
                                   a x
  
\end{euleroutput}
\begin{eulercomment}
menggunakan variabel global 
\end{eulercomment}
\begin{eulerprompt}
>a=6; expr(2.5)
\end{eulerprompt}
\begin{euleroutput}
  585.9375
\end{euleroutput}
\begin{eulercomment}
menggunakan variabel lokal
\end{eulercomment}
\begin{eulerprompt}
>expr(2.5,a=6)
\end{eulerprompt}
\begin{euleroutput}
  585.9375
\end{euleroutput}
\begin{eulercomment}
evaluasi langsung
\end{eulercomment}
\begin{eulerprompt}
>"a*x^5"(3,4)
\end{eulerprompt}
\begin{euleroutput}
  1458
\end{euleroutput}
\begin{eulercomment}
Oleh karena itu, banyak algoritma EMT yang dapat menggunakan ekspresi
dalam x, bukan fungsi. Namun jika parameter tambahan yang tidak
bersifat global dilibatkan, fungsi harus diutamakan.

menggunakan variabel  global "a"
\end{eulercomment}
\begin{eulerprompt}
>a=5; plot2d("a*x^3-x",0,1):
>function f(x,a) := a*x^3-x
\end{eulerprompt}
\begin{eulercomment}
gunakan "a=6" sebagai parameter
\end{eulercomment}
\begin{eulerprompt}
>plot2d("f",0,1;6):
\end{eulerprompt}
\begin{eulercomment}
alternatif lain
\end{eulercomment}
\begin{eulerprompt}
>plot2d(\{\{"f",6\}\},0,1):
\end{eulerprompt}
\begin{eulercomment}
alternatif lain 
\end{eulercomment}
\begin{eulerprompt}
>plot2d("f(x,6)",0,1):
\end{eulerprompt}
\eulerheading{Sub Bab 3}
\begin{eulercomment}
Menggambar Fungsi Simbolik

Fungsi Plot yang paling penting untuk plot planar adalah plot2d().
Fungsi ini diimplementasikan dalam bahasa Euler dalam file "plot.e",
yang dimuat diawal program.

plot2d() menerima ekspresi, fungsi, dan data.

Rentang plot diatur dengan parameter yang ditetapkan ssbagai berikut\\
- a,b: rentang x (default -2,2)\\
- -c,d: rentang y (default: skala dengan nilai)\\
- r: alternatifnya radius di sekitar pusat plot\\
- cx,cy: koordinat pusat plot (default 0,0)

Keterangan:(menggambar grafik fungsi satu variabel yang fungsinya
didefinisikan sebagai fungsi simbolik)\\
- \&: untuk menampilkan variabel pada teks

Berikut adalah beberapa contoh menggunakan fungsi. Seperti biasa di
EMT, fungsi yang berfungsi untuk fungsi atau ekspresi lain, jadi kita
dapat meneruskan parameter tambahan (selain x) yang bukan variabel
global ke fungsi dengan parameter titik koma atau dengan koleksi
panggilan.
\end{eulercomment}
\begin{eulerprompt}
>plot2d("f",0,1;0.4): // plot with a=0.4
>plot2d(\{\{"f",0.2\}\},0,1); 
>plot2d(\{\{"f(x,b)",b=0.1\}\},0,1):
>function f(x) := x^3-x;...
>plot2d("f",r=1):
>plot2d("exp(-a*x^2)/a"):
\end{eulerprompt}
\begin{eulercomment}
Berikut merupakan ringkasan dari fungsi yang diterima\\
- ekspresi atau ekspresi simbolik dalam x\\
- fungsi atau fungsi simbolis dengan nama sebagai "f"\\
- fungsi simbolis hanya dengan nama f\\
\end{eulercomment}
\begin{eulerttcomment}
 
\end{eulerttcomment}
\begin{eulercomment}
Fungsi plot2d() juga menerima fungsi simbolis. Untuk fungsi simbolis,
hanya nama saja yang berfungsi.
\end{eulercomment}
\begin{eulerprompt}
>function f(x) &= diff(x^x,x)
\end{eulerprompt}
\begin{euleroutput}
  
                              x
                             x  (log(x) + 1)
  
\end{euleroutput}
\begin{eulerprompt}
>plot2d(f,0,2):
>$&expr = sin (x)*exp(-x)
>plot2d(expr,0,3pi):
>plot2d("cos(x)","sin(3*x)"):
\end{eulerprompt}
\eulerheading{Sub Bab 4 }
\begin{eulercomment}
Menggambar Fungsi Numerik 

Fungsi Numerik adalah sebuah fungsi dengan himpunan bilangan cacah
sebagai domain dan himpunan mendasar yang melibatkan hubungan
matematis antara bilangan yang menjadi domain dan bilangan sebagai
kodomain.
\end{eulercomment}
\begin{eulerprompt}
> 
\end{eulerprompt}
\begin{eulercomment}
Fungsi numerik  memiliki  1  atau  lebih  variabel  independen, yang
sering dilambangkan sebagai "X". Variabel X adalah nilai atau
parameter yang dapat berubah, dan fungsi numerik menggambarkan
bagaimana variabel ini memengaruhi variabel dependen. Variabel
dependen adalah hasil perhitungan atau keluaran dari fungsi numerik
yang bergantung pada nilai atau perubahan dalam variabel independen.

\end{eulercomment}
\begin{eulercomment}
Dalam EMT cara mendefinisikan fungsi menggunakan syntak function.
untuk mendefinisikan fungsi numerik menggunakan tanda ":="

Fungsi  numerik  menjelaskan bagaimana bilangan  dalam  domain
berhubungan dengan bilangan sebagai kodomain, biasanya diberikan dalam
bentuk rumus matematik(persamaan) atau aturan yang memetakan setiap
domain kedalam kodomain yang sesuai. contoh:

f(x)=2x+3
\end{eulercomment}
\begin{eulerprompt}
> 
\end{eulerprompt}
\begin{eulercomment}
(x)(variabel dependen) adalah fungsi yang memetakan setiap nilai
x(variabel independen)kedalam nilai 2x+3. Terdapat berbagai jenis
fungsi yang termasuk ke dalam fungsi numerik, diantaranya:

Fungsi linier dengan bentuk umum\\
f (x) = ax + b
\end{eulercomment}
\begin{eulercomment}
Fungsi kuadrat dengan bentuk umum

f (x) = ax2 + bx + c
\end{eulercomment}
\begin{eulercomment}
Fungsi eksponensial dengan bentuk umum

f (x) = ax
\end{eulercomment}
\begin{eulercomment}
Fungsi logaritma dengan bentuk umum

f (x) = log a(x)

\end{eulercomment}
\begin{eulercomment}
Fungsi trigonometri dengan bentuk umum

f (x) = sin(x), f (x) = cos(x)

\end{eulercomment}
\begin{eulercomment}
Salah satu  cara  yang  umum  digunakan  untuk  memvisualisasikan
fungsi numerik adalah dengan menggambar grafiknya. Grafik ini
menggambarkan bagaimana variabel dependen berubah seiring perubahan
variabel independen dan membantu dalam memahami sifat-sifat fungsi,
seperti titik ekstrim
\end{eulercomment}
\eulersubheading{Contoh soal}
\begin{eulerprompt}
>function r(x):= abs(x-10)
>function s(x):= abs(sin(x))
>r(-5)
\end{eulerprompt}
\begin{euleroutput}
  15
\end{euleroutput}
\begin{eulerprompt}
>function t(x):=log(x*(2+sin(x/1000)))
>function u(x):=integrate("(sin(x)*exp(-x^2)"0,x)
>function v(x):=logbase((x^2),2)
>plot2d("v"):
>plot2d("s"):
>plot2d("t",-2,2):
>function P(x):=x*cos(x)
>plot2d("P",-2*pi,2*pi):
\end{eulerprompt}
\begin{eulercomment}
Fungsi plot2d() adalah fungsi serbaguna untuk membuat grafik dalam
bidang (grafik 2D). Fungsi ini dapat digunakan untuk membuat grafik
fungsi-fungsi satu variabel, grafik data,  kurva-kurva  dalam  bidang,
grafik batang (bar plots), grid dari bilangan kompleks, dan grafik
implisit dari fungsi dua variabel.

Parameter\\
x,y : persamaan, fungsi, atau vektor data a,b,c,d : area plot (default
a=-2, b=2)\\
r  :  jika  r  diatur,  maka  a=cx-r,  b=cx+r,  c=cy-r,  d=cy+r r bisa
berupa vektor [rx,ry] atau vektor [rx1,rx2,ry1,ry2]. xmin,xmax :
rentang parameter untuk kurva\\
auto : tentukan rentang y secara otomatis (default)\\
square : jika benar, mencoba menjaga rentang x-y tetap persegi n :
jumlah interval (default adalah adaptif)\\
grid : 0 = tanpa grid dan label, 1 = hanya sumbu,\\
2 = grid normal (lihat di bawah untuk jumlah garis grid) 3 = di dalam
sumbu\\
4 = tanpa grid\\
5 = grid penuh termasuk margin 6 = tanda di pinggiran\\
7 = hanya sumbu\\
8 = hanya sumbu, sub-ticks frame : 0 = tanpa bingkai\\
framecolor: warna bingkai dan grid\\
margin : angka antara 0 dan 0,4 untuk margin di sekitar plot color :
Warna kurva. Jika ini adalah vektor warna,akan digunakan untuk setiap
baris matriks plot. Dalam  hal grafik titik, harus berupa vektor
kolom. Jika vektor baris atau matriks penuh warna digunakan untuk
grafik titik, akan digunakan untuk setiap titik data.\\
thickness : ketebalan garis untuk kurva

Nilai ini dapat lebih kecil dari 1 untuk garis yang sangat tipis. \\
style: Gaya plot untuk garis, penanda, dan isian.

Untuk titik gunakan\\
"[]", "\textless{}\textgreater{}", ".", "..", "...", "*", "+", " ", "-", "o"\\
"[]", "\textless{}\textgreater{}", "o" (bentuk terisi)\\
"[]w", "\textless{}\textgreater{}w", "ow" (tidak transparan)

Untuk garis gunakan\\
"-", "-", "-.", ".", ".-.", "-.-", "-\textgreater{}"

Untuk poligon terisi atau plot batang gunakan\\
"", "O", "O", "/", "", "/","+", " ", "-", "t"

points : plot titik tunggal sebagai gantinya garis segmen addpoints :
jika benar, plot segmen garis dan titik\\
add : tambahkan plot ke plot yang ada\\
user : aktifkan interaksi pengguna untuk fungsi delta : ukuran langkah
untuk interaksi pengguna\\
bar : plot batang (x adalah batas interval, y adalah nilai interval)
histogram : plot frekuensi x dalam n subinterval\\
distribusi=n : plot distribusi x dengan n subinterval even : gunakan
nilai antar untuk histogram otomatis. steps : plot fungsi sebagai
fungsi langkah (steps=1,2)\\
adaptive : gunakan plot adaptif (n adalah jumlah minimal langkah)
level : plot garis level dari fungsi implisit dua variabel\\
outline : menggambar batas rentang level.
\end{eulercomment}
\begin{eulerprompt}
>function s(x):=(x-10)
>function r(x):=abs(sin(x))
>s(-5)
\end{eulerprompt}
\begin{euleroutput}
  -15
\end{euleroutput}
\begin{eulerprompt}
>function t(x):=log(x*(2+sin(x/1000)))
>function u(x):=integrate("(sin(x)*exp(-x^2)"),0,x)
>function v(x):=logbase((x^2),2)
>plot2d("v"):
>plot2d("s"):
>function P(x):=x*cos(x)
>plot2d("P", -2*pi,2*pi):
\end{eulerprompt}
\eulerheading{Sub Bab 5 }
\begin{eulercomment}
Menggambar Beberapa Kurva Sekaligus 


Dalam subtopik ini, kita akan membahas mengenai cara menggambar
beberapa kurva sekaligus. Dalam hal ini kita dapat menggambar beberapa
kurva dalam jendela grafik yang berbeda secara bersama-sama. Untuk
membuat ini kita dapat menggunakan perintah figure(). Berikut contoh
dari menggambar beberapa kurva sekaligus

Menggambar plot fungsi\\
\end{eulercomment}
\begin{eulerformula}
\[
x^n, 1 \leq n \leq 4
\]
\end{eulerformula}
\begin{eulerprompt}
>reset;
>figure(2,2);...
>for n=1 to 4; figure(n); plot2d("x^"+n); end;...
>figure(0):
\end{eulerprompt}
\begin{eulercomment}
Penjelasan sintaks dari plot fungsi

\end{eulercomment}
\begin{eulerformula}
\[
x^n,  1 \leq n \leq 4
\]
\end{eulerformula}
\begin{eulercomment}
- reset;\\
Perintah ini berguna untuk menghapus grafik yang telah ada sebelumnya,
sehingga kita dapat memulai dari awal untuk menggambar grafik\\
- figure(2x2);\\
Perintah figure() digunakan untuk membuat jendela grafik dengan ukuran\\
axb. Dalam kasus ini perintah figure(2,2) memiliki makna bahwa jendela
grafik yang dibuat berukuran 2x2. Artinya, akan ada empat jendela
grafik yang akan ditampilkan dengan tata letak 2 baris dan 2 kolom.\\
- for n=1 to 4;\\
Perintah ini digunakan untuk melakukan pengulangan (looping) perintah
sebanyak empat kali, yaitu dari 1 hingga 4.\\
- figure(n);\\
Perintah ini digunakan untuk beralih dari jendela grafik satu ke
jendela grafik lainnya (jendela grafik ke-n).\\
- plot2d("x\textasciicircum{}"+n);\\
Perintah plot2d() digunakan untuk membuat plot fungsi matematika.\\
Dalam hal ini fungsi yang diplot adalah x\textasciicircum{}n, di mana n adalah nilai
dari variabel yang sedang diulang. Dengan kata lain, ini akan membuat\\
plot dari x\textasciicircum{}1, x\textasciicircum{}2, x\textasciicircum{}3, dan x\textasciicircum{}4 dalam jendela grafik yang sesuai\\
- end;\\
Perintah ini menandakan akhir dari looping.\\
- figure(0);\\
Perintah ini digunakan untuk beralih kembali ke jendela grafik utama.
\end{eulercomment}
\begin{eulercomment}
Dari sini dapat kita perhatikan untuk membuat kurva fungsi x\textasciicircum{}n (x
pangkat n) perintahnya tidak ditulis dengan (x\textasciicircum{}n) melainkan ditulis
dengan ("x\textasciicircum{}"+n). Tanda petik dua ("...") digunakan untuk
mengidentifikasi bahwa teks tersebut merupakan ekspresi matematika.\\
Sedangkan tanda (+) digunakan untuk menggabungkan string dengan nilai
yang berubah-ubah atau variabel.

Contoh lain:\\
Menggambar plot fungsi\\
\end{eulercomment}
\begin{eulerformula}
\[
f(x)=x^3-x, -2<x<2
\]
\end{eulerformula}
\begin{eulerprompt}
>reset;
>figure(3,3);...
>for k=1:9; figure(k); plot2d("x^3-x",-2,2,grid=k); end;...
>figure(0):
\end{eulerprompt}
\begin{eulerttcomment}
 Penjelasan sintaks dari plot fungsi
\end{eulerttcomment}
\begin{eulerformula}
\[
f(x)=x^3-x, -2<x<2
\]
\end{eulerformula}
\begin{eulercomment}
- reset;\\
Perintah ini berguna untuk menghapus grafik yang telah ada sebelumnya,
sehingga kita dapat memulai dari awal untuk menggambar grafik\\
- figure (3,3);\\
Perintah ini digunakan untuk membuat jendela grafik dengan ukuran 3x3.
Artinya, akan ada empat jendela grafik yang akan ditampilkan dengan
tata letak 3 baris dan 3 kolom.\\
- for k=1:9;\\
Perintah ini digunakan untuk melakukan pengulangan (looping) perintah
sebanyak sembilan kali.\\
- figure(n);\\
Perintah ini digunakan untuk beralih dari jendela grafik satu ke\\
\end{eulercomment}
\begin{eulerttcomment}
 jendela grafik lainnya (jendela grafik ke-n).
\end{eulerttcomment}
\begin{eulercomment}
- plot2d("x\textasciicircum{}3-x",-2,2,grid=k);\\
Perintah plot2d() digunakan untuk membuat plot fungsi matematika.\\
Dalam hal ini fungsi yang diplot adalah x\textasciicircum{}3-x, dengan batas sumbu x
dari -2 hingga 2. Argumen grid=k digunakan untuk mengaktifkan grid
pada jendela grafik ke-k.\\
- end;\\
Perintah ini menandakan akhir dari looping.\\
- figure(0);\\
Perintah ini digunakan untuk beralih kembali ke jendela grafik utama.

Dari contoh diatas dapat kita perhatikan bahwa tampilan plot dari yang
ke-1 hingga ke-9 memiliki tampilan yang berbeda-beda. Dalam EMT
memiliki berbagai gaya plot 2D yang dapat dijalankan menggunakan
perintah grid=n dimana n adalah jumlah langkah minimal. Setiap nilai n
memiliki tampilan plot adaptif yang berbeda dalam plot 2D, diantaranya
yaitu:\\
0 : tidak ada grid (kisi), frame, sumbu, dan label, hanya kurva saja\\
1 : dengan sumbu, label-label sumbu di luar frame jendela grafik\\
2 : tampilan default\\
3 : dengan grid pada sumbu x dan y, label-label sumbu berada di dalam
jendela grafik\\
4 : tidak ada grid (kisi), sumbu x dan y, dan label berada di luar
frame jendela grafik\\
5 : tampilan default tanpa margin di sekitar plot\\
6 : hanya dengan sumbu x y dan label, tanpa grid\\
7 : hanya dengan sumbu x y dan tanda-tanda pada sumbu.\\
8 : hanya dengan sumbu dan tanda-tanda pada sumbu, dengan tanda-tanda
yang lebih halus pada sumbu.\\
9 : tampilan default dengan tanda-tanda kecil di dalam jendela\\
10: hanya dengan sumbu-sumbu, tanpa tanda

Contoh lain:\\
Menggambar plot fungsi\\
\end{eulercomment}
\begin{eulerformula}
\[
g(x)=2x^3-x
\]
\end{eulerformula}
\begin{eulerprompt}
>reset;
>aspect(1.2);
>figure(3,4); ...
> figure(2); plot2d("2x^3-x",grid=1); ... // x-y-axis
> figure(3); plot2d("2x^3-x",grid=2); ... // default ticks
>figure(4); plot2d("2x^3-x",grid=3); ... // x-y- axis with labels inside
> figure(5); plot2d("2x^3-x",grid=4); ... // no ticks, only labels
>figure(6); plot2d("2x^3-x",grid=5); ... // default, but no margin
>figure(7); plot2d("2x^3-x",grid=6); ... // axes only
>figure(8); plot2d("2x^3-x",grid=7); ... // axes only, ticks at axis
>figure(9); plot2d("2x^3-x",grid=8); ... // axes only, finer ticks at axis
>figure(10); plot2d("2x^3-x",grid=9); ... // default, small ticks inside
>figure(11); plot2d("2x^3-x",grid=10); ...// no ticks, axes only
>figure(0):
\end{eulerprompt}
\begin{eulercomment}
Penjelasan sintaks dari plot fungsi\\
\end{eulercomment}
\begin{eulerformula}
\[
g(x)=2x^3-x
\]
\end{eulerformula}
\begin{eulercomment}
- aspect(1.2);\\
Perintah aspect() digunakan untuk mengatur rasio aspek dari jendela
grafik. Hal ini berarti perintah aspect(1.2); akan menghasilkan plot
dengan perbandingan rasio panjang dan lebar 2:1.\\
- figure(3,4);\\
Perintah ini digunakan untuk membuat jendela grafik dengan ukuran 3x4.\\
Jadi, akan ada total 12 jendela grafik yang akan ditampilkan dalam
tata letak 3 baris dan 4 kolom.\\
- figure(1); plot2d("x\textasciicircum{}3-x",grid=0); ...\\
Adalah perintah untuk beralih ke jendela grafik pertama dan menggambar
plot dari fungsi x\textasciicircum{}3 - x tanpa grid, frame, atau sumbu.\\
- figure(2); plot2d("x\textasciicircum{}3-x",grid=1); ...\\
Adalah perintah untuk beralih ke jendela grafik kedua dan menggambar
plot dari fungsi x\textasciicircum{}3 - x dengan grid hanya pada sumbu x dan y.\\
- figure(3); plot2d("x\textasciicircum{}3-x",grid=2); ...\\
Adalah perintah untuk beralih ke jendela grafik ketiga dan menggambar
plot dari fungsi x\textasciicircum{}3 - x dengan tampilan default, termasuk tanda-tanda
default pada sumbu.\\
- figure(4); plot2d("x\textasciicircum{}3-x",grid=3); ...\\
Adalah perintah untuk beralih ke jendela grafik keempat dan menggambar
plot dari fungsi x\textasciicircum{}3 - x dengan grid pada sumbu x dan y, serta
label-label sumbu yang ada di dalam jendela.\\
- figure(5); plot2d("x\textasciicircum{}3-x",grid=4); ...\\
Adalah perintah untuk beralih ke jendela grafik kelima dan menggambar
plot dari fungsi x\textasciicircum{}3 - x tanpa tanda-tanda sumbu, hanya label-label
yang ada.\\
- figure(6); plot2d("x\textasciicircum{}3-x",grid=5); ...\\
Adalah perintah untuk beralih ke jendela grafik keenam dan menggambar
plot dari fungsi x\textasciicircum{}3 - x dengan tampilan default, tetapi tanpa margin
di sekitar plot.\\
- figure(7); plot2d("x\textasciicircum{}3-x",grid=6); ...\\
Adalah perintah untuk beralih ke jendela grafik ketujuh dan menggambar
plot dari fungsi x\textasciicircum{}3 - x hanya dengan sumbu-sumbu (tanpa grid atau
label).\\
- figure(8); plot2d("x\textasciicircum{}3-x",grid=7); ...\\
Adalah perintah untuk beralih ke jendela grafik kedelapan dan
menggambar plot dari fungsi x\textasciicircum{}3 - x hanya dengan sumbu-sumbu dan
tanda-tanda pada sumbu.\\
- figure(9); plot2d("x\textasciicircum{}3-x",grid=8); ...\\
Adalah perintah untuk beralih ke jendela grafik kesembilan dan
menggambar plot dari fungsi x\textasciicircum{}3 - x hanya dengan sumbu-sumbu dan
tanda-tanda pada sumbu, dengan tanda-tanda yang lebih halus pada
sumbu.\\
- figure(10); plot2d("x\textasciicircum{}3-x",grid=9); ...\\
Adalah perintah untuk beralih ke jendela grafik kesepuluh dan
menggambar plot dari fungsi x\textasciicircum{}3 - x dengan tanda-tanda default kecil
di dalam jendela.\\
- figure(11); plot2d("x\textasciicircum{}3-x",grid=10); ...\\
Adalah perintah untuk beralih ke jendela grafik kesebelas dan
menggambar plot dari fungsi x\textasciicircum{}3 - x hanya dengan sumbu-sumbu, tanpa
tanda-tanda.\\
- figure(0);\\
Adalah perintah untuk beralih kembali ke jendela grafik utama atau
jendela grafik dengan nomor 0 setelah semua perintah dalam urutan
selesai dieksekusi.

Dari ketiga contoh di atas, dapat kita katakan bahwa untuk menggambar
beberapa kurva sekaligus itu dapat dilakukan dengan satu baris
perintah ataupun dengan cara mendefinisikannya 1 per 1.

Terlihat beberapa jenis grid memiliki tampilan yang mirip atau sama,
seperti 1 dan 2, 2 dan 5, 4 dan 9, 7 dan 8, untuk dapat membedakannya
secara lebih jelas, ubah grid dari contoh di bawah ini.
\end{eulercomment}
\begin{eulerprompt}
>reset;
>aspect(1.3);
>figure(1,3);...
>figure (1); plot2d("x^2*exp(-x)",0,10);...
>figure (2); plot2d("2*exp(x)",-5,5);...
>figure (3); plot2d("exp(x^2)",-2,2);...
>figure (0):
\end{eulerprompt}
\begin{eulercomment}
Contoh lain:
\end{eulercomment}
\begin{eulerprompt}
>reset;
>aspect(3/4);
>figure(2,1);...
>for a=1:2; figure(a); plot2d("2*x*log(x^2)",0,3,grid=a); end;...
>figure(0):
\end{eulerprompt}
\eulerheading{Sub Bab 6 }
\begin{eulercomment}
Menggambar Beberapa Kurva pada bidang koordinat yang sama 

Plot lebih dari satu fungsi (multiple function) ke dalam satu jendela
dapat dilakukan dengan berbagai cara. Salah satu caranya adalah
menggunakan \textgreater{}add untuk beberapa panggilan ke plot2d secara
keseluruhan, kecuali panggilan pertama.

Berikut contohnya:\\
menggambar kurva\\
\end{eulercomment}
\begin{eulerformula}
\[
 f(x)=cos(x)
\]
\end{eulerformula}
\begin{eulerformula}
\[
f(x)= x^2
\]
\end{eulerformula}
\begin{eulerprompt}
>aspect(); plot2d("cos(x)",r=3); plot2d("x^2",style=".",>add):
\end{eulerprompt}
\begin{eulerformula}
\[
f(x)=cos(x)-1
\]
\end{eulerformula}
\begin{eulerformula}
\[
f(x)= sin(x)-1
\]
\end{eulerformula}
\begin{eulerprompt}
>aspect(2); plot2d("cos(x)-1",-1,6); plot2d("sin(x)-1",style="--",>add):
\end{eulerprompt}
\begin{eulercomment}
Selain menggunakan \textgreater{}add kita juga bisa menambahkannya secara langsung

Berikut contohnya:\\
Menggambar kurva\\
\end{eulercomment}
\begin{eulerformula}
\[
f(x)= 2x+1
\]
\end{eulerformula}
\begin{eulerformula}
\[
f(x)= -2x+1
\]
\end{eulerformula}
\begin{eulerprompt}
>plot2d(["2x+1","x"],0,8):
\end{eulerprompt}
\begin{eulerformula}
\[
f(x)=sin(2x)
\]
\end{eulerformula}
\begin{eulerformula}
\[
f(x)=cos(3x)
\]
\end{eulerformula}
\begin{eulerprompt}
>aspect(1.5); plot2d(["sin(2x)","cos(3x)"],0,8):
\end{eulerprompt}
\begin{eulercomment}
Kegunaan \textgreater{}add yang lain juga bisa untuk menambahkan titik pada kurva.

Berikut contohnya:\\
Menambahkan sebuah titik di\\
\end{eulercomment}
\begin{eulerformula}
\[
f(x)= x+4
\]
\end{eulerformula}
\begin{eulerprompt}
>aspect(); plot2d("x+4",-2,5,); plot2d(2,6,>points,>add):
\end{eulerprompt}
\begin{eulercomment}
Kita juga bisa mencari titik perpotongan dengan cara berikut:

\end{eulercomment}
\begin{eulerformula}
\[
sin(x)=2x
\]
\end{eulerformula}
\begin{eulerprompt}
>plot2d(["sin(x)","2x"],r=2,cx=1,cy=1, ...
>  color=[black,blue],style=["-","."], ...
>  grid=1);
>x0=solve("sin(x)-2x",1);  ...
>  plot2d(x0,x0,>points,>add);  ...
>  label("sin(x) = 2x",x0,x0,pos="cl",offset=20):
>function f(x,a) := x^2+a*x-x/a; ...
>plot2d("f",-10,10;1,title="a=1"):
> plot2d(\{\{"f",1\}\},-10,10); ...
>for a=1:10; plot2d(\{\{"f",a\}\},>add); end:
>function f(x,a) := x^2*exp(-x^2/a); ...
>plot2d("f",-10,10;5,thickness=2,title="a=5"):
>plot2d(\{\{"f",1\}\},-8,8); ...
>for a=2:5; plot2d(\{\{"f",a\}\},>add,thickness=2); end:
>aspect(2.1); &plot2d(1/x,[x,-1,1]):
>x=linspace(-1,1,50);...
>plot2d("1/x"):
\end{eulerprompt}
\eulerheading{Sub Bab 7 }
\begin{eulercomment}
Menuliskan Label koordinat,label kurva, dan keterangan 

kurva(legend) Dalam EMT, untuk menambahkan judul dapat dilakukan
dengan title="..."\\
untuk menambahkan sumbu x dan sumbu y dapat dilakukan dengan x1="...",
y1="..."\\
sebagai contoh:
\end{eulercomment}
\begin{eulerprompt}
>plot2d("x^2-4*x"):
\end{eulerprompt}
\begin{eulercomment}
untuk menambahkan judul dapat dilakukan dengan title="..."\\
untuk menambahkan sumbu x dan sumbu y dapat dilakukan dengan x1="...",
y1="..."
\end{eulercomment}
\begin{eulerprompt}
>plot2d("x^2-4*x",title="FUNGSI y=x^2-4*x",yl="Sumbu y",xl="Sumbu x"):
\end{eulerprompt}
\begin{eulercomment}
Selain itu juga dapat dengan cara lain seperti contoh berikut:
\end{eulercomment}
\begin{eulerprompt}
>expr := "x^3-x"; ...
>  plot2d(expr,title="y="+expr,xl="Sumbu x",yl="Sumbu y"); ...
>  label("(1,0)",1,0);  label("Max",E,expr(E),pos="lc"): 
\end{eulerprompt}
\eulerheading{Sub Bab 8 }
\begin{eulercomment}
Mengatur ukuran gambar,format(style),dan warna kurva 


Untuk mengubah ukuran, dapat dilakukan dengan menggunakan
aspect="...", semakin besar nilai aspect, maka ukuran kurva akan
semakin kecil, begitupun sebaliknya

untuk mengganti style, dapat dipilih dengan berbagai pilihan\\
style="...", dapat dipilih dari, misal : "-","\_',"-.",".-.","-.-".

untuk warna dapat dipilih sebagai salah satu warna default\\
color="...", warna default= red,green,blue,yellow, dll

sebagai contoh:
\end{eulercomment}
\begin{eulerprompt}
>aspect(1); plot2d("exp(x^2-3)"):
\end{eulerprompt}
\begin{eulercomment}
ukuran kurva dapat diganti dengan mengganti nilai aspect="...",
semakin besar nilai aspect, maka ukuran kurva akan semakin kecil Untuk
mengganti warna dapat ditambahkan dengan color="...", sedangkan untuk
mengganti format(style) dapat dilakukan dengan menambahkan style="..."
\end{eulercomment}
\begin{eulerprompt}
>aspect(2); plot2d("exp(x^2-3)", color=red, style="--"):
\end{eulerprompt}
\begin{eulercomment}
Berikut adalah tampilan warna EMT yang telah ditentukan
\end{eulercomment}
\begin{eulerprompt}
>aspect (1) ; columnsplot (ones(1,16),lab=0:15,grid=0, color=0:15) :
\end{eulerprompt}
\begin{eulercomment}
selain menggunakan warna default, untuk mengubah warna dapat juga
dengan menggunakan kode warna di atas\\
sebagai contoh:
\end{eulercomment}
\begin{eulerprompt}
>aspect(1); plot2d("exp(x^3+2*x)",r=3, color=1, style="--"):
\end{eulerprompt}
\eulerheading{Sub Bab 9 }
\begin{eulercomment}
Menggambar Sekumpulan Kurva dalam satu perintah plot2d. 


Dalam pembahasan sub-bab 9 kali ini akan membahas mengenai bagaimana
menggambar sekumpulan kurva dalam satu perintah plot2d. Menggambar
sekumpulan kurva dalam satu perintah plot2d adalah teknik yang
digunakan untuk memvisualisasikan beberapa fungsi dalam satu grafik.
Ini memudahkan perbandingan antara beberapa kurva.\\
\end{eulercomment}
\eulersubheading{}
\eulersubheading{Contoh}
\begin{eulerprompt}
>plot2d(["x^2","2*x"],-3,3):
\end{eulerprompt}
\begin{eulerttcomment}
 - Dalam contoh ini, merupakan gambar dua kurva sekaligus, yaitu x^2
\end{eulerttcomment}
\begin{eulercomment}
dan 2x, pada rentang -3 hingga 3.

\end{eulercomment}
\begin{eulerttcomment}
 - Hasilnya akan menunjukkan grafik dari kedua fungsi tersebut, dan
\end{eulerttcomment}
\begin{eulercomment}
titik-titik potongan antara keduanya adalah solusi dari persamaan
kuadrat.
\end{eulercomment}
\begin{eulerprompt}
>plot2d(["sin(x)","cos(x)"],0,2pi):
\end{eulerprompt}
\begin{eulercomment}
- Pada contoh ini, merupakan gambar dua fungsi trigonometri, sin(x)
dan cos(x), pada rentang 0 hingga 2p.

- Ini akan menghasilkan dua grafik yang memperlihatkan hubungan antara
sin(x) dan cos(x) dalam rentang tersebut.
\end{eulercomment}
\begin{eulerprompt}
>plot2d(["sin(x)","cos(x)"],0,2pi,color=red:green):
\end{eulerprompt}
\begin{eulercomment}
Sama seperti contoh kedua, gambar sin(x) dan cos(x) pada rentang 0
hingga 2phi, tetapi Anda juga memberikan warna yang berbeda pada kedua
grafik (sin(x) berwarna merah dan cos(x) berwana hijau.
\end{eulercomment}
\begin{eulerprompt}
>plot2d(["sin(x)","cos(x)"],xmin=0,xmax=2pi):
\end{eulerprompt}
\begin{eulercomment}
Dalam contoh ini, menggunakan parameter `xmin` dan `xmax` untuk
mengatur rentang tampilan grafik pada 0 hingga 2p.

\end{eulercomment}
\begin{eulerprompt}
> 
>plot2d(["cos(x)","sin(3*x)"],xmin=0,xmax=2pi):
\end{eulerprompt}
\begin{eulercomment}
- ini Merupakan gambar dua fungsi, yaitu cos(x) dan sin(3*x), dalam
rentang 0 hingga 2p.\\
\end{eulercomment}
\begin{eulerttcomment}
  
\end{eulerttcomment}
\begin{eulercomment}
- Penjelasan mencakup konsep bahwa grafik berulang dalam rentang
tertentu karena fungsi-fungsi ini memiliki frekuensi, periode, dan
amplitudo yang berbeda.
\end{eulercomment}
\begin{eulerprompt}
>plot2d("cos(x)","sin(3*x)",xmin=0,xmax=2pi):
\end{eulerprompt}
\begin{eulercomment}
Sintaks diatas lebih menjelaskan bahaimana hubungan periodik grafik
fungsi dari 2 fungsi yaitu cos x dan sin 3x dari rentang khusus dimana
xmin dari 0 sampai 2pi, hal tersebut dapat terjadi karena fungsi cos x
dan fungsi sin 3x memiliki frekuensi, periode, dan amplitudo yang
berbeda. grafik akan berulang pada rentang tertentu dan menghasilkan
sebuah pola.
\end{eulercomment}
\begin{eulerprompt}
>x=linspace(0,2pi,1000); plot2d(sin(5x),cos(7x)):
\end{eulerprompt}
\begin{eulercomment}
sintaks linspace digunakan untuk menghasilkan vektor x dari rentang
yang telah ditentukan yaitu o sampai 2pi yang berisi 1000 nilai yang
teratur
\end{eulercomment}
\begin{eulerprompt}
>a:=5.6; f &= exp(-a*x^2)/a;
>plot2d(f,r=1,thickness=2):
\end{eulerprompt}
\begin{eulercomment}
- Fungsi f(x) yang merupakan hasil dari ekspresi exp(-a*x\textasciicircum{}2)/a. Fungsi
ini memiliki parameter a yang bergantung pada nilai yang diterapkan
sebelumnya.\\
- Menggunakan perintah plot2d untuk menggambar grafik dari fungsi
f(x).\\
Parameter r digunakan untuk mengatur rentang plot dan parameter
thickness digunakan untuk mengatur ketebalan garis grafik.
\end{eulercomment}
\begin{eulerprompt}
>plot2d(&diff(f,x),>add,style="--",color=red):
\end{eulerprompt}
\begin{eulercomment}
ini adalah grafik fungsi f dan grafik turunan pertama dari fungsi f.
sintaks r=1 digunakan untuk mengatur rentang yang akan ditampilkan
pada plot, r=1 berarti rentang dari -1 sampai 1.\\
sintaks \textgreater{}add digunakan untuk menambahkan grafik kedalam jendela grafik
yang sudah ada sebelumnya.
\end{eulercomment}
\begin{eulerprompt}
>plot2d("x^2",0,1,steps=1,color=red,n=10):
>plot2d("x^2",>add,steps=2,color=blue,n=10):
\end{eulerprompt}
\begin{eulercomment}
sintaks steps digunakan untuk mengatur jumlah langkah atau titik-titik
yang digunakan dalam plot.\\
dan sintaks n digunakan untuk mengatur jumlah step yang akan
digunakan. semakin bayak n, maka bentuk grafik akan semakin mendekati
aslinya.
\end{eulercomment}
\begin{eulerprompt}
>function f(x) &= x^x;
>plot2d(f,r=1,cx=1,cy=1,color=blue,thickness=2);
>plot2d(&diff(f(x),x),>add,color=red,style="-.-"):
\end{eulerprompt}
\begin{eulercomment}
sintaks cx=1, cy=1 digunakan untuk mengatur pusat tampilan grafik,
maka plot akan diatur dengan titik pusat (1,1).
\end{eulercomment}
\begin{eulerprompt}
>plot2d("(1-x)^10",0,1);
>for i=1 to 10; plot2d("bin(10,i)*x^i*(1-x)^(10-i)",>add); end;
>insimg;
\end{eulerprompt}
\begin{eulercomment}
dalam contoh kita menggambar serangkaian plot yang menggambarkan
distribusi binomial dengan berbagai nilai i dari 1 hingga 10. kali ini
kita menggunakan sintaks untuk melakukan looping pada fungsi yang
berasosiasi dengan koefisien binomial dengan kombinasi 10 item. ini
memungkinkan untuk memahami bagaimana distribusu binomial berubah
dengan berbagai parameter.
\end{eulercomment}
\begin{eulerprompt}
>x=linspace(0,1,500);
>n=10; k=(0:n)';
>y=bin(n,k)*x^k*(1-x)^(n-k);
>plot2d(x,y):
\end{eulerprompt}
\begin{eulercomment}
n adalah vektor baris\\
k adalah vektor kolom\\
y adalah matrik dari vektor baris dan vektor kolom tersebut dengan
menggunakanfungsi binomial.
\end{eulercomment}
\begin{eulerprompt}
>x=linspace(0,1,200); y=x^(1:10)'; plot2d(x,y,color=1:10):
>n=(1:10)'; plot2d("x^n",0,1,color=1:10):
>  
>function f(x,a) := 1/a*exp(-x^2/a); ...
>plot2d("f",-10,10;5,thickness=2,title="a=5"):
>plot2d(\{\{"f",1\}\},-10,10); ...
>for a=2:10; plot2d(\{\{"f",a\}\},>add); end:
\end{eulerprompt}
\eulerheading{Sub Bab 10 }
\begin{eulercomment}
Membuat Gambar Kurva yang Bersifat Interaktif 


Kode ini, menggunakan `plot2d` untuk membuat plot dari fungsi
matematika `2*x\textasciicircum{}3-a*x` dengan parameter `a`. Flag `\textgreater{}user` memungkinkan
interaksi pengguna. Setelah plot ditampilkan, pengguna dapat melakukan
beberapa tindakan interaktif.\\
Saat plot ditampilkan dengan flag `\textgreater{}user`, pengguna dapat melakukan
beberapa tindakan interaktif sebagai berikut:

- Perbesar dengan + atau -: Pengguna dapat memperbesar atau
memperkecil plot dengan menggunakan tombol + atau - pada keyboard.

- Pindahkan Plot dengan Tombol Kursor: Pengguna dapat menggeser plot
dengan menggunakan tombol kursor (panning).\\
- Pilih Jendela Plot dengan Mouse: Pengguna dapat memilih area
tertentu dalam plot dengan menggunakan mouse.

- Atur Ulang Tampilan dengan Spasi: Jika pengguna menekan tombol
spasi, maka tampilan plot akan diatur ulang ke jendela plot.\\
- Keluar dengan Kembali: Jika pengguna menekan tombol kembali, maka
pengguna dapat keluar dari interaksi plot.
\end{eulercomment}
\begin{eulerprompt}
>plot2d(\{\{"2*x^3-a*x",a=1\}\},>user,title="Press any key!"); ...
>insimg;  
> plot2d("exp(x)*sin(x)",user=true, ...
>  title="+/- or cursor keys (return to exit)"):
\end{eulerprompt}
\begin{eulercomment}
Berikut ini menunjukkan cara interaksi pengguna tingkat lanjut

Ini adalah pemanggilan fungsi plot2d yang digunakan untuk membuat plot
dari fungsi matematika exp(x)*sin(x). Parameter user=true menunjukkan
bahwa ini adalah plot yang interaktif, yang berarti pengguna dapat
berinteraksi dengan plot ini.

title="+/- or cursor keys (return to exit)": Ini adalah judul yang
akan ditampilkan di atas plot. Pesan ini memberi petunjuk kepada
pengguna tentang bagaimana mereka dapat berinteraksi dengan plot ini.
Mereka dapat menggunakan tombol + atau - atau tombol kursor untuk
berinteraksi dengan plot, dan tombol return (Enter) untuk keluar dari
interaksi.

Berikut ini menunjukkan cara interaksi pengguna tingkat lanjut:\\
- mousedrag(): Ini adalah fungsi bawaan yang digunakan untuk menunggu
event mouse atau keyboard. Fungsi ini dapat mendeteksi kejadian
seperti klik mouse, pergerakan mouse, atau penekanan tombol.\\
- dragpoints(): Fungsi ini memanfaatkan mousedrag() untuk memungkinkan
pengguna menyeret titik-titik pada plot. Ini berarti pengguna dapat
mengklik dan menarik titik-titik dalam plot sesuai dengan preferensi
mereka.

Kita membutuhkan fungsi plot terlebih dahulu. Sebagai contoh, kita
interpolasi dalam 5 titik dengan polinomial. Fungsi harus diplot ke
area plot tetap.
\end{eulercomment}
\begin{eulerprompt}
>function plotf(xp,yp,select) ...
\end{eulerprompt}
\begin{eulerudf}
    d=interp(xp,yp);
    plot2d("interpval(xp,d,x)";d,xp,r=2);
    plot2d(xp,yp,>points,>add);
    if select>0 then
      plot2d(xp[select],yp[select],color=red,>points,>add);
    endif;
    title("Drag one point, or press space or return!");
  endfunction
\end{eulerudf}
\begin{eulercomment}
Perhatikan parameter titik koma di plot2d (d dan xp), yang diteruskan
ke evaluasi fungsi interp(). Tanpa ini, kita harus menulis fungsi
plotinterp() terlebih dahulu, mengakses nilai secara global.

Sekarang kita menghasilkan beberapa nilai acak, dan membiarkan
pengguna menyeret poin.

kode berikut digunakan untuk menghasilkan beberapa nilai acak t dan
membiarkan pengguna menyeret titik-titik pada plot dengan menggunakan
fungsi dragpoints():
\end{eulercomment}
\begin{eulerprompt}
>t=-1:0.5:1; dragpoints("plotf",t,random(size(t))-0.5):
\end{eulerprompt}
\begin{eulercomment}
Ada juga fungsi, yang memplot fungsi lain tergantung pada vektor
parameter, dan memungkinkan pengguna menyesuaikan parameter ini.

Pertama kita membutuhkan fungsi plot.
\end{eulercomment}
\begin{eulerprompt}
>function plotf([a,b]) := plot2d("exp(a*x)*cos(2pi*b*x)",0,2pi;a,b);
\end{eulerprompt}
\begin{eulercomment}
Kemudian kita membutuhkan nama untuk parameter, nilai awal dan matriks
rentang nx2, opsional baris judul.\\
Ada slider interaktif, yang dapat mengatur nilai oleh pengguna. Fungsi
dragvalues() menyediakan ini.
\end{eulercomment}
\begin{eulerprompt}
>dragvalues("plotf",["a","b"],[-1,2],[[-2,2];[1,10]], ...
>  heading="Drag these values:",hcolor=black):
\end{eulerprompt}
\begin{eulercomment}
Dimungkinkan untuk membatasi nilai yang diseret ke bilangan bulat.
Sebagai contoh, kita menulis fungsi plot, yang memplot polinomial
Taylor derajat n ke fungsi kosinus.
\end{eulercomment}
\begin{eulerprompt}
>function plotf(n) ...
\end{eulerprompt}
\begin{eulerudf}
  plot2d("cos(x)",0,2pi,>square,grid=6);
  plot2d(&"taylor(cos(x),x,0,@n)",color=blue,>add);
  textbox("Taylor polynomial of degree "+n,0.1,0.02,style="t",>left);
  endfunction
\end{eulerudf}
\begin{eulercomment}
Sekarang kami mengizinkan derajat n bervariasi dari 0 hingga 20 dalam
20 pemberhentian. Hasil dragvalues() digunakan untuk memplot sketsa
dengan n ini, dan untuk memasukkan plot ke dalam buku catatan.
\end{eulercomment}
\begin{eulerprompt}
>nd=dragvalues("plotf","degree",3,[0,10],10,y=0.8, ...
>   heading="Drag the value:"); ...
>plotf(nd):
\end{eulerprompt}
\begin{eulercomment}
Berikut ini adalah demonstrasi sederhana dari fungsi tersebut.
Pengguna dapat menggambar di atas jendela plot, meninggalkan jejak
poin.
\end{eulercomment}
\begin{eulerprompt}
>function dragtest ...
\end{eulerprompt}
\begin{eulerudf}
    plot2d(none,r=1,title="Drag with the mouse, or press any key!");
    start=0;
    repeat
      \{flag,m,time\}=mousedrag();
      if flag==0 then return; endif;
      if flag==2 then
        hold on; mark(m[1],m[2]); hold off;
      endif;
    end
  endfunction
\end{eulerudf}
\eulerheading{Sub Bab 11 }
\begin{eulercomment}
Menggambar Kurva Fungsi Parametrik 


Kita telah terbiasa dengan kurva yang didefinisikan oleh sebuah
persamaan yang menghubungkan koordinat x dan y Contohnya\\
\end{eulercomment}
\begin{eulerformula}
\[
y=x^2
\]
\end{eulerformula}
\begin{eulercomment}
Atau\\
\end{eulercomment}
\begin{eulerformula}
\[
x^2+y^2=13
\]
\end{eulerformula}
\begin{eulercomment}
dimana persamaan-persamaan ini tidak dikaitkan dengan panjang kurva s
, waktu t, dan besaran lainnya. Besaran besaran ini disebut parameter\\
persamaan parametrik adalah persamaan yang menyatakan hubungan
variabel x, y dituliskan dengan\\
\end{eulercomment}
\begin{eulerformula}
\[
x=f(t)
\]
\end{eulerformula}
\begin{eulerformula}
\[
y=g(t)
\]
\end{eulerformula}
\begin{eulercomment}
dengan a\textless{}=t\textless{}=b tiap nilai t menentukan titik(x,y) pada kurva. Jadi ,
dengan berubahnya nilai t. titik\\
\end{eulercomment}
\begin{eulerformula}
\[
(x,y) = (f(t),g(t))
\]
\end{eulerformula}
\begin{eulercomment}
bergerak sepanjang kurva yang disebut kurva parametrik


Dalam contoh berikut, kita memplot spiral

\end{eulercomment}
\begin{eulerformula}
\[
\gamma(t) = t \cdot (\cos(2\pi t),\sin(2\pi t))
\]
\end{eulerformula}
\begin{eulercomment}
Kita perlu menggunakan banyak titik untuk tampilan yang halus
\end{eulercomment}
\begin{eulerprompt}
>t=linspace(0,1,1000); ...
>plot2d(t*cos(2*pi*t),t*sin(2*pi*t),r=1):
\end{eulerprompt}
\begin{eulerttcomment}
 r digunakan untuk mengatur radius marker titik-titik yang akan
\end{eulerttcomment}
\begin{eulercomment}
digunakan dalam plot.



Sebagai alternatif, dimungkinkan untuk menggunakan dua ekspresi untuk
kurva. Berikut ini plot kurva yang sama seperti di atas.
\end{eulercomment}
\begin{eulerprompt}
>plot2d("x*cos(2*pi*x)","x*sin(2*pi*x)",xmin=0,xmax=1,r=1):
\end{eulerprompt}
\begin{eulercomment}
Perintah linspace digunakan untuk membuat array nilai yang
terdistribusi secara merata antara dua angka tertentu. Fungsi ini
sangat berguna untuk menentukan rentang nilai yang ingin digunakan
pada sumbu x atau y ketika membuat plot.

\end{eulercomment}
\begin{eulerttcomment}
    0  : Nilai awal dari rentang.
    1  : Nilai akhir dari rentang.
 
\end{eulerttcomment}
\begin{eulercomment}

Perintah linspace akan menghasilkan array dengan n elemen yang
terdistribusi merata antara start dan stop.
\end{eulercomment}
\begin{eulerprompt}
>t=linspace(0,1,1000); r=exp(-t); x=r*cos(2pi*t); y=r*sin(2pi*t);
>plot2d(x,y,r=1):
\end{eulerprompt}
\begin{eulercomment}
exp(-t) menghasilkan nilai yang semakin mendekati nol seiring dengan
pertambahan nilai t, karena eksponensial dari nilai negatif semakin
mendekati nol saat nilai t semakin besar.\\
Jadi, r = exp(-t) memberikan suatu fungsi yang menurun dengan nilai t.
Dalam konteks program ini, r digunakan untuk mengontrol jari-jari dari
kurva spiral dalam plot 2D. Jari-jari ini semakin kecil seiring dengan
pertambahan nilai t, menciptakan efek spiral yang semakin rapat ke
pusat pada bagian ujung kurva.





Pada contoh berikutnya, kita memplot kurvanya

\end{eulercomment}
\begin{eulerformula}
\[
\gamma(t) = (r(t) \cos(t), r(t) \sin(t))
\]
\end{eulerformula}
\begin{eulercomment}
dengan

\end{eulercomment}
\begin{eulerformula}
\[
r(t) = 1 + \dfrac{\sin(3t)}{2}.
\]
\end{eulerformula}
\begin{eulerprompt}
>t=linspace(0,2pi,1000); r=1+sin(3*t)/2; x=r*cos(t); y=r*sin(t); ...
>plot2d(x,y,>filled,fillcolor=red,style="/",r=1.5):
\end{eulerprompt}
\eulersubheading{Contoh lain}
\begin{eulerprompt}
>t=linspace(-3,3,1000); x=2*t+1; y=t^2-1;
>plot2d(x,y,r=8):
>t=linspace(0,2pi,1000); r=3; x=r*cos(t); y=r*sin(t);...
>plot2d(x,y,>filled,fillcolor=green,style="/",r=5):
>t=linspace(-1,1,1000); x=t^2; y=2*t;...
>plot2d(x,y):
>t=linspace(0,2pi,1000); x=3*cos(t); y=2*sin(t);...
>plot2d(x,y,>filled,fillcolor=green,style="/",r=3):
\end{eulerprompt}
\eulerheading{Sub Bab 12 }
\begin{eulercomment}
Menggambar Kurva Fungsi Implisit 


Fungsi implisit adalah fungsi yang memuat lebih dari satu variabel,
berjenis variabel bebas dan variabel terikat yang berada dalam satu
ruas sehingga tidak bisa dipisahkan pada ruas yang berbeda.

Untuk fungsi implisit, harus berupa fungsi atau ekspresi dari
parameter x dan y.

\end{eulercomment}
\begin{eulerformula}
\[
f(x,y)=c
\]
\end{eulerformula}
\begin{eulercomment}
Untuk menggambar himpunan f(x,y)=c untuk satu atau lebih konstanta c,
dapat menggunakan\\
plot2d().

Fungsi implisit juga dapat diisi dengan persamaan tingkat

\end{eulercomment}
\begin{eulerformula}
\[
a<=f(x,y)<=b
\]
\end{eulerformula}
\begin{eulercomment}
Untuk fungsi ini harus berupa matriks 2xn dimana baris pertama berisi
awal dan baris kedua adalah akhir dari setiap interval.

Plot implisit menunjukkan garis level yang menyelesaikan f(x,y)=level,
di mana "level" dapat berupa nilai tunggal atau vektor nilai. Jika
level="auto", akan ada garis level nc, yang akan menyebar antara
fungsi minimum dan maksimum secara merata. Warna yang lebih gelap atau
lebih terang dapat ditambahkan dengan \textgreater{}hue untuk menunjukkan nilai
fungsi. Untuk fungsi implisit, xv harus berupa fungsi atau ekspresi
dari parameter x dan y, atau, sebagai alternatif, xv dapat berupa
matriks nilai.

Euler dapat menandai garis level

\end{eulercomment}
\begin{eulerformula}
\[
f(x,y) = c
\]
\end{eulerformula}
\begin{eulercomment}
dari fungsi apapun.

Untuk menggambar himpunan f(x,y)=c untuk satu atau lebih konstanta c,
Anda dapat menggunakan plot2d() dengan plot implisitnya di dalam
bidang. Parameter untuk c adalah level=c, di mana c dapat berupa
vektor garis level. Selain itu, skema warna dapat digambar di latar
belakang untuk menunjukkan nilai fungsi untuk setiap titik dalam plot.
Parameter "n" menentukan kehalusan plot.

\end{eulercomment}
\eulersubheading{Contoh Soal}
\begin{eulercomment}
\end{eulercomment}
\begin{eulerformula}
\[
x^2+y^2-xy-x = 0
\]
\end{eulerformula}
\begin{eulerprompt}
>aspect(2)
>plot2d("x^2+y^2-x*y-x",r=1.5,level=0,contourcolor=green):
\end{eulerprompt}
\begin{eulerformula}
\[
2x^2+xy+3y^4+y = 0
\]
\end{eulerformula}
\begin{eulerprompt}
>expr := "2*x^2+x*y+3*y^4+y"; // define an expression f(x,y)
>plot2d(expr,level=0,contourcolor=green): // Solutions of f(x,y)=0
>plot2d(expr,level=0:0.5:20,>hue,contourcolor=white,n=200): // nice
\end{eulerprompt}
\begin{eulercomment}
Parameter \textgreater{}hue digunakan untuk memberikan warna pada kontur sesuai
dengan levelnya. Kontur dengan level yang lebih tinggi akan memiliki
warna yang berbeda.
\end{eulercomment}
\begin{eulerprompt}
>plot2d(expr,level=0:0.5:20,>hue,>spectral,n=200,grid=4): // nicer
\end{eulerprompt}
\begin{eulercomment}
\textgreater{}spectral digunakan untuk mengatur palet warna yang akan digunakan
pada kontur. Dalam hal ini, digunakan palet warna "spectral".
\end{eulercomment}
\begin{eulerprompt}
>x=-2:0.05:1; y=x'; z=expr(x,y);
>plot2d(z,level=0,a=-1,b=2,c=-2,d=1,>hue):
>plot2d("x^3-y^2",>contour,>hue,>spectral):
\end{eulerprompt}
\begin{eulercomment}
Perintah \textgreater{}contour adalah cara untuk menghasilkan plot kontur, yaitu
plot yang menunjukkan garis-garis kontur yang mewakili tingkat-tingkat
dari suatu fungsi. Jumlah dan posisi garis kontur akan secara otomatis
diatur oleh Euler Math Toolbox berdasarkan distribusi nilai-nilai
fungsi.

Penggunaan level memungkinkan Anda secara eksplisit menentukan tingkat
kontur yang ingin kita tampilkan pada plot. kita dapat mengatur level
kontur sesuai dengan preferensi kita, dan plot akan menampilkan garis
kontur pada tingkat-tingkat yang kita tentukan.
\end{eulercomment}
\begin{eulerprompt}
>plot2d("x^3-y^2",level=0,contourwidth=3,>add,contourcolor=red):
>z=z+normal(size(z))*0.2;
>plot2d(z,level=0.5,a=-1,b=2,c=-2,d=1):
\end{eulerprompt}
\begin{eulercomment}
normal(size(z)) menghasilkan matriks dengan ukuran yang sama dengan
matriks z, dan setiap elemennya diambil dari distribusi normal standar
(mean 0, deviasi standar 1). Kemudian, matriks z diubah dengan
menambahkan nilai-nilai acak ini, yang telah dikalikan dengan 0.2. Ini
menciptakan variasi acak dalam matriks z.

\end{eulercomment}
\begin{eulerprompt}
>plot2d(expr,level=[0:0.2:5;0.05:0.2:5.05],color=lightgray):
>plot2d("x^2+y^3+x*y",level=1,r=4,n=100):
>plot2d("x^2+2*y^2-x*y",level=0:0.1:10,n=100,contourcolor=white,>hue):
>plot2d(expr,level=[0;1],style="-",color=blue): // 0 <= f(x,y) <= 1
>plot2d("x^4+y^4",r=1.5,level=[0;1],color=blue,style="/"):
>plot2d("x^2+y^3+x*y",level=[0,2,4;1,3,5],style="/",r=2,n=100):
>plot2d("x^2+y^3+x*y",level=-10:20,r=2,style="-",dl=0.1,n=100):
\end{eulerprompt}
\begin{eulercomment}
dl Parameter ini mengatur tingkat penghalusan pada plot. Semakin kecil
nilai ini, semakin halus plotnya.
\end{eulercomment}
\begin{eulerprompt}
>plot2d("sin(x)*cos(y)",r=pi,>hue,>levels,n=100):
>plot2d("(x^2+y^2-1)^3-x^2*y^3",r=1.3, ...
>style="/",color=red,<outline, ...
>level=[-2;0],n=100):
\end{eulerprompt}
\begin{eulercomment}
\textless{}outline: Parameter ini mengatur plot agar hanya memiliki kontur saja
tanpa diisi.



Misal plot solusi dari persamaan

\end{eulercomment}
\begin{eulerformula}
\[
x^3-xy+x^2y^2=6
\]
\end{eulerformula}
\begin{eulerprompt}
>plot2d("x^3-x*y+x^2*y^2",r=6,level=6,n=100):
>plot2d("x^2+y^2-1",level=0):
>plot2d("x^2+y^2-1",r=3,level=0:1:10,n=200):
>plot2d("x^2+y^2-1",r=3,level=0:1:10,>hue,contourcolor=white):
>plot2d("x^2+y^2-1",r=3,level=0:1:20,>hue,>spectral,n=200,grid=4):
>plot2d("x^2+y^2-1",level=0,a=-2,b=2,c=-2,d=2,>hue):
>plot2d("x^2+y^2-1",>contour,>hue,>spectral):
>plot2d("x^2+y^2-1",level=[0:0.2:5;0.05:0.2:5.05],color=lightgray):
>plot2d("x^2+y^2-1",r=2,level=[0;1],style="-",color=blue): // 0 <= f(x,y) <= 1
>plot2d("x^2+y^2",r=1.5,level=[0;1],color=blue,style="/"):
>plot2d("x^2+y^2-1",level=[0,2,4;1,3,5],style="/",r=2,n=100):
>plot2d("x^2+y^2-1",level=-10:20,r=3,style="-",dl=0.1,n=100):
\end{eulerprompt}
\eulerheading{Sub bab 13 }
\begin{eulercomment}
Menggambar Grafik Bilangan Kompleks 


Bilangan kompleks secara visual dapat direpresentasikan sebagai
sepasang angka (a, b) membentuk vektor pada diagram yang disebut
diagram Argand, mewakili yang bidang kompleks. Sumbu-x adalah sumbu
nyata dan sumbu-y adalah sumbu imajiner.

Menggambar kurva fungsi kompleks sendiri adalah proses visualisasi
grafis dari fungsi matematika kompleks (yaitu fungsi yang melibatkan
bilangan kompleks, yaitu bilangan dengan bagian real dan imajiner)
berperilaku dalam koordinas kompleks. Hal tersebut memungkinkan untuk
melihat bagaimana pola, bentuk, dan sifat dari fungsi kompleks
tersebut.

Array bilangan kompleks juga dapat diplot. Kemudian titik-titik grid
akan terhubung. Jika sejumlah garis kisi ditentukan (atau vektor garis
kisi 1x2) dalam argumen cgrid, hanya garis kisi tersebut yang
terlihat.

Matriks bilangan kompleks akan secara otomatis diplot sebagai kisi di
bidang kompleks.

\textgreater{} Definisi fungsi kompleks, mendefinisikan fungsi kompleks yang
dianalisis atau digambarkan. Fungsi ini memiliki variabel kompleks z,
yang melibatkan bagian real dan imajiner.\\
\textgreater{} Selanjutnya kita dapat menggunakan fungsi linspace. Fungsi linspace
sendiri adalah salah satu fungsi yang umum digunakan dalam
pemrograman, terutama dalam konteks pemrograman numerik dan ilmu data.
Ini sering digunakan untuk menghasilkan urutan nilai dalam rentang
tertentu dengan jumlah titik yang sama di antara dua titik ujungnya.
Penggunaannya tidak terbatas pada pemrosesan sinyal atau
elektromagnetik, tetapi bisa digunakan dalam berbagai konteks di mana
Anda perlu membuat urutan nilai.\\
\textgreater{} Penentuan rentang, memilih rentang nilai z yang ingin ditampilkan di
dalam plot. Rentang ini mencakup wilayah kompleks tertentu yang ingin
diamati.\\
\textgreater{} Menggunakan sintaks plot2d.\\
\textgreater{} Penyesuaian plot, mengubah plot sesuai yang diinginkan (mengubah
warna, format (style), dan sebagainya).

Dalam contoh berikut, kami memplot gambar lingkaran satuan di bawah
fungsi eksponensial. Parameter cgrid menyembunyikan beberapa kurva
grid.

\begin{eulercomment}
\eulerheading{Contoh}
\begin{eulerprompt}
>aspect(); r=linspace(0,1,50); a=linspace(0,2pi,80)'; z=r*exp(I*a);...
>plot2d(z,a=-1.25,b=1.25,c=-1.25,d=1.25,cgrid=10):
\end{eulerprompt}
\begin{eulercomment}
Penjelasan sintaks

z       : sebuah ekspresi atau fungsi yang akan digambar dalam
koordinat kompleks.\\
a,b,c,d : parameter-parameter yang digunakan untuk mengatur jendela
tampilan (viewport) dalam koordinat kompleks. Parameter-parameter ini
akan menentukan rentang sumbu x dan sumbu y yang akan ditampilkan di
dalam plot.\\
cgrid   : parameter ini mengontrol tampilan grid pada plot. Jika
cgrid=n, maka grid akan ditampilkan, jika cgrid=0, maka grid akan
disembunyikan.

\begin{eulercomment}
\eulerheading{Bentuk lain}
\begin{eulerprompt}
>aspect(1.25); r=linspace(0,1,50); a=linspace(0,2pi,200)'; z=r*exp(I*a);
>plot2d(exp(z),cgrid=[40,10]):
\end{eulerprompt}
\begin{eulercomment}
Penjelasan :\\
Perintah tersebut merupakan perintah untuk menggambar kurva dari
fungsi kompleks eksponensial "exp(z)" dalam koordinat kompleks. Dalam
perintah tersebut juga menggunakan parameter cgrid dengan nilai
[40,10] untuk mengatur grid pada plot.\\
Dalam sintaks ini,\\
exp(z) : fungsi eksponensial kompleks yang akan digambar\\
cgrid=[40,10] : mengatur grid pada plot. cgrid tersebut adalah jumlah
garis grid yang akan digunakan pada sumbu x dan sumbu y. Nah di dalam
plot ini, akan ada 40 garis grid pada sumbu x dan 10 grid pada sumbu
y.

\begin{eulercomment}
\eulerheading{Bentuk lain}
\begin{eulerprompt}
>r=linspace(0,1,10); a=linspace(0,2pi,40)'; z=r*exp(I*a);
>plot2d(exp(z),>points,>add):
\end{eulerprompt}
\begin{eulercomment}
Sebuah vektor bilangan kompleks secara otomatis diplot sebagai kurva
pada bidang kompleks dengan bagian real dan bagian imajiner.

Penjelasan :\\
Perintah plot2d di atas adalah perintah untuk menggambar kurva fungsi
kompleks dalam koordinat kompleks, namun dengan opsi yang berbeda,\\
exp(z): fungsi kompleks yang akan digambar\\
\textgreater{}points : opsi ini mengubah cara plot untuk dilakukan. Dengan
menggunakan \textgreater{}points, plot ini akan menggunakan titik-titik diskrit
untuk merepresentasikan fungsi ke dalam bentuk titik,titik\\
\textgreater{}add    : sintaks ini menginstrusikan perintah untuk menambahkan plot
ini ke plot sebelumnya jika ada.

\begin{eulercomment}
\eulerheading{Contoh}
\begin{eulerprompt}
>t=linspace(0,2pi,1000); ...
>plot2d(exp(I*t)+exp(10*I*t),r=3):
\end{eulerprompt}
\begin{eulercomment}
Penjelasan :\\
Perintah plot2d di atas menggambarkan kurva dari fungsi kompleks yang
diberikan dalam koordinat kompleks dengan parameter-parameter
tertentu.\\
Sintaks yang digunakan yaitu,\\
exp(I*t)+exp(10*I*t) : fungsi kompleks yang akan digambar. Fungsi ini
terdiri dari dua bagian yang masing-masing merupakan fungsi kompleks
eksponensial. Dengan 10 adalah berapa kali putaran dalam gambar
tersebut.\\
r : parameter r digunakan untuk menentukan rentang nilai dari variabel
t. Dalam contoh ini, r=3, yaitu mengatur rentang nilai t dari 3 hingga
3.

\begin{eulercomment}
\eulerheading{Sub Bab 14 }
\begin{eulercomment}
Menggambar Daerah Yang Dibatasi Kurva 


Plot data benar-benar poligon. Kita juga dapat memplot kurva atau
kurva terisi.

Pada subtopik sebelumnya telah kita ketahui dan pelajari bersama bahwa
EMT dapat melakukan visualisasi plot mulai dari bentuk ekspresi
langsung hingga plot dari fungsi-fungsi. Pada subtopik ini merupakan
kelanjutan dari subtopik sebelumnya, bahwa kita dapat
membentuk/menggambar daerah dari perpotongan beberapa kurva yang telah
didefinisikan. Hal ini dapat bermanfaat untuk membantu dalam
menyelesaikan permasalahan dalam matematika, salah satu contohnya
seperti optimasi program linear, dimana disajikan beberapa
fungsi-fungsi kendala beserta dengan fungsi tujuannya dan perlu
divisualisasikan dalam bentuk grafik untuk melihat dimana letak daerah
layaknya untuk menentukan nilai optimum.

Dalam EMT ada beberapa perintah yang digunakan untuk menggambar daerah
yang dibatasi oleh beberapa kuva, diantaranya yaitu:

- plot2d\\
\end{eulercomment}
\begin{eulerttcomment}
  Digunakan untuk melakukan plotting.
\end{eulerttcomment}
\begin{eulercomment}

- filled=true\\
\end{eulercomment}
\begin{eulerttcomment}
  Digunakan untuk memberikan isian/arsiran pada daerah/area di bawah
\end{eulerttcomment}
\begin{eulercomment}
kurva saat plotting.

- style="..."\\
\end{eulercomment}
\begin{eulerttcomment}
  Digunakan untuk memilih gaya kurva yang akan digunakan saat
\end{eulerttcomment}
\begin{eulercomment}
plotting. Anda dapat memilih dari beberapa gaya, seperti "#", "/",
"\textbackslash{}", atau "-". Dan hal ini mempengaruhi tampilan daerah kurva yang
terbentuk.

- fillcolor\\
\end{eulercomment}
\begin{eulerttcomment}
  Digunakan untuk menentukan warna isian yang akan digunakan untuk
\end{eulerttcomment}
\begin{eulercomment}
mengiri area di bawah kurva.

\end{eulercomment}
\eulersubheading{Contoh}
\begin{eulerprompt}
>t=linspace(0,2pi,1000); // parameter for curve
>x=sin(t)*exp(t/pi); y=cos(t)*exp(t/pi); // x(t) and y(t)
>figure(1,2); aspect(16/9)
>figure(1); plot2d(x,y,r=10); // plot curve
>figure(2); plot2d(x,y,r=10,>filled,style="/",fillcolor=red); // fill curve
>figure(0):
\end{eulerprompt}
\begin{eulercomment}
Penjelasan:

- t=linspace(0,2pi,1000);\\
Pada langkah pertama yaitu mendefinisikan parameter t sebagai
serangkaian 1000 titik antara 0 dan 2pi. Parameter t ini akan
digunakan sebagai parameter untuk menggambar kurva.

- x=sin(t)*exp(t/pi); y=cos(t)*exp(t/pi); // x(t) and y(t)\\
Kemudian kita definisika dua vektor x dan y yang merupakan koordinat x
dan y dari kurva yang akan digambar. Fungsi\\
\end{eulercomment}
\begin{eulerformula}
\[
sin(t)*exp(t/pi)
\]
\end{eulerformula}
\begin{eulercomment}
digunakan untuk menghitung komponen x (x(t)), dan\\
\end{eulercomment}
\begin{eulerformula}
\[
cos(t)*exp(t/pi)
\]
\end{eulerformula}
\begin{eulercomment}
digunakan untuk menghitung komponen y (y(t)) dari kurva.

- figure(1,2); aspect(16/9)\\
Perintah ini digunakan untuk mengatur tampilan gambar. Perintah
figure(1,2) digunakan membuat dua gambar (1 dan 2) dalam satu jendela
gambar. Dan perintah aspect(16/9) mengatur rasio aspek gambar menjadi
16:9, yang mempengaruhi bentuk dan ukuran gambar yang akan digambar.

- figure(1); plot2d(x,y,r=10); // membuat plot kurva\\
Perintah ini memilih gambar pertama (1) dan menggunakan perintah
plot2d untuk menggambar kurva yang dihitung sebelumnya. Parameter r=10
mengatur lebar garis plot. Ini menghasilkan kurva tanpa adanya isi
atau arsiran di dalamnya.

- figure(2); plot2d(x,y,r=10,\textgreater{}filled,style="/",fillcolor=red); // fill
curve\\
Selanjutnya pada perintah ini beralih ke gambar kedua (2) dan
menggunakan perintah plot2d lagi untuk menggambar kurva yang sama
dengan pengisian area di bawahnya. Perintah \textgreater{}filled digunakan untuk
mengisi area di bawah kurva, style="/" digunakan untuk mengatur gaya
garis menjadi garis miring, dan fillcolor=red digunakan untuk mengatur
warna isian menjadi merah.

-figure(0):\\
Baris perintah ini digunakan untuk mengakhiri gambar dan kembali ke
tampilan biasa tanpa gambar. Ini berfungsi untuk menyelesaikan proses
penggambaran.

\end{eulercomment}
\eulersubheading{Contoh}
\begin{eulerprompt}
>x=linspace(0,2pi,1000); plot2d(cos(x),sin(x)*0.5,r=1,>filled,style="\(\backslash\)"):
\end{eulerprompt}
\begin{eulercomment}
Penjelasan:

- x=linspace(0,2pi,100);\\
Mendefinisikan vektor x dengan menggunakan perintah linspace. linspace
digunakan untuk membuat vektor dengan 100 titik yang secara merata
tersebar antara 0 dan 2phi. Dalam konteks ini, vektor x akan digunakan
sebagai parameter saat menggambar kurva.

- plot2d(cos(x),sin(x)*0.5,r=1,\textgreater{}filled,style="\textbackslash{}"):\\
Ini merupakan perintah utama yang digunakan untuk menggambar plot.
Perintah ini memiliki beberapa parameter sebagai berikut:\\
\textgreater{} cos(x) adalah komponen x dari kurva. Ini adalah hasil dari fungsi
kosinus yang dihitung pada vektor x.\\
\textgreater{} sin(x)*0.5 adalah komponen y dari kurva. Ini adalah hasil dari
fungsi sinus yang dihitung pada vektor x dan kemudian dikalikan dengan
0,5, yang mengubah amplitudonya.\\
\textgreater{} r=1 mengatur lebar garis plot menjadi 1.\\
\textgreater{} filled digunakan untuk mengisi area di bawah kurva, sehingga
menciptakan daerah yang terisi.\\
\textgreater{} style="\textbackslash{}" mengatur gaya garis kurva untuk membentuk garis miring
yang gunanya menutupi semua bagian kurva dengan garis miring.

\end{eulercomment}
\eulersubheading{Contoh}
\begin{eulerprompt}
>t=linspace(0,2pi,6); ...
>plot2d(cos(t),sin(t),>filled,style="/",fillcolor=red,r=1.5):
\end{eulerprompt}
\begin{eulercomment}
Penjelasan:

- t=linspace(0,2pi,6); ...\\
\end{eulercomment}
\begin{eulerttcomment}
  Pada perintah ini, kita definisikan vektor t dengan menggunakan
\end{eulerttcomment}
\begin{eulercomment}
perintah linspace. linspace digunakan untuk membuat vektor dengan 6
titik yang terletak secara merata antara 0 dan 2pi. Dalam konteks ini,
vektor t akan digunakan sebagai parameter saat menggambar kurva.

- plot2d(cos(t),sin(t),\textgreater{}filled,style="/",fillcolor=red,r=1.5):\\
\end{eulercomment}
\begin{eulerttcomment}
  Ini adalah perintah utama yang digunakan untuk menggambar plot.
\end{eulerttcomment}
\begin{eulercomment}
Perintah ini memiliki beberapa parameter sebagai berikut:\\
\textgreater{} cos(t) adalah komponen x dari kurva.\\
\textgreater{} sin(t) adalah komponen y dari kurva.\\
\textgreater{} filled digunakan untuk mengisi area di bawah kurva, sehingga
menciptakan bentuk yang terisi. Ini berarti daerah di bawah kurva akan
diwarnai.\\
\textgreater{} style="/" mengatur gaya garis kurva menjadi garis miring ("/").\\
\textgreater{} fillcolor=orange mengatur warna isian daerah di bawah kurva menjadi
oranye.\\
\textgreater{} r=1.5 mengatur lebar garis plot menjadi 1.5.

\end{eulercomment}
\eulersubheading{Contoh}
\begin{eulerprompt}
>t=linspace(0,2pi,6); plot2d(cos(t),sin(t),>filled,style="#"):
\end{eulerprompt}
\begin{eulercomment}
Penjelasan:\\
- t=linspace(0,2pi,6);\\
Pada perintah ini, kita definisikan vektor t dengan menggunakan
perintah linspace. linspace digunakan untuk membuat vektor dengan 6
titik yang terletak secara merata antara 0 dan 2phi. Dalam konteks
ini, vektor t akan digunakan sebagai parameter saat menggambar kurva.

- plot2d(cos(t),sin(t),\textgreater{}filled,style="#"):\\
Ini adalah perintah utama yang digunakan untuk menggambar plot.
Perintah ini memiliki beberapa parameter sebagai berikut:\\
\textgreater{} cos(t) adalah komponen x dari kurva.\\
\textgreater{} sin(t) adalah komponen y dari kurva.\\
\textgreater{} filled digunakan untuk mengisi area di bawah kurva, sehingga
menciptakan bentuk yang terisi. Ini berarti daerah di bawah kurva akan
diisi dengan warna atau pola tertentu.\\
\textgreater{} style="#" mengatur isian kurva menjadi warna solid dengan
menggunakan simbol tanda pagar ("#")

Pada contoh ini tidak ada perintah untuk mengatur warna, maka warna
yang dihasilkan pada plot ini akan mengikuti pada warna yang disetting
pada bagian sebelumnya.

\end{eulercomment}
\eulersubheading{Contoh}
\begin{eulercomment}
Contoh lainnya adalah segi empat, yang kita buat dengan 7 titik pada
lingkaran satuan.
\end{eulercomment}
\begin{eulerprompt}
>t=linspace(0,2pi,7);  ...
>plot2d(cos(t),sin(t),r=1,>filled,style="/",fillcolor=orange):
\end{eulerprompt}
\begin{eulercomment}
Penjelasan:

- t=linspace(0,2pi,7);:\\
Fungsi linspace digunakan untuk membuat array berisi sejumlah nilai
yang merata dalam rentang tertentu. Dalam hal ini, rentangnya adalah
dari 0 hingga 2p (dua kali nilai p) dan sebanyak 7 titik akan
dihasilkan. Ini akan digunakan sebagai sudut dalam koordinat polar
untuk menggambarkan data.

- plot2d(cos(t),sin(t),r=1,\textgreater{}filled,style="/",fillcolor=orange):\\
Ini adalah perintah untuk melakukan plotting data. Terdapat beberapa
argumen di sini:\\
\textgreater{} cos(t): Ini adalah nilai kosinus dari setiap elemen dalam array t.
Ini akan digunakan sebagai komponen sumbu Y dalam koordinat polar.\\
\textgreater{} sin(t): Ini adalah nilai sinus dari setiap elemen dalam array t. Ini
akan digunakan sebagai komponen sumbu X dalam koordinat polar.\\
\textgreater{} r=1: Ini adalah argumen opsional yang menentukan radius plot. Dalam
hal ini, radiusnya diatur menjadi 1.\\
\textgreater{} filled: Ini adalah argumen yang menginstruksikan untuk mengisi area
di dalam kurva plot.\\
\textgreater{} style="/": Ini adalah argumen yang menentukan gaya garis yang
digunakan untuk plot. Di sini, garisnya akan berbentuk garis miring
("/").\\
\textgreater{} fillcolor=orange: Ini adalah argumen yang menentukan warna pengisian
untuk area di dalam kurva plot. Dalam hal ini, warnanya diatur menjadi
oren.

\end{eulercomment}
\eulersubheading{Contoh}
\begin{eulerprompt}
>A=[2,1;1,2;-1,0;0,-1];
>function f(x,y) := max([x,y].A');
>plot2d("f",r=4,level=[0;3],color=red,n=111):
\end{eulerprompt}
\begin{eulercomment}
Penjelasan:

- A=[2,1;1,2;-1,0;0,-1];\\
Ini adalah perintah untuk membuat matriks A. Matriks ini memiliki
dimensi 4x2, yang berarti memiliki 4 baris dan 2 kolom. Isinya adalah:\\
\end{eulercomment}
\begin{eulerformula}
\[
\begin {bmatrix} 2 \hspace{10pt} 1 \\ 1 \hspace{10pt} 2 \\ \end{bmatrix}
\]
\end{eulerformula}
\begin{eulercomment}
- function f(x,y) := max([x,y].A');\\
Ini adalah perintah untuk mendefinisikan sebuah fungsi bernama f(x,
y). Fungsi ini mengambil dua argumen input, yaitu x dan y. Fungsi ini
melakukan operasi berikut:

\textgreater{} [x, y] menghasilkan vektor baris dengan elemen [x, y].\\
\textgreater{} [x, y].A' adalah perkalian dot (dot product) dari vektor baris [x,
y] dengan transpose dari matriks A.\\
\textgreater{} max([x, y].A') menghitung nilai maksimum dari hasil perkalian dot
tersebut.

Dengan kata lain, fungsi `f(x, y)` mengambil vektor `[x, y]` sebagai
input, mengalikannya dengan matriks `A`, dan mengembalikan nilai
maksimum dari hasil perkalian tersebut.

- plot2d("f",r=4,level=[0;3],color=red,n=111):\\
Ini adalah perintah untuk membuat plot 2D dari fungsi `f(x, y)` yang
telah didefinisikan. Rincian perintah ini adalah sebagai berikut:

\textgreater{} "f" adalah nama fungsi yang akan diplot.\\
\textgreater{} r=4 menentukan rentang plot, yang dalam hal ini adalah [-4, 4] untuk
kedua sumbu x dan y.\\
\textgreater{} level=[0;3] menentukan tingkat kontur (contour levels) yang akan
digunakan dalam plot. Ada dua tingkat kontur: 0 dan 3.\\
\textgreater{} color=green mengatur warna kontur plot menjadi merah.\\
\textgreater{} n=111 mengendalikan jumlah titik yang digunakan dalam plot.

Hasilnya akan menjadi sebuah grafik kontur 2D dari fungsi `f(x, y)`
dengan kontur berwarna merah pada tingkat 0 dan 3, yang mencakup
rentang -4 hingga 4 pada kedua sumbu x dan y.

\end{eulercomment}
\eulersubheading{Contoh}
\begin{eulerprompt}
>t=linspace(0,2pi,1000); x=cos(3*t); y=sin(4*t);
>plot2d(x,y,<grid,<frame,>filled):
\end{eulerprompt}
\begin{eulercomment}
Penjelasan:

- t = linspace(0, 2*pi, 1000);\\
Ini adalah perintah untuk membuat vektor t yang berisi 1000 nilai yang
merata terdistribusi antara 0 hingga 2pi. Vektor t ini akan digunakan
sebagai parameter waktu atau sudut dalam parameterisasi lingkaran.\\
linspace(0, 2*pi, 1000) membuat 1000 titik antara 0 hingga 2pi,
memberikan sudut-sudut yang merata di sepanjang satu putaran
lingkaran.

- x = cos(3*t); y = sin(4*t);\\
Ini adalah perintah untuk menghitung vektor x dan y yang menggambarkan
lintasan dalam koordinat polar.

\textgreater{} x = cos(3*t); menghitung nilai x sebagai hasil dari fungsi kosinus
dari 3 kali nilai t. Ini akan menghasilkan osilasi yang lebih cepat
pada sumbu x.\\
\textgreater{} y = sin(4*t); menghitung nilai y sebagai hasil dari fungsi sinus
dari 4 kali nilai t. Ini akan menghasilkan osilasi yang lebih cepat
pada sumbu y.

- plot2d(x, y, \textless{}grid, \textless{}frame, \textgreater{}filled);\\
Ini adalah perintah untuk membuat plot dari vektor x dan y. Berikut
adalah rincian perintah ini:

x adalah vektor yang digunakan sebagai data untuk sumbu x.\\
y adalah vektor yang digunakan sebagai data untuk sumbu y.\\
\textless{}grid mengaktifkan garis-garis koordinat (grid) di latar belakang
plot, membantu dalam visualisasi.\\
\textless{}frame mengaktifkan bingkai (frame) di sekitar plot.\\
\textgreater{}filled mengisi area di bawah kurva dengan warna, membuat plot menjadi
lebih berwarna.

\end{eulercomment}
\eulersubheading{Contoh}
\begin{eulercomment}
Sebuah vektor interval diplot terhadap nilai x sebagai daerah terisi\\
antara nilai interval bawah dan atas.

Ini dapat berguna untuk memplot kesalahan perhitungan. Tapi itu bisa\\
juga digunakan untuk memplot kesalahan statistik.
\end{eulercomment}
\begin{eulerprompt}
>t=0:0.1:1; ...
>plot2d(t,interval(t-random(size(t)),t+random(size(t))),style="|");  ...
>plot2d(t,t,add=true):
\end{eulerprompt}
\begin{eulercomment}
Penjelasan:

- t = 0:0.1:1;\\
Ini adalah perintah untuk membuat vektor t yang berisi nilai-nilai
dari 0 hingga 1 dengan interval 0.1. Hasilnya adalah vektor [0, 0.1,
0.2, 0.3, ..., 0.9, 1].

- plot2d(t, interval(t - random(size(t)), t + random(size(t))),
style="\textbar{}");\\
Ini adalah perintah untuk membuat plot pertama. Rincian perintah ini
adalah sebagai berikut:

\textgreater{} interval(t - random(size(t)), t + random(size(t))) adalah interval
yang digunakan untuk menggambar "garis" pada plot. Setiap titik pada
sumbu x (t) akan dihubungkan oleh dua garis vertikal yang dibuat
secara acak di sekitar titik tersebut menggunakan random(size(t)).
Hasilnya adalah plot dengan garis-garis vertikal yang mewakili
interval acak di sekitar setiap titik pada sumbu x.\\
\textgreater{} style="\textbar{}" mengatur gaya plot menjadi garis vertikal ("\textbar{}").

- plot2d(t, t, add=true);\\
Ini adalah perintah untuk membuat plot kedua dan menambahkannya ke
dalam plot yang sudah ada dari perintah sebelumnya. Rincian perintah
ini adalah sebagai berikut:

\textgreater{} t adalah sumbu x dan y plot ini, sehingga plot ini akan menjadi plot
garis diagonal dengan kemiringan 45 derajat.\\
\textgreater{} add=true digunakan untuk menambahkan plot ini ke dalam plot
sebelumnya, sehingga kedua plot akan ditampilkan dalam satu plot yang
sama.

\end{eulercomment}
\eulersubheading{Contoh}
\begin{eulercomment}
Jika x adalah vektor yang diurutkan, dan y adalah vektor interval,
maka plot2d akan memplot rentang interval yang terisi dalam bidang.
Gaya isian sama dengan gaya poligon.
\end{eulercomment}
\begin{eulerprompt}
>t=-1:0.01:1; x=~t-0.01,t+0.01~; y=x^3-x;
>plot2d(t,y):
\end{eulerprompt}
\begin{eulercomment}
Penjelasan:

- t = -1:0.01:1;\\
Ini adalah perintah untuk membuat vektor t yang berisi nilai-nilai
dari -1 hingga 1 dengan interval 0.01. Hasilnya adalah vektor t yang
berisi nilai-nilai seperti [-1, -0.99, -0.98, ..., 0.99, 1]. Vektor t
ini akan digunakan sebagai sumbu x pada plot.

- x = ~t - 0.01, t + 0.01~;\\
Ini adalah perintah yang menghitung vektor x. Tanda ~ digunakan di
sini untuk mendefinisikan dua interval, yaitu [~t - 0.01, t + 0.01~].
Ini menghasilkan vektor x yang memiliki dua interval, satu yang kurang
dari t - 0.01 dan satu yang lebih dari t + 0.01.

- y = x\textasciicircum{}3 - x;\\
Ini adalah perintah yang menghitung vektor y sebagai fungsi dari x.
Fungsi ini menghitung nilai y dengan memasukkan setiap nilai x ke
dalam rumus x\textasciicircum{}3 - x.

- plot2d(t, y);\\
Ini adalah perintah untuk membuat plot dari fungsi y sebagai fungsi
dari t. Rincian perintah ini adalah sebagai berikut:\\
\textgreater{} t adalah sumbu x pada plot, yang berisi vektor t yang telah
didefinisikan sebelumnya.\\
\textgreater{} y adalah sumbu y pada plot, yang berisi vektor y yang dihitung dari
rumus x\textasciicircum{}3 - x.

\end{eulercomment}
\eulersubheading{Contoh}
\begin{eulerprompt}
>expr := "2*x^2+x*y+3*y^4+y"; // define an expression f(x,y)
>plot2d(expr,level=[0;1],style="-",color=blue): // 0 <= f(x,y) <= 1
\end{eulerprompt}
\begin{eulercomment}
Penjelasan:

- expr := "2*x\textasciicircum{}2+x*y+3*y\textasciicircum{}4+y";\\
Ini adalah perintah untuk mendefinisikan ekspresi matematika yang
disimpan dalam variabel expr. Ekspresi ini merupakan suatu fungsi f(x,
y) yang tergantung pada dua variabel, yaitu x dan y. Ekspresi ini
memiliki bentuk matematika yang terdiri dari berbagai suku, seperti
kuadrat dari x, perkalian x*y, kuadrat dari y, dan lainnya.

- plot2d(expr, level=[0;1], style="-", color=blue);\\
Ini adalah perintah untuk membuat plot dari fungsi f(x, y) yang telah
didefinisikan sebelumnya. Berikut adalah rincian perintah ini:

\textgreater{} expr adalah ekspresi yang akan digunakan sebagai fungsi yang akan
diplotkan. Dalam hal ini, ekspresi 2*x\textasciicircum{}2+x*y+3*y\textasciicircum{}4+y adalah fungsi
f(x, y) yang telah didefinisikan sebelumnya.

\textgreater{} level=[0;1] mengatur tingkat kontur (contour levels) yang akan
digunakan dalam plot. Dalam hal ini, tingkat kontur adalah 0 hingga 1,
yang berarti plot akan menunjukkan wilayah di mana f(x, y) memiliki
nilai antara 0 hingga 1.

\textgreater{} style="-" mengatur gaya plot menjadi garis berjenis -, yang akan
menghasilkan plot kontur.

\textgreater{} color=blue mengatur warna garis plot menjadi biru.

\end{eulercomment}
\eulersubheading{Contoh}
\begin{eulerprompt}
>plot2d("(x^2+y^2)^2-x^2+y^2",r=1.2,level=[-1;0],style="/"):
\end{eulerprompt}
\begin{eulercomment}
Penjelasan:

plot2d("(x\textasciicircum{}2+y\textasciicircum{}2)\textasciicircum{}2-x\textasciicircum{}2+y\textasciicircum{}2", r=1.2, level=[-1;0], style="/");\\
Ini adalah perintah untuk membuat plot dari fungsi matematika yang
didefinisikan dalam bentuk string: "(x\textasciicircum{}2+y\textasciicircum{}2)\textasciicircum{}2-x\textasciicircum{}2+y\textasciicircum{}2". Fungsi ini
tergantung pada dua variabel, yaitu x dan y.

(x\textasciicircum{}2+y\textasciicircum{}2)\textasciicircum{}2-x\textasciicircum{}2+y\textasciicircum{}2 adalah rumus dari fungsi matematika yang akan
diplotkan.

r=1.2 mengatur rentang (range) plot untuk kedua sumbu x dan y. Dalam
hal ini, rentangnya adalah [-1.2, 1.2], yang berarti plot akan berada
dalam wilayah ini.

level=[-1;0] mengatur tingkat kontur (contour levels) yang akan
digunakan dalam plot. Dalam hal ini, ada dua tingkat kontur: -1 dan 0.
Ini akan menentukan wilayah kontur dalam plot.

style="/" mengatur gaya plot menjadi garis miring ("/"). Ini akan
menghasilkan plot dengan garis-garis miring yang menggambarkan kontur
fungsi.

\end{eulercomment}
\eulersubheading{Contoh}
\begin{eulerprompt}
>plot2d("cos(x)","sin(x)^3",xmin=0,xmax=2pi,>filled,style="/"):
\end{eulerprompt}
\begin{eulercomment}
Penjelasan:

plot2d("sin(x)\textasciicircum{}3", "cos(x)", xmin=0, xmax=2*pi, \textgreater{}filled, style="/");\\
Ini adalah perintah untuk membuat plot dari dua fungsi matematika,
yaitu sin(x)\textasciicircum{}3 dan cos(x), dalam satu plot yang sama. Berikut adalah
rincian perintah ini:

"sin(x)\textasciicircum{}3" adalah ekspresi pertama yang akan diplotkan. Ini adalah
fungsi trigonometri sin(x) yang dipangkatkan tiga. Fungsi ini
tergantung pada variabel x.

"cos(x)" adalah ekspresi kedua yang akan diplotkan. Ini adalah fungsi
trigonometri cos(x). Fungsi ini juga tergantung pada variabel x.

xmin=0 dan xmax=2*pi mengatur rentang (range) plot untuk sumbu x dari
0 hingga 2p. Ini adalah rentang yang akan ditampilkan dalam plot.

\textgreater{}filled mengisi area di bawah kurva fungsi dengan warna, sehingga area
di bawah kurva fungsi akan diisi dengan warna.

style="/" mengatur gaya plot menjadi garis miring ("\textbackslash{}"). Ini akan
menghasilkan plot dengan garis-garis miring.
\end{eulercomment}
\eulerheading{Sub Bab 15 }
\begin{eulercomment}
Menggambar Segi Banyak 


Data plot merupakan poligon atau segi banyak. Kita juga dapat membuat
kurva atau mengisi kurva.\\
Fungsi perintah yang digunakan untuk menggambar segi banyak atau
poligon.

Membentuk poligon dengan fungsi:\\
x=linspace(0,2pi,n); plot2d(cos(x),sin(x),r=1,\textgreater{}filled,style="..."):\\
atau\\
x=linspace(0,2pi,n);
plot2d(sin(x),cos(x),r=1,\textgreater{}filled,style="...",fillcolor=red):

Keterangan\\
- filled=true, mengisi plot.\\
- style="...": Pilih dari "#", "/", "\textbackslash{}", "\textbackslash{}/" dan gaya gaya lainnya.\\
- fillcolor: untuk memeberikan warna.

Warna isian ditentukan oleh argumen "fillcolor", dan pada \textless{}outline
opsional mencegah menggambar batas untuk semua gaya kecuali yang
default.

Poligon dalam EMT dapat digambar dengan fungsi maksimal. Dengan fungsi
maksimal ini, poligon yang dihasilkan dapat berupa poligon tak
beraturan.

A=[2,1;1,2;-1,0;0,-1];\\
function f(x,y) := max([x,y].A');\\
plot2d("f",r=4,level=[0;3],color=green,n=111):

Keterangan:\\
-A adalah titik koordinat dari poligon yang akan dibuat.\\
-"r" untuk menentukan ukuran bidang koordinat.

Berikut adalah himpunan nilai maksimal dari empat kondisi linear yang
kurang dari atau sama dengan 3. Ini merupakan A[k].v\textless{}=3 untuk semua
baris A. Untuk mendapatkan sudut yang bagus, kita menggunakan n yang
relatif besar.


1. Menggambar Segitiga
\end{eulercomment}
\begin{eulerprompt}
>x=linspace(0,2pi,3); ...
>plot2d(sin(x),cos(x),r=1):
\end{eulerprompt}
\begin{eulercomment}
Segitiga diatas digambar dari kurva tertutup dengan 3 titik.

Kita dapat membuat segitiga dengan gaya yang berbeda-beda. Seperti
pada contoh berikut ini.
\end{eulercomment}
\begin{eulerprompt}
>x=linspace(0,2pi,3); ...
>plot2d(sin(x),cos(x),>filled,style="/",fillcolor=red,r=1):
>x=linspace(0,2pi,3); ...
>plot2d(sin(x),cos(x),>filled,style="#",fillcolor=blue,r=2):
\end{eulerprompt}
\begin{eulercomment}
Dua gambar segitiga diatas memiliki gaya yang berbeda, dengan
menggunakan fungsi perintah "style=". Gambar segitiga juga dapat
dibuat dengan posisi yang berbeda, tergantung pada fungsi yang akan
diplot.


2. Menggambar Segiempat
\end{eulercomment}
\begin{eulerprompt}
>x=linspace(0,2pi,4); ...
>plot2d(cos(x),sin(x),r=1.5):
>x=linspace(0,2pi,4); 
>plot2d(cos(x),sin(x),r=2,>filled,outline=1):
\end{eulerprompt}
\begin{eulercomment}
Gambar diatas merupakan salah satu contoh segiempat yang dapat
digambar di EMT. Fungsi perintah yang digunakan masih sama seperti
fungsi perintah untuk menggambar segitiga. 

Selain fungsi perintah diatas, untuk menggambar segi banyak, dapat
menggunakan fungsi maksimum.
\end{eulercomment}
\begin{eulerprompt}
>A=[2,1;1,2;-1,0;0,-1];
>function f(x,y) := max([x,y].A');
>plot2d("f",r=4,level=[0;3],color=yellow,n=111):
>A=[1,1;-1,1;-1,-1;1,-1];
>function f(x,y) := max([x,y].A');
>plot2d("f",r=1,level=[0;1],color=gray,n=90):
\end{eulerprompt}
\begin{eulercomment}
Dengan fungsi maksimal ini, kita dapat menggambar segiempat atau segi
banyak sebarang.


3. Menggambar Segilima
\end{eulercomment}
\begin{eulerprompt}
>t=linspace(0,2pi,5); plot2d(sin(t),cos(t),r=1.5):
>t=linspace(0,2pi,5); ...
>plot2d(sin(t),cos(t),r=1.5,>filled,style="\(\backslash\)",fillcolor=orange):
>A=[0,5;3,2;1,-4;-1,-4;-3,2];
>function f(x,y) := max([x,y].A');
>plot2d("f",r=1,level=[0;2],color=cyan,n=111):
\end{eulerprompt}
\begin{eulercomment}
4. Menggambar Segienam
\end{eulercomment}
\begin{eulerprompt}
>t=linspace(0,2pi,6); ...
>plot2d(cos(t),sin(t),r=1.2):
>t=linspace(0,2pi,6); ...
>plot2d(cos(t),sin(t),>filled,style="/",fillcolor=olive,r=1.2):
\end{eulerprompt}
\begin{eulercomment}
5. Menggambar dekagon
\end{eulercomment}
\begin{eulerprompt}
>t=linspace(0,2pi,10); ...
>plot2d(cos(t),sin(t),r=1.2):
>t=linspace(0,2pi,10); ...
\end{eulerprompt}
\end{eulernotebook}
\end{document}


\newpage
\chapter{KB Pekan 5: Menggunakan EMT untuk mengambar grafik 3 dimensi (3D)}
\documentclass[a4paper,10pt]{article}
\usepackage{eumat}

\begin{document}
\begin{eulernotebook}
\eulerheading{Menggambar Plot 3D dengan EMT}
\begin{eulercomment}
Rasyid Shalahuddin\\
22305144016\\
Matematika E 2022 

This is an introduction to 3D plots in Euler. We need a 3D plot to
visualize a function of two variables.

Euler draws such functions using a sorting algorithm to hide parts in
the background. In general, Euler uses a central projection. The
default is from the positive x-y quadrant towards the origin x=y=z=0,
but angle=0° looks from into the direction of the y-axis. The view
angle and height can be changed.

Euler can plot

- surfaces with shading and level lines or level ranges,\\
- clouds of points,\\
- parametric curves,\\
- implicit surfaces.

A 3D plot of a function uses plot3d. The easiest way is to plot an
expression in x and y. The parameter r set the range of the plot
around (0,0).
\end{eulercomment}
\begin{eulerprompt}
>aspect(1.5); plot3d("x^2+sin(y)",-5,5,0,6*pi):
>plot3d("x^2+x*sin(y)",-5,5,0,6*pi):
\end{eulerprompt}
\begin{eulercomment}
Silakan lakukan modifikasi agar gambar "talang bergelombang" tersebut tidak lurus melainkan melengkung/melingkar, baik
melingkar secara mendatar maupun melingkar turun/naik (seperti papan peluncur pada kolam renang. Temukan rumusnya.
\end{eulercomment}
\eulerheading{Functions of two Variables}
\begin{eulercomment}
For the graph of a function, use

- a simple expression in x and y,\\
- the name of a function of two variablesl\\
- or data matrices.

The default is a filled wire grid with different colors on both sides. Note that the default number of grid intervals is
10, but the plot uses the default number of 40x40 rectangles to construct the surface. This can be changed.

- n=40, n=[40,40]: number of grid lines in each direction\\
- grid=10, grid=[10,10]: number of grid lines in each direction.

We use the default n=40 and grid=10.
\end{eulercomment}
\begin{eulerprompt}
>plot3d("x^2+y^2"):
\end{eulerprompt}
\begin{eulercomment}
User interaction is possible with the \textgreater{}user parameter. The user can press the following keys.

- left,right,up,down: turn the viewing angle\\
- +,-: zoom in or out\\
- a: produce an anaglyph (see below)\\
- l: toggle turning the light source (see below)\\
- space: reset to default\\
- return: end interaction
\end{eulercomment}
\begin{eulerprompt}
>plot3d("exp(-x^2+y^2)",>user, ...
>  title="Turn with the vector keys (press return to finish)"):
\end{eulerprompt}
\begin{eulercomment}
The plot range for functions can be specified with

- a,b: the x-range\\
- c,d: the y-range\\
- r: a symmetric square around (0,0).\\
- n: number of subintervals for the plot.

There are some parameters to scale the function or change the look of the graph.

fscale: scales to function values (default is \textless{}fscale).\\
scale: number or 1x2 vector to scale into x- and y-direction.\\
frame: type of frame (default 1).
\end{eulercomment}
\begin{eulerprompt}
>plot3d("exp(-(x^2+y^2)/5)",r=10,n=80,fscale=4,scale=1.2,frame=3,>user):
\end{eulerprompt}
\begin{eulercomment}
The view can be changed in many different ways.

- distance: the viewing distance to the plot.\\
- zoom: the zoom value.\\
- angle: the angle to the negative y-axis in radians.\\
- height: the height of the view in radians.

The default values can be inspected or changed with the function view(). It returns the parameters in the order above.
\end{eulercomment}
\begin{eulerprompt}
>view
\end{eulerprompt}
\begin{euleroutput}
  [5,  2.6,  2,  0.4]
\end{euleroutput}
\begin{eulercomment}
A closer distance needs less zoom. The effect is more like a wide
angle lens.

In the following example, angle=0 and height=0 look from the negative
y-axis. The axis labels for y are hidden in this case.
\end{eulercomment}
\begin{eulerprompt}
>plot3d("x^2+y",distance=3,zoom=1,angle=pi/2,height=0):
\end{eulerprompt}
\begin{eulercomment}
The plot looks always to the center of the plot cube. You can move the center with the center parameter.
\end{eulercomment}
\begin{eulerprompt}
>plot3d("x^4+y^2",a=0,b=1,c=-1,d=1,angle=-20°,height=20°, ...
>  center=[0.4,0,0],zoom=5):
\end{eulerprompt}
\begin{eulercomment}
The plot is scaled to fit into a unit cube for viewing. So there is no need to change the distance or zoom depending on
the size of the plot. The labels refer to the actual size, however.

If you turn this off with scale=false, you need to take care, that the plot still fits into the plotting window, by
changing the viewing distance or zoom, and moving the center.
\end{eulercomment}
\begin{eulerprompt}
>plot3d("5*exp(-x^2-y^2)",r=2,<fscale,<scale,distance=13,height=50°, ...
>  center=[0,0,-2],frame=3):
\end{eulerprompt}
\begin{eulercomment}
A polar plot is also available. The parameter polar=true draws a polar plot. The function must still be a function of x and y. The
parameter "fscale" scales the function with an own scale. Otherwise the function is scaled to fit into a cube.
\end{eulercomment}
\begin{eulerprompt}
>plot3d("1/(x^2+y^2+1)",r=5,>polar, ...
>fscale=2,>hue,n=100,zoom=4,>contour,color=blue):
>function f(r) := exp(-r/2)*cos(r); ...
>plot3d("f(x^2+y^2)",>polar,scale=[1,1,0.4],r=pi,frame=3,zoom=4):
\end{eulerprompt}
\begin{eulercomment}
The parameter rotate rotates a function in x around the x-axis.

- rotate=1: Uses the x-axis\\
- rotate=2: Uses the z-axis
\end{eulercomment}
\begin{eulerprompt}
>plot3d("x^2+1",a=-1,b=1,rotate=true,grid=5):
>plot3d("x^2+1",a=-1,b=1,rotate=2,grid=5):
>plot3d("sqrt(25-x^2)",a=0,b=5,rotate=1):
>plot3d("x*sin(x)",a=0,b=6pi,rotate=2):
\end{eulerprompt}
\begin{eulercomment}
Here is a plot with three functions.
\end{eulercomment}
\begin{eulerprompt}
>plot3d("x","x^2+y^2","y",r=2,zoom=3.5,frame=3):
\end{eulerprompt}
\eulerheading{Contour Plots}
\begin{eulercomment}
For the plot, Euler adds grid lines. Instead it is possible to use level lines and a
one-color hue or a spectral colored hue. Euler can draw the heights of functions on a
plot with shading. In all 3D plots Euler can produce red/cyan anaglyphs.

- \textgreater{}hue: Turns on light shading instead of wires.\\
- \textgreater{}contour: Plots automatic contour lines on a plot.\\
- level=... (or levels): A vector of values for the contour lines.

The default is level="auto", which computes some level lines automatically. As you
see in the plot, the levels are in fact ranges of levels.

The default style can be changed. For the following contour plot, we use a finer grid
fo 100x100 points, scale the function and the plot, and use different angle of view.
\end{eulercomment}
\begin{eulerprompt}
>plot3d("exp(-x^2-y^2)",r=2,n=100,level="thin", ...
> >contour,>spectral,fscale=1,scale=1.1,angle=45°,height=20°):
>plot3d("exp(x*y)",angle=100°,>contour,color=green):
\end{eulerprompt}
\begin{eulercomment}
The default shading uses a gray color. But a spectral range of colors is also available.

- \textgreater{}spectral: Used the default spectral scheme\\
- color=...: Uses special colors or spectral schemes

For the following plot, we use the default spectral scheme and increase the number of points to get a very smooth look.
\end{eulercomment}
\begin{eulerprompt}
>plot3d("x^2+y^2",>spectral,>contour,n=100):
\end{eulerprompt}
\begin{eulercomment}
Instead of automatic level lines, we can also set values of the level lines. This will produce thin level lines instead
of ranges of levels.
\end{eulercomment}
\begin{eulerprompt}
>plot3d("x^2-y^2",0,5,0,5,level=-1:0.1:1,color=redgreen):
\end{eulerprompt}
\begin{eulercomment}
In the following plot, we use two very broad level bands from -0.1 to 1, and from 0.9 to 1. This is entered as a matrix
with level bounds as columns.

Moreover, we overlay a grid with 10 intervals in each direction.
\end{eulercomment}
\begin{eulerprompt}
>plot3d("x^2+y^3",level=[-0.1,0.9;0,1], ...
>  >spectral,angle=30°,grid=10,contourcolor=gray):
\end{eulerprompt}
\begin{eulercomment}
In the following example, we plot the set, where

\end{eulercomment}
\begin{eulerformula}
\[
f(x,y) = x^y-y^x = 0
\]
\end{eulerformula}
\begin{eulercomment}
We use a single thin line for the level line.
\end{eulercomment}
\begin{eulerprompt}
>plot3d("x^y-y^x",level=0,a=0,b=6,c=0,d=6,contourcolor=red,n=100):
\end{eulerprompt}
\begin{eulercomment}
It is possible to show a contour plane below the plot. A color and
distance to the plot can be specified.
\end{eulercomment}
\begin{eulerprompt}
>plot3d("x^2+y^4",>cp,cpcolor=green,cpdelta=0.2):
\end{eulerprompt}
\begin{eulercomment}
Here are a few more styles. We always turn off the frame, and use
various color schemes for the plot and the grid.
\end{eulercomment}
\begin{eulerprompt}
>figure(2,2); ...
>expr="y^3-x^2"; ...
>figure(1);  ...
>  plot3d(expr,<frame,>cp,cpcolor=spectral); ...
>figure(2);  ...
>  plot3d(expr,<frame,>spectral,grid=10,cp=2); ...
>figure(3);  ...
>  plot3d(expr,<frame,>contour,color=gray,nc=5,cp=3,cpcolor=greenred); ...
>figure(4);  ...
>  plot3d(expr,<frame,>hue,grid=10,>transparent,>cp,cpcolor=gray); ...
>figure(0):
\end{eulerprompt}
\begin{eulercomment}
There are some other spectral schemes, numbered from 1 to 9. But you can also use the color=value, where value

- spectral: for a range from blue to red\\
- white: for a fainter range\\
- yellowblue,purplegreen,blueyellow,greenred\\
- blueyellow, greenpurple,yellowblue,redgreen
\end{eulercomment}
\begin{eulerprompt}
>figure(3,3); ...
>for i=1:9;  ...
>  figure(i); plot3d("x^2+y^2",spectral=i,>contour,>cp,<frame,zoom=4);  ...
>end; ...
>figure(0):
\end{eulerprompt}
\begin{eulercomment}
The light source can be changed with l and the cursor keys during the user interaction. It can also be set with
parameters.

- light: a direction for the light\\
- amb: ambient light between 0 and 1

Note that the program does not make a difference between the sides of the plot. There are no shadows. For this you would
need Povray.
\end{eulercomment}
\begin{eulerprompt}
>plot3d("-x^2-y^2", ...
>  hue=true,light=[0,1,1],amb=0,user=true, ...
>  title="Press l and cursor keys (return to exit)"):
\end{eulerprompt}
\begin{eulercomment}
The color parameter changes the color of the surface. The color of the level lines can also be changed.
\end{eulercomment}
\begin{eulerprompt}
>plot3d("-x^2-y^2",color=rgb(0.2,0.2,0),hue=true,frame=false, ...
>  zoom=3,contourcolor=red,level=-2:0.1:1,dl=0.01):
\end{eulerprompt}
\begin{eulercomment}
The color 0 gives a special rainbow effect.
\end{eulercomment}
\begin{eulerprompt}
>plot3d("x^2/(x^2+y^2+1)",color=0,hue=true,grid=10):
\end{eulerprompt}
\begin{eulercomment}
The surface can also be transparent.
\end{eulercomment}
\begin{eulerprompt}
>plot3d("x^2+y^2",>transparent,grid=10,wirecolor=red):
\end{eulerprompt}
\eulerheading{Implicit Plots}
\begin{eulercomment}
There are also implicit plots in three dimensions. Euler generates cuts through the objects. The features of plot3d
include implicit plots. These plots show the zero set of a function in three variables.\\
The solutions of

\end{eulercomment}
\begin{eulerformula}
\[
f(x,y,z) = 0
\]
\end{eulerformula}
\begin{eulercomment}
can be visualized in cuts parallel to the x-y-, the x-z- and the y-z-plane.

- implicit=1: cut parallel to the y-z-plane\\
- implicit=2: cut parallel to the x-z-plane\\
- implicit=4: cut parallel to the x-y-plane

Add these values, if you like. In the example we plot

\end{eulercomment}
\begin{eulerformula}
\[
M = \{ (x,y,z) : x^2+y^3+zy=1 \}
\]
\end{eulerformula}
\begin{eulerprompt}
>plot3d("x^2+y^3+z*y-1",r=5,implicit=3):
>c=1; d=1;
>plot3d("((x^2+y^2-c^2)^2+(z^2-1)^2)*((y^2+z^2-c^2)^2+(x^2-1)^2)*((z^2+x^2-c^2)^2+(y^2-1)^2)-d",r=2,<frame,>implicit,>user): 
\end{eulerprompt}
\begin{euleroutput}
  Cannot combine a 41x41 and a 1x81 matrix for +!
  Error in expression: ((x^2+y^2-c^2)^2+(z^2-1)^2)*((y^2+z^2-c^2)^2+(x^2-1)^2)*((z^2+x^2-c^2)^2+(y^2-1)^2)-d
  Try "trace errors" to inspect local variables after errors.
  pov3d:
      z=f(x,y;args());
\end{euleroutput}
\begin{eulerprompt}
>plot3d("x^2+y^2+4*x*z+z^3",>implicit,r=2,zoom=2.5):
\end{eulerprompt}
\eulerheading{Plotting 3D Data}
\begin{eulercomment}
Just as plot2d, plot3d accepts data. For 3D objects, you need to provide a matrix of x-, y- and z-values, or three
functions or expressions fx(x,y), fy(x,y), fz(x,y).

\end{eulercomment}
\begin{eulerformula}
\[
\gamma(t,s) = (x(t,s),y(t,s),z(t,s))
\]
\end{eulerformula}
\begin{eulercomment}
Since x,y,z are matrices, we assume that (t,s) run through a square grid. As a result, you can plot images of rectangles
in space.

You can use the Euler matrix language to produce the coordinates effectively.

In the following example, we use a vector of t values and a column vector of s values to parameterize the surface of the
ball. In the drawing we can mark regions, in our case the polar region.
\end{eulercomment}
\begin{eulerprompt}
>t=linspace(0,2pi,180); s=linspace(-pi/2,pi/2,90)'; ...
>x=cos(s)*cos(t); y=cos(s)*sin(t); z=sin(s); ...
>plot3d(x,y,z,>hue, ...
>color=blue,<frame,grid=[10,20], ...
>values=s,contourcolor=red,level=[90°-24°;90°-22°], ...
>scale=1.4,height=50°):
\end{eulerprompt}
\begin{eulercomment}
Here is an example, which is the graph of a function.
\end{eulercomment}
\begin{eulerprompt}
>t=-1:0.1:1; s=(-1:0.1:1)'; plot3d(t,s,t*s,grid=10):
\end{eulerprompt}
\begin{eulercomment}
However, we can make all sorts of surfaces. Here is the same surface
as a function

\end{eulercomment}
\begin{eulerformula}
\[
x = y \, z
\]
\end{eulerformula}
\begin{eulerprompt}
>plot3d(t*s,t,s,angle=180°,grid=10):
\end{eulerprompt}
\begin{eulercomment}
With more effort, we can produce many surfaces.

In the following example we make a shaded view of a distorted ball. The usual coordinates for the ball are

\end{eulercomment}
\begin{eulerformula}
\[
\gamma(t,s) = (\cos(t)\cos(s),\sin(t)\sin(s),\cos(s))
\]
\end{eulerformula}
\begin{eulercomment}
with

\end{eulercomment}
\begin{eulerformula}
\[
0 \le t \le 2\pi, \quad \frac{-\pi}{2} \le s \le \frac{\pi}{2}.
\]
\end{eulerformula}
\begin{eulercomment}
We distored this with a factor

\end{eulercomment}
\begin{eulerformula}
\[
d(t,s) = \frac{\cos(4t)+\cos(8s)}{4}.
\]
\end{eulerformula}
\begin{eulerprompt}
>t=linspace(0,2pi,320); s=linspace(-pi/2,pi/2,160)'; ...
>d=1+0.2*(cos(4*t)+cos(8*s)); ...
>plot3d(cos(t)*cos(s)*d,sin(t)*cos(s)*d,sin(s)*d,hue=1, ...
>  light=[1,0,1],frame=0,zoom=5):
\end{eulerprompt}
\begin{eulercomment}
Of course, a point cloud is also possible. To plot point data in the space, we need three vectors for the coordinates of
the points.

The styles are just as in plot2d with points=true;
\end{eulercomment}
\begin{eulerprompt}
>n=500;  ...
>  plot3d(normal(1,n),normal(1,n),normal(1,n),points=true,style="."):
\end{eulerprompt}
\begin{eulercomment}
It is also possible to plot a curve in 3D. In this case, it is easier to precompute
the points of the curve. For curves in the plane we use a sequence of coordinates and
the parameter wire=true.
\end{eulercomment}
\begin{eulerprompt}
>t=linspace(0,8pi,500); ...
>plot3d(sin(t),cos(t),t/10,>wire,zoom=3):
>t=linspace(0,4pi,1000); plot3d(cos(t),sin(t),t/2pi,>wire, ...
>linewidth=3,wirecolor=blue):
>X=cumsum(normal(3,100)); ...
> plot3d(X[1],X[2],X[3],>anaglyph,>wire):
\end{eulerprompt}
\begin{eulercomment}
EMT can also plot in anaglyph mode. To view such a plot, you need red/cyan glasses.
\end{eulercomment}
\begin{eulerprompt}
> plot3d("x^2+y^3",>anaglyph,>contour,angle=30°):
\end{eulerprompt}
\begin{eulercomment}
Often, a spectral color scheme is used for plots. This emphasizes the heights of the
function.
\end{eulercomment}
\begin{eulerprompt}
>plot3d("x^2*y^3-y",>spectral,>contour,zoom=3.2):
\end{eulerprompt}
\begin{eulercomment}
Euler can plot parameterized surfaces too, when the parameters are the x-, y-, and
z-values of an image of a rectangular grid in the space.

For the following demo, we setup u- and v- parameters, and generate space coordinates
from these.
\end{eulercomment}
\begin{eulerprompt}
>u=linspace(-1,1,10); v=linspace(0,2*pi,50)'; ...
>X=(3+u*cos(v/2))*cos(v); Y=(3+u*cos(v/2))*sin(v); Z=u*sin(v/2); ...
>plot3d(X,Y,Z,>anaglyph,<frame,>wire,scale=2.3):
\end{eulerprompt}
\begin{eulercomment}
Here is a more complicated example, which is majestic with red/cyan glasses.
\end{eulercomment}
\begin{eulerprompt}
>u:=linspace(-pi,pi,160); v:=linspace(-pi,pi,400)';  ...
>x:=(4*(1+.25*sin(3*v))+cos(u))*cos(2*v); ...
>y:=(4*(1+.25*sin(3*v))+cos(u))*sin(2*v); ...
> z=sin(u)+2*cos(3*v); ...
>plot3d(x,y,z,frame=0,scale=1.5,hue=1,light=[1,0,-1],zoom=2.8,>anaglyph):
\end{eulerprompt}
\eulerheading{Statistical Plots}
\begin{eulercomment}
Bar plots are possible too. For this, we have to provide

- x: row vector with n+1 elements\\
- y: column vector with n+1 elements\\
- z: nxn matrix of values.

z can be larger, but only nxn values will be used.

In the example, we first compute the values. Then we adjust x and y, so that the vectors center at the values used.
\end{eulercomment}
\begin{eulerprompt}
>x=-1:0.1:1; y=x'; z=x^2+y^2; ...
>xa=(x|1.1)-0.05; ya=(y_1.1)-0.05; ...
>plot3d(xa,ya,z,bar=true):
\end{eulerprompt}
\begin{eulercomment}
It is possible to split the plot of a surface in two or more parts.
\end{eulercomment}
\begin{eulerprompt}
>x=-1:0.1:1; y=x'; z=x+y; d=zeros(size(x)); ...
>plot3d(x,y,z,disconnect=2:2:20):
\end{eulerprompt}
\begin{eulercomment}
If load or generate a data matrix M from a file and need to plot it in
3D you can either scale the matrix to [-1,1] with scale(M), or scale
the matrix with \textgreater{}zscale. This can be combined with individual scaling
factors which are applied additionally.
\end{eulercomment}
\begin{eulerprompt}
>i=1:20; j=i'; ...
>plot3d(i*j^2+100*normal(20,20),>zscale,scale=[1,1,1.5],angle=-40°,zoom=1.8):
>Z=intrandom(5,100,6); v=zeros(5,6); ...
>loop 1 to 5; v[#]=getmultiplicities(1:6,Z[#]); end; ...
>columnsplot3d(v',scols=1:5,ccols=[1:5]):
\end{eulerprompt}
\eulerheading{Permukaan Benda Putar}
\begin{eulerprompt}
>plot2d("(x^2+y^2-1)^3-x^2*y^3",r=1.3, ...
>style="#",color=red,<outline, ...
>level=[-2;0],n=100):
>ekspresi &= (x^2+y^2-1)^3-x^2*y^3; $ekspresi
\end{eulerprompt}
\begin{eulercomment}
We wish to turn the heart curve around the y-axis. Here is the expression, which
defines the heart:

\end{eulercomment}
\begin{eulerformula}
\[
f(x,y)=(x^2+y^2-1)^3-x^2.y^3.
\]
\end{eulerformula}
\begin{eulercomment}
Next we set

\end{eulercomment}
\begin{eulerformula}
\[
x=r.cos(a),\quad y=r.sin(a).
\]
\end{eulerformula}
\begin{eulerprompt}
>function fr(r,a) &= ekspresi with [x=r*cos(a),y=r*sin(a)] | trigreduce; $fr(r,a)
\end{eulerprompt}
\begin{eulercomment}
This allows to define a numerical function, which solves for r, if a is given. With
that function we can plot the turned heart as a parametric surface.
\end{eulercomment}
\begin{eulerprompt}
>function map f(a) := bisect("fr",0,2;a); ...
>t=linspace(-pi/2,pi/2,100); r=f(t);  ...
>s=linspace(pi,2pi,100)'; ...
>plot3d(r*cos(t)*sin(s),r*cos(t)*cos(s),r*sin(t), ...
>>hue,<frame,color=red,zoom=4,amb=0,max=0.7,grid=12,height=50°):
\end{eulerprompt}
\begin{eulercomment}
The following is a 3D plot of the figure above rotated around the z-axis. We define
the function, which describes the object.
\end{eulercomment}
\begin{eulerprompt}
>function f(x,y,z) ...
\end{eulerprompt}
\begin{eulerudf}
  r=x^2+y^2;
  return (r+z^2-1)^3-r*z^3;
   endfunction
\end{eulerudf}
\begin{eulerprompt}
>plot3d("f(x,y,z)", ...
>xmin=0,xmax=1.2,ymin=-1.2,ymax=1.2,zmin=-1.2,zmax=1.4, ...
>implicit=1,angle=-30°,zoom=2.5,n=[10,100,60],>anaglyph):
\end{eulerprompt}
\eulerheading{Special 3D Plots}
\begin{eulercomment}
The plot3d function is nice to have, but it does not satisfy all needs. Besides more basic routines, it is possible to get a
framed plot of any object you like.

Though Euler is not a 3D program, it can combine some basic objects. We try to visualize a paraboloid and its tangent.
\end{eulercomment}
\begin{eulerprompt}
>function myplot ...
\end{eulerprompt}
\begin{eulerudf}
    y=-1:0.01:1; x=(-1:0.01:1)';
    plot3d(x,y,0.2*(x-0.1)/2,<scale,<frame,>hue, ..
      hues=0.5,>contour,color=orange);
    h=holding(1);
    plot3d(x,y,(x^2+y^2)/2,<scale,<frame,>contour,>hue);
    holding(h);
  endfunction
\end{eulerudf}
\begin{eulercomment}
Now framedplot() provides the frames, and sets the views.
\end{eulercomment}
\begin{eulerprompt}
>framedplot("myplot",[-1,1,-1,1,0,1],height=0,angle=-30°, ...
>  center=[0,0,-0.7],zoom=3):
\end{eulerprompt}
\begin{eulercomment}
In the same way, you can plot the contour plane manually. Note that plot3d() sets the window to fullwindow() by default, but
plotcontourplane() assumes that.
\end{eulercomment}
\begin{eulerprompt}
>x=-1:0.02:1.1; y=x'; z=x^2-y^4;
>function myplot (x,y,z) ...
\end{eulerprompt}
\begin{eulerudf}
    zoom(2);
    wi=fullwindow();
    plotcontourplane(x,y,z,level="auto",<scale);
    plot3d(x,y,z,>hue,<scale,>add,color=white,level="thin");
    window(wi);
    reset();
  endfunction
\end{eulerudf}
\begin{eulerprompt}
>myplot(x,y,z):
\end{eulerprompt}
\eulerheading{Animation}
\begin{eulercomment}
Euler can use frames to pre-compute the animation.

One function, which makes use of this technique is rotate. It can
change the angle of view and redraw a 3D plot. The function calls
addpage() for each new plot. Finally it animates the plots.

Please study the source of rotate to see more details.
\end{eulercomment}
\begin{eulerprompt}
>function testplot () := plot3d("x^2+y^3"); ...
>rotate("testplot"); testplot():
\end{eulerprompt}
\eulerheading{Menggambar Povray}
\begin{eulercomment}
With the help of the Euler file povray.e, Euler can generate Povray files. The results are very nice to look at.

You need to install Povray (32bit or 64bit) from http://www.povray.org/, and put the sub-directory "bin" of Povray into
the environment path, or set the variable "defaultpovray" with full path pointing to "pvengine.exe".

The Povray interface of Euler generates Povray files in the home directory of the user, and calls Povray to parse these
files. The default file name is current.pov, and the default directory is eulerhome(), usually c:\textbackslash{}Users\textbackslash{}Username\textbackslash{}Euler.
Povray generates a PNG file, which can be loaded by Euler into a notebook. To clean up these files, use povclear().

The pov3d function is in the same spirit as plot3d. It can generate the graph of a function f(x,y), or a surface with
coordinates X,Y,Z in matrices, including optional level lines. This function starts the raytracer automatically, and
loads the scene into the Euler notebook.

Besides pov3d(), there are many functions, which generate Povray objects. These functions return strings, containing the
Povray code for the objects. To use these functions, start the Povray file with povstart(). Then use writeln(...) to
write the objects to the scene file. Finally, end the file with povend(). By default, the raytracer will start, and the
PNG will be inserted into the Euler notebook.

The object functions have a parameter called "look", which needs a string with Povray code for the texture and the finish
of the object. The function povlook() can be used to produce this string. It has parameters for the color, the
transparency, Phong Shading etc.

Note that the Povray universe has another coordinate system. This interface translates all coordinates to the Povray
system. So you can keep thinking in the Euler coordinate system with z pointing vertically upwards,a nd x,y,z axes in
right hand sense.\\
You need to load the povray file.
\end{eulercomment}
\begin{eulerprompt}
>load povray;
\end{eulerprompt}
\begin{eulercomment}
Make sure, the Povray bin directory is in the path. If it is not edit the following variable so that it contains the path
to the povray executable.
\end{eulercomment}
\begin{eulerprompt}
>defaultpovray="C:\(\backslash\)Program Files\(\backslash\)POV-Ray\(\backslash\)v3.7\(\backslash\)bin\(\backslash\)pvengine.exe"
\end{eulerprompt}
\begin{euleroutput}
  C:\(\backslash\)Program Files\(\backslash\)POV-Ray\(\backslash\)v3.7\(\backslash\)bin\(\backslash\)pvengine.exe
\end{euleroutput}
\begin{eulercomment}
For a first impression, we plot a simple function. The following command generates a povray file in your user directory,
and runs Povray for ray tracing this file.

If you start the following command, the Povray GUI should open, run the file, and close automatically. Due to security
reasons, you will be asked, if you want to allow the exe file to run. You can press cancel to stop further questions. You
may have to press OK in the Povray window to acknowledge the start-up dialog of Povray.
\end{eulercomment}
\begin{eulerprompt}
>plot3d("x^2+y^2",zoom=2):
>pov3d("x^2+y^2",zoom=3);
\end{eulerprompt}
\begin{eulercomment}
We can make the function transparent and add another finish. We can
also add level lines to the function plot.
\end{eulercomment}
\begin{eulerprompt}
>pov3d("x^2+y^3",axiscolor=red,angle=-45°,>anaglyph, ...
>  look=povlook(cyan,0.2),level=-1:0.5:1,zoom=3.8);
\end{eulerprompt}
\begin{eulercomment}
Sometimes it is necessary to prevent the scaling of the function, and scale the function by hand.

We plot the set of points in the complex plane, where the product of the distances to 1 and -1 is equal to 1.
\end{eulercomment}
\begin{eulerprompt}
>pov3d("((x-1)^2+y^2)*((x+1)^2+y^2)/40",r=2, ...
>  angle=-120°,level=1/40,dlevel=0.005,light=[-1,1,1],height=10°,n=50, ...
>  <fscale,zoom=3.8);
\end{eulerprompt}
\eulerheading{Plotting with Coordinates}
\begin{eulercomment}
Instead of functions, we can plot with coordinates. As in plot3d, we need three matrices to define the object.

In the example we turn a function around the z-axis.
\end{eulercomment}
\begin{eulerprompt}
>function f(x) := x^3-x+1; ...
>x=-1:0.01:1; t=linspace(0,2pi,50)'; ...
>Z=x; X=cos(t)*f(x); Y=sin(t)*f(x); ...
>pov3d(X,Y,Z,angle=40°,look=povlook(red,0.1),height=50°,axis=0,zoom=4,light=[10,5,15]);
\end{eulerprompt}
\begin{eulercomment}
In the following example, we plot a damped wave. We generate the wave with the matrix language of Euler.

We also show, how an additional object can be added to a pov3d scene. For the generation of objects, see the following
examples. Note that plot3d scales the plot, so that it fits into the unit cube.
\end{eulercomment}
\begin{eulerprompt}
>r=linspace(0,1,80); phi=linspace(0,2pi,80)'; ...
>x=r*cos(phi); y=r*sin(phi); z=exp(-5*r)*cos(8*pi*r)/3;  ...
>pov3d(x,y,z,zoom=6,axis=0,height=30°,add=povsphere([0.5,0,0.25],0.15,povlook(red)), ...
>  w=500,h=300);
\end{eulerprompt}
\begin{eulercomment}
With the advanced shading method of Povray, very few points can
produce very smooth surfaces. Only at the boundaries and in shadows
the trick might become obvious.

For this, we need to add normal vectors in each matrix point.
\end{eulercomment}
\begin{eulerprompt}
>Z &= x^2*y^3
\end{eulerprompt}
\begin{euleroutput}
  
                                                             2  3
                                                            x  y
  
\end{euleroutput}
\begin{eulercomment}
The equation of the surface is [x,y,Z]. We compute the two derivatives
to x and y of this and take the cross product as the normal.
\end{eulercomment}
\begin{eulerprompt}
>dx &= diff([x,y,Z],x); dy &= diff([x,y,Z],y);
\end{eulerprompt}
\begin{eulercomment}
We define the normal as the cross product of these derivatives, and
define coordinate functions.
\end{eulercomment}
\begin{eulerprompt}
>N &= crossproduct(dx,dy); NX &= N[1]; NY &= N[2]; NZ &= N[3]; N,
\end{eulerprompt}
\begin{euleroutput}
  
                                                          3       2  2
                                                  [- 2 x y , - 3 x  y , 1]
  
\end{euleroutput}
\begin{eulercomment}
We use only 25 points.
\end{eulercomment}
\begin{eulerprompt}
>x=-1:0.5:1; y=x';
>pov3d(x,y,Z(x,y),angle=10°, ...
>  xv=NX(x,y),yv=NY(x,y),zv=NZ(x,y),<shadow);
\end{eulerprompt}
\begin{eulercomment}
The following is the Trefoil knot done by A. Busser in Povray. There
is an improved version of this in the examples.

See: Examples\textbackslash{}Trefoil Knot \textbar{} Trefoil Knot

For a good look with not too many points, we add normal vectors here.
We use Maxima to compute the normals for us. First, the three
functions for the coordinates as symbolic expressions.
\end{eulercomment}
\begin{eulerprompt}
>X &= ((4+sin(3*y))+cos(x))*cos(2*y); ...
>Y &= ((4+sin(3*y))+cos(x))*sin(2*y); ...
>Z &= sin(x)+2*cos(3*y);
\end{eulerprompt}
\begin{eulercomment}
Then the two derivative vectors to x and y.
\end{eulercomment}
\begin{eulerprompt}
>dx &= diff([X,Y,Z],x); dy &= diff([X,Y,Z],y);
\end{eulerprompt}
\begin{eulercomment}
Now the normal, which is the cross product of the two derivatives.
\end{eulercomment}
\begin{eulerprompt}
>dn &= crossproduct(dx,dy);
\end{eulerprompt}
\begin{eulercomment}
We now evaluate all this numerically.
\end{eulercomment}
\begin{eulerprompt}
>x:=linspace(-%pi,%pi,40); y:=linspace(-%pi,%pi,100)';
\end{eulerprompt}
\begin{eulercomment}
The normal vectors are evaluations of the symbolic expressions dn[i]
for i=1,2,3. The syntax for this is \&"expression"(parameters). This is
an alternative to the method in the previous example, where we defined
symbolic expressions NX, NY, NZ first.
\end{eulercomment}
\begin{eulerprompt}
>pov3d(X(x,y),Y(x,y),Z(x,y),>anaglyph,axis=0,zoom=5,w=450,h=350, ...
>  <shadow,look=povlook(blue), ...
>  xv=&"dn[1]"(x,y), yv=&"dn[2]"(x,y), zv=&"dn[3]"(x,y));
\end{eulerprompt}
\begin{eulercomment}
We can also generate a grid in 3D.
\end{eulercomment}
\begin{eulerprompt}
>povstart(zoom=4); ...
>x=-1:0.5:1; r=1-(x+1)^2/6; ...
>t=(0°:30°:360°)'; y=r*cos(t); z=r*sin(t); ...
>writeln(povgrid(x,y,z,d=0.02,dballs=0.05)); ...
>povend();
\end{eulerprompt}
\begin{eulercomment}
With povgrid(), curves are possible.
\end{eulercomment}
\begin{eulerprompt}
>povstart(center=[0,0,1],zoom=3.6); ...
>t=linspace(0,2,1000); r=exp(-t); ...
>x=cos(2*pi*10*t)*r; y=sin(2*pi*10*t)*r; z=t; ...
>writeln(povgrid(x,y,z,povlook(red))); ...
>writeAxis(0,2,axis=3); ...
>povend();
\end{eulerprompt}
\eulerheading{Povray Objects}
\begin{eulercomment}
Above, we used pov3d to plot surfaces. The povray interface in Euler can also generate Povray objects. These objects are
stored as strings in Euler, and need to be written to a Povray file.

We start the output with povstart().
\end{eulercomment}
\begin{eulerprompt}
>povstart(zoom=4);
\end{eulerprompt}
\begin{eulercomment}
First we define the three cylinders, and store them in strings in Euler.

The functions povx() etc. simply returns the vector [1,0,0], which could be used instead.
\end{eulercomment}
\begin{eulerprompt}
>c1=povcylinder(-povx,povx,1,povlook(red)); ...
>c2=povcylinder(-povy,povy,1,povlook(yellow)); ...
>c3=povcylinder(-povz,povz,1,povlook(blue)); ...
\end{eulerprompt}
\begin{eulercomment}
The strings contain Povray code, which we need not understand at that
point.
\end{eulercomment}
\begin{eulerprompt}
>c2
\end{eulerprompt}
\begin{euleroutput}
  cylinder \{ <0,0,-1>, <0,0,1>, 1
   texture \{ pigment \{ color rgb <0.941176,0.941176,0.392157> \}  \} 
   finish \{ ambient 0.2 \} 
   \}
\end{euleroutput}
\begin{eulercomment}
As you see, we added texture to the objects in three different colors.

That is done by povlook(), which returns a string with the relevant
Povray code. We can use the default Euler colors, or define our own
color. We can also add transparency, or change the ambient light.
\end{eulercomment}
\begin{eulerprompt}
>povlook(rgb(0.1,0.2,0.3),0.1,0.5)
\end{eulerprompt}
\begin{euleroutput}
   texture \{ pigment \{ color rgbf <0.101961,0.2,0.301961,0.1> \}  \} 
   finish \{ ambient 0.5 \} 
  
\end{euleroutput}
\begin{eulercomment}
Now we define an intersection object, and write the result to the
file.
\end{eulercomment}
\begin{eulerprompt}
>writeln(povintersection([c1,c2,c3]));
\end{eulerprompt}
\begin{eulercomment}
The intersection of three cylinders is hard to visualize, if you never
saw it before.
\end{eulercomment}
\begin{eulerprompt}
>povend;
\end{eulerprompt}
\begin{eulercomment}
The following functions generate a fractal recursively.

The first function shows, how Euler handles simple Povray objects. The
function povbox() returns a string, containing the box coordinates,
the texture and the finish.
\end{eulercomment}
\begin{eulerprompt}
>function onebox(x,y,z,d) := povbox([x,y,z],[x+d,y+d,z+d],povlook());
>function fractal (x,y,z,h,n) ...
\end{eulerprompt}
\begin{eulerudf}
   if n==1 then writeln(onebox(x,y,z,h));
   else
     h=h/3;
     fractal(x,y,z,h,n-1);
     fractal(x+2*h,y,z,h,n-1);
     fractal(x,y+2*h,z,h,n-1);
     fractal(x,y,z+2*h,h,n-1);
     fractal(x+2*h,y+2*h,z,h,n-1);
     fractal(x+2*h,y,z+2*h,h,n-1);
     fractal(x,y+2*h,z+2*h,h,n-1);
     fractal(x+2*h,y+2*h,z+2*h,h,n-1);
     fractal(x+h,y+h,z+h,h,n-1);
   endif;
  endfunction
\end{eulerudf}
\begin{eulerprompt}
>povstart(fade=10,<shadow);
>fractal(-1,-1,-1,2,4);
>povend();
\end{eulerprompt}
\begin{eulercomment}
Differences allow cutting off one object from another. Like
intersections, there are part of the CSG objects of Povray.
\end{eulercomment}
\begin{eulerprompt}
>povstart(light=[5,-5,5],fade=10);
\end{eulerprompt}
\begin{eulercomment}
For this demonstration, we define an object in Povray, instead of
using a string in Euler. Definitions are written to the file
immediately.

A box coordinate of -1 just means [-1,-1,-1].
\end{eulercomment}
\begin{eulerprompt}
>povdefine("mycube",povbox(-1,1));
\end{eulerprompt}
\begin{eulercomment}
We can use this object in povobject(), which returns a string as
usual.
\end{eulercomment}
\begin{eulerprompt}
>c1=povobject("mycube",povlook(red));
\end{eulerprompt}
\begin{eulercomment}
We generate a second cube, and rotate and scale it a bit.
\end{eulercomment}
\begin{eulerprompt}
>c2=povobject("mycube",povlook(yellow),translate=[1,1,1], ...
>  rotate=xrotate(10°)+yrotate(10°), scale=1.2);
\end{eulerprompt}
\begin{eulercomment}
Then we take the difference of the two objects.
\end{eulercomment}
\begin{eulerprompt}
>writeln(povdifference(c1,c2));
\end{eulerprompt}
\begin{eulercomment}
Now add three axes.
\end{eulercomment}
\begin{eulerprompt}
>writeAxis(-1.2,1.2,axis=1); ...
>writeAxis(-1.2,1.2,axis=2); ...
>writeAxis(-1.2,1.2,axis=4); ...
>povend();
\end{eulerprompt}
\eulerheading{Implicit Functions}
\begin{eulercomment}
Povray can plot the set where f(x,y,z)=0, just like the implicit parameter in plot3d. The results looks much better,
however.

The syntax for the functions is a bit different. You cannot use the output of Maxima or Euler expressions.

\end{eulercomment}
\begin{eulerformula}
\[
((x^2+y^2-c^2)^2+(z^2-1)^2)*((y^2+z^2-c^2)^2+(x^2-1)^2)*((z^2+x^2-c^2)^2+(y^2-1)^2)=d
\]
\end{eulerformula}
\begin{eulerprompt}
>povstart(angle=70°,height=50°,zoom=4);
>c=0.1; d=0.1; ...
>writeln(povsurface("(pow(pow(x,2)+pow(y,2)-pow(c,2),2)+pow(pow(z,2)-1,2))*(pow(pow(y,2)+pow(z,2)-pow(c,2),2)+pow(pow(x,2)-1,2))*(pow(pow(z,2)+pow(x,2)-pow(c,2),2)+pow(pow(y,2)-1,2))-d",povlook(red))); ...
>povend();
\end{eulerprompt}
\begin{euleroutput}
  Error : Povray error!
  
  Error generated by error() command
  
  povray:
      error("Povray error!");
  Try "trace errors" to inspect local variables after errors.
  povend:
      povray(file,w,h,aspect,exit); 
\end{euleroutput}
\begin{eulerprompt}
>povstart(angle=25°,height=10°); 
>writeln(povsurface("pow(x,2)+pow(y,2)*pow(z,2)-1",povlook(blue),povbox(-2,2,"")));
>povend();
>povstart(angle=70°,height=50°,zoom=4);
\end{eulerprompt}
\begin{eulercomment}
Create the implicit surface. Note the different syntax in the
expression.
\end{eulercomment}
\begin{eulerprompt}
>writeln(povsurface("pow(x,2)*y-pow(y,3)-pow(z,2)",povlook(green))); ...
>writeAxes(); ...
>povend();
\end{eulerprompt}
\eulerheading{Mesh Object}
\begin{eulercomment}
In this example, we show how to create a mesh object, and draw it with additional information.

We like to maximize xy under the condition x+y=1 and demonstrate the tangential touching of the level lines.
\end{eulercomment}
\begin{eulerprompt}
>povstart(angle=-10°,center=[0.5,0.5,0.5],zoom=7);
\end{eulerprompt}
\begin{eulercomment}
We cannot store the object in a string as before, since is too large. So we define the object in a Povray file using
#declare. The function povtriangle() does this automatically. It can accept normal vectors just like pov3d().

The following defines the mesh object, and writes it immediately into the file.
\end{eulercomment}
\begin{eulerprompt}
>x=0:0.02:1; y=x'; z=x*y; vx=-y; vy=-x; vz=1;
>mesh=povtriangles(x,y,z,"",vx,vy,vz);
\end{eulerprompt}
\begin{eulercomment}
Now we define two discs, which will be intersected with the surface.
\end{eulercomment}
\begin{eulerprompt}
>cl=povdisc([0.5,0.5,0],[1,1,0],2); ...
>ll=povdisc([0,0,1/4],[0,0,1],2);
\end{eulerprompt}
\begin{eulercomment}
Write the surface minus the two discs.
\end{eulercomment}
\begin{eulerprompt}
>writeln(povdifference(mesh,povunion([cl,ll]),povlook(green)));
\end{eulerprompt}
\begin{eulercomment}
Write the two intersections.
\end{eulercomment}
\begin{eulerprompt}
>writeln(povintersection([mesh,cl],povlook(red))); ...
>writeln(povintersection([mesh,ll],povlook(gray)));
\end{eulerprompt}
\begin{eulercomment}
Write a point at the maximum.
\end{eulercomment}
\begin{eulerprompt}
>writeln(povpoint([1/2,1/2,1/4],povlook(gray),size=2*defaultpointsize));
\end{eulerprompt}
\begin{eulercomment}
Add axes and finish.
\end{eulercomment}
\begin{eulerprompt}
>writeAxes(0,1,0,1,0,1,d=0.015); ...
>povend();
\end{eulerprompt}
\eulerheading{Anaglyphs in Povray}
\begin{eulercomment}
To generate an anaglyph for a red/cyan glasses, Povray must run twice
from different camera positions. It generates two Povray files and two
PNG files, which are loaded with the function loadanaglyph().

Of course, you need red/cyan glasses to view the following examples
properly.

The function pov3d() has a simple switch to generate anaglyphs.
\end{eulercomment}
\begin{eulerprompt}
>pov3d("-exp(-x^2-y^2)/2",r=2,height=45°,>anaglyph, ...
>  center=[0,0,0.5],zoom=3.5);
\end{eulerprompt}
\begin{eulercomment}
If you create a scene with objects, you need to put the generation of
the scene into a function, and run it twice with different values for
the anaglyph parameter.
\end{eulercomment}
\begin{eulerprompt}
>function myscene ...
\end{eulerprompt}
\begin{eulerudf}
    s=povsphere(povc,1);
    cl=povcylinder(-povz,povz,0.5);
    clx=povobject(cl,rotate=xrotate(90°));
    cly=povobject(cl,rotate=yrotate(90°));
    c=povbox([-1,-1,0],1);
    un=povunion([cl,clx,cly,c]);
    obj=povdifference(s,un,povlook(red));
    writeln(obj);
    writeAxes();
  endfunction
\end{eulerudf}
\begin{eulercomment}
The function povanaglyph() does all this. The parameters are like in
povstart() and povend() combined.
\end{eulercomment}
\begin{eulerprompt}
>povanaglyph("myscene",zoom=4.5);
\end{eulerprompt}
\eulerheading{Defining own Objects}
\begin{eulercomment}
The povray interface of Euler contains a lot of objects. But you are
not restricted to these. You can create own objects, which combine
other objects, or are completely new objects.

We demonstrate a torus. The Povray command for this is "torus". So we
return a string with this command and its parameters. Note that the
torus is always centered at the origin.
\end{eulercomment}
\begin{eulerprompt}
>function povdonat (r1,r2,look="") ...
\end{eulerprompt}
\begin{eulerudf}
    return "torus \{"+r1+","+r2+look+"\}";
  endfunction
\end{eulerudf}
\begin{eulercomment}
Here is our first torus.
\end{eulercomment}
\begin{eulerprompt}
>t1=povdonat(0.8,0.2)
\end{eulerprompt}
\begin{euleroutput}
  torus \{0.8,0.2\}
\end{euleroutput}
\begin{eulercomment}
Let us use this object to create a second torus, translated and
rotated.
\end{eulercomment}
\begin{eulerprompt}
>t2=povobject(t1,rotate=xrotate(90°),translate=[0.8,0,0])
\end{eulerprompt}
\begin{euleroutput}
  object \{ torus \{0.8,0.2\}
   rotate 90 *x 
   translate <0.8,0,0>
   \}
\end{euleroutput}
\begin{eulercomment}
Now we place these objects into a scene. For the look, we use Phong
Shading.
\end{eulercomment}
\begin{eulerprompt}
>povstart(center=[0.4,0,0],angle=0°,zoom=3.8,aspect=1.5); ...
>writeln(povobject(t1,povlook(green,phong=1))); ...
>writeln(povobject(t2,povlook(green,phong=1))); ...
\end{eulerprompt}
\begin{eulerttcomment}
 >povend();
\end{eulerttcomment}
\begin{eulercomment}
calls the Povray program. However, in case of errors, it does not
display the error. You should therefore use

\end{eulercomment}
\begin{eulerttcomment}
 >povend(<exit);
\end{eulerttcomment}
\begin{eulercomment}

if anything did not work. This will leave the Povray window open.
\end{eulercomment}
\begin{eulerprompt}
>povend(h=320,w=480);
\end{eulerprompt}
\begin{euleroutput}
  Function povstart not found.
  Try list ... to find functions!
  Error in:
  povstart(center=[0.4,0,0],angle=0°,zoom=3.8,aspect=1.5); writeln(povobject(t1,povlook(green,phong=1))); writeln(povobj ...
                                                         ^
\end{euleroutput}
\begin{eulercomment}
Here is a more elaborate example. We solve

\end{eulercomment}
\begin{eulerformula}
\[
Ax \le b, \quad x \ge 0, \quad c.x \to \text{Max.}
\]
\end{eulerformula}
\begin{eulercomment}
and show the feasible points and the optimum in a 3D plot.
\end{eulercomment}
\begin{eulerprompt}
>A=[10,8,4;5,6,8;6,3,2;9,5,6];
>b=[10,10,10,10]';
>c=[1,1,1];
\end{eulerprompt}
\begin{eulercomment}
First, let us check, if this example has a solution at all.
\end{eulercomment}
\begin{eulerprompt}
>x=simplex(A,b,c,>max,>check)'
\end{eulerprompt}
\begin{euleroutput}
  [0,  1,  0.5]
\end{euleroutput}
\begin{eulercomment}
Yes, it has.

Next we define two objects. The first is the plane

\end{eulercomment}
\begin{eulerformula}
\[
a \cdot x \le b
\]
\end{eulerformula}
\begin{eulerprompt}
>function oneplane (a,b,look="") ...
\end{eulerprompt}
\begin{eulerudf}
    return povplane(a,b,look)
  endfunction
\end{eulerudf}
\begin{eulercomment}
Then we define the intersection of all half spaces and a cube.
\end{eulercomment}
\begin{eulerprompt}
>function adm (A, b, r, look="") ...
\end{eulerprompt}
\begin{eulerudf}
    ol=[];
    loop 1 to rows(A); ol=ol|oneplane(A[#],b[#]); end;
    ol=ol|povbox([0,0,0],[r,r,r]);
    return povintersection(ol,look);
  endfunction
\end{eulerudf}
\begin{eulercomment}
We can now plot the scene.
\end{eulercomment}
\begin{eulerprompt}
>povstart(angle=120°,center=[0.5,0.5,0.5],zoom=3.5); ...
>writeln(adm(A,b,2,povlook(green,0.4))); ...
>writeAxes(0,1.3,0,1.6,0,1.5); ...
\end{eulerprompt}
\begin{eulercomment}
The following is a circle around the optimum.
\end{eulercomment}
\begin{eulerprompt}
>writeln(povintersection([povsphere(x,0.5),povplane(c,c.x')], ...
>  povlook(red,0.9)));
\end{eulerprompt}
\begin{eulercomment}
And an error in the direction of the optimum.
\end{eulercomment}
\begin{eulerprompt}
>writeln(povarrow(x,c*0.5,povlook(red)));
\end{eulerprompt}
\begin{eulercomment}
We add text to the screen. Text is just a 3D object. We need to place
and turn it according to our view.
\end{eulercomment}
\begin{eulerprompt}
>writeln(povtext("Linear Problem",[0,0.2,1.3],size=0.05,rotate=5°)); ...
>povend();
\end{eulerprompt}
\eulerheading{More Examples}
\begin{eulercomment}
You can find some more examples for Povray in Euler in the following
files.

See: Examples/Dandelin Spheres\\
See: Examples/Donat Math\\
See: Examples/Trefoil Knot\\
See: Examples/Optimization by Affine Scaling
\end{eulercomment}
\end{eulernotebook}
\end{document}


\newpage
\chapter{KB Pekan 6-7: Menggunakan EMT untuk kalkulus}
\documentclass[a4paper,10pt]{article}
\usepackage{eumat}

\begin{document}
\begin{eulernotebook}
\begin{eulercomment}
Rasyid Shalahuddin\\
22305144016\\
Matematika E 2022


\begin{eulercomment}
\eulerheading{Kalkulus dengan EMT}
\begin{eulercomment}
Materi Kalkulus mencakup di antaranya:

- Fungsi (fungsi aljabar, trigonometri, eksponensial, logaritma,
komposisi fungsi)\\
- Limit Fungsi,\\
- Turunan Fungsi,\\
- Integral Tak Tentu,\\
- Integral Tentu dan Aplikasinya,\\
- Barisan dan Deret (kekonvergenan barisan dan deret).

EMT (bersama Maxima) dapat digunakan untuk melakukan semua perhitungan
di dalam kalkulus, baik secara numerik maupun analitik (eksak).

\end{eulercomment}
\eulersubheading{Mendefinisikan Fungsi}
\begin{eulercomment}
Terdapat beberapa cara untuk mendefinisikan fungsi pada EMT, yakni:

- Menggunakan format nama\_fungsi := rumus fungsi (untuk fungsi
numerik),\\
- Menggunakan format nama\_fungsi \&= rumus fungsi (untuk fungsi
simbolik, namun dapat dihitung secara numerik),\\
- Menggunakan format nama\_fungsi \&\&= rumus fungsi (untuk fungsi
simbolik murni, tidak dapat dihitung langsung),\\
- Fungsi sebagai program EMT.

Setiap format harus diawali dengan perintah function (bukan sebagai
ekspresi).

Berikut adalah adalah beberapa contoh cara mendefinisikan fungsi.
\end{eulercomment}
\begin{eulerprompt}
>function f(x) := 3*x^2+exp(sin(x)) // fungsi numerik
>f(0), f(1), f(pi)
\end{eulerprompt}
\begin{euleroutput}
  1
  5.31977682472
  30.6088132033
\end{euleroutput}
\begin{eulerprompt}
>function g(x) := sqrt(x^2-4*x)/(x+1)
>g(4)
\end{eulerprompt}
\begin{euleroutput}
  0
\end{euleroutput}
\begin{eulerprompt}
>g(0)
\end{eulerprompt}
\begin{euleroutput}
  0
\end{euleroutput}
\begin{eulerprompt}
>f(g(5)) // komposisi fungsi
\end{eulerprompt}
\begin{euleroutput}
  1.85590048506
\end{euleroutput}
\begin{eulerprompt}
>g(f(5))
\end{eulerprompt}
\begin{euleroutput}
  0.960367527714
\end{euleroutput}
\begin{eulerprompt}
>f(0:10) // nilai-nilai f(1), f(2), ..., f(10)
\end{eulerprompt}
\begin{euleroutput}
  [1,  5.31978,  14.4826,  28.1516,  48.4692,  75.3833,  108.756,
  148.929,  194.69,  244.51,  300.58]
\end{euleroutput}
\begin{eulerprompt}
>fmap(0:10) // sama dengan f(0:10), berlaku untuk semua fungsi
\end{eulerprompt}
\begin{euleroutput}
  [1,  5.31978,  14.4826,  28.1516,  48.4692,  75.3833,  108.756,
  148.929,  194.69,  244.51,  300.58]
\end{euleroutput}
\begin{eulercomment}
Misalkan kita akan mendefinisikan fungsi

\end{eulercomment}
\begin{eulerformula}
\[
f(x) = \begin{cases} x^3 & x>0 \\ x^2 & x\le 0. \end{cases}
\]
\end{eulerformula}
\begin{eulercomment}
Fungsi tersebut tidak dapat didefinisikan sebagai fungsi numerik
secara "inline" menggunakan format :=, melainkan didefinisikan sebagai
program. Perhatikan, kata "map" digunakan agar fungsi dapat menerima
vektor sebagai input, dan hasilnya berupa vektor. Jika tanpa kata
"map" fungsinya hanya dapat menerima input satu nilai.
\end{eulercomment}
\begin{eulerprompt}
>function map f(x) ...
\end{eulerprompt}
\begin{eulerudf}
    if x>0 then return x^3
    else return x^2
    endif;
  endfunction
\end{eulerudf}
\begin{eulerprompt}
>f(1)
\end{eulerprompt}
\begin{euleroutput}
  1
\end{euleroutput}
\begin{eulercomment}
karena 1\textgreater{}0 maka fungsi akan memunculkan x\textasciicircum{}3 yang berarti 1\textasciicircum{}3=1
\end{eulercomment}
\begin{eulerprompt}
>f(-2)
\end{eulerprompt}
\begin{euleroutput}
  4
\end{euleroutput}
\begin{eulercomment}
karena -2\textless{}0 maka fungsi akan memunculkan x\textasciicircum{}2 yang berarti (-2)\textasciicircum{}2=4
\end{eulercomment}
\begin{eulerprompt}
>f(-5:5) // fungsi ini memberikan nilai nilai f(-5) sampai f(5)
\end{eulerprompt}
\begin{euleroutput}
  [25,  16,  9,  4,  1,  0,  1,  8,  27,  64,  125]
\end{euleroutput}
\begin{eulerprompt}
>aspect(1.5); plot2d("f(x)",-5,5):
\end{eulerprompt}
\eulerimg{12}{images/EMTKalkulus_Rasyid Shalahuddin_22305144016-001.png}
\begin{eulerprompt}
>function f(x) &= 2*E^x // fungsi simbolik
\end{eulerprompt}
\begin{euleroutput}
  
                                      x
                                   2 E
  
\end{euleroutput}
\begin{eulerprompt}
>function g(x) &= 3*x+1
\end{eulerprompt}
\begin{euleroutput}
  
                                 3 x + 1
  
\end{euleroutput}
\begin{eulerprompt}
>function h(x) &= f(g(x)) // komposisi fungsi
\end{eulerprompt}
\begin{euleroutput}
  
                                   3 x + 1
                                2 E
  
\end{euleroutput}
\eulerheading{Latihan}
\begin{eulercomment}
Bukalah buku Kalkulus. Cari dan pilih beberapa (paling sedikit 5
fungsi berbeda tipe/bentuk/jenis) fungsi dari buku tersebut, kemudian
definisikan di EMT pada baris-baris perintah berikut (jika perlu
tambahkan lagi). Untuk setiap fungsi, hitung beberapa nilainya, baik
untuk satu nilai maupun vektor. Gambar grafik tersebut.

Juga, carilah fungsi beberapa (dua) variabel. Lakukan hal sama seperti
di atas.

Jawab:\\
\end{eulercomment}
\begin{eulerformula}
\[
\text{A). FUNGSI 1 VARIABEL}
\]
\end{eulerformula}
\begin{eulercomment}
1. Fungsi 1
\end{eulercomment}
\begin{eulerprompt}
>function k(x) := 2*x*(x^3-9)^2
>k(5), k(6), k(7)
\end{eulerprompt}
\begin{euleroutput}
  134560
  514188
  1561784
\end{euleroutput}
\begin{eulerprompt}
>kmap(-4:3)
\end{eulerprompt}
\begin{euleroutput}
  [-42632,  -7776,  -1156,  -200,  0,  128,  4,  1944]
\end{euleroutput}
\begin{eulerprompt}
>plot2d("k(x)"):
\end{eulerprompt}
\eulerimg{12}{images/EMTKalkulus_Rasyid Shalahuddin_22305144016-002.png}
\begin{eulercomment}
2. Fungsi 2
\end{eulercomment}
\begin{eulerprompt}
>function y(x) := 3*(x)^3/(4-x^3) 
>y(0), y(-1), y(3)
\end{eulerprompt}
\begin{euleroutput}
  0
  -0.6
  -3.52173913043
\end{euleroutput}
\begin{eulerprompt}
>ymap(-5:-5)
\end{eulerprompt}
\begin{euleroutput}
  -2.90697674419
\end{euleroutput}
\begin{eulerprompt}
>plot2d("y(x)"):
\end{eulerprompt}
\eulerimg{12}{images/EMTKalkulus_Rasyid Shalahuddin_22305144016-003.png}
\begin{eulercomment}
3. Fungsi 3
\end{eulercomment}
\begin{eulerprompt}
>function g(x) := 8*x/(2*x^2+5)+2
>g(2), g(-1), g(-3), g(4)
\end{eulerprompt}
\begin{euleroutput}
  3.23076923077
  0.857142857143
  0.95652173913
  2.86486486486
\end{euleroutput}
\begin{eulerprompt}
>gmap(2:5)
\end{eulerprompt}
\begin{euleroutput}
  [3.23077,  3.04348,  2.86486,  2.72727]
\end{euleroutput}
\begin{eulerprompt}
>plot2d("g(x)"):
\end{eulerprompt}
\eulerimg{12}{images/EMTKalkulus_Rasyid Shalahuddin_22305144016-004.png}
\begin{eulercomment}
4. Fungsi 4
\end{eulercomment}
\begin{eulerprompt}
>function j(x) := 8*x^5/(x^3-3)
>j(5), j(4), j(3)
\end{eulerprompt}
\begin{euleroutput}
  204.918032787
  134.295081967
  81
\end{euleroutput}
\begin{eulerprompt}
>jmap(5:8)
\end{eulerprompt}
\begin{euleroutput}
  [204.918,  292.056,  395.459,  515.018]
\end{euleroutput}
\begin{eulerprompt}
>plot2d("j(x)",-3,3,-600,600):
\end{eulerprompt}
\eulerimg{12}{images/EMTKalkulus_Rasyid Shalahuddin_22305144016-005.png}
\begin{eulercomment}
5. Fungsi 5
\end{eulercomment}
\begin{eulerprompt}
>function l(x) := (sin(x))*-cos(8*x)
>l(pi), l(0), l(pi/3)
\end{eulerprompt}
\begin{euleroutput}
  0
  0
  0.433012701892
\end{euleroutput}
\begin{eulerprompt}
>lmap(0:3pi)
\end{eulerprompt}
\begin{euleroutput}
  [0,  0.122434,  0.870797,  -0.0598601,  0.631342,  -0.639543,
  -0.178866,  -0.560554,  -0.387687,  0.398622]
\end{euleroutput}
\begin{eulerprompt}
>plot2d("j(x)"):
\end{eulerprompt}
\eulerimg{12}{images/EMTKalkulus_Rasyid Shalahuddin_22305144016-006.png}
\begin{eulercomment}
6. Fungsi 6
\end{eulercomment}
\begin{eulerprompt}
>function z(x) := x*sqrt(6x+12)
>z(11), z(9), z(8)
\end{eulerprompt}
\begin{euleroutput}
  97.1493695296
  73.1163456417
  61.9677335393
\end{euleroutput}
\begin{eulerprompt}
>zmap(3:12)
\end{eulerprompt}
\begin{euleroutput}
  [16.4317,  24,  32.4037,  41.5692,  51.4393,  61.9677,  73.1163,
  84.8528,  97.1494,  109.982]
\end{euleroutput}
\begin{eulerprompt}
>plot2d("z(x)"):
\end{eulerprompt}
\eulerimg{12}{images/EMTKalkulus_Rasyid Shalahuddin_22305144016-007.png}
\begin{eulercomment}
\end{eulercomment}
\begin{eulerformula}
\[
\text{B). FUNGSI 2 VARIABEL}
\]
\end{eulerformula}
\begin{eulercomment}
1. Fungsi 1
\end{eulercomment}
\begin{eulerprompt}
>function a(x,y) ...
\end{eulerprompt}
\begin{eulerudf}
  return x^2+y^2-24
  endfunction
\end{eulerudf}
\begin{eulerprompt}
>a(2,1), a(5,4), a(2,4)
\end{eulerprompt}
\begin{euleroutput}
  -19
  17
  -4
\end{euleroutput}
\begin{eulerprompt}
>amap(-2:2,3:3)
\end{eulerprompt}
\begin{euleroutput}
  [-11,  -14,  -15,  -14,  -11]
\end{euleroutput}
\begin{eulerprompt}
>aspect=1.5; plot3d("a(x,y)",a=-100,b=100,c=-80,d=80,angle=35°,height=30°,r=pi,n=100):
\end{eulerprompt}
\eulerimg{12}{images/EMTKalkulus_Rasyid Shalahuddin_22305144016-008.png}
\begin{eulercomment}
2. Fungsi 2
\end{eulercomment}
\begin{eulerprompt}
>function q(x,y) ...
\end{eulerprompt}
\begin{eulerudf}
  return y^2/(x^2/3)
  endfunction
\end{eulerudf}
\begin{eulerprompt}
>q(4,2), q(2,3), q(4,3)
\end{eulerprompt}
\begin{euleroutput}
  0.75
  6.75
  1.6875
\end{euleroutput}
\begin{eulerprompt}
>qmap(2:2,-2:2)
\end{eulerprompt}
\begin{euleroutput}
  [3,  0.75,  0,  0.75,  3]
\end{euleroutput}
\begin{eulerprompt}
>aspect=1.5; plot3d("q(x,y)",a=-100,b=100,c=-80,d=80,angle=35°,height=30°,r=pi,n=100):
\end{eulerprompt}
\eulerimg{12}{images/EMTKalkulus_Rasyid Shalahuddin_22305144016-009.png}
\eulerheading{Menghitung Limit}
\begin{eulercomment}
Perhitungan limit pada EMT dapat dilakukan dengan menggunakan fungsi
Maxima, yakni "limit". Fungsi "limit" dapat digunakan untuk menghitung
limit fungsi dalam bentuk ekspresi maupun fungsi yang sudah
didefinisikan sebelumnya. Nilai limit dapat dihitung pada sebarang
nilai atau pada tak hingga (-inf, minf, dan inf). Limit kiri dan limit
kanan juga dapat dihitung, dengan cara memberi opsi "plus" atau
"minus". Hasil limit dapat berupa nilai, "und' (tak definisi), "ind"
(tak tentu namun terbatas), "infinity" (kompleks tak hingga).

Perhatikan beberapa contoh berikut. Perhatikan cara menampilkan
perhitungan secara lengkap, tidak hanya menampilkan hasilnya saja.
\end{eulercomment}
\begin{eulerprompt}
>$showev('limit(1/(2*x-1),x,0))
>$showev('limit((x^2-3*x-10)/(x-5),x,5))
>$showev('limit(sin(x)/x,x,0))
>plot2d("sin(x)/x",-pi,pi):
\end{eulerprompt}
\eulerimg{12}{images/EMTKalkulus_Rasyid Shalahuddin_22305144016-010.png}
\begin{eulerprompt}
>$showev('limit(sin(x^3)/x,x,0))
>$showev('limit(log(x), x, minf))
>$showev('limit((-2)^x,x, inf))
>$showev('limit(t-sqrt(2-t),t,2,minus))
>$showev('limit(t-sqrt(2-t),t,5,plus)) // Perhatikan hasilnya
>plot2d("x-sqrt(2-x)",-2,5):
\end{eulerprompt}
\eulerimg{12}{images/EMTKalkulus_Rasyid Shalahuddin_22305144016-011.png}
\begin{eulerprompt}
>$showev('limit((x^2-9)/(2*x^2-5*x-3),x,3))
>$showev('limit((1-cos(x))/x,x,0))
>$showev('limit((x^2+abs(x))/(x^2-abs(x)),x,0))
>$showev('limit((1+1/x)^x,x,inf))
>$showev('limit((1+k/x)^x,x,inf))
>$showev('limit((1+x)^(1/x),x,0))
>$showev('limit((x/(x+k))^x,x,inf))
>$showev('limit(sin(1/x),x,0))
>$showev('limit(sin(1/x),x,inf))
>plot2d("sin(1/x)",-5,5):
\end{eulerprompt}
\eulerimg{12}{images/EMTKalkulus_Rasyid Shalahuddin_22305144016-012.png}
\eulerheading{Latihan}
\begin{eulercomment}
Bukalah buku Kalkulus. Cari dan pilih beberapa (paling sedikit 5
fungsi berbeda tipe/bentuk/jenis) fungsi dari buku tersebut, kemudian
definisikan di EMT pada baris-baris perintah berikut (jika perlu
tambahkan lagi). Untuk setiap fungsi, hitung nilai limit fungsi
tersebut di beberapa nilai dan di tak hingga. Gambar grafik fungsi
tersebut untuk mengkonfirmasi nilai-nilai limit tersebut.

Jawab:\\
1. Fungsi 1\\
\end{eulercomment}
\begin{eulerformula}
\[
\text{$f(x)=\frac{3x-6}{x+2}$}
\]
\end{eulerformula}
\begin{eulerprompt}
>$showev('limit((3*x-6)/(x+2),x,2))
>plot2d("(3*x-6)/(x+2)",-2,3.5,-1,5):
\end{eulerprompt}
\eulerimg{12}{images/EMTKalkulus_Rasyid Shalahuddin_22305144016-013.png}
\begin{eulercomment}
2. Fungsi 2\\
\end{eulercomment}
\begin{eulerformula}
\[
\text{$f(x)=\frac{cos 2x}{sin x - cos x}$}
\]
\end{eulerformula}
\begin{eulerprompt}
>$showev('limit(cos(2*x)/(sin(x) - cos (x)),x,0))
>plot2d("cos(2*x)/(sin(x) - cos (x))",-1,1):
\end{eulerprompt}
\eulerimg{12}{images/EMTKalkulus_Rasyid Shalahuddin_22305144016-014.png}
\begin{eulercomment}
3. Fungsi 3\\
\end{eulercomment}
\begin{eulerformula}
\[
\text{$f(x)=\frac{2x^2-2x+5}{3x^2+x-6}$}
\]
\end{eulerformula}
\begin{eulerprompt}
>$showev('limit(((2*x^2-2*x+5)/(3*x^2+x-6)),x,3))
>plot2d("(2*x^2-2*x+5)/(3*x^2+x-6)",-2,10,-10,5):
\end{eulerprompt}
\eulerimg{12}{images/EMTKalkulus_Rasyid Shalahuddin_22305144016-015.png}
\begin{eulercomment}
4. Fungsi 4\\
\end{eulercomment}
\begin{eulerformula}
\[
\text{$f(x)=4x^2-3$}
\]
\end{eulerformula}
\begin{eulerprompt}
>$showev('limit((4*x^2-3),x,0))
>plot2d("(4*x^2-3)"):
\end{eulerprompt}
\eulerimg{12}{images/EMTKalkulus_Rasyid Shalahuddin_22305144016-016.png}
\begin{eulercomment}
5. Fungsi 5\\
\end{eulercomment}
\begin{eulerformula}
\[
\text{$f(x)=x^{x^{x}}$}
\]
\end{eulerformula}
\begin{eulerprompt}
>$showev('limit((x^(x^(x))),x,0,plus))
>plot2d("(x^(x^(x)))",-3,3,-1,7):
\end{eulerprompt}
\eulerimg{12}{images/EMTKalkulus_Rasyid Shalahuddin_22305144016-017.png}
\begin{eulercomment}
6. Fungsi 6\\
\end{eulercomment}
\begin{eulerformula}
\[
\text{$f(x)=\frac{3xtanx}{1-cos4x}$}
\]
\end{eulerformula}
\begin{eulerprompt}
>$showev('limit((3*x*tan(x))/(1-cos(4*x)),x,0))
>plot2d("(3*x*tan(x))/(1-cos(4*x))",-pi/2,2pi,0,2pi):
\end{eulerprompt}
\eulerimg{12}{images/EMTKalkulus_Rasyid Shalahuddin_22305144016-018.png}
\eulerheading{Turunan Fungsi}
\begin{eulercomment}
Definisi turunan:

\end{eulercomment}
\begin{eulerformula}
\[
f'(x) = \lim_{h\to 0} \frac{f(x+h)-f(x)}{h}
\]
\end{eulerformula}
\begin{eulercomment}
Berikut adalah contoh-contoh menentukan turunan fungsi dengan
menggunakan definisi turunan (limit).
\end{eulercomment}
\begin{eulerprompt}
>$showev('limit(((x+h)^n-x^n)/h,h,0)) // turunan x^n
\end{eulerprompt}
\begin{eulercomment}
Mengapa hasilnya seperti itu? Tuliskan atau tunjukkan bahwa hasil
limit tersebut benar, sehingga benar turunan fungsinya benar.  Tulis
penjelasan Anda di komentar ini.

Sebagai petunjuk, ekspansikan (x+h)\textasciicircum{}n dengan menggunakan teorema
binomial.\\
Jawab:\\
\end{eulercomment}
\begin{eulerformula}
\[
\text{Akan ditunjukkan bahwa \: $f'(x)=\lim_{h\to 0} \frac{(x+h)^n-x^n}{h}=nx^{n-1}$}
\]
\end{eulerformula}
\begin{eulercomment}
\end{eulercomment}
\begin{eulerformula}
\[
\text{Pertama, ekspansikan $(x+h)^n$, yakni: }
\]
\end{eulerformula}
\begin{eulercomment}
\end{eulercomment}
\begin{eulerformula}
\[
\text{$(x+h)^n=\sum_{k=0}^{n} \binom{n}{k}x^{n-k}h^k$}
\]
\end{eulerformula}
\begin{eulercomment}
\end{eulercomment}
\begin{eulerformula}
\[
\text{$\Leftrightarrow \: (x+h)^n=\binom{n}{0}x^{n}+\binom{n}{1}x^{n-1}h+\binom{n}{2}x^{n-2}h^2+ ...+\binom{n}{n}h^n$}
\]
\end{eulerformula}
\begin{eulercomment}
\end{eulercomment}
\begin{eulerformula}
\[
\text{$\Leftrightarrow \: (x+h)^n=x^{n}+nx^{n-1}h+\binom{n}{2}x^{n-2}h^2+\binom{n}{3}x^{n-3}h^3+ ...+h^n$}
\]
\end{eulerformula}
\begin{eulercomment}
\end{eulercomment}
\begin{eulerformula}
\[
\text{Sehingga, $f'(x)$ menjadi:\: $f'(x)=\lim_{h\to 0} \frac{(x+h)^n-x^n}{h}$}
\]
\end{eulerformula}
\begin{eulercomment}
\end{eulercomment}
\begin{eulerformula}
\[
\text{$\Leftrightarrow f'(x)=\lim_{h\to 0} \frac{x^{n}+nx^{n-1}h+\binom{n}{2}x^{n-2}h^2+\binom{n}{3}x^{n-3}h^3+ ...+h^n-x^n}{h}$}
\]
\end{eulerformula}
\begin{eulercomment}
\end{eulercomment}
\begin{eulerformula}
\[
\text{$\Leftrightarrow f'(x)=\lim_{h\to 0} nx^{n-1}+\binom{n}{2}x^{n-2}h+\binom{n}{3}x^{n-3}h^2+ ...+h^{n-1}$}
\]
\end{eulerformula}
\begin{eulercomment}
\end{eulercomment}
\begin{eulerformula}
\[
\text{$\Leftrightarrow f'(x)=nx^{n-1}$. Terbukti.}
\]
\end{eulerformula}
\begin{eulerprompt}
>$showev('limit((sin(x+h)-sin(x))/h,h,0)) // turunan sin(x)
\end{eulerprompt}
\begin{eulercomment}
Mengapa hasilnya seperti itu? Tuliskan atau tunjukkan bahwa hasil
limit tersebut\\
benar, sehingga benar turunan fungsinya benar.  Tulis penjelasan Anda
di komentar ini.

Sebagai petunjuk, ekspansikan sin(x+h) dengan menggunakan rumus jumlah
dua sudut.\\
Jawab:\\
\end{eulercomment}
\begin{eulerformula}
\[
\text{Akan ditunjukkan bahwa\: $\lim_{h\to 0} \frac{\sin(x+h)-\sin x}{h}=\cos x$}
\]
\end{eulerformula}
\begin{eulercomment}
\end{eulercomment}
\begin{eulerformula}
\[
\text{Diketahui bahwa:}
\]
\end{eulerformula}
\begin{eulercomment}
\end{eulercomment}
\begin{eulerformula}
\[
\text{$1).\: \sin(x+h)=\sin x\cos h+\cos x\sin h$}
\]
\end{eulerformula}
\begin{eulerformula}
\[
\text{$2).\: \lim_{h\to 0} \frac{1-\cos h}{h}=0$}
\]
\end{eulerformula}
\begin{eulerformula}
\[
\text{$3).\: \lim_{h\to 0} \frac{\sin h}{h}=1$}
\]
\end{eulerformula}
\begin{eulercomment}
\end{eulercomment}
\begin{eulerformula}
\[
\text{$\lim_{h\to 0} \frac{\sin(x+h)-\sin x}{h}$}
\]
\end{eulerformula}
\begin{eulercomment}
\end{eulercomment}
\begin{eulerformula}
\[
\text{$=\lim_{h\to 0} \frac{\sin x\cos h+\cos x\sin h-\sin x}{h}$}
\]
\end{eulerformula}
\begin{eulercomment}
\end{eulercomment}
\begin{eulerformula}
\[
\text{$=\lim_{h\to 0} \left[-\sin x\cdot\frac{1-\cos h}{h}+\cos x\cdot\frac{\sin h}{h}\right]$}
\]
\end{eulerformula}
\begin{eulercomment}
\end{eulercomment}
\begin{eulerformula}
\[
\text{$=(-\sin x)\left[\lim_{h\to 0} \frac{1-\cos h}{h}+(\cos x)\lim_{h\to 0} \frac{\sin h}{h}\right]$}
\]
\end{eulerformula}
\begin{eulercomment}
\end{eulercomment}
\begin{eulerformula}
\[
\text{$=(-\sin x)(0)+(\cos x)(1)=\cos x$. Terbukti.}
\]
\end{eulerformula}
\begin{eulerprompt}
>$showev('limit((log(x+h)-log(x))/h,h,0)) // turunan log(x)
\end{eulerprompt}
\begin{eulercomment}
Mengapa hasilnya seperti itu? Tuliskan atau tunjukkan bahwa hasil
limit tersebut\\
benar, sehingga benar turunan fungsinya benar.  Tulis penjelasan Anda
di komentar ini.

Sebagai petunjuk, gunakan sifat-sifat logaritma dan hasil limit pada
bagian sebelumnya di atas.\\
Jawab:\\
Bukti:\\
\end{eulercomment}
\begin{eulerformula}
\[
\text{Ambil $f(x)=^a\log x$.}
\]
\end{eulerformula}
\begin{eulercomment}
\end{eulercomment}
\begin{eulerformula}
\[
\text{$\lim_{h\to 0} \frac{^a\log (x+h)-^a\log x}{h}$}
\]
\end{eulerformula}
\begin{eulercomment}
\end{eulercomment}
\begin{eulerformula}
\[
\text{$=\lim _{h\to 0} \frac{^a\log \frac{(x+h)}{x}}{h}$}
\]
\end{eulerformula}
\begin{eulercomment}
\end{eulercomment}
\begin{eulerformula}
\[
\text{$=\lim_{h\to 0} \frac{^a\log (1+\frac{h}{x})}{h}$}
\]
\end{eulerformula}
\begin{eulercomment}
\end{eulercomment}
\begin{eulerformula}
\[
\text{$=\lim_{h\to 0} \frac{^a\log (1+\frac{h}{x})}{\frac{h}{x}x}$}
\]
\end{eulerformula}
\begin{eulercomment}
\end{eulercomment}
\begin{eulerformula}
\[
\text{$=\lim_{h\to 0} \frac{\frac{x}{h}\cdot ^a\log (1+\frac{h}{x})}{x}$}
\]
\end{eulerformula}
\begin{eulercomment}
\end{eulercomment}
\begin{eulerformula}
\[
\text{$=\lim_{h\to 0} \frac{^a\log (1+\frac{h}{x})^\frac{x}{h}}{x}$}
\]
\end{eulerformula}
\begin{eulercomment}
\end{eulercomment}
\begin{eulerformula}
\[
\text{$=\frac{\lim_{h\to 0}\: ^a\log (1+\frac{h}{x})^\frac{x}{h}}{\lim _{h\to 0}\: x}$}
\]
\end{eulerformula}
\begin{eulercomment}
\end{eulercomment}
\begin{eulerformula}
\[
\text{$=\frac{1}{x\cdot ^e\log a}$}
\]
\end{eulerformula}
\begin{eulercomment}
\end{eulercomment}
\begin{eulerformula}
\[
\text{$=\frac{1}{x\cdot \ln a}$}
\]
\end{eulerformula}
\begin{eulercomment}
\end{eulercomment}
\begin{eulerformula}
\[
\text{Menggunakan hasil di atas, maka:}
\]
\end{eulerformula}
\begin{eulercomment}
\end{eulercomment}
\begin{eulerformula}
\[
\text{$\frac{d\: \ln x}{dx}=\frac{d\: ^e\log x}{dx}=\frac{1}{x\cdot \ln e}=\frac{1}{x}.$ Terbukti.}
\]
\end{eulerformula}
\begin{eulerprompt}
>$showev('limit((1/(x+h)-1/x)/h,h,0)) // turunan 1/x
>$showev('limit((E^(x+h)-E^x)/h,h,0)) // turunan f(x)=e^x
\end{eulerprompt}
\begin{euleroutput}
  Answering "Is x an integer?" with "integer"
  Answering "Is x an integer?" with "integer"
  Answering "Is x an integer?" with "integer"
  Answering "Is x an integer?" with "integer"
  Answering "Is x an integer?" with "integer"
  Maxima is asking
  Acceptable answers are: yes, y, Y, no, n, N, unknown, uk
  Is x an integer?
  
  Use assume!
  Error in:
   $showev('limit((E^(x+h)-E^x)/h,h,0)) // turunan f(x)=e^x ...
                                       ^
\end{euleroutput}
\begin{eulercomment}
Maxima bermasalah dengan limit:

\end{eulercomment}
\begin{eulerformula}
\[
\lim_{h\to 0}\frac{e^{x+h}-e^x}{h}.
\]
\end{eulerformula}
\begin{eulercomment}
Oleh karena itu diperlukan trik khusus agar hasilnya benar.
\end{eulercomment}
\begin{eulerprompt}
>$showev('limit((E^h-1)/h,h,0))
>$factor(E^(x+h)-E^x)
>$showev('limit(factor((E^(x+h)-E^x)/h),h,0)) // turunan f(x)=e^x
>function f(x) &= x^x
\end{eulerprompt}
\begin{euleroutput}
  
                                     x
                                    x
  
\end{euleroutput}
\begin{eulerprompt}
>$showev('limit((f(x+h)-f(x))/h,h,0)) // turunan f(x)=x^x
\end{eulerprompt}
\begin{eulercomment}
Di sini Maxima juga bermasalah terkait limit:

\end{eulercomment}
\begin{eulerformula}
\[
lim_{h\to 0} \frac{(x+h)^{x+h}-x^x}{h}.
\]
\end{eulerformula}
\begin{eulercomment}
Dalam hal ini diperlukan asumsi nilai x.
\end{eulercomment}
\begin{eulerprompt}
>&assume(x>0); $showev('limit((f(x+h)-f(x))/h,h,0)) // turunan f(x)=x^x
>&forget(x>0) // jangan lupa, lupakan asumsi untuk kembali ke semula
\end{eulerprompt}
\begin{euleroutput}
  
                                 [x > 0]
  
\end{euleroutput}
\begin{eulerprompt}
>&forget(x<0)
\end{eulerprompt}
\begin{euleroutput}
  
                                 [x < 0]
  
\end{euleroutput}
\begin{eulerprompt}
>&facts()
\end{eulerprompt}
\begin{euleroutput}
  
                                    []
  
\end{euleroutput}
\begin{eulerprompt}
>$showev('limit((asin(x+h)-asin(x))/h,h,0)) // turunan arcsin(x)
\end{eulerprompt}
\begin{euleroutput}
  
\end{euleroutput}
\begin{eulerprompt}
>$showev('limit((tan(x+h)-tan(x))/h,h,0)) // turunan tan(x)
>function f(x) &= sinh(x) // definisikan f(x)=sinh(x)
\end{eulerprompt}
\begin{euleroutput}
  
                                 sinh(x)
  
\end{euleroutput}
\begin{eulerprompt}
>function df(x) &= limit((f(x+h)-f(x))/h,h,0); $df(x) // df(x) = f'(x)
\end{eulerprompt}
\begin{eulercomment}
Hasilnya adalah cosh(x), karena

\end{eulercomment}
\begin{eulerformula}
\[
\frac{e^x+e^{-x}}{2}=\cosh(x).
\]
\end{eulerformula}
\begin{eulerprompt}
>plot2d(["f(x)","df(x)"],-pi,pi,color=[blue,red]):
\end{eulerprompt}
\eulerheading{Latihan}
\begin{eulercomment}
Bukalah buku Kalkulus. Cari dan pilih beberapa (paling sedikit 5
fungsi berbeda tipe/bentuk/jenis) fungsi dari buku tersebut, kemudian
definisikan di EMT pada baris-baris perintah berikut (jika perlu
tambahkan lagi). Untuk setiap fungsi, tentukan turunannya dengan
menggunakan definisi turunan (limit), seperti contoh-contoh tersebut.
Gambar grafik fungsi asli dan fungsi turunannya pada sumbu koordinat
yang sama.

Jawab:\\
1. Fungsi 1
\end{eulercomment}
\begin{eulerprompt}
>function f(x) := x^2
>$showev('limit((((x+h)^2-x^2)/h),h,0)) // turunan x^2
>function df(x) &= limit((((x+h)^2-x^2)/h),h,0);  $df(x)// df(x) = f'(x)
>plot2d(["f(x)","df(x)"],-pi,pi,color=[blue,red]), label("f(x)",2,0.6), label("df(x)",2,0.17):
\end{eulerprompt}
\begin{eulercomment}
2. Fungsi 2
\end{eulercomment}
\begin{eulerprompt}
>function f(x) := sin(x)*cos(x)
>$showev('limit(((sin(x+h)*cos(x+h))-sin(x)*cos(x))/h,h,0)) // turunan sin(x)*cos(x)
>function df(x) &= limit(((sin(x+h)*cos(x+h))-sin(x)*cos(x))/h,h,0);  $df(x)// df(x) = f'(x)
>plot2d(["f(x)","df(x)"],-pi,pi,color=[blue,red]), label("f(x)",1,0), label("df(x)",2.3,1.2):
\end{eulerprompt}
\begin{eulercomment}
3. Fungsi 3
\end{eulercomment}
\begin{eulerprompt}
>function f(x) := sqrt(x)*4
>$showev('limit((sqrt(x+h)*4-sqrt(x)*4)/h,h,0)) // turunan sqrt(x)*4
>function df(x) &= limit((sqrt(x+h)*4-sqrt(x)*4)/h,h,0);  $df(x)// df(x) = f'(x)
>plot2d(["f(x)","df(x)"],-pi,pi,color=[blue,red]), label("f(x)",-2,11), label("df(x)",-2,-10):
\end{eulerprompt}
\begin{eulercomment}
4. Fungsi 4
\end{eulercomment}
\begin{eulerprompt}
>function f(x) := cos(1/x)
>$showev('limit((cos(1/(x+h))-cos(1/x))/h,h,0)) // turunan cos(1/x)
>function df(x) &= limit((cos(1/(x+h))-cos(1/x))/h,h,0);  $df(x)// df(x) = f'(x)
>plot2d(["f(x)","df(x)"],-pi,pi,color=[blue,red]), label("f(x)",2,0.4), label("df(x)",1,-0.5):
\end{eulerprompt}
\begin{eulercomment}
5. Fungsi 5
\end{eulercomment}
\begin{eulerprompt}
>function f(x) := (log(x))^5
>$showev('limit(((log(x+h))^5-(log(x))^5)/h,h,0)) // turunan (log(x))^5
>function df(x) &= limit(((log(x+h))^5-(log(x))^5)/h,h,0);  $df(x)// df(x) = f'(x)
>plot2d(["f(x)","df(x)"],-50,100,-10,50,color=[blue,red]), label("f(x)",25,35), label("df(x)",50,1):
\end{eulerprompt}
\begin{eulercomment}
6. Fungsi 6
\end{eulercomment}
\begin{eulerprompt}
>function f(x) := sqrt(tan(x))
>$showev('limit((sqrt(tan(x+h))-sqrt(tan(x)))/h,h,0)) // turunan exp(x)*cos(x)
>function df(x) &= limit((sqrt(tan(x+h))-sqrt(tan(x)))/h,h,0);  $df(x)// df(x) = f'(x)
>plot2d(["f(x)","df(x)"],-10,10,-10,10,color=[blue,red]), label("f(x)",4.5,0), label("df(x)",5.5,5):
\end{eulerprompt}
\eulerheading{Integral}
\begin{eulercomment}
EMT dapat digunakan untuk menghitung integral, baik integral tak tentu
maupun integral tentu. Untuk integral tak tentu (simbolik) sudah tentu
EMT menggunakan Maxima, sedangkan untuk perhitungan integral tentu EMT
sudah menyediakan beberapa fungsi yang mengimplementasikan algoritma
kuadratur (perhitungan integral tentu menggunakan metode numerik).

Pada notebook ini akan ditunjukkan perhitungan integral tentu dengan
menggunakan Teorema Dasar Kalkulus:

\end{eulercomment}
\begin{eulerformula}
\[
\int_a^b f(x)\ dx = F(b)-F(a), \quad \text{ dengan  } F'(x) = f(x).
\]
\end{eulerformula}
\begin{eulercomment}
Fungsi untuk menentukan integral adalah integrate. Fungsi ini dapat
digunakan untuk menentukan, baik integral tentu maupun tak tentu (jika
fungsinya memiliki antiderivatif). Untuk perhitungan integral tentu
fungsi integrate menggunakan metode numerik (kecuali fungsinya tidak
integrabel, kita tidak akan menggunakan metode ini).
\end{eulercomment}
\begin{eulerprompt}
>$showev('integrate(x^n,x))
\end{eulerprompt}
\begin{euleroutput}
  Answering "Is n equal to -1?" with "no"
\end{euleroutput}
\begin{eulerprompt}
>$showev('integrate(1/(1+x),x))
>$showev('integrate(1/(1+x^2),x))
>$showev('integrate(1/sqrt(1-x^2),x))
>$showev('integrate(sin(x),x,0,pi))
>$showev('integrate(sin(x),x,a,b))
>$showev('integrate(x^n,x,a,b))
\end{eulerprompt}
\begin{euleroutput}
  Answering "Is n positive, negative or zero?" with "positive"
\end{euleroutput}
\begin{eulerprompt}
>$showev('integrate(x^2*sqrt(2*x+1),x))
>$showev('integrate(x^2*sqrt(2*x+1),x,0,2))
>$ratsimp(%)
>$showev('integrate((sin(sqrt(x)+a)*E^sqrt(x))/sqrt(x),x,0,pi^2))
>$factor(%)
>function map f(x) &= E^(-x^2); $f(x)
>$showev('integrate(f(x),x))
\end{eulerprompt}
\begin{eulercomment}
Fungsi f tidak memiliki antiturunan, integralnya masih memuat integral
lain.

\end{eulercomment}
\begin{eulerformula}
\[
erf(x) = \int \frac{e^{-x^2}}{\sqrt{\pi}} \ dx.
\]
\end{eulerformula}
\begin{eulercomment}
Kita tidak dapat menggunakan teorema Dasar kalkulus untuk menghitung
integral tentu fungsi tersebut jika semua batasnya berhingga. Dalam
hal ini dapat digunakan metode numerik (rumus kuadratur).

Misalkan kita akan menghitung:

maxima: 'integrate(f(x),x,0,pi)
\end{eulercomment}
\begin{eulerprompt}
>x=0:0.1:pi-0.1; plot2d(x,f(x+0.1),>bar); plot2d("f(x)",0,pi,>add):
\end{eulerprompt}
\begin{eulercomment}
Integral tentu

maxima: 'integrate(f(x),x,0,pi)

dapat dihampiri dengan jumlah luas persegi-persegi panjang di bawah
kurva y=f(x) tersebut. Langkah-langkahnya adalah sebagai berikut.
\end{eulercomment}
\begin{eulerprompt}
>t &= makelist(a,a,0,pi-0.1,0.1); // t sebagai list untuk menyimpan nilai-nilai x
>fx &= makelist(f(t[i]+0.1),i,1,length(t)); // simpan nilai-nilai f(x)
>// jangan menggunakan x sebagai list, kecuali Anda pakar Maxima!
\end{eulerprompt}
\begin{eulercomment}
Hasilnya adalah:

maxima: 'integrate(f(x),x,0,pi) = 0.1*sum(fx[i],i,1,length(fx))

Jumlah tersebut diperoleh dari hasil kali lebar sub-subinterval (=0.1)
dan jumlah nilai-nilai f(x) untuk x = 0.1, 0.2, 0.3, ..., 3.2.
\end{eulercomment}
\begin{eulerprompt}
>0.1*sum(f(x+0.1)) // cek langsung dengan perhitungan numerik EMT
\end{eulerprompt}
\begin{euleroutput}
  0.836219610253
\end{euleroutput}
\begin{eulercomment}
Untuk mendapatkan nilai integral tentu yang mendekati nilai sebenarnya, lebar
sub-intervalnya dapat diperkecil lagi, sehingga daerah di bawah kurva tertutup
semuanya, misalnya dapat digunakan lebar subinterval 0.001. (Silakan dicoba!)

Meskipun Maxima tidak dapat menghitung integral tentu fungsi tersebut untuk
batas-batas yang berhingga, namun integral tersebut dapat dihitung secara eksak jika
batas-batasnya tak hingga. Ini adalah salah satu keajaiban di dalam matematika, yang
terbatas tidak dapat dihitung secara eksak, namun yang tak hingga malah dapat
dihitung secara eksak.
\end{eulercomment}
\begin{eulerprompt}
>$showev('integrate(f(x),x,0,inf))
\end{eulerprompt}
\begin{eulercomment}
Berikut adalah contoh lain fungsi yang tidak memiliki antiderivatif, sehingga
integral tentunya hanya dapat dihitung dengan metode numerik.
\end{eulercomment}
\begin{eulerprompt}
>function f(x) &= x^x; $f(x)
>$showev('integrate(f(x),x,0,1))
>x=0:0.1:1-0.01; plot2d(x,f(x+0.01),>bar); plot2d("f(x)",0,1,>add):
\end{eulerprompt}
\begin{eulercomment}
Maxima gagal menghitung integral tentu tersebut secara langsung menggunakan perintah
integrate. Berikut kita lakukan seperti contoh sebelumnya untuk mendapat hasil atau
pendekatan nilai integral tentu tersebut.
\end{eulercomment}
\begin{eulerprompt}
>t &= makelist(a,a,0,1-0.01,0.01);
>fx &= makelist(f(t[i]+0.01),i,1,length(t));
\end{eulerprompt}
\eulerheading{Latihan}
\begin{eulercomment}
- Bukalah buku Kalkulus.\\
- Cari dan pilih beberapa (paling sedikit 5 fungsi berbeda
tipe/bentuk/jenis) fungsi dari buku tersebut, kemudian definisikan di
EMT pada baris-baris perintah berikut (jika perlu tambahkan lagi).\\
- Untuk setiap fungsi, tentukan anti turunannya (jika ada), hitunglah
integral tentu dengan batas-batas yang menarik (Anda tentukan
sendiri), seperti contoh-contoh tersebut.\\
- Lakukan hal yang sama untuk fungsi-fungsi yang tidak dapat
diintegralkan (cari sedikitnya 3 fungsi).\\
- Gambar grafik fungsi dan daerah integrasinya pada sumbu koordinat
yang sama.\\
- Gunakan integral tentu untuk mencari luas daerah yang dibatasi oleh
dua kurva yang berpotongan di dua titik. (Cari dan gambar kedua kurva
dan arsir (warnai) daerah yang dibatasi oleh keduanya.)\\
- Gunakan integral tentu untuk menghitung volume benda putar kurva y=
f(x) yang diputar mengelilingi sumbu x dari x=a sampai x=b, yakni

\end{eulercomment}
\begin{eulerformula}
\[
V = \int_a^b \pi (f(x))^2\ dx.
\]
\end{eulerformula}
\begin{eulercomment}
(Pilih fungsinya dan gambar kurva dan benda putar yang dihasilkan.
Anda dapat mencari contoh-contoh bagaimana cara menggambar benda hasil
perputaran suatu kurva.)\\
- Gunakan integral tentu untuk menghitung panjang kurva y=f(x) dari
x=a sampai x=b dengan menggunakan rumus:

\end{eulercomment}
\begin{eulerformula}
\[
S = \int_a^b \sqrt{1+(f'(x))^2} \ dx.
\]
\end{eulerformula}
\begin{eulercomment}
(Pilih fungsi dan gambar kurvanya.)

Jawab:\\
1. Fungsi 1
\end{eulercomment}
\begin{eulerprompt}
>function f(x) &= 5*x^2; $f(x)
>$showev('integrate(f(x),x))
>$showev('integrate(f(x),x,2,3))
>x=0.01:0.03:4; plot2d(x,f(x+0.01),>bar); plot2d("f(x)",2,3,>add):
\end{eulerprompt}
\begin{eulercomment}
2. Fungsi 2
\end{eulercomment}
\begin{eulerprompt}
>function f(x) &= cos(2*x+5); $f(x)
>$showev('integrate(f(x),x))
>$showev('integrate(f(x),x,pi,2*pi))
>x=0:0.05:pi-0.1; plot2d(x,f(x+0.03),>bar); plot2d("f(x)",pi,2*pi,>add):
\end{eulerprompt}
\begin{eulercomment}
3. Fungsi 3
\end{eulercomment}
\begin{eulerprompt}
>function f(x) &= (sin(x))*(cos((x)))^2; $f(x)
>$showev('integrate(f(x),x))
>$showev('integrate(f(x),x,0,pi))
>x=-pi:0.04:pi; plot2d(x,f(x+0.01),>bar); plot2d("f(x)",0,pi,>add):
\end{eulerprompt}
\begin{eulercomment}
4. Fungsi 4
\end{eulercomment}
\begin{eulerprompt}
>function f(x) &= (x^2*(2-x^3)^(1/2)); $f(x)
>$showev('integrate(f(x),x))
>$showev('integrate(f(x),x,0,1))
>x=-1:0.04:1; plot2d(x,f(x+0.01),>bar); plot2d("f(x)",0,1,>add):
\end{eulerprompt}
\begin{eulercomment}
5. Fungsi 5
\end{eulercomment}
\begin{eulerprompt}
>function f(x) &= sqrt(24-x^2); $f(x)
>$showev('integrate(f(x),x))
>$showev('integrate(f(x),x,1,2))
>x=-2:0.04:1; plot2d(x,f(x+0.01),>bar); plot2d("f(x)",1,2,>add):
\end{eulerprompt}
\begin{eulercomment}
6. Fungsi 6
\end{eulercomment}
\begin{eulerprompt}
>t &= makelist(a,a,0,1-0.01,0.01);
>fx &= makelist(f(t[i]+0.01),i,1,length(t));
>function f(x) &= x^2+50; $f(x)
>x=0:0.1:pi-0.01; plot2d(x,f(x+0.01),>bar); plot2d("f(x)",0,pi,>add):
>0.01*sum(f(x+0.01))
\end{eulerprompt}
\begin{euleroutput}
  17.051552
\end{euleroutput}
\begin{eulercomment}
7. Fungsi 7
\end{eulercomment}
\begin{eulerprompt}
>t &= makelist(a,a,0,1-0.01,0.01);
>fx &= makelist(f(t[i]+0.01),i,1,length(t));
>function f(x) &= cos(x)/x; $f(x)
>x=-pi:0.07:pi-0.01; plot2d(x,f(x+0.01),>bar); plot2d("f(x)",0,pi,>add):
>0.01*sum(f(x+0.01))
\end{eulerprompt}
\begin{euleroutput}
  0.415163991256
\end{euleroutput}
\begin{eulercomment}
8. Fungsi 8
\end{eulercomment}
\begin{eulerprompt}
>t &= makelist(a,a,0,1-0.01,0.01);
>fx &= makelist(f(t[i]+0.01),i,1,length(t));
>function f(x) &= sqrt(x^2-1); $f(x)
>x=3:0.04:pi-0.01; plot2d(x,f(x+0.01),>bar); plot2d("f(x)",0,2,>add):
>0.01*sum(f(x+0.01))
\end{eulerprompt}
\begin{euleroutput}
  0.11610107668
\end{euleroutput}
\eulersubheading{Luas daerah dibatasi 2 kurva}
\begin{eulercomment}
1). Fungsi 1
\end{eulercomment}
\begin{eulerprompt}
>function f(x) &= x^3; $f(x)
>function g(x) &= x; $g(x)
>plot2d(["x^4","x^3"],-2,2,-1,2):
>function h(x) &= f(x)-g(x); $h(x)
>$showev('integrate(h(x),x))
>$&solve(f(x)=g(x))
>$showev('integrate(h(x),x,0,1)) // menghitung luas daerah yang dibatasi 2 kurva
\end{eulerprompt}
\begin{eulercomment}
\end{eulercomment}
\begin{eulerformula}
\[
\text{Arsiran daerah yang dibatasi kurva $f(x)$ dan $g(x)$ sebagai berikut:}
\]
\end{eulerformula}
\begin{eulerprompt}
>x=-1:0.01:1; plot2d(x,f(x),>bar,>filled,style="-",fillcolor=orange,>grid); plot2d(x,g(x),>bar,>add,>filled,style="-",fillcolor=white); label("f(x)",0,2.1); label("g(x)",0.5,0.3):
\end{eulerprompt}
\begin{eulercomment}
2). Fungsi 2
\end{eulercomment}
\begin{eulerprompt}
>function f(x) &= x^3+1; $f(x)
>function g(x) &= x^2; $g(x)
>plot2d(["-x^2+2","x^2"],-2,2,-1,2):
>function h(x) &= f(x)-g(x); $h(x)
>$&solve(f(x)=g(x))
>$showev('integrate(h(x),x,-1,1)) // menghitung luas daerah yang dibatasi 2 kurva
\end{eulerprompt}
\begin{eulercomment}
\end{eulercomment}
\begin{eulerformula}
\[
\text{Arsiran daerah yang dibatasi kurva $f(x)$ dan $g(x)$ sebagai berikut:}
\]
\end{eulerformula}
\begin{eulerprompt}
>x=-1:0.01:1; plot2d(x,f(x),>bar,>filled,style="-",fillcolor=orange,>grid); plot2d(x,g(x),>bar,>add,>filled,style="-",fillcolor=white); label("f(x)",0,2.1); label("g(x)",0.5,0.3):
\end{eulerprompt}
\eulersubheading{Volume benda putar}
\begin{eulercomment}
Menghitung volume hasil perputaran kurva\\
\end{eulercomment}
\begin{eulerformula}
\[
m(x)=x^3+1
\]
\end{eulerformula}
\begin{eulercomment}
dari x=-1 sampai x=0. Diputar terhadap sumbu-x.\\
Jawab:
\end{eulercomment}
\begin{eulerprompt}
>function m(x) &= x^4+3; $m(x)
>$showev('integrate(pi*(m(x))^2,x,-1,0)) // Menghitung volume hasil perputaran m(x)
\end{eulerprompt}
\begin{eulercomment}
Daerah di bawah kurva yang akan dirotasi terhadap sumbu x sebagai
berikut:
\end{eulercomment}
\begin{eulerprompt}
>plot2d("m(x)",-1,0,-1,2,grid=7,>filled, style="/\(\backslash\)"): 
\end{eulerprompt}
\begin{eulercomment}
Hasil perputaran m(x) terhadap sumbu x sebagai berikut:
\end{eulercomment}
\begin{eulerprompt}
>plot3d("m(x)",-1,0,-1,1,>rotate,angle=6.3,>hue,>contour,color=redgreen,height=11):
\end{eulerprompt}
\begin{eulercomment}
\end{eulercomment}
\eulersubheading{Menghitung panjang kurva}
\begin{eulercomment}
Menghitung panjang kurva\\
\end{eulercomment}
\begin{eulerformula}
\[
\text{$y=x^2-x+1$}
\]
\end{eulerformula}
\begin{eulercomment}
dari x=1 sampai x=3.
\end{eulercomment}
\begin{eulerprompt}
>function d(x) &= x^2-x+1; $d(x)
>plot2d("d(x)",-5,6): // gambar kurva d(x)
>$showev('limit((d(x+h)-d(x))/h,h,0))
>function dd(x) &= limit((d(x+h)-d(x))/h,h,0); $dd(x)
>function q(x) &= ((dd(x))^2); $q(x)
>$showev('integrate(sqrt(1+q(x)),x,1,3)) // menghitung panjang kurva
\end{eulerprompt}
\begin{eulercomment}
Jadi, panjang kurva\\
\end{eulercomment}
\begin{eulerformula}
\[
\text{$y=x^2-x+1$}
\]
\end{eulerformula}
\begin{eulercomment}
dari x=0 sampai x=4 adalah\\
\end{eulercomment}
\begin{eulerformula}
\[
\text{$S=\frac{asinh 5+5sqrt(26)}{4}-\frac{asinh(1)+sqrt(2)}{4}$}.
\]
\end{eulerformula}
\begin{eulercomment}
\begin{eulercomment}
\eulerheading{Barisan dan Deret}
\begin{eulercomment}
(Catatan: bagian ini belum lengkap. Anda dapat membaca contoh-contoh
pengguanaan EMT dan Maxima untuk menghitung limit barisan, rumus
jumlah parsial suatu deret, jumlah tak hingga suatu deret konvergen,
dan sebagainya. Anda dapat mengeksplor contoh-contoh di EMT atau
perbagai panduan penggunaan Maxima di software Maxima atau dari
Internet.)

Barisan dapat didefinisikan dengan beberapa cara di dalam EMT, di
antaranya:

- dengan cara yang sama seperti mendefinisikan vektor dengan
elemen-elemen beraturan (menggunakan titik dua ":");\\
- menggunakan perintah "sequence" dan rumus barisan (suku ke -n);\\
- menggunakan perintah "iterate" atau "niterate";\\
- menggunakan fungsi Maxima "create\_list" atau "makelist" untuk
menghasilkan barisan simbolik;\\
- menggunakan fungsi biasa yang inputnya vektor atau barisan;\\
- menggunakan fungsi rekursif.

EMT menyediakan beberapa perintah (fungsi) terkait barisan, yakni:

- sum: menghitung jumlah semua elemen suatu barisan\\
- cumsum: jumlah kumulatif suatu barisan\\
- differences: selisih antar elemen-elemen berturutan

EMT juga dapat digunakan untuk menghitung jumlah deret berhingga
maupun deret tak hingga, dengan menggunakan perintah (fungsi) "sum".
Perhitungan dapat dilakukan secara numerik maupun simbolik dan eksak.

Berikut adalah beberapa contoh perhitungan barisan dan deret
menggunakan EMT.
\end{eulercomment}
\begin{eulerprompt}
>1:10 // barisan sederhana
\end{eulerprompt}
\begin{euleroutput}
  [1,  2,  3,  4,  5,  6,  7,  8,  9,  10]
\end{euleroutput}
\begin{eulerprompt}
>1:2:30
\end{eulerprompt}
\begin{euleroutput}
  [1,  3,  5,  7,  9,  11,  13,  15,  17,  19,  21,  23,  25,  27,  29]
\end{euleroutput}
\begin{eulerprompt}
>sum(1:2:30), sum(1/(1:2:30))
\end{eulerprompt}
\begin{euleroutput}
  225
  2.33587263431
\end{euleroutput}
\begin{eulerprompt}
>$'sum(k, k, 1, n) = factor(ev(sum(k, k, 1, n),simpsum=true)) // simpsum:menghitung deret secara simbolik
>$'sum(1/(3^k+k), k, 0, inf) = factor(ev(sum(1/(3^k+k), k, 0, inf),simpsum=true))
\end{eulerprompt}
\begin{eulercomment}
Di sini masih gagal, hasilnya tidak dihitung.
\end{eulercomment}
\begin{eulerprompt}
>$'sum(1/x^2, x, 1, inf)= ev(sum(1/x^2, x, 1, inf),simpsum=true) // ev: menghitung nilai ekspresi
>$'sum((-1)^(k-1)/k, k, 1, inf) = factor(ev(sum((-1)^(x-1)/x, x, 1, inf),simpsum=true))
\end{eulerprompt}
\begin{eulercomment}
Di sini masih gagal, hasilnya tidak dihitung.
\end{eulercomment}
\begin{eulerprompt}
>$'sum((-1)^k/(2*k-1), k, 1, inf) = factor(ev(sum((-1)^k/(2*k-1), k, 1, inf),simpsum=true))
>$ev(sum(1/n!, n, 0, inf),simpsum=true)
\end{eulerprompt}
\begin{eulercomment}
Di sini masih gagal, hasilnya tidak dihitung, harusnya hasilnya e.
\end{eulercomment}
\begin{eulerprompt}
>&assume(abs(x)<1); $'sum(a*x^k, k, 0, inf)=ev(sum(a*x^k, k, 0, inf),simpsum=true), &forget(abs(x)<1);
\end{eulerprompt}
\begin{eulercomment}
Deret geometri tak hingga, dengan asumsi rasional antara -1 dan 1.
\end{eulercomment}
\eulerheading{Deret Taylor}
\begin{eulercomment}
Deret Taylor suatu fungsi f yang diferensiabel sampai tak hingga di
sekitar x=a adalah:

\end{eulercomment}
\begin{eulerformula}
\[
f(x) = \sum_{k=0}^\infty \frac{(x-a)^k f^{(k)}(a)}{k!}.
\]
\end{eulerformula}
\begin{eulerprompt}
>$'e^x =taylor(exp(x),x,0,10) // deret Taylor e^x di sekitar x=0, sampai suku ke-11
>$'log(x)=taylor(log(x),x,1,10)// deret log(x) di sekitar x=1
\end{eulerprompt}
\end{eulernotebook}
\end{document}


\newpage
\chapter{KB Pekan 8: Menggunakan EMT untuk Geometri}
\documentclass[a4paper,10pt]{article}
\usepackage{eumat}

\begin{document}
\begin{eulernotebook}
\eulersubheading{EMT geometri}
\begin{eulercomment}
Rasyid Shalahuddin\\
22305144016\\
Matematika E

\end{eulercomment}
\eulersubheading{Visualisasi dan Perhitungan Geometri dengan EMT}
\begin{eulercomment}
Euler menyediakan beberapa fungsi untuk melakukan visualisasi dan
perhitungan geometri, baik secara numerik maupun analitik (seperti
biasanya tentunya, menggunakan Maxima). Fungsi-fungsi untuk
visualisasi dan perhitungan geometeri tersebut disimpan di dalam file
program "geometry.e", sehingga file tersebut harus dipanggil sebelum
menggunakan fungsi-fungsi atau perintah-perintah untuk geometri.
\end{eulercomment}
\begin{eulerprompt}
>load geometry
\end{eulerprompt}
\begin{euleroutput}
  Numerical and symbolic geometry.
\end{euleroutput}
\eulersubheading{Fungsi-fungsi Geometri}
\begin{eulercomment}
Fungsi-fungsi untuk Menggambar Objek Geometri antara lain:

\end{eulercomment}
\begin{eulerttcomment}
  defaultd:=textheight()*1.5: nilai asli untuk parameter d
\end{eulerttcomment}
\begin{eulercomment}

\end{eulercomment}
\begin{eulerttcomment}
  setPlotrange(x1,x2,y1,y2): menentukan rentang x dan y pada bidang
\end{eulerttcomment}
\begin{eulercomment}
koordinat

\end{eulercomment}
\begin{eulerttcomment}
  setPlotRange(r): pusat bidang koordinat (0,0) dan batas-batas
\end{eulerttcomment}
\begin{eulercomment}
sumbu-x dan y adalah -r sd r\\
\end{eulercomment}
\begin{eulerttcomment}
  plotPoint (P, "P"): menggambar titik P dan diberi label "P"
\end{eulerttcomment}
\begin{eulercomment}

\end{eulercomment}
\begin{eulerttcomment}
  plotSegment (A,B, "AB", d): menggambar ruas garis AB, diberi label
\end{eulerttcomment}
\begin{eulercomment}
"AB" sejauh d

\end{eulercomment}
\begin{eulerttcomment}
  plotLine (g, "g", d): menggambar garis g diberi label "g" sejauh d
\end{eulerttcomment}
\begin{eulercomment}

\end{eulercomment}
\begin{eulerttcomment}
  plotCircle (c,"c",v,d): Menggambar lingkaran c dan diberi label "c"
\end{eulerttcomment}
\begin{eulercomment}

\end{eulercomment}
\begin{eulerttcomment}
  plotLabel (label, P, V, d): menuliskan label pada posisi P
\end{eulerttcomment}
\begin{eulercomment}

Fungsi-fungsi Geometri Analitik (numerik maupun simbolik):

\end{eulercomment}
\begin{eulerttcomment}
  turn(v, phi): memutar vektor v sejauh phi
  turnLeft(v):   memutar vektor v ke kiri
  turnRight(v):  memutar vektor v ke kanan
  normalize(v): normal vektor v
  crossProduct(v, w): hasil kali silang vektorv dan w.
  lineThrough(A, B): garis melalui A dan B, hasilnya [a,b,c] sdh.
\end{eulerttcomment}
\begin{eulercomment}
ax+by=c.\\
\end{eulercomment}
\begin{eulerttcomment}
  lineWithDirection(A,v): garis melalui A searah vektor v
  getLineDirection(g): vektor arah (gradien) garis g
  getNormal(g): vektor normal (tegak lurus) garis g
  getPointOnLine(g):  titik pada garis g
  perpendicular(A, g):  garis melalui A tegak lurus garis g
  parallel (A, g):  garis melalui A sejajar garis g
  lineIntersection(g, h):  titik potong garis g dan h
  projectToLine(A, g):   proyeksi titik A pada garis g
  distance(A, B):  jarak titik A dan B
  distanceSquared(A, B):  kuadrat jarak A dan B
  quadrance(A, B): kuadrat jarak A dan B
  areaTriangle(A, B, C):  luas segitiga ABC
  computeAngle(A, B, C):   besar sudut <ABC
  angleBisector(A, B, C): garis bagi sudut <ABC
  circleWithCenter (A, r): lingkaran dengan pusat A dan jari-jari r
  getCircleCenter(c):  pusat lingkaran c
  getCircleRadius(c):  jari-jari lingkaran c
  circleThrough(A,B,C):  lingkaran melalui A, B, C
  middlePerpendicular(A, B): titik tengah AB
  lineCircleIntersections(g, c): titik potong garis g dan lingkran c
  circleCircleIntersections (c1, c2):  titik potong lingkaran c1 dan
\end{eulerttcomment}
\begin{eulercomment}
c2\\
\end{eulercomment}
\begin{eulerttcomment}
  planeThrough(A, B, C):  bidang melalui titik A, B, C
\end{eulerttcomment}
\begin{eulercomment}

Fungsi-fungsi Khusus Untuk Geometri Simbolik:

\end{eulercomment}
\begin{eulerttcomment}
  getLineEquation (g,x,y): persamaan garis g dinyatakan dalam x dan y
  getHesseForm (g,x,y,A): bentuk Hesse garis g dinyatakan dalam x dan
\end{eulerttcomment}
\begin{eulercomment}
y dengan titik A pada\\
\end{eulercomment}
\begin{eulerttcomment}
  sisi positif (kanan/atas) garis
  quad(A,B): kuadrat jarak AB
  spread(a,b,c): Spread segitiga dengan panjang sisi-sisi a,b,c, yakni
\end{eulerttcomment}
\begin{eulercomment}
sin(alpha)\textasciicircum{}2 dengan\\
\end{eulercomment}
\begin{eulerttcomment}
  alpha sudut yang menghadap sisi a.
  crosslaw(a,b,c,sa): persamaan 3 quads dan 1 spread pada segitiga
\end{eulerttcomment}
\begin{eulercomment}
dengan panjang sisi a, b, c.\\
\end{eulercomment}
\begin{eulerttcomment}
  triplespread(sa,sb,sc): persamaan 3 spread sa,sb,sc yang memebntuk
\end{eulerttcomment}
\begin{eulercomment}
suatu segitiga\\
\end{eulercomment}
\begin{eulerttcomment}
  doublespread(sa): Spread sudut rangkap Spread 2*phi, dengan
\end{eulerttcomment}
\begin{eulercomment}
sa=sin(phi)\textasciicircum{}2 spread a.

\end{eulercomment}
\eulersubheading{Contoh 1: Luas, Lingkaran Luar, Lingkaran Dalam Segitiga}
\begin{eulercomment}
Untuk menggambar objek-objek geometri, langkah pertama adalah
menentukan rentang sumbu-sumbu koordinat. Semua objek geometri akan
digambar pada satu bidang koordinat, sampai didefinisikan bidang
koordinat yang baru.
\end{eulercomment}
\begin{eulerprompt}
>setPlotRange(-0.5,2.5,-0.5,2.5); // mendefinisikan bidang koordinat baru 
\end{eulerprompt}
\begin{eulercomment}
Sekarang tetapkan tiga poin dan plot mereka.
\end{eulercomment}
\begin{eulerprompt}
>A=[1,0]; plotPoint(A,"A"); // definisi dan gambar tiga titik
>B=[-0.5,1.5]; plotPoint(B,"B");
>C=[2.5,1.5]; plotPoint(C,"C");
\end{eulerprompt}
\begin{eulercomment}
Kemudian tiga segmen.
\end{eulercomment}
\begin{eulerprompt}
>plotSegment(A,B,"c"); // c=AB
>plotSegment(B,C,"a"); // a=BC
>plotSegment(A,C,"b"); // b=AC
\end{eulerprompt}
\begin{eulercomment}
Fungsi geometri meliputi fungsi untuk membuat garis dan lingkaran.
Format garis adalah [a,b,c], yang mewakili garis dengan persamaan
ax+by=c.
\end{eulercomment}
\begin{eulerprompt}
>lineThrough(B,C) // garis yang melalui B dan C
\end{eulerprompt}
\begin{euleroutput}
  [0,  3,  4.5]
\end{euleroutput}
\begin{eulercomment}
Hitunglah garis tegak lurus yang melalui A pada BC.
\end{eulercomment}
\begin{eulerprompt}
>h=perpendicular(A,lineThrough(B,C)); // garis h tegak lurus BC melalui A
\end{eulerprompt}
\begin{eulercomment}
Dan persimpangannya dengan BC.
\end{eulercomment}
\begin{eulerprompt}
>D=lineIntersection(h,lineThrough(B,C)); // D adalah titik potong h dan BC
\end{eulerprompt}
\begin{eulercomment}
Plot itu.
\end{eulercomment}
\begin{eulerprompt}
>plotPoint(D,value=1); // koordinat D ditampilkan
>aspect(1); plotSegment(A,D): // tampilkan semua gambar hasil plot...()
\end{eulerprompt}
\eulerimg{29}{images/Rasyid Shalahuddin_22305144016_EMT geometri-001.png}
\begin{eulercomment}
Hitung luas ABC:

\end{eulercomment}
\begin{eulerformula}
\[
L_{\triangle ABC}= \frac{1}{2}AD.BC.
\]
\end{eulerformula}
\begin{eulerprompt}
>norm(A-D)*norm(B-C)/2 // AD=norm(A-D), BC=norm(B-C)
\end{eulerprompt}
\begin{euleroutput}
  2.25
\end{euleroutput}
\begin{eulercomment}
Bandingkan dengan rumus determinan.
\end{eulercomment}
\begin{eulerprompt}
>areaTriangle(A,B,C) // hitung luas segitiga langusng dengan fungsi
\end{eulerprompt}
\begin{euleroutput}
  2.25
\end{euleroutput}
\begin{eulercomment}
Cara lain menghitung luas segitigas ABC:
\end{eulercomment}
\begin{eulerprompt}
>distance(A,D)*distance(B,C)/2
\end{eulerprompt}
\begin{euleroutput}
  2.25
\end{euleroutput}
\begin{eulercomment}
Sudut di C
\end{eulercomment}
\begin{eulerprompt}
>degprint(computeAngle(B,C,A))
\end{eulerprompt}
\begin{euleroutput}
  45°
\end{euleroutput}
\begin{eulercomment}
Sekarang lingkaran luar segitiga.
\end{eulercomment}
\begin{eulerprompt}
>c=circleThrough(A,B,C); // lingkaran luar segitiga ABC
>R=getCircleRadius(c); // jari2 lingkaran luar 
>O=getCircleCenter(c); // titik pusat lingkaran c 
>plotPoint(O,"O"); // gambar titik "O"
>plotCircle(c,"Lingkaran luar segitiga ABC"):
\end{eulerprompt}
\eulerimg{29}{images/Rasyid Shalahuddin_22305144016_EMT geometri-002.png}
\begin{eulercomment}
Tampilkan koordinat titik pusat dan jari-jari lingkaran luar.
\end{eulercomment}
\begin{eulerprompt}
>O, R
\end{eulerprompt}
\begin{euleroutput}
  [1,  1.5]
  1.5
\end{euleroutput}
\begin{eulercomment}
Sekarang akan digambar lingkaran dalam segitiga ABC. Titik pusat lingkaran dalam adalah
titik potong garis-garis bagi sudut.
\end{eulercomment}
\begin{eulerprompt}
>l=angleBisector(A,C,B); // garis bagi <ACB
>g=angleBisector(C,A,B); // garis bagi <CAB
>P=lineIntersection(l,g) // titik potong kedua garis bagi sudut
\end{eulerprompt}
\begin{euleroutput}
  [1,  0.87868]
\end{euleroutput}
\begin{eulercomment}
Tambahkan semuanya ke plot.
\end{eulercomment}
\begin{eulerprompt}
>color(5); plotLine(l); plotLine(g); color(1); // gambar kedua garis bagi sudut
>plotPoint(P,"P"); // gambar titik potongnya
>r=norm(P-projectToLine(P,lineThrough(A,B))) // jari-jari lingkaran dalam
\end{eulerprompt}
\begin{euleroutput}
  0.62132034356
\end{euleroutput}
\begin{eulerprompt}
>plotCircle(circleWithCenter(P,r),"Lingkaran dalam segitiga ABC"): // gambar lingkaran dalam
\end{eulerprompt}
\eulerimg{29}{images/Rasyid Shalahuddin_22305144016_EMT geometri-003.png}
\eulersubheading{Latihan}
\begin{eulercomment}
1. Tentukan Luas segitiga ABC.
\end{eulercomment}
\begin{eulerprompt}
>setPlotRange(-2.5,4.5,-2.5,4.5);
>A=[-2.5,1]; plotPoint(A,"A");
>B=[2,-2]; plotPoint(B,"B");
>C=[4,4]; plotPoint(C,"C");
>norm(A-D)*norm(B-C)/2 // AD=norm(A-D), BC=norm(B-C)
\end{eulerprompt}
\begin{euleroutput}
  11.1803398875
\end{euleroutput}
\begin{eulercomment}
2. Gambar segitiga dengan titik-titik sudut ketiga titik singgung
tersebut.
\end{eulercomment}
\begin{eulerprompt}
>plotSegment(A,B,"c")
>plotSegment(B,C,"a")
>plotSegment(A,C,"b")
>aspect(1):
\end{eulerprompt}
\eulerimg{29}{images/Rasyid Shalahuddin_22305144016_EMT geometri-004.png}
\begin{eulercomment}
3. Tunjukkan bahwa garis bagi sudut yang ke tiga juga melalui titik
pusat lingkaran dalam.
\end{eulercomment}
\begin{eulerprompt}
>l=angleBisector(A,C,B);
>g=angleBisector(C,A,B);
>P=lineIntersection(l,g)
\end{eulerprompt}
\begin{euleroutput}
  [0.581139,  0.581139]
\end{euleroutput}
\begin{eulerprompt}
>color(5); plotLine(l); plotLine(g); color(1);
>plotPoint(P,"P");
>r=norm(P-projectToLine(P,lineThrough(A,B)))
\end{eulerprompt}
\begin{euleroutput}
  1.52896119631
\end{euleroutput}
\begin{eulerprompt}
>plotCircle(circleWithCenter(P,r),"Lingkaran dalam segitiga ABC"):
\end{eulerprompt}
\eulerimg{29}{images/Rasyid Shalahuddin_22305144016_EMT geometri-005.png}
\begin{eulercomment}
Jadi, terbukti bahwa garis bagi sudut yang ketiga juga melalui titik
pusat lingkaran dalam.

4. Gambar jari-jari lingkaran dalam.
\end{eulercomment}
\begin{eulerprompt}
>r=norm(P-projectToLine(P,lineThrough(A,B)))
\end{eulerprompt}
\begin{euleroutput}
  1.52896119631
\end{euleroutput}
\begin{eulerprompt}
>plotCircle(circleWithCenter(P,r),"Lingkaran dalam segitiga ABC"):
\end{eulerprompt}
\eulerimg{29}{images/Rasyid Shalahuddin_22305144016_EMT geometri-006.png}
\eulersubheading{Contoh 2: Geometri Simbolik}
\begin{eulercomment}
Kita dapat menghitung geometri eksak dan simbolik menggunakan Maxima.

File geometri.e menyediakan fungsi yang sama (dan lebih banyak lagi)
di Maxima. Namun, kita dapat menggunakan perhitungan simbolis
sekarang.
\end{eulercomment}
\begin{eulerprompt}
>A &= [1,0]; B &= [0,1]; C &= [2,2]; // menentukan tiga titik A, B, C
\end{eulerprompt}
\begin{eulercomment}
Fungsi untuk garis dan lingkaran bekerja seperti fungsi Euler, tetapi
memberikan perhitungan simbolis.
\end{eulercomment}
\begin{eulerprompt}
>c &= lineThrough(B,C) // c=BC
\end{eulerprompt}
\begin{euleroutput}
  
                               [- 1, 2, 2]
  
\end{euleroutput}
\begin{eulercomment}
Kita bisa mendapatkan persamaan garis dengan mudah.
\end{eulercomment}
\begin{eulerprompt}
>$getLineEquation(c,x,y), $solve(%,y) | expand // persamaan garis c
\end{eulerprompt}
\begin{eulerformula}
\[
\left[ y=\frac{x}{2}+1 \right] 
\]
\end{eulerformula}
\eulerimg{1}{images/Rasyid Shalahuddin_22305144016_EMT geometri-008-large.png}
\begin{eulerprompt}
>$getLineEquation(lineThrough(A,[x1,y1]),x,y) // persamaan garis melalui A dan (x1, y1)
>h &= perpendicular(A,lineThrough(B,C)) // h melalui A tegak lurus BC
\end{eulerprompt}
\begin{euleroutput}
  
                                [2, 1, 2]
  
\end{euleroutput}
\begin{eulerprompt}
>Q &= lineIntersection(c,h) // Q titik potong garis c=BC dan h
\end{eulerprompt}
\begin{euleroutput}
  
                                   2  6
                                  [-, -]
                                   5  5
  
\end{euleroutput}
\begin{eulerprompt}
>$projectToLine(A,lineThrough(B,C)) // proyeksi A pada BC
>$distance(A,Q) // jarak AQ
>cc &= circleThrough(A,B,C); $cc // (titik pusat dan jari-jari) lingkaran melalui A, B, C
>r&=getCircleRadius(cc); $r , $float(r) // tampilkan nilai jari-jari
>$computeAngle(A,C,B) // nilai <ACB
>$solve(getLineEquation(angleBisector(A,C,B),x,y),y)[1] // persamaan garis bagi <ACB
>P &= lineIntersection(angleBisector(A,C,B),angleBisector(C,B,A)); $P // titik potong 2 garis bagi sudut
>P() // hasilnya sama dengan perhitungan sebelumnya
\end{eulerprompt}
\begin{euleroutput}
  [0.86038,  0.86038]
\end{euleroutput}
\eulersubheading{Garis dan Lingkaran yang Berpotongan}
\begin{eulercomment}
Tentu saja, kita juga dapat memotong garis dengan lingkaran, dan
lingkaran dengan lingkaran.
\end{eulercomment}
\begin{eulerprompt}
>A &:= [1,0]; c=circleWithCenter(A,4);
>B &:= [1,2]; C &:= [2,1]; l=lineThrough(B,C);
>setPlotRange(5); plotCircle(c); plotLine(l);
\end{eulerprompt}
\begin{eulercomment}
Perpotongan garis dengan lingkaran menghasilkan dua titik dan jumlah
titik potong.
\end{eulercomment}
\begin{eulerprompt}
>\{P1,P2,f\}=lineCircleIntersections(l,c);
>P1, P2,
\end{eulerprompt}
\begin{euleroutput}
  [4.64575,  -1.64575]
  [-0.645751,  3.64575]
\end{euleroutput}
\begin{eulerprompt}
>plotPoint(P1); plotPoint(P2):
\end{eulerprompt}
\begin{eulercomment}
Begitu pula di Maxima.
\end{eulercomment}
\begin{eulerprompt}
>c &= circleWithCenter(A,4) // lingkaran dengan pusat A jari-jari 4
\end{eulerprompt}
\begin{euleroutput}
  
                                [1, 0, 4]
  
\end{euleroutput}
\begin{eulerprompt}
>l &= lineThrough(B,C) // garis l melalui B dan C
\end{eulerprompt}
\begin{euleroutput}
  
                                [1, 1, 3]
  
\end{euleroutput}
\begin{eulerprompt}
>$lineCircleIntersections(l,c) | radcan, // titik potong lingkaran c dan garis l
\end{eulerprompt}
\begin{eulercomment}
Akan ditunjukkan bahwa sudut-sudut yang menghadap bsuusr yang sama adalah sama besar.
\end{eulercomment}
\begin{eulerprompt}
>C=A+normalize([-2,-3])*4; plotPoint(C); plotSegment(P1,C); plotSegment(P2,C);
>degprint(computeAngle(P1,C,P2))
\end{eulerprompt}
\begin{euleroutput}
  69°17'42.68''
\end{euleroutput}
\begin{eulerprompt}
>C=A+normalize([-4,-3])*4; plotPoint(C); plotSegment(P1,C); plotSegment(P2,C);
>degprint(computeAngle(P1,C,P2))
\end{eulerprompt}
\begin{euleroutput}
  69°17'42.68''
\end{euleroutput}
\begin{eulerprompt}
>insimg;
\end{eulerprompt}
\eulersubheading{Garis Sumbu}
\begin{eulercomment}
Berikut adalah langkah-langkah menggambar garis sumbu ruas garis AB:

1. Gambar lingkaran dengan pusat A melalui B.\\
2. Gambar lingkaran dengan pusat B melalui A.\\
3. Tarik garis melallui kedua titik potong kedua lingkaran tersebut. Garis ini merupakan
garis sumbu (melalui titik tengah dan tegak lurus) AB.
\end{eulercomment}
\begin{eulerprompt}
>A=[2,2]; B=[-1,-2];
>c1=circleWithCenter(A,distance(A,B));
>c2=circleWithCenter(B,distance(A,B));
>\{P1,P2,f\}=circleCircleIntersections(c1,c2);
>l=lineThrough(P1,P2);
>setPlotRange(5); plotCircle(c1); plotCircle(c2);
>plotPoint(A); plotPoint(B); plotSegment(A,B); plotLine(l):
\end{eulerprompt}
\begin{eulercomment}
Selanjutnya, kami melakukan hal yang sama di Maxima dengan koordinat
umum.
\end{eulercomment}
\begin{eulerprompt}
>A &= [a1,a2]; B &= [b1,b2];
>c1 &= circleWithCenter(A,distance(A,B));
>c2 &= circleWithCenter(B,distance(A,B));
>P &= circleCircleIntersections(c1,c2); P1 &= P[1]; P2 &= P[2];
\end{eulerprompt}
\begin{eulercomment}
Persamaan untuk persimpangan cukup terlibat. Tetapi kita dapat
menyederhanakannya, jika kita memecahkan y.
\end{eulercomment}
\begin{eulerprompt}
>g &= getLineEquation(lineThrough(P1,P2),x,y);
>$solve(g,y)
\end{eulerprompt}
\begin{eulercomment}
Ini memang sama dengan tegak lurus tengah, yang dihitung dengan cara
yang sama sekali berbeda.
\end{eulercomment}
\begin{eulerprompt}
>$solve(getLineEquation(middlePerpendicular(A,B),x,y),y)
>h &=getLineEquation(lineThrough(A,B),x,y);
>$solve(h,y)
\end{eulerprompt}
\begin{eulercomment}
Perhatikan hasil kali gradien garis g dan h adalah:

\end{eulercomment}
\begin{eulerformula}
\[
\frac{-(b_1-a_1)}{(b_2-a_2)}\times \frac{(b_2-a_2)}{(b_1-a_1)} = -1.
\]
\end{eulerformula}
\begin{eulercomment}
Artinya kedua garis tegak lurus.

\end{eulercomment}
\eulersubheading{Contoh 3: Rumus Heron}
\begin{eulercomment}
Rumus Heron menyatakan bahwa luas segitiga dengan panjang sisi-sisi a,
b dan c adalah:

\end{eulercomment}
\begin{eulerformula}
\[
L = \sqrt{s(s-a)(s-b)(s-c)}\quad \text{ dengan } s=(a+b+c)/2,
\]
\end{eulerformula}
\begin{eulercomment}
Untuk membuktikan hal ini kita misalkan C(0,0), B(a,0) dan A(x,y),
b=AC, c=AB. Luas segitiga ABC adalah

\end{eulercomment}
\begin{eulerformula}
\[
L_{\triangle ABC}=\frac{1}{2}a\times y.
\]
\end{eulerformula}
\begin{eulercomment}
Nilai y didapat dengan menyelesaikan sistem persamaan:

\end{eulercomment}
\begin{eulerformula}
\[
x^2+y^2=b^2, \quad (x-a)^2+y^2=c^2.
\]
\end{eulerformula}
\begin{eulerprompt}
>sol &= solve([x^2+y^2=b^2,(x-a)^2+y^2=c^2],[x,y])
\end{eulerprompt}
\begin{euleroutput}
  
                                    []
  
\end{euleroutput}
\begin{eulerprompt}
>setPlotRange(-1,10,-1,8); plotPoint([0,0], "C(0,0)"); plotPoint([5.5,0], "B(a,0)");  ...
>plotPoint([7.5,6], "A(x,y)");
>plotSegment([0,0],[5.5,0], "a",25); plotSegment([5.5,0],[7.5,6],"c",15);  ...
>plotSegment([0,0],[7.5,6],"b",25); 
>plotSegment([7.5,6],[7.5,0],"t=y",25):
>sol &= solve([x^2+y^2=b^2,(x-a)^2+y^2=c^2],[x,y])
\end{eulerprompt}
\begin{euleroutput}
  
                                    []
  
\end{euleroutput}
\begin{eulercomment}
Ekstrak solusi y.
\end{eulercomment}
\begin{eulerprompt}
>ysol &= y with sol[2][2]; $ysol
\end{eulerprompt}
\begin{euleroutput}
  Maxima said:
  part: invalid index of list or matrix.
   -- an error. To debug this try: debugmode(true);
  
  Error in:
  ysol &= y with sol[2][2]; $ysol ...
                          ^
\end{euleroutput}
\begin{eulercomment}
Kami mendapatkan rumus Heron.
\end{eulercomment}
\begin{eulerprompt}
>function H(a,b,c) &= sqrt(factor((ysol*a/2)^2)); $'H(a,b,c)=H(a,b,c)
>$'Luas=H(3,4,5) // luas segitiga dengan panjang sisi-sisi 3, 4, 5
\end{eulerprompt}
\begin{eulercomment}
Tentu saja, setiap segitiga persegi panjang adalah kasus yang
terkenal.
\end{eulercomment}
\begin{eulerprompt}
>H(3,4,5) //luas segitiga siku-siku dengan panjang sisi 3, 4, 5
\end{eulerprompt}
\begin{euleroutput}
  Variable or function ysol not found.
  Try "trace errors" to inspect local variables after errors.
  H:
      useglobal; return abs(a)*abs(ysol)/2 
  Error in:
  H(3,4,5) //luas segitiga siku-siku dengan panjang sisi 3, 4, 5 ...
          ^
\end{euleroutput}
\begin{eulercomment}
Dan juga jelas, bahwa ini adalah segitiga dengan luas maksimal dan dua
sisi 3 dan 4.
\end{eulercomment}
\begin{eulerprompt}
>aspect (1.5); plot2d(&H(3,4,x),1,7): // Kurva luas segitiga sengan panjang sisi 3, 4, x (1<= x <=7)
\end{eulerprompt}
\begin{euleroutput}
  Variable or function ysol not found.
  Error in expression: 3*abs(ysol)/2
   %ploteval:
      y0=f$(x[1],args());
  adaptiveevalone:
      s=%ploteval(g$,t;args());
  Try "trace errors" to inspect local variables after errors.
  plot2d:
      dw/n,dw/n^2,dw/n,auto;args());
\end{euleroutput}
\begin{eulercomment}
Kasus umum juga berfungsi.
\end{eulercomment}
\begin{eulerprompt}
>$solve(diff(H(a,b,c)^2,c)=0,c)
\end{eulerprompt}
\begin{euleroutput}
  Maxima said:
  diff: second argument must be a variable; found [1,0,4]
   -- an error. To debug this try: debugmode(true);
  
  Error in:
   $solve(diff(H(a,b,c)^2,c)=0,c) ...
                                ^
\end{euleroutput}
\begin{eulercomment}
Sekarang mari kita cari himpunan semua titik di mana b+c=d untuk
beberapa konstanta d. Diketahui bahwa ini adalah elips.
\end{eulercomment}
\begin{eulerprompt}
>s1 &= subst(d-c,b,sol[2]); $s1
\end{eulerprompt}
\begin{euleroutput}
  Maxima said:
  part: invalid index of list or matrix.
   -- an error. To debug this try: debugmode(true);
  
  Error in:
  s1 &= subst(d-c,b,sol[2]); $s1 ...
                           ^
\end{euleroutput}
\begin{eulercomment}
Dan buat fungsi ini.
\end{eulercomment}
\begin{eulerprompt}
>function fx(a,c,d) &= rhs(s1[1]); $fx(a,c,d), function fy(a,c,d) &= rhs(s1[2]); $fy(a,c,d)
\end{eulerprompt}
\begin{eulercomment}
Sekarang kita bisa menggambar setnya. Sisi b bervariasi dari 1 hingga
4. Diketahui bahwa kita mendapatkan elips.
\end{eulercomment}
\begin{eulerprompt}
>aspect(1); plot2d(&fx(3,x,5),&fy(3,x,5),xmin=1,xmax=4,square=1):
\end{eulerprompt}
\begin{eulercomment}
Kita dapat memeriksa persamaan umum untuk elips ini, yaitu.

\end{eulercomment}
\begin{eulerformula}
\[
\frac{(x-x_m)^2}{u^2}+\frac{(y-y_m)}{v^2}=1,
\]
\end{eulerformula}
\begin{eulercomment}
di mana (xm,ym) adalah pusat, dan u dan v adalah setengah sumbu.
\end{eulercomment}
\begin{eulerprompt}
>$ratsimp((fx(a,c,d)-a/2)^2/u^2+fy(a,c,d)^2/v^2 with [u=d/2,v=sqrt(d^2-a^2)/2])
\end{eulerprompt}
\begin{eulercomment}
Kita lihat bahwa tinggi dan luas segitiga adalah maksimal untuk x=0.
Jadi luas segitiga dengan a+b+c=d maksimal jika segitiga sama sisi.
Kami ingin menurunkan ini secara analitis.
\end{eulercomment}
\begin{eulerprompt}
>eqns &= [diff(H(a,b,d-(a+b))^2,a)=0,diff(H(a,b,d-(a+b))^2,b)=0]; $eqns
\end{eulerprompt}
\begin{eulercomment}
Kami mendapatkan beberapa minima, yang termasuk dalam segitiga dengan
satu sisi 0, dan solusinya a=b=c=d/3.
\end{eulercomment}
\begin{eulerprompt}
>$solve(eqns,[a,b])
\end{eulerprompt}
\begin{eulercomment}
Ada juga metode Lagrange, memaksimalkan H(a,b,c)\textasciicircum{}2 terhadap a+b+d=d.
\end{eulercomment}
\begin{eulerprompt}
>&solve([diff(H(a,b,c)^2,a)=la,diff(H(a,b,c)^2,b)=la, ...
>   diff(H(a,b,c)^2,c)=la,a+b+c=d],[a,b,c,la])
\end{eulerprompt}
\begin{euleroutput}
  Maxima said:
  diff: second argument must be a variable; found [1,0,4]
   -- an error. To debug this try: debugmode(true);
  
  Error in:
  ... la,    diff(H(a,b,c)^2,c)=la,a+b+c=d],[a,b,c,la]) ...
                                                       ^
\end{euleroutput}
\begin{eulercomment}
Kita bisa membuat plot situasinya
\end{eulercomment}
\begin{eulercomment}
Pertama-tama atur poin di Maxima.
\end{eulercomment}
\begin{eulerprompt}
>A &= at([x,y],sol[2]); $A
\end{eulerprompt}
\begin{euleroutput}
  Maxima said:
  part: invalid index of list or matrix.
   -- an error. To debug this try: debugmode(true);
  
  Error in:
  A &= at([x,y],sol[2]); $A ...
                       ^
\end{euleroutput}
\begin{eulerprompt}
>B &= [0,0]; $B, C &= [a,0]; $C
\end{eulerprompt}
\begin{eulercomment}
Kemudian atur rentang plot, dan plot titik-titiknya.
\end{eulercomment}
\begin{eulerprompt}
>setPlotRange(0,5,-2,3); ...
>a=4; b=3; c=2; ...
>plotPoint(mxmeval("B"),"B"); plotPoint(mxmeval("C"),"C"); ...
>plotPoint(mxmeval("A"),"A"):
\end{eulerprompt}
\begin{euleroutput}
  Variable a1 not found!
  Use global variables or parameters for string evaluation.
  Error in Evaluate, superfluous characters found.
  Try "trace errors" to inspect local variables after errors.
  mxmeval:
      return evaluate(mxm(s));
  Error in:
  ... otPoint(mxmeval("C"),"C"); plotPoint(mxmeval("A"),"A"): ...
                                                       ^
\end{euleroutput}
\begin{eulercomment}
Plot segmen.
\end{eulercomment}
\begin{eulerprompt}
>plotSegment(mxmeval("A"),mxmeval("C")); ...
>plotSegment(mxmeval("B"),mxmeval("C")); ...
>plotSegment(mxmeval("B"),mxmeval("A")):
\end{eulerprompt}
\begin{euleroutput}
  Variable a1 not found!
  Use global variables or parameters for string evaluation.
  Error in Evaluate, superfluous characters found.
  Try "trace errors" to inspect local variables after errors.
  mxmeval:
      return evaluate(mxm(s));
  Error in:
  plotSegment(mxmeval("A"),mxmeval("C")); plotSegment(mxmeval("B ...
                          ^
\end{euleroutput}
\begin{eulercomment}
Hitung tegak lurus tengah di Maxima.
\end{eulercomment}
\begin{eulerprompt}
>h &= middlePerpendicular(A,B); g &= middlePerpendicular(B,C);
\end{eulerprompt}
\begin{eulercomment}
Dan pusat lingkaran.
\end{eulercomment}
\begin{eulerprompt}
>U &= lineIntersection(h,g);
\end{eulerprompt}
\begin{eulercomment}
Kami mendapatkan rumus untuk jari-jari lingkaran.
\end{eulercomment}
\begin{eulerprompt}
>&assume(a>0,b>0,c>0); $distance(U,B) | radcan
\end{eulerprompt}
\begin{eulercomment}
Mari kita tambahkan ini ke plot.
\end{eulercomment}
\begin{eulerprompt}
>plotPoint(U()); ...
>plotCircle(circleWithCenter(mxmeval("U"),mxmeval("distance(U,C)"))):
\end{eulerprompt}
\begin{euleroutput}
  Variable a2 not found!
  Use global variables or parameters for string evaluation.
  Error in ^
  Error in expression: [a/2,(a2^2+a1^2-a*a1)/(2*a2)]
  Error in:
  plotPoint(U()); plotCircle(circleWithCenter(mxmeval("U"),mxmev ...
               ^
\end{euleroutput}
\begin{eulercomment}
Menggunakan geometri, kami memperoleh rumus sederhana

\end{eulercomment}
\begin{eulerformula}
\[
\frac{a}{\sin(\alpha)}=2r
\]
\end{eulerformula}
\begin{eulercomment}
untuk radiusnya. Kami dapat memeriksa, apakah ini benar dengan Maxima.
Maxima akan memfaktorkan ini hanya jika kita kuadratkan.
\end{eulercomment}
\begin{eulerprompt}
>$c^2/sin(computeAngle(A,B,C))^2  | factor
\end{eulerprompt}
\eulersubheading{Contoh 4: Garis Euler dan Parabola}
\begin{eulercomment}
Garis Euler adalah garis yang ditentukan dari sembarang segitiga yang
tidak sama sisi. Ini adalah garis tengah segitiga, dan melewati
beberapa titik penting yang ditentukan dari segitiga, termasuk
orthocenter, circumcenter, centroid, titik Exeter dan pusat lingkaran
sembilan titik segitiga.

Untuk demonstrasi, kami menghitung dan memplot garis Euler dalam
sebuah segitiga.

Pertama, kita mendefinisikan sudut-sudut segitiga di Euler. Kami
menggunakan definisi, yang terlihat dalam ekspresi simbolis.
\end{eulercomment}
\begin{eulerprompt}
>A::=[-1,-1]; B::=[2,0]; C::=[1,2];
\end{eulerprompt}
\begin{eulercomment}
Untuk memplot objek geometris, kami menyiapkan area plot, dan
menambahkan titik ke sana. Semua plot objek geometris ditambahkan ke
plot saat ini.
\end{eulercomment}
\begin{eulerprompt}
>setPlotRange(3); plotPoint(A,"A"); plotPoint(B,"B"); plotPoint(C,"C");
\end{eulerprompt}
\begin{eulercomment}
Kita juga bisa menambahkan sisi segitiga.
\end{eulercomment}
\begin{eulerprompt}
>plotSegment(A,B,""); plotSegment(B,C,""); plotSegment(C,A,""):
\end{eulerprompt}
\begin{eulercomment}
Berikut adalah luas segitiga, menggunakan rumus determinan. Tentu
saja, kita harus mengambil nilai absolut dari hasil ini.
\end{eulercomment}
\begin{eulerprompt}
>$areaTriangle(A,B,C)
\end{eulerprompt}
\begin{eulercomment}
Kita dapat menghitung koefisien sisi c.
\end{eulercomment}
\begin{eulerprompt}
>c &= lineThrough(A,B)
\end{eulerprompt}
\begin{euleroutput}
  
                              [- 1, 3, - 2]
  
\end{euleroutput}
\begin{eulercomment}
Dan juga dapatkan rumus untuk baris ini.
\end{eulercomment}
\begin{eulerprompt}
>$getLineEquation(c,x,y)
\end{eulerprompt}
\begin{eulercomment}
Untuk bentuk Hesse, kita perlu menentukan sebuah titik, sehingga titik
tersebut berada di sisi positif dari bentuk Hesse. Memasukkan titik
menghasilkan jarak positif ke garis.
\end{eulercomment}
\begin{eulerprompt}
>$getHesseForm(c,x,y,C), $at(%,[x=C[1],y=C[2]])
\end{eulerprompt}
\begin{eulercomment}
Sekarang kita hitung lingkaran luar ABC.
\end{eulercomment}
\begin{eulerprompt}
>LL &= circleThrough(A,B,C); $getCircleEquation(LL,x,y)
>O &= getCircleCenter(LL); $O
\end{eulerprompt}
\begin{eulercomment}
Gambarkan lingkaran dan pusatnya. Cu dan U adalah simbolis. Kami
mengevaluasi ekspresi ini untuk Euler.
\end{eulercomment}
\begin{eulerprompt}
>plotCircle(LL()); plotPoint(O(),"O"):
\end{eulerprompt}
\begin{eulercomment}
Kita dapat menghitung perpotongan ketinggian di ABC (orthocenter)
secara numerik dengan perintah berikut.
\end{eulercomment}
\begin{eulerprompt}
>H &= lineIntersection(perpendicular(A,lineThrough(C,B)),...
>  perpendicular(B,lineThrough(A,C))); $H
\end{eulerprompt}
\begin{eulercomment}
Sekarang kita dapat menghitung garis Euler dari segitiga.
\end{eulercomment}
\begin{eulerprompt}
>el &= lineThrough(H,O); $getLineEquation(el,x,y)
\end{eulerprompt}
\begin{eulercomment}
Tambahkan ke plot kami.
\end{eulercomment}
\begin{eulerprompt}
>plotPoint(H(),"H"); plotLine(el(),"Garis Euler"):
\end{eulerprompt}
\begin{eulercomment}
Pusat gravitasi harus berada di garis ini.
\end{eulercomment}
\begin{eulerprompt}
>M &= (A+B+C)/3; $getLineEquation(el,x,y) with [x=M[1],y=M[2]]
>plotPoint(M(),"M"): // titik berat
\end{eulerprompt}
\begin{eulercomment}
Teorinya memberitahu kita MH=2*MO. Kita perlu menyederhanakan dengan
radcan untuk mencapai ini.
\end{eulercomment}
\begin{eulerprompt}
>$distance(M,H)/distance(M,O)|radcan
\end{eulerprompt}
\begin{eulercomment}
Fungsi termasuk fungsi untuk sudut juga.
\end{eulercomment}
\begin{eulerprompt}
>$computeAngle(A,C,B), degprint(%())
\end{eulerprompt}
\begin{euleroutput}
  60°15'18.43''
\end{euleroutput}
\begin{eulercomment}
Persamaan untuk pusat incircle tidak terlalu bagus.
\end{eulercomment}
\begin{eulerprompt}
>Q &= lineIntersection(angleBisector(A,C,B),angleBisector(C,B,A))|radcan; $Q
\end{eulerprompt}
\begin{eulercomment}
Mari kita hitung juga ekspresi untuk jari-jari lingkaran yang
tertulis.
\end{eulercomment}
\begin{eulerprompt}
>r &= distance(Q,projectToLine(Q,lineThrough(A,B)))|ratsimp; $r
>LD &=  circleWithCenter(Q,r); // Lingkaran dalam
\end{eulerprompt}
\begin{eulercomment}
Mari kita tambahkan ini ke plot.
\end{eulercomment}
\begin{eulerprompt}
>color(5); plotCircle(LD()):
\end{eulerprompt}
\eulersubheading{Parabola}
\begin{eulercomment}
Selanjutnya akan dicari persamaan tempat kedudukan titik-titik yang berjarak sama ke titik C
dan ke garis AB.
\end{eulercomment}
\begin{eulerprompt}
>p &= getHesseForm(lineThrough(A,B),x,y,C)-distance([x,y],C); $p='0
\end{eulerprompt}
\begin{eulercomment}
Persamaan tersebut dapat digambar menjadi satu dengan gambar sebelumnya.
\end{eulercomment}
\begin{eulerprompt}
>plot2d(p,level=0,add=1,contourcolor=6):
\end{eulerprompt}
\begin{eulercomment}
Ini seharusnya menjadi beberapa fungsi, tetapi pemecah default Maxima
hanya dapat menemukan solusinya, jika kita kuadratkan persamaannya.
Akibatnya, kami mendapatkan solusi palsu.
\end{eulercomment}
\begin{eulerprompt}
>akar &= solve(getHesseForm(lineThrough(A,B),x,y,C)^2-distance([x,y],C)^2,y)
\end{eulerprompt}
\begin{euleroutput}
  
          [y = - 3 x - sqrt(70) sqrt(9 - 2 x) + 26, 
                                y = - 3 x + sqrt(70) sqrt(9 - 2 x) + 26]
  
\end{euleroutput}
\begin{eulercomment}
Solusi pertama adalah

maxima: akar[1]

Menambahkan solusi pertama ke plot menunjukkan, bahwa itu memang jalan
yang kita cari. Teorinya memberi tahu kita bahwa itu adalah parabola
yang diputar.
\end{eulercomment}
\begin{eulerprompt}
>plot2d(&rhs(akar[1]),add=1):
>function g(x) &= rhs(akar[1]); $'g(x)= g(x)// fungsi yang mendefinisikan kurva di atas
>T &=[-1, g(-1)]; // ambil sebarang titik pada kurva tersebut
>dTC &= distance(T,C); $fullratsimp(dTC), $float(%) // jarak T ke C
>U &= projectToLine(T,lineThrough(A,B)); $U // proyeksi T pada garis AB 
>dU2AB &= distance(T,U); $fullratsimp(dU2AB), $float(%) // jatak T ke AB
\end{eulerprompt}
\begin{eulercomment}
Ternyata jarak T ke C sama dengan jarak T ke AB. Coba Anda pilih titik T yang lain dan
ulangi perhitungan-perhitungan di atas untuk menunjukkan bahwa hasilnya juga sama.
\end{eulercomment}
\eulersubheading{Contoh 5: Trigonometri Rasional}
\begin{eulercomment}
Ini terinspirasi dari ceramah N.J.Wildberger. Dalam bukunya "Divine
Proportions", Wildberger mengusulkan untuk mengganti pengertian klasik
tentang jarak dan sudut dengan kuadrat dan penyebaran. Dengan
menggunakan ini, memang mungkin untuk menghindari fungsi trigonometri
dalam banyak contoh, dan tetap "rasional".

Berikut ini, saya memperkenalkan konsep, dan memecahkan beberapa
masalah. Saya menggunakan perhitungan simbolik Maxima di sini, yang
menyembunyikan keuntungan utama dari trigonometri rasional bahwa
perhitungan hanya dapat dilakukan dengan kertas dan pensil. Anda
diundang untuk memeriksa hasil tanpa komputer.

Intinya adalah bahwa perhitungan rasional simbolis sering kali
menghasilkan hasil yang sederhana. Sebaliknya, trigonometri klasik
menghasilkan hasil trigonometri yang rumit, yang hanya mengevaluasi
perkiraan numerik.
\end{eulercomment}
\begin{eulerprompt}
>load geometry;
\end{eulerprompt}
\begin{eulercomment}
Untuk pengenalan pertama, kami menggunakan segitiga persegi panjang
dengan proporsi Mesir terkenal 3, 4 dan 5. Perintah berikut adalah
perintah Euler untuk merencanakan geometri bidang yang terdapat dalam
file Euler "geometry.e".
\end{eulercomment}
\begin{eulerprompt}
>C&:=[0,0]; A&:=[4,0]; B&:=[0,3]; ...
>setPlotRange(-1,5,-1,5); ...
>plotPoint(A,"A"); plotPoint(B,"B"); plotPoint(C,"C"); ...
>plotSegment(B,A,"c"); plotSegment(A,C,"b"); plotSegment(C,B,"a"); ...
>insimg(30);
\end{eulerprompt}
\begin{eulercomment}
Tentu saja,

\end{eulercomment}
\begin{eulerformula}
\[
\sin(w_a)=\frac{a}{c},
\]
\end{eulerformula}
\begin{eulercomment}
di mana wa adalah sudut di A. Cara yang biasa untuk menghitung sudut
ini, adalah dengan mengambil invers dari fungsi sinus. Hasilnya adalah
sudut yang tidak dapat dicerna, yang hanya dapat dicetak kira-kira.
\end{eulercomment}
\begin{eulerprompt}
>wa := arcsin(3/5); degprint(wa)
\end{eulerprompt}
\begin{euleroutput}
  36°52'11.63''
\end{euleroutput}
\begin{eulercomment}
Trigonometri rasional mencoba menghindari hal ini.

Gagasan pertama trigonometri rasional adalah kuadran, yang
menggantikan jarak. Sebenarnya, itu hanya jarak kuadrat. Berikut ini,
a, b, dan c menunjukkan kuadrat dari sisi-sisinya.

Teorema Pythogoras menjadi a+b=c.
\end{eulercomment}
\begin{eulerprompt}
>a &= 3^2; b &= 4^2; c &= 5^2; &a+b=c
\end{eulerprompt}
\begin{euleroutput}
  
                                 25 = 25
  
\end{euleroutput}
\begin{eulercomment}
Pengertian kedua dari trigonometri rasional adalah penyebaran. Spread
mengukur pembukaan antar baris. Ini adalah 0, jika garis-garisnya
sejajar, dan 1, jika garis-garisnya persegi panjang. Ini adalah
kuadrat sinus sudut antara dua garis.

Penyebaran garis AB dan AC pada gambar di atas didefinisikan sebagai:

\end{eulercomment}
\begin{eulerformula}
\[
s_a = \sin(\alpha)^2 = \frac{a}{c},
\]
\end{eulerformula}
\begin{eulercomment}
di mana a dan c adalah kuadrat dari sembarang segitiga siku-siku
dengan salah satu sudut di A.
\end{eulercomment}
\begin{eulerprompt}
>sa &= a/c; $sa
\end{eulerprompt}
\begin{eulercomment}
Ini lebih mudah dihitung daripada sudut, tentu saja. Tetapi Anda
kehilangan properti bahwa sudut dapat ditambahkan dengan mudah.

Tentu saja, kita dapat mengonversi nilai perkiraan untuk sudut wa
menjadi sprad, dan mencetaknya sebagai pecahan.
\end{eulercomment}
\begin{eulerprompt}
>fracprint(sin(wa)^2)
\end{eulerprompt}
\begin{euleroutput}
  9/25
\end{euleroutput}
\begin{eulercomment}
Hukum kosinus trgonometri klasik diterjemahkan menjadi "hukum silang"
berikut.

\end{eulercomment}
\begin{eulerformula}
\[
(c+b-a)^2 = 4 b c \, (1-s_a)
\]
\end{eulerformula}
\begin{eulercomment}
Di sini a, b, dan c adalah kuadrat dari sisi-sisi segitiga, dan sa
adalah penyebaran sudut A. Sisi a, seperti biasa, berhadapan dengan
sudut A.

Hukum ini diimplementasikan dalam file geometri.e yang kami muat ke
Euler.
\end{eulercomment}
\begin{eulerprompt}
>$crosslaw(aa,bb,cc,saa)
\end{eulerprompt}
\begin{eulercomment}
Dalam kasus kami, kami mendapatkan
\end{eulercomment}
\begin{eulerprompt}
>$crosslaw(a,b,c,sa)
\end{eulerprompt}
\begin{eulercomment}
Mari kita gunakan crosslaw ini untuk mencari spread di A. Untuk
melakukan ini, kita buat crosslaw untuk kuadran a, b, dan c, dan
selesaikan untuk spread yang tidak diketahui sa.

Anda dapat melakukannya dengan tangan dengan mudah, tetapi saya
menggunakan Maxima. Tentu saja, kami mendapatkan hasilnya, kami sudah
memilikinya.
\end{eulercomment}
\begin{eulerprompt}
>$crosslaw(a,b,c,x), $solve(%,x)
\end{eulerprompt}
\begin{eulercomment}
Kita sudah tahu ini. Definisi spread adalah kasus khusus dari
crosslaw.

Kita juga dapat menyelesaikan ini untuk umum a,b,c. Hasilnya adalah
rumus yang menghitung penyebaran sudut segitiga yang diberikan kuadrat
dari ketiga sisinya.
\end{eulercomment}
\begin{eulerprompt}
>$solve(crosslaw(aa,bb,cc,x),x)
\end{eulerprompt}
\begin{eulercomment}
Kita bisa membuat fungsi dari hasilnya. Fungsi seperti itu sudah
didefinisikan dalam file geometri.e dari Euler.
\end{eulercomment}
\begin{eulerprompt}
>$spread(a,b,c)
\end{eulerprompt}
\begin{eulercomment}
Sebagai contoh, kita dapat menggunakannya untuk menghitung sudut
segitiga dengan sisi

\end{eulercomment}
\begin{eulerformula}
\[
a, \quad a, \quad \frac{4a}{7}
\]
\end{eulerformula}
\begin{eulercomment}
Hasilnya rasional, yang tidak begitu mudah didapat jika kita
menggunakan trigonometri klasik.
\end{eulercomment}
\begin{eulerprompt}
>$spread(a,a,4*a/7)
\end{eulerprompt}
\begin{eulercomment}
Ini adalah sudut dalam derajat.
\end{eulercomment}
\begin{eulerprompt}
>degprint(arcsin(sqrt(6/7)))
\end{eulerprompt}
\begin{euleroutput}
  67°47'32.44''
\end{euleroutput}
\eulersubheading{Contoh lain}
\begin{eulercomment}
Sekarang, mari kita coba contoh yang lebih maju.

Kami mengatur tiga sudut segitiga sebagai berikut.
\end{eulercomment}
\begin{eulerprompt}
>A&:=[1,2]; B&:=[4,3]; C&:=[0,4]; ...
>setPlotRange(-1,5,1,7); ...
>plotPoint(A,"A"); plotPoint(B,"B"); plotPoint(C,"C"); ...
>plotSegment(B,A,"c"); plotSegment(A,C,"b"); plotSegment(C,B,"a"); ...
>insimg;
\end{eulerprompt}
\begin{eulercomment}
Menggunakan Pythogoras, mudah untuk menghitung jarak antara dua titik.
Saya pertama kali menggunakan jarak fungsi file Euler untuk geometri.
Jarak fungsi menggunakan geometri klasik.
\end{eulercomment}
\begin{eulerprompt}
>$distance(A,B)
\end{eulerprompt}
\begin{eulercomment}
Euler juga mengandung fungsi untuk kuadran antara dua titik.

Dalam contoh berikut, karena c+b bukan a, maka segitiga itu bukan
persegi panjang.
\end{eulercomment}
\begin{eulerprompt}
>c &= quad(A,B); $c, b &= quad(A,C); $b, a &= quad(B,C); $a,
\end{eulerprompt}
\begin{eulercomment}
Pertama, mari kita hitung sudut tradisional. Fungsi computeAngle
menggunakan metode biasa berdasarkan hasil kali titik dua vektor.
Hasilnya adalah beberapa pendekatan floating point.

\end{eulercomment}
\begin{eulerformula}
\[
A=<1,2>\quad B=<4,3>,\quad C=<0,4>
\]
\end{eulerformula}
\begin{eulerformula}
\[
\mathbf{a}=C-B=<-4,1>,\quad \mathbf{c}=A-B=<-3,-1>,\quad \beta=\angle ABC
\]
\end{eulerformula}
\begin{eulerformula}
\[
\mathbf{a}.\mathbf{c}=|\mathbf{a}|.|\mathbf{c}|\cos \beta
\]
\end{eulerformula}
\begin{eulerformula}
\[
\cos \angle ABC =\cos\beta=\frac{\mathbf{a}.\mathbf{c}}{|\mathbf{a}|.|\mathbf{c}|}=\frac{12-1}{\sqrt{17}\sqrt{10}}=\frac{11}{\sqrt{17}\sqrt{10}}
\]
\end{eulerformula}
\begin{eulerprompt}
>wb &= computeAngle(A,B,C); $wb, $(wb/pi*180)()
\end{eulerprompt}
\begin{euleroutput}
  32.4711922908
\end{euleroutput}
\begin{eulercomment}
Dengan menggunakan pensil dan kertas, kita dapat melakukan hal yang
sama dengan hukum silang. Kami memasukkan kuadran a, b, dan c ke dalam
hukum silang dan menyelesaikan x.
\end{eulercomment}
\begin{eulerprompt}
>$crosslaw(a,b,c,x), $solve(%,x), //(b+c-a)^=4b.c(1-x)
\end{eulerprompt}
\begin{eulercomment}
Yaitu, apa yang dilakukan oleh penyebaran fungsi yang didefinisikan
dalam "geometry.e".
\end{eulercomment}
\begin{eulerprompt}
>sb &= spread(b,a,c); $sb
\end{eulerprompt}
\begin{eulercomment}
Maxima mendapatkan hasil yang sama menggunakan trigonometri biasa,
jika kita memaksanya. Itu menyelesaikan istilah sin(arccos(...))
menjadi hasil pecahan. Sebagian besar siswa tidak dapat melakukan ini.
\end{eulercomment}
\begin{eulerprompt}
>$sin(computeAngle(A,B,C))^2
\end{eulerprompt}
\begin{eulercomment}
Setelah kita memiliki spread di B, kita dapat menghitung tinggi ha di
sisi a. Ingat bahwa

\end{eulercomment}
\begin{eulerformula}
\[
s_b=\frac{h_a}{c}
\]
\end{eulerformula}
\begin{eulercomment}
Menurut definisi.
\end{eulercomment}
\begin{eulerprompt}
>ha &= c*sb; $ha
\end{eulerprompt}
\begin{eulercomment}
Gambar berikut telah dihasilkan dengan program geometri C.a.R., yang
dapat menggambar kuadrat dan menyebar.

image: (20) Rational\_Geometry\_CaR.png

Menurut definisi, panjang ha adalah akar kuadrat dari kuadratnya.
\end{eulercomment}
\begin{eulerprompt}
>$sqrt(ha)
\end{eulerprompt}
\begin{eulercomment}
Sekarang kita dapat menghitung luas segitiga. Jangan lupa, bahwa kita
berhadapan dengan kuadrat!
\end{eulercomment}
\begin{eulerprompt}
>$sqrt(ha)*sqrt(a)/2
\end{eulerprompt}
\begin{eulercomment}
Rumus determinan biasa menghasilkan hasil yang sama.
\end{eulercomment}
\begin{eulerprompt}
>$areaTriangle(B,A,C)
\end{eulerprompt}
\eulersubheading{Rumus Bangau}
\begin{eulercomment}
Sekarang, mari kita selesaikan masalah ini secara umum!
\end{eulercomment}
\begin{eulerprompt}
>&remvalue(a,b,c,sb,ha);
\end{eulerprompt}
\begin{eulercomment}
Pertama kita hitung spread di B untuk segitiga dengan sisi a, b, dan
c. Kemudian kita menghitung luas kuadrat ("quadrea"?), faktorkan
dengan Maxima, dan kita mendapatkan rumus Heron yang terkenal.

Memang, ini sulit dilakukan dengan pensil dan kertas.
\end{eulercomment}
\begin{eulerprompt}
>$spread(b^2,c^2,a^2), $factor(%*c^2*a^2/4)
\end{eulerprompt}
\eulersubheading{Aturan Triple Spread}
\begin{eulercomment}
Kerugian dari spread adalah mereka tidak lagi hanya menambahkan sudut
yang sama.

Namun, tiga spread dari sebuah segitiga memenuhi aturan "triple
spread" berikut.
\end{eulercomment}
\begin{eulerprompt}
>&remvalue(sa,sb,sc); $triplespread(sa,sb,sc)
\end{eulerprompt}
\begin{eulercomment}
Aturan ini berlaku untuk setiap tiga sudut yang menambah 180 °.

\end{eulercomment}
\begin{eulerformula}
\[
\alpha+\beta+\gamma=\pi
\]
\end{eulerformula}
\begin{eulercomment}
Sejak menyebar

\end{eulercomment}
\begin{eulerformula}
\[
\alpha, \pi-\alpha
\]
\end{eulerformula}
\begin{eulercomment}
sama, aturan triple spread juga benar, jika

\end{eulercomment}
\begin{eulerformula}
\[
\alpha+\beta=\gamma
\]
\end{eulerformula}
\begin{eulercomment}
Karena penyebaran sudut negatif adalah sama, aturan penyebaran rangkap
tiga juga berlaku, jika

\end{eulercomment}
\begin{eulerformula}
\[
\alpha+\beta+\gamma=0
\]
\end{eulerformula}
\begin{eulercomment}
Misalnya, kita dapat menghitung penyebaran sudut 60°. Ini 3/4.
Persamaan memiliki solusi kedua, bagaimanapun, di mana semua spread
adalah 0.
\end{eulercomment}
\begin{eulerprompt}
>$solve(triplespread(x,x,x),x)
\end{eulerprompt}
\begin{eulercomment}
Sebaran 90° jelas 1. Jika dua sudut dijumlahkan menjadi 90°,
sebarannya menyelesaikan persamaan sebaran rangkap tiga dengan a,b,1.
Dengan perhitungan berikut kita mendapatkan a+b=1.
\end{eulercomment}
\begin{eulerprompt}
>$triplespread(x,y,1), $solve(%,x)
\end{eulerprompt}
\begin{eulercomment}
Karena sebaran 180°-t sama dengan sebaran t, rumus sebaran rangkap
tiga juga berlaku, jika satu sudut adalah jumlah atau selisih dua
sudut lainnya.

Jadi kita dapat menemukan penyebaran sudut berlipat ganda. Perhatikan
bahwa ada dua solusi lagi. Kami membuat ini fungsi.
\end{eulercomment}
\begin{eulerprompt}
>$solve(triplespread(a,a,x),x), function doublespread(a) &= factor(rhs(%[1]))
\end{eulerprompt}
\begin{euleroutput}
  
                              - 4 (a - 1) a
  
\end{euleroutput}
\eulersubheading{Pembagi Sudut}
\begin{eulercomment}
Ini situasinya, kita sudah tahu.
\end{eulercomment}
\begin{eulerprompt}
>C&:=[0,0]; A&:=[4,0]; B&:=[0,3]; ...
>setPlotRange(-1,5,-1,5); ...
>plotPoint(A,"A"); plotPoint(B,"B"); plotPoint(C,"C"); ...
>plotSegment(B,A,"c"); plotSegment(A,C,"b"); plotSegment(C,B,"a"); ...
>insimg;
\end{eulerprompt}
\begin{eulercomment}
Mari kita hitung panjang garis bagi sudut di A. Tetapi kita ingin
menyelesaikannya untuk umum a,b,c.
\end{eulercomment}
\begin{eulerprompt}
>&remvalue(a,b,c);
\end{eulerprompt}
\begin{eulercomment}
Jadi pertama-tama kita hitung penyebaran sudut yang dibagi dua di A,
dengan menggunakan rumus sebaran rangkap tiga.

Masalah dengan rumus ini muncul lagi. Ini memiliki dua solusi. Kita
harus memilih yang benar. Solusi lainnya mengacu pada sudut terbelah
180 °-wa.
\end{eulercomment}
\begin{eulerprompt}
>$triplespread(x,x,a/(a+b)), $solve(%,x), sa2 &= rhs(%[1]); $sa2
\end{eulerprompt}
\begin{eulercomment}
Mari kita periksa persegi panjang Mesir.
\end{eulercomment}
\begin{eulerprompt}
>$sa2 with [a=3^2,b=4^2]
\end{eulerprompt}
\begin{eulercomment}
Kami dapat mencetak sudut dalam Euler, setelah mentransfer penyebaran
ke radian.
\end{eulercomment}
\begin{eulerprompt}
>wa2 := arcsin(sqrt(1/10)); degprint(wa2)
\end{eulerprompt}
\begin{euleroutput}
  18°26'5.82''
\end{euleroutput}
\begin{eulercomment}
Titik P adalah perpotongan garis bagi sudut dengan sumbu y.
\end{eulercomment}
\begin{eulerprompt}
>P := [0,tan(wa2)*4]
\end{eulerprompt}
\begin{euleroutput}
  [0,  1.33333]
\end{euleroutput}
\begin{eulerprompt}
>plotPoint(P,"P"); plotSegment(A,P):
\end{eulerprompt}
\begin{eulercomment}
Mari kita periksa sudut dalam contoh spesifik kita.
\end{eulercomment}
\begin{eulerprompt}
>computeAngle(C,A,P), computeAngle(P,A,B)
\end{eulerprompt}
\begin{euleroutput}
  0.321750554397
  0.321750554397
\end{euleroutput}
\begin{eulercomment}
Sekarang kita hitung panjang garis bagi AP.

Kami menggunakan teorema sinus dalam segitiga APC. Teorema ini
menyatakan bahwa

\end{eulercomment}
\begin{eulerformula}
\[
\frac{BC}{\sin(w_a)} = \frac{AC}{\sin(w_b)} = \frac{AB}{\sin(w_c)}
\]
\end{eulerformula}
\begin{eulercomment}
berlaku dalam segitiga apa pun. Kuadratkan, itu diterjemahkan ke dalam
apa yang disebut "hukum penyebaran"

\end{eulercomment}
\begin{eulerformula}
\[
\frac{a}{s_a} = \frac{b}{s_b} = \frac{c}{s_b}
\]
\end{eulerformula}
\begin{eulercomment}
di mana a,b,c menunjukkan qudrances.

Karena spread CPA adalah 1-sa2, kita dapatkan darinya bisa/1=b/(1-sa2)
dan dapat menghitung bisa (kuadran dari garis-bagi sudut).
\end{eulercomment}
\begin{eulerprompt}
>&factor(ratsimp(b/(1-sa2))); bisa &= %; $bisa
\end{eulerprompt}
\begin{eulercomment}
Mari kita periksa rumus ini untuk nilai-nilai Mesir kita.
\end{eulercomment}
\begin{eulerprompt}
>sqrt(mxmeval("at(bisa,[a=3^2,b=4^2])")), distance(A,P)
\end{eulerprompt}
\begin{euleroutput}
  4.21637021356
  4.21637021356
\end{euleroutput}
\begin{eulercomment}
Kita juga dapat menghitung P menggunakan rumus spread.
\end{eulercomment}
\begin{eulerprompt}
>py&=factor(ratsimp(sa2*bisa)); $py
\end{eulerprompt}
\begin{eulercomment}
Nilainya sama dengan yang kita dapatkan dengan rumus trigonometri.
\end{eulercomment}
\begin{eulerprompt}
>sqrt(mxmeval("at(py,[a=3^2,b=4^2])"))
\end{eulerprompt}
\begin{euleroutput}
  1.33333333333
\end{euleroutput}
\eulersubheading{Sudut Akord}
\begin{eulercomment}
Perhatikan situasi berikut.
\end{eulercomment}
\begin{eulerprompt}
>setPlotRange(1.2); ...
>color(1); plotCircle(circleWithCenter([0,0],1)); ...
>A:=[cos(1),sin(1)]; B:=[cos(2),sin(2)]; C:=[cos(6),sin(6)]; ...
>plotPoint(A,"A"); plotPoint(B,"B"); plotPoint(C,"C"); ...
>color(3); plotSegment(A,B,"c"); plotSegment(A,C,"b"); plotSegment(C,B,"a"); ...
>color(1); O:=[0,0];  plotPoint(O,"0"); ...
>plotSegment(A,O); plotSegment(B,O); plotSegment(C,O,"r"); ...
>insimg;
\end{eulerprompt}
\begin{eulercomment}
Kita dapat menggunakan Maxima untuk menyelesaikan rumus penyebaran
rangkap tiga untuk sudut-sudut di pusat O untuk r. Jadi kita
mendapatkan rumus untuk jari-jari kuadrat dari pericircle dalam hal
kuadrat dari sisi.

Kali ini, Maxima menghasilkan beberapa nol kompleks, yang kita
abaikan.
\end{eulercomment}
\begin{eulerprompt}
>&remvalue(a,b,c,r); // hapus nilai-nilai sebelumnya untuk perhitungan baru
>rabc &= rhs(solve(triplespread(spread(b,r,r),spread(a,r,r),spread(c,r,r)),r)[4]); $rabc
\end{eulerprompt}
\begin{eulercomment}
Kita dapat menjadikannya sebagai fungsi Euler.
\end{eulercomment}
\begin{eulerprompt}
>function periradius(a,b,c) &= rabc;
\end{eulerprompt}
\begin{eulercomment}
Mari kita periksa hasilnya untuk poin A,B,C.
\end{eulercomment}
\begin{eulerprompt}
>a:=quadrance(B,C); b:=quadrance(A,C); c:=quadrance(A,B);
\end{eulerprompt}
\begin{eulercomment}
Jari-jarinya memang 1.
\end{eulercomment}
\begin{eulerprompt}
>periradius(a,b,c)
\end{eulerprompt}
\begin{euleroutput}
  1
\end{euleroutput}
\begin{eulercomment}
Faktanya, spread CBA hanya bergantung pada b dan c. Ini adalah teorema
sudut chord.
\end{eulercomment}
\begin{eulerprompt}
>$spread(b,a,c)*rabc | ratsimp
\end{eulerprompt}
\begin{eulercomment}
Sebenarnya spreadnya adalah b/(4r), dan kita melihat bahwa sudut chord
dari chord b adalah setengah dari sudut pusat.
\end{eulercomment}
\begin{eulerprompt}
>$doublespread(b/(4*r))-spread(b,r,r) | ratsimp
\end{eulerprompt}
\eulersubheading{Contoh 6: Jarak Minimal pada Bidang}
\begin{eulercomment}
\end{eulercomment}
\eulersubheading{Catatan awal}
\begin{eulercomment}
Fungsi yang, ke titik M di bidang, menetapkan jarak AM antara titik
tetap A dan M, memiliki garis level yang agak sederhana: lingkaran
berpusat di A.
\end{eulercomment}
\begin{eulerprompt}
>&remvalue();
>A=[-1,-1];
>function d1(x,y):=sqrt((x-A[1])^2+(y-A[2])^2)
>fcontour("d1",xmin=-2,xmax=0,ymin=-2,ymax=0,hue=1, ...
>title="If you see ellipses, please set your window square"):
\end{eulerprompt}
\begin{eulercomment}
dan grafiknya juga agak sederhana: bagian atas kerucut:
\end{eulercomment}
\begin{eulerprompt}
>plot3d("d1",xmin=-2,xmax=0,ymin=-2,ymax=0):
\end{eulerprompt}
\begin{eulercomment}
Tentu saja minimal 0 dicapai di A.

\end{eulercomment}
\eulersubheading{Dua poin}
\begin{eulercomment}
Sekarang kita lihat fungsi MA+MB dimana A dan B adalah dua titik
(tetap). Ini adalah "fakta yang diketahui" bahwa kurva level adalah
elips, titik fokusnya adalah A dan B; kecuali untuk AB minimum yang
konstan pada segmen [AB]:
\end{eulercomment}
\begin{eulerprompt}
>B=[1,-1];
>function d2(x,y):=d1(x,y)+sqrt((x-B[1])^2+(y-B[2])^2)
>fcontour("d2",xmin=-2,xmax=2,ymin=-3,ymax=1,hue=1):
\end{eulerprompt}
\begin{eulercomment}
Grafiknya lebih menarik:
\end{eulercomment}
\begin{eulerprompt}
>plot3d("d2",xmin=-2,xmax=2,ymin=-3,ymax=1):
\end{eulerprompt}
\begin{eulercomment}
Pembatasan garis (AB) lebih terkenal:
\end{eulercomment}
\begin{eulerprompt}
>plot2d("abs(x+1)+abs(x-1)",xmin=-3,xmax=3):
\end{eulerprompt}
\eulersubheading{Tiga poin}
\begin{eulercomment}
Sekarang hal-hal yang kurang sederhana: Ini sedikit kurang terkenal
bahwa MA+MB+MC mencapai minimum pada satu titik pesawat tetapi untuk
menentukan itu kurang sederhana:

1) Jika salah satu sudut segitiga ABC lebih dari 120° (katakanlah di
A), maka minimum dicapai pada titik ini (misalnya AB+AC).

Contoh:
\end{eulercomment}
\begin{eulerprompt}
>C=[-4,1];
>function d3(x,y):=d2(x,y)+sqrt((x-C[1])^2+(y-C[2])^2)
>plot3d("d3",xmin=-5,xmax=3,ymin=-4,ymax=4);
>insimg;
>fcontour("d3",xmin=-4,xmax=1,ymin=-2,ymax=2,hue=1,title="The minimum is on A");
>P=(A_B_C_A)'; plot2d(P[1],P[2],add=1,color=12);
>insimg;
\end{eulerprompt}
\begin{eulercomment}
2) Tetapi jika semua sudut segitiga ABC kurang dari 120 °, minimumnya
adalah pada titik F di bagian dalam segitiga, yang merupakan
satu-satunya titik yang melihat sisi-sisi ABC dengan sudut yang sama
(maka masing-masing 120 ° ):
\end{eulercomment}
\begin{eulerprompt}
>C=[-0.5,1];
>plot3d("d3",xmin=-2,xmax=2,ymin=-2,ymax=2):
>fcontour("d3",xmin=-2,xmax=2,ymin=-2,ymax=2,hue=1,title="The Fermat point");
>P=(A_B_C_A)'; plot2d(P[1],P[2],add=1,color=12);
>insimg;
\end{eulerprompt}
\begin{eulercomment}
Merupakan kegiatan yang menarik untuk mewujudkan gambar di atas dengan
perangkat lunak geometri; misalnya, saya tahu soft yang ditulis di
Jawa yang memiliki instruksi "garis kontur" ...

Semua ini di atas telah ditemukan oleh seorang hakim Perancis bernama
Pierre de Fermat; dia menulis surat kepada dilettants lain seperti
pendeta Marin Mersenne dan Blaise Pascal yang bekerja di pajak
penghasilan. Jadi titik unik F sedemikian rupa sehingga FA+FB+FC
minimal, disebut titik Fermat segitiga. Tetapi tampaknya beberapa
tahun sebelumnya, Torriccelli Italia telah menemukan titik ini sebelum
Fermat melakukannya! Bagaimanapun tradisinya adalah mencatat poin ini
F...

\end{eulercomment}
\eulersubheading{Empat poin}
\begin{eulercomment}
Langkah selanjutnya adalah menambahkan 4 titik D dan mencoba
meminimalkan MA+MB+MC+MD; katakan bahwa Anda adalah operator TV kabel
dan ingin mencari di bidang mana Anda harus meletakkan antena sehingga
Anda dapat memberi makan empat desa dan menggunakan panjang kabel
sesedikit mungkin!
\end{eulercomment}
\begin{eulerprompt}
>D=[1,1];
>function d4(x,y):=d3(x,y)+sqrt((x-D[1])^2+(y-D[2])^2)
>plot3d("d4",xmin=-1.5,xmax=1.5,ymin=-1.5,ymax=1.5):
>fcontour("d4",xmin=-1.5,xmax=1.5,ymin=-1.5,ymax=1.5,hue=1);
>P=(A_B_C_D)'; plot2d(P[1],P[2],points=1,add=1,color=12);
>insimg;
\end{eulerprompt}
\begin{eulercomment}
Masih ada minimum dan tidak tercapai di salah satu simpul A, B, C atau
D:
\end{eulercomment}
\begin{eulerprompt}
>function f(x):=d4(x[1],x[2])
>neldermin("f",[0.2,0.2])
\end{eulerprompt}
\begin{euleroutput}
  [0.142858,  0.142857]
\end{euleroutput}
\begin{eulercomment}
Tampaknya dalam kasus ini, koordinat titik optimal adalah rasional
atau mendekati rasional...

Sekarang ABCD adalah persegi, kami berharap bahwa titik optimal akan
menjadi pusat ABCD:
\end{eulercomment}
\begin{eulerprompt}
>C=[-1,1];
>plot3d("d4",xmin=-1,xmax=1,ymin=-1,ymax=1):
>fcontour("d4",xmin=-1.5,xmax=1.5,ymin=-1.5,ymax=1.5,hue=1);
>P=(A_B_C_D)'; plot2d(P[1],P[2],add=1,color=12,points=1);
>insimg;
\end{eulerprompt}
\eulersubheading{Contoh 7: Bola Dandelin dengan Povray}
\begin{eulercomment}
Anda dapat menjalankan demonstrasi ini, jika Anda telah menginstal
Povray, dan pvengine.exe di jalur program.

Pertama kita hitung jari-jari bola.

Jika Anda melihat gambar di bawah, Anda melihat bahwa kita membutuhkan
dua lingkaran yang menyentuh dua garis yang membentuk kerucut, dan
satu garis yang membentuk bidang yang memotong kerucut.

Kami menggunakan file geometri.e dari Euler untuk ini.
\end{eulercomment}
\begin{eulerprompt}
>load geometry;
\end{eulerprompt}
\begin{eulercomment}
Pertama dua garis yang membentuk kerucut.
\end{eulercomment}
\begin{eulerprompt}
>g1 &= lineThrough([0,0],[1,a])
\end{eulerprompt}
\begin{euleroutput}
  
                               [- a, 1, 0]
  
\end{euleroutput}
\begin{eulerprompt}
>g2 &= lineThrough([0,0],[-1,a])
\end{eulerprompt}
\begin{euleroutput}
  
                              [- a, - 1, 0]
  
\end{euleroutput}
\begin{eulercomment}
Kemudian saya baris ketiga.
\end{eulercomment}
\begin{eulerprompt}
>g &= lineThrough([-1,0],[1,1])
\end{eulerprompt}
\begin{euleroutput}
  
                               [- 1, 2, 1]
  
\end{euleroutput}
\begin{eulercomment}
Kami merencanakan semuanya sejauh ini.
\end{eulercomment}
\begin{eulerprompt}
>setPlotRange(-1,1,0,2);
>color(black); plotLine(g(),"")
>a:=2; color(blue); plotLine(g1(),""), plotLine(g2(),""):
\end{eulerprompt}
\begin{eulercomment}
Sekarang kita ambil titik umum pada sumbu y.
\end{eulercomment}
\begin{eulerprompt}
>P &= [0,u]
\end{eulerprompt}
\begin{euleroutput}
  
                                  [0, u]
  
\end{euleroutput}
\begin{eulercomment}
Hitung jarak ke g1.
\end{eulercomment}
\begin{eulerprompt}
>d1 &= distance(P,projectToLine(P,g1)); $d1
\end{eulerprompt}
\begin{eulercomment}
Hitung jarak ke g.
\end{eulercomment}
\begin{eulerprompt}
>d &= distance(P,projectToLine(P,g)); $d
\end{eulerprompt}
\begin{eulercomment}
Dan temukan pusat kedua lingkaran yang jaraknya sama.
\end{eulercomment}
\begin{eulerprompt}
>sol &= solve(d1^2=d^2,u); $sol
\end{eulerprompt}
\begin{eulercomment}
Ada dua solusi.

Kami mengevaluasi solusi simbolis, dan menemukan kedua pusat, dan
kedua jarak.
\end{eulercomment}
\begin{eulerprompt}
>u := sol()
\end{eulerprompt}
\begin{euleroutput}
  [0.333333,  1]
\end{euleroutput}
\begin{eulerprompt}
>dd := d()
\end{eulerprompt}
\begin{euleroutput}
  [0.149071,  0.447214]
\end{euleroutput}
\begin{eulercomment}
Plot lingkaran ke dalam gambar.
\end{eulercomment}
\begin{eulerprompt}
>color(red);
>plotCircle(circleWithCenter([0,u[1]],dd[1]),"");
>plotCircle(circleWithCenter([0,u[2]],dd[2]),"");
>insimg;
\end{eulerprompt}
\eulersubheading{Plot dengan Povray}
\begin{eulercomment}
Selanjutnya kami merencanakan semuanya dengan Povray. Perhatikan bahwa
Anda mengubah perintah apa pun dalam urutan perintah Povray berikut,
dan menjalankan kembali semua perintah dengan Shift-Return.

Pertama kita memuat fungsi povray.
\end{eulercomment}
\begin{eulerprompt}
>load povray;
>defaultpovray="C:\(\backslash\)Program Files\(\backslash\)POV-Ray\(\backslash\)v3.7\(\backslash\)bin\(\backslash\)pvengine.exe"
\end{eulerprompt}
\begin{euleroutput}
  C:\(\backslash\)Program Files\(\backslash\)POV-Ray\(\backslash\)v3.7\(\backslash\)bin\(\backslash\)pvengine.exe
\end{euleroutput}
\begin{eulercomment}
Kami mengatur adegan dengan tepat.
\end{eulercomment}
\begin{eulerprompt}
>povstart(zoom=11,center=[0,0,0.5],height=10°,angle=140°);
\end{eulerprompt}
\begin{eulercomment}
Selanjutnya kita menulis dua bidang ke file Povray.
\end{eulercomment}
\begin{eulerprompt}
>writeln(povsphere([0,0,u[1]],dd[1],povlook(red)));
>writeln(povsphere([0,0,u[2]],dd[2],povlook(red)));
\end{eulerprompt}
\begin{eulercomment}
Dan kerucutnya, transparan.
\end{eulercomment}
\begin{eulerprompt}
>writeln(povcone([0,0,0],0,[0,0,a],1,povlook(lightgray,1)));
\end{eulerprompt}
\begin{eulercomment}
Kami menghasilkan bidang terbatas pada kerucut.
\end{eulercomment}
\begin{eulerprompt}
>gp=g();
>pc=povcone([0,0,0],0,[0,0,a],1,"");
>vp=[gp[1],0,gp[2]]; dp=gp[3];
>writeln(povplane(vp,dp,povlook(blue,0.5),pc));
\end{eulerprompt}
\begin{eulercomment}
Sekarang kita menghasilkan dua titik pada lingkaran, di mana bola
menyentuh kerucut.
\end{eulercomment}
\begin{eulerprompt}
>function turnz(v) := return [-v[2],v[1],v[3]]
>P1=projectToLine([0,u[1]],g1()); P1=turnz([P1[1],0,P1[2]]);
>writeln(povpoint(P1,povlook(yellow)));
>P2=projectToLine([0,u[2]],g1()); P2=turnz([P2[1],0,P2[2]]);
>writeln(povpoint(P2,povlook(yellow)));
\end{eulerprompt}
\begin{eulercomment}
Kemudian kami menghasilkan dua titik di mana bola menyentuh bidang.
Ini adalah fokus dari elips.
\end{eulercomment}
\begin{eulerprompt}
>P3=projectToLine([0,u[1]],g()); P3=[P3[1],0,P3[2]];
>writeln(povpoint(P3,povlook(yellow)));
>P4=projectToLine([0,u[2]],g()); P4=[P4[1],0,P4[2]];
>writeln(povpoint(P4,povlook(yellow)));
\end{eulerprompt}
\begin{eulercomment}
Selanjutnya kita hitung perpotongan P1P2 dengan bidang.
\end{eulercomment}
\begin{eulerprompt}
>t1=scalp(vp,P1)-dp; t2=scalp(vp,P2)-dp; P5=P1+t1/(t1-t2)*(P2-P1);
>writeln(povpoint(P5,povlook(yellow)));
\end{eulerprompt}
\begin{eulercomment}
Kami menghubungkan titik-titik dengan segmen garis.
\end{eulercomment}
\begin{eulerprompt}
>writeln(povsegment(P1,P2,povlook(yellow)));
>writeln(povsegment(P5,P3,povlook(yellow)));
>writeln(povsegment(P5,P4,povlook(yellow)));
\end{eulerprompt}
\begin{eulercomment}
Sekarang kita menghasilkan pita abu-abu, di mana bola menyentuh
kerucut.
\end{eulercomment}
\begin{eulerprompt}
>pcw=povcone([0,0,0],0,[0,0,a],1.01);
>pc1=povcylinder([0,0,P1[3]-defaultpointsize/2],[0,0,P1[3]+defaultpointsize/2],1);
>writeln(povintersection([pcw,pc1],povlook(gray)));
>pc2=povcylinder([0,0,P2[3]-defaultpointsize/2],[0,0,P2[3]+defaultpointsize/2],1);
>writeln(povintersection([pcw,pc2],povlook(gray)));
\end{eulerprompt}
\begin{eulercomment}
Mulai program Povray.
\end{eulercomment}
\begin{eulerprompt}
>povend();
\end{eulerprompt}
\begin{euleroutput}
  
\end{euleroutput}
\begin{eulercomment}
Untuk mendapatkan Anaglyph ini kita perlu memasukkan semuanya ke dalam
fungsi scene. Fungsi ini akan digunakan dua kali kemudian.
\end{eulercomment}
\begin{eulerprompt}
>function scene () ...
\end{eulerprompt}
\begin{eulerudf}
  global a,u,dd,g,g1,defaultpointsize;
  writeln(povsphere([0,0,u[1]],dd[1],povlook(red)));
  writeln(povsphere([0,0,u[2]],dd[2],povlook(red)));
  writeln(povcone([0,0,0],0,[0,0,a],1,povlook(lightgray,1)));
  gp=g();
  pc=povcone([0,0,0],0,[0,0,a],1,"");
  vp=[gp[1],0,gp[2]]; dp=gp[3];
  writeln(povplane(vp,dp,povlook(blue,0.5),pc));
  P1=projectToLine([0,u[1]],g1()); P1=turnz([P1[1],0,P1[2]]);
  writeln(povpoint(P1,povlook(yellow)));
  P2=projectToLine([0,u[2]],g1()); P2=turnz([P2[1],0,P2[2]]);
  writeln(povpoint(P2,povlook(yellow)));
  P3=projectToLine([0,u[1]],g()); P3=[P3[1],0,P3[2]];
  writeln(povpoint(P3,povlook(yellow)));
  P4=projectToLine([0,u[2]],g()); P4=[P4[1],0,P4[2]];
  writeln(povpoint(P4,povlook(yellow)));
  t1=scalp(vp,P1)-dp; t2=scalp(vp,P2)-dp; P5=P1+t1/(t1-t2)*(P2-P1);
  writeln(povpoint(P5,povlook(yellow)));
  writeln(povsegment(P1,P2,povlook(yellow)));
  writeln(povsegment(P5,P3,povlook(yellow)));
  writeln(povsegment(P5,P4,povlook(yellow)));
  pcw=povcone([0,0,0],0,[0,0,a],1.01);
  pc1=povcylinder([0,0,P1[3]-defaultpointsize/2],[0,0,P1[3]+defaultpointsize/2],1);
  writeln(povintersection([pcw,pc1],povlook(gray)));
  pc2=povcylinder([0,0,P2[3]-defaultpointsize/2],[0,0,P2[3]+defaultpointsize/2],1);
  writeln(povintersection([pcw,pc2],povlook(gray)));
  endfunction
\end{eulerudf}
\begin{eulercomment}
Anda membutuhkan kacamata merah/sian untuk menghargai efek berikut.
\end{eulercomment}
\begin{eulerprompt}
>povanaglyph("scene",zoom=11,center=[0,0,0.5],height=10°,angle=140°);
\end{eulerprompt}
\begin{euleroutput}
  exec:
      return _exec(program,param,dir,print,hidden,wait);
  povray:
      exec(program,params,defaulthome);
  Try "trace errors" to inspect local variables after errors.
  povanaglyph:
      povray(currentfile,w,h,aspect,exit); 
\end{euleroutput}
\eulersubheading{Contoh 8: Geometri Bumi}
\begin{eulercomment}
Dalam buku catatan ini, kami ingin melakukan beberapa perhitungan
sferis. Fungsi-fungsi tersebut terdapat dalam file "spherical.e" di
folder contoh. Kita perlu memuat file itu terlebih dahulu.
\end{eulercomment}
\begin{eulerprompt}
>load "spherical.e";
\end{eulerprompt}
\begin{eulercomment}
Untuk memasukkan posisi geografis, kami menggunakan vektor dengan dua
koordinat dalam radian (utara dan timur, nilai negatif untuk selatan
dan barat). Berikut koordinat Kampus FMIPA UNY.
\end{eulercomment}
\begin{eulerprompt}
>FMIPA=[rad(-7,-46.467),rad(110,23.05)]
\end{eulerprompt}
\begin{euleroutput}
  [-0.13569,  1.92657]
\end{euleroutput}
\begin{eulercomment}
Anda dapat mencetak posisi ini dengan sposprint (cetak posisi
spherical).
\end{eulercomment}
\begin{eulerprompt}
>sposprint(FMIPA) // posisi garis lintang dan garis bujur FMIPA UNY
\end{eulerprompt}
\begin{euleroutput}
  S 7°46.467' E 110°23.050'
\end{euleroutput}
\begin{eulercomment}
Mari kita tambahkan dua kota lagi, Solo dan Semarang.
\end{eulercomment}
\begin{eulerprompt}
>Solo=[rad(-7,-34.333),rad(110,49.683)]; Semarang=[rad(-6,-59.05),rad(110,24.533)];
>sposprint(Solo), sposprint(Semarang),
\end{eulerprompt}
\begin{euleroutput}
  S 7°34.333' E 110°49.683'
  S 6°59.050' E 110°24.533'
\end{euleroutput}
\begin{eulercomment}
Pertama kita menghitung vektor dari satu ke yang lain pada bola ideal.
Vektor ini [pos,jarak] dalam radian. Untuk menghitung jarak di bumi,
kita kalikan dengan jari-jari bumi pada garis lintang 7°.
\end{eulercomment}
\begin{eulerprompt}
>br=svector(FMIPA,Solo); degprint(br[1]), br[2]*rearth(7°)->km // perkiraan jarak FMIPA-Solo
\end{eulerprompt}
\begin{euleroutput}
  65°20'26.60''
  53.8945384608
\end{euleroutput}
\begin{eulercomment}
Ini adalah perkiraan yang baik. Rutinitas berikut menggunakan
perkiraan yang lebih baik. Pada jarak yang begitu pendek hasilnya
hampir sama.
\end{eulercomment}
\begin{eulerprompt}
>esdist(FMIPA,Semarang)->" km", // perkiraan jarak FMIPA-Semarang
\end{eulerprompt}
\begin{euleroutput}
  88.0114026318 km
\end{euleroutput}
\begin{eulercomment}
Ada fungsi untuk heading, dengan mempertimbangkan bentuk elips bumi.
Sekali lagi, kami mencetak dengan cara yang canggih.
\end{eulercomment}
\begin{eulerprompt}
>sdegprint(esdir(FMIPA,Solo))
\end{eulerprompt}
\begin{euleroutput}
       65.34°
\end{euleroutput}
\begin{eulercomment}
Sudut segitiga melebihi 180° pada bola.
\end{eulercomment}
\begin{eulerprompt}
>asum=sangle(Solo,FMIPA,Semarang)+sangle(FMIPA,Solo,Semarang)+sangle(FMIPA,Semarang,Solo); degprint(asum)
\end{eulerprompt}
\begin{euleroutput}
  180°0'10.77''
\end{euleroutput}
\begin{eulercomment}
Ini dapat digunakan untuk menghitung luas segitiga. Catatan: Untuk
segitiga kecil, ini tidak akurat karena kesalahan pengurangan dalam
asum-pi.
\end{eulercomment}
\begin{eulerprompt}
>(asum-pi)*rearth(48°)^2->" km^2", // perkiraan luas segitiga FMIPA-Solo-Semarang
\end{eulerprompt}
\begin{euleroutput}
  2116.02948749 km^2
\end{euleroutput}
\begin{eulercomment}
Ada fungsi untuk ini, yang menggunakan garis lintang rata-rata
segitiga untuk menghitung jari-jari bumi, dan menangani kesalahan
pembulatan untuk segitiga yang sangat kecil.
\end{eulercomment}
\begin{eulerprompt}
>esarea(Solo,FMIPA,Semarang)->" km^2", //perkiraan yang sama dengan fungsi esarea()
\end{eulerprompt}
\begin{euleroutput}
  2123.64310526 km^2
\end{euleroutput}
\begin{eulercomment}
Kita juga dapat menambahkan vektor ke posisi. Sebuah vektor berisi
heading dan jarak, keduanya dalam radian. Untuk mendapatkan vektor,
kami menggunakan vektor. Untuk menambahkan vektor ke posisi, kami
menggunakan vektor sadd.
\end{eulercomment}
\begin{eulerprompt}
>v=svector(FMIPA,Solo); sposprint(saddvector(FMIPA,v)), sposprint(Solo),
\end{eulerprompt}
\begin{euleroutput}
  S 7°34.333' E 110°49.683'
  S 7°34.333' E 110°49.683'
\end{euleroutput}
\begin{eulercomment}
Fungsi-fungsi ini mengasumsikan bola yang ideal. Hal yang sama di
bumi.
\end{eulercomment}
\begin{eulerprompt}
>sposprint(esadd(FMIPA,esdir(FMIPA,Solo),esdist(FMIPA,Solo))), sposprint(Solo),
\end{eulerprompt}
\begin{euleroutput}
  S 7°34.333' E 110°49.683'
  S 7°34.333' E 110°49.683'
\end{euleroutput}
\begin{eulercomment}
Mari kita beralih ke contoh yang lebih besar, Tugu Jogja dan Monas
Jakarta (menggunakan Google Earth untuk mencari koordinatnya).
\end{eulercomment}
\begin{eulerprompt}
>Tugu=[-7.7833°,110.3661°]; Monas=[-6.175°,106.811944°];
>sposprint(Tugu), sposprint(Monas)
\end{eulerprompt}
\begin{euleroutput}
  S 7°46.998' E 110°21.966'
  S 6°10.500' E 106°48.717'
\end{euleroutput}
\begin{eulercomment}
Menurut Google Earth, jaraknya adalah 429,66 km. Kami mendapatkan
pendekatan yang baik.
\end{eulercomment}
\begin{eulerprompt}
>esdist(Tugu,Monas)->" km", // perkiraan jarak Tugu Jogja - Monas Jakarta
\end{eulerprompt}
\begin{euleroutput}
  431.565659488 km
\end{euleroutput}
\begin{eulercomment}
Judulnya sama dengan judul yang dihitung di Google Earth.
\end{eulercomment}
\begin{eulerprompt}
>degprint(esdir(Tugu,Monas))
\end{eulerprompt}
\begin{euleroutput}
  294°17'2.85''
\end{euleroutput}
\begin{eulercomment}
Namun, kita tidak lagi mendapatkan posisi target yang tepat, jika kita
menambahkan heading dan jarak ke posisi semula. Hal ini terjadi,
karena kita tidak menghitung fungsi invers secara tepat, tetapi
mengambil perkiraan jari-jari bumi di sepanjang jalan.
\end{eulercomment}
\begin{eulerprompt}
>sposprint(esadd(Tugu,esdir(Tugu,Monas),esdist(Tugu,Monas)))
\end{eulerprompt}
\begin{euleroutput}
  S 6°10.500' E 106°48.717'
\end{euleroutput}
\begin{eulercomment}
Namun, kesalahannya tidak besar.
\end{eulercomment}
\begin{eulerprompt}
>sposprint(Monas),
\end{eulerprompt}
\begin{euleroutput}
  S 6°10.500' E 106°48.717'
\end{euleroutput}
\begin{eulercomment}
Tentu kita tidak bisa berlayar dengan tujuan yang sama dari satu
tujuan ke tujuan lainnya, jika kita ingin menempuh jalur terpendek.
Bayangkan, Anda terbang NE mulai dari titik mana pun di bumi. Kemudian
Anda akan berputar ke kutub utara. Lingkaran besar tidak mengikuti
heading yang konstan!

Perhitungan berikut menunjukkan bahwa kami jauh dari tujuan yang
benar, jika kami menggunakan pos yang sama selama perjalanan kami.
\end{eulercomment}
\begin{eulerprompt}
>dist=esdist(Tugu,Monas); hd=esdir(Tugu,Monas);
\end{eulerprompt}
\begin{eulercomment}
Sekarang kita tambahkan 10 kali sepersepuluh dari jarak, menggunakan
pos ke Monas, kita sampai di Tugu.
\end{eulercomment}
\begin{eulerprompt}
>p=Tugu; loop 1 to 10; p=esadd(p,hd,dist/10); end;
\end{eulerprompt}
\begin{eulercomment}
Hasilnya jauh.
\end{eulercomment}
\begin{eulerprompt}
>sposprint(p), skmprint(esdist(p,Monas))
\end{eulerprompt}
\begin{euleroutput}
  S 6°11.250' E 106°48.372'
       1.529km
\end{euleroutput}
\begin{eulercomment}
Sebagai contoh lain, mari kita ambil dua titik di bumi pada garis
lintang yang sama.
\end{eulercomment}
\begin{eulerprompt}
>P1=[30°,10°]; P2=[30°,50°];
\end{eulerprompt}
\begin{eulercomment}
Jalur terpendek dari P1 ke P2 bukanlah lingkaran garis lintang 30°,
melainkan jalur terpendek yang dimulai 10° lebih jauh ke utara di P1.
\end{eulercomment}
\begin{eulerprompt}
>sdegprint(esdir(P1,P2))
\end{eulerprompt}
\begin{euleroutput}
       79.69°
\end{euleroutput}
\begin{eulercomment}
Tapi, jika kita mengikuti pembacaan kompas ini, kita akan berputar ke
kutub utara! Jadi kita harus menyesuaikan arah kita di sepanjang
jalan. Untuk tujuan kasar, kami menyesuaikannya pada 1/10 dari total
jarak.
\end{eulercomment}
\begin{eulerprompt}
>p=P1;  dist=esdist(P1,P2); ...
>  loop 1 to 10; dir=esdir(p,P2); sdegprint(dir), p=esadd(p,dir,dist/10); end;
\end{eulerprompt}
\begin{euleroutput}
       79.69°
       81.67°
       83.71°
       85.78°
       87.89°
       90.00°
       92.12°
       94.22°
       96.29°
       98.33°
\end{euleroutput}
\begin{eulercomment}
Jaraknya tidak tepat, karena kita akan menambahkan sedikit kesalahan,
jika kita mengikuti heading yang sama terlalu lama.
\end{eulercomment}
\begin{eulerprompt}
>skmprint(esdist(p,P2))
\end{eulerprompt}
\begin{euleroutput}
       0.203km
\end{euleroutput}
\begin{eulercomment}
Kami mendapatkan perkiraan yang baik, jika kami menyesuaikan pos
setelah setiap 1/100 dari total jarak dari Tugu ke Monas.
\end{eulercomment}
\begin{eulerprompt}
>p=Tugu; dist=esdist(Tugu,Monas); ...
>  loop 1 to 100; p=esadd(p,esdir(p,Monas),dist/100); end;
>skmprint(esdist(p,Monas))
\end{eulerprompt}
\begin{euleroutput}
       0.000km
\end{euleroutput}
\begin{eulercomment}
Untuk keperluan navigasi, kita bisa mendapatkan urutan posisi GPS di
sepanjang lingkaran besar menuju Monas dengan fungsi navigasi.
\end{eulercomment}
\begin{eulerprompt}
>load spherical; v=navigate(Tugu,Monas,10); ...
>  loop 1 to rows(v); sposprint(v[#]), end;
\end{eulerprompt}
\begin{euleroutput}
  S 7°46.998' E 110°21.966'
  S 7°37.422' E 110°0.573'
  S 7°27.829' E 109°39.196'
  S 7°18.219' E 109°17.834'
  S 7°8.592' E 108°56.488'
  S 6°58.948' E 108°35.157'
  S 6°49.289' E 108°13.841'
  S 6°39.614' E 107°52.539'
  S 6°29.924' E 107°31.251'
  S 6°20.219' E 107°9.977'
  S 6°10.500' E 106°48.717'
\end{euleroutput}
\begin{eulercomment}
Kami menulis sebuah fungsi, yang memplot bumi, dua posisi, dan posisi
di antaranya.
\end{eulercomment}
\begin{eulerprompt}
>function testplot ...
\end{eulerprompt}
\begin{eulerudf}
  useglobal;
  plotearth;
  plotpos(Tugu,"Tugu Jogja"); plotpos(Monas,"Tugu Monas");
  plotposline(v);
  endfunction
\end{eulerudf}
\begin{eulercomment}
Sekarang rencanakan semuanya.
\end{eulercomment}
\begin{eulerprompt}
>plot3d("testplot",angle=25, height=6,>own,>user,zoom=4):
\end{eulerprompt}
\begin{eulercomment}
Atau gunakan plot3d untuk mendapatkan tampilan anaglyph. Ini terlihat
sangat bagus dengan kacamata merah/sian.
\end{eulercomment}
\begin{eulerprompt}
>plot3d("testplot",angle=25,height=6,distance=5,own=1,anaglyph=1,zoom=4):
\end{eulerprompt}
\eulersubheading{MENCOBA RUMUS-RUMUS PADA MATEI DI ATAS}
\eulersubheading{Geometri Simbolik}
\begin{eulerprompt}
>A &= [2,0]; B &= [0,2]; C &= [3,3]; // menentukan tiga titik A, B, C
>c &= lineThrough(B,C) // c=BC
\end{eulerprompt}
\begin{euleroutput}
  
                       lineThrough([0, 2], [3, 3])
  
\end{euleroutput}
\begin{eulerprompt}
>$getLineEquation(c,x,y), $solve(%,y) | expand // persamaan garis c
>h &= perpendicular(A,lineThrough(B,C)) // h melalui A tegak lurus BC
\end{eulerprompt}
\begin{euleroutput}
  
            perpendicular([2, 0], lineThrough([0, 2], [3, 3]))
  
\end{euleroutput}
\begin{eulerprompt}
>Q &= lineIntersection(c,h) // Q titik potong garis c=BC dan h
\end{eulerprompt}
\begin{euleroutput}
  
         lineIntersection(lineThrough([0, 2], [3, 3]), 
                     perpendicular([2, 0], lineThrough([0, 2], [3, 3])))
  
\end{euleroutput}
\begin{eulerprompt}
>$projectToLine(A,lineThrough(B,C)) // proyeksi A pada BC
>$distance(A,Q) // jarak AQ
>cc &= circleThrough(A,B,C); $cc // (titik pusat dan jari-jari) lingkaran melalui A, B, C
>r&=getCircleRadius(cc); $r , $float(r) // tampilkan nilai jari-jari
>$computeAngle(A,C,B) // nilai <ACB
>$solve(getLineEquation(angleBisector(A,C,B),x,y),y)[1] // persamaan garis bagi <ACB
>P &= lineIntersection(angleBisector(A,C,B),angleBisector(C,B,A)); $P // titik potong 2
>P() // 
\end{eulerprompt}
\begin{euleroutput}
  Function angleBisector not found.
  Try list ... to find functions!
  Error in expression: lineIntersection(angleBisector([2,0],[3,3],[0,2]),angleBisector([3,3],[0,2],[2,0]))
  Error in:
  P() //  ...
     ^
\end{euleroutput}
\eulersubheading{Garis dan Lingkaran yang berpotongan}
\begin{eulerprompt}
>A &:= [2,0]; c=circleWithCenter(A,4);
\end{eulerprompt}
\begin{euleroutput}
  Function circleWithCenter not found.
  Try list ... to find functions!
  Error in:
  A &:= [2,0]; c=circleWithCenter(A,4); ...
                                      ^
\end{euleroutput}
\begin{eulerprompt}
>B &:= [2,3]; C &:= [3,2]; l=lineThrough(B,C);
\end{eulerprompt}
\begin{euleroutput}
  Function lineThrough not found.
  Try list ... to find functions!
  Error in:
  B &:= [2,3]; C &:= [3,2]; l=lineThrough(B,C); ...
                                              ^
\end{euleroutput}
\begin{eulerprompt}
>setPlotRange(5); plotCircle(c); plotLine(l);
\end{eulerprompt}
\begin{euleroutput}
  Function setPlotRange not found.
  Try list ... to find functions!
  Error in:
  setPlotRange(5); plotCircle(c); plotLine(l); ...
                 ^
\end{euleroutput}
\begin{eulerprompt}
>\{P1,P2,f\}=lineCircleIntersections(l,c);
\end{eulerprompt}
\begin{euleroutput}
  Variable or function l not found.
  Error in:
  \{P1,P2,f\}=lineCircleIntersections(l,c); ...
                                     ^
\end{euleroutput}
\begin{eulerprompt}
>P1, P2,
\end{eulerprompt}
\begin{euleroutput}
  [0.523599,  0.174533]
  [0.523599,  0.872665]
\end{euleroutput}
\begin{eulerprompt}
>plotPoint(P1); plotPoint(P2):
\end{eulerprompt}
\begin{euleroutput}
  Function plotPoint not found.
  Try list ... to find functions!
  Error in:
  plotPoint(P1); plotPoint(P2): ...
               ^
\end{euleroutput}
\begin{eulercomment}
\end{eulercomment}
\begin{eulerprompt}
>c &= circleWithCenter(A,4) // lingkaran dengan pusat A jari-jari 4
\end{eulerprompt}
\begin{euleroutput}
  
                       circleWithCenter([2, 0], 4)
  
\end{euleroutput}
\begin{eulerprompt}
>l &= lineThrough(B,C) // garis l melalui B dan C
\end{eulerprompt}
\begin{euleroutput}
  
                       lineThrough([2, 3], [3, 2])
  
\end{euleroutput}
\begin{eulerprompt}
>$lineCircleIntersections(l,c) | radcan, // titik potong lingkaran c dan garis l
\end{eulerprompt}
\begin{eulercomment}
\end{eulercomment}
\begin{eulerprompt}
>C=A+normalize([-3,-4])*4; plotPoint(C); plotSegment(P1,C); plotSegment(P2,C);
\end{eulerprompt}
\begin{euleroutput}
  Function normalize not found.
  Try list ... to find functions!
  Error in:
  C=A+normalize([-3,-4])*4; plotPoint(C); plotSegment(P1,C); plo ...
                        ^
\end{euleroutput}
\begin{eulerprompt}
>degprint(computeAngle(P1,C,P2))
\end{eulerprompt}
\begin{euleroutput}
  Function computeAngle not found.
  Try list ... to find functions!
  Error in:
  degprint(computeAngle(P1,C,P2)) ...
                                ^
\end{euleroutput}
\begin{eulerprompt}
>C=A+normalize([-4,-5])*4; plotPoint(C); plotSegment(P1,C); plotSegment(P2,C);
\end{eulerprompt}
\begin{euleroutput}
  Function normalize not found.
  Try list ... to find functions!
  Error in:
  C=A+normalize([-4,-5])*4; plotPoint(C); plotSegment(P1,C); plo ...
                        ^
\end{euleroutput}
\begin{eulerprompt}
>degprint(computeAngle(P1,C,P2))
\end{eulerprompt}
\begin{euleroutput}
  Function computeAngle not found.
  Try list ... to find functions!
  Error in:
  degprint(computeAngle(P1,C,P2)) ...
                                ^
\end{euleroutput}
\begin{eulerprompt}
>insimg;
\end{eulerprompt}
\eulersubheading{ Garis Sumbu}
\begin{eulerprompt}
>A=[3,3]; B=[-2,-3];
>c1=circleWithCenter(A,distance(A,B));
\end{eulerprompt}
\begin{euleroutput}
  Function distance not found.
  Try list ... to find functions!
  Error in:
  c1=circleWithCenter(A,distance(A,B)); ...
                                     ^
\end{euleroutput}
\begin{eulerprompt}
>c2=circleWithCenter(B,distance(A,B));
\end{eulerprompt}
\begin{euleroutput}
  Function distance not found.
  Try list ... to find functions!
  Error in:
  c2=circleWithCenter(B,distance(A,B)); ...
                                     ^
\end{euleroutput}
\begin{eulerprompt}
>\{P1,P2,f\}=circleCircleIntersections(c1,c2);
\end{eulerprompt}
\begin{euleroutput}
  Variable or function c1 not found.
  Error in:
  \{P1,P2,f\}=circleCircleIntersections(c1,c2); ...
                                        ^
\end{euleroutput}
\begin{eulerprompt}
>l=lineThrough(P1,P2);
\end{eulerprompt}
\begin{euleroutput}
  Function lineThrough not found.
  Try list ... to find functions!
  Error in:
  l=lineThrough(P1,P2); ...
                      ^
\end{euleroutput}
\begin{eulerprompt}
>setPlotRange(5); plotCircle(c1); plotCircle(c2);
\end{eulerprompt}
\begin{euleroutput}
  Function setPlotRange not found.
  Try list ... to find functions!
  Error in:
  setPlotRange(5); plotCircle(c1); plotCircle(c2); ...
                 ^
\end{euleroutput}
\begin{eulerprompt}
>plotPoint(A); plotPoint(B); plotSegment(A,B); plotLine(l):
\end{eulerprompt}
\begin{euleroutput}
  Function plotPoint not found.
  Try list ... to find functions!
  Error in:
  plotPoint(A); plotPoint(B); plotSegment(A,B); plotLine(l): ...
              ^
\end{euleroutput}
\begin{eulerprompt}
>A &= [a1,a2]; B &= [b1,b2];
>c1 &= circleWithCenter(A,distance(A,B));
>c2 &= circleWithCenter(B,distance(A,B));
>P &= circleCircleIntersections(c1,c2); P1 &= P[1]; P2 &= P[2];
>g &= getLineEquation(lineThrough(P1,P2),x,y);
>$solve(g,y)
>$solve(getLineEquation(middlePerpendicular(A,B),x,y),y)
>h &=getLineEquation(lineThrough(A,B),x,y);
>$solve(h,y)
\end{eulerprompt}
\eulersubheading{Garis Euler dan Parabola}
\begin{eulerprompt}
>A::=[-1.5,-1.5]; B::=[3,0]; C::=[1.5,3];
>setPlotRange(3); plotPoint(A,"A"); plotPoint(B,"B"); plotPoint(C,"C");
\end{eulerprompt}
\begin{euleroutput}
  Function setPlotRange not found.
  Try list ... to find functions!
  Error in:
  setPlotRange(3); plotPoint(A,"A"); plotPoint(B,"B"); plotPoint ...
                 ^
\end{euleroutput}
\begin{eulercomment}
\end{eulercomment}
\begin{eulerprompt}
>plotSegment(A,B,""); plotSegment(B,C,""); plotSegment(C,A,""):
\end{eulerprompt}
\begin{euleroutput}
  Function plotSegment not found.
  Try list ... to find functions!
  Error in:
  plotSegment(A,B,""); plotSegment(B,C,""); plotSegment(C,A,""): ...
                     ^
\end{euleroutput}
\begin{eulerprompt}
>$areaTriangle(A,B,C)
\end{eulerprompt}
\begin{eulercomment}
\end{eulercomment}
\begin{eulerprompt}
>c &= lineThrough(A,B)
\end{eulerprompt}
\begin{euleroutput}
  
                                    3    3
                     lineThrough([- -, - -], [3, 0])
                                    2    2
  
\end{euleroutput}
\begin{eulercomment}
\end{eulercomment}
\begin{eulerprompt}
>$getLineEquation(c,x,y)
>$getHesseForm(c,x,y,C), $at(%,[x=C[1],y=C[2]])
\end{eulerprompt}
\begin{eulercomment}
\end{eulercomment}
\begin{eulerprompt}
>LL &= circleThrough(A,B,C); $getCircleEquation(LL,x,y)
>O &= getCircleCenter(LL); $O
>plotCircle(LL()); plotPoint(O(),"O"):
\end{eulerprompt}
\begin{euleroutput}
  Function circleThrough not found.
  Try list ... to find functions!
  Error in expression: circleThrough([-3/2,-3/2],[3,0],[3/2,3])
  Error in:
  plotCircle(LL()); plotPoint(O(),"O"): ...
                 ^
\end{euleroutput}
\begin{eulerprompt}
>H &= lineIntersection(perpendicular(A,lineThrough(C,B)),...
>  perpendicular(B,lineThrough(A,C))); $H
\end{eulerprompt}
\begin{eulercomment}
\end{eulercomment}
\begin{eulerprompt}
>el &= lineThrough(H,O); $getLineEquation(el,x,y)
\end{eulerprompt}
\begin{eulercomment}
\end{eulercomment}
\begin{eulerprompt}
>plotPoint(H(),"H"); plotLine(el(),"Garis Euler"):
\end{eulerprompt}
\begin{euleroutput}
  Function lineThrough not found.
  Try list ... to find functions!
  Error in expression: lineIntersection(perpendicular([-3/2,-3/2],lineThrough([3/2,3],[3,0])),perpendicular([3,0],lineThrough([-3/2,-3/2],[3/2,3])))
  Error in:
  plotPoint(H(),"H"); plotLine(el(),"Garis Euler"): ...
               ^
\end{euleroutput}
\begin{eulercomment}
\end{eulercomment}
\begin{eulerprompt}
>M &= (A+B+C)/3; $getLineEquation(el,x,y) with [x=M[1],y=M[2]]
>plotPoint(M(),"M"): // titik berat
\end{eulerprompt}
\begin{euleroutput}
  Function plotPoint not found.
  Try list ... to find functions!
  Error in:
  plotPoint(M(),"M"): // titik berat ...
                    ^
\end{euleroutput}
\begin{eulerprompt}
>$distance(M,H)/distance(M,O)|radcan
\end{eulerprompt}
\begin{eulercomment}
\end{eulercomment}
\begin{eulerprompt}
>$computeAngle(A,C,B), degprint(%())
\end{eulerprompt}
\begin{euleroutput}
  Function computeAngle not found.
  Try list ... to find functions!
  Error in expression: computeAngle([-3/2,-3/2],[3/2,3],[3,0])
  Error in:
   $computeAngle(A,C,B), degprint(%()) ...
                                    ^
\end{euleroutput}
\begin{eulerprompt}
>Q &= lineIntersection(angleBisector(A,C,B),angleBisector(C,B,A))|radcan; $Q
>r &= distance(Q,projectToLine(Q,lineThrough(A,B)))|ratsimp; $r
>LD &=  circleWithCenter(Q,r); // Lingkaran dalam
\end{eulerprompt}
\begin{eulercomment}
\end{eulercomment}
\begin{eulerprompt}
>color(5); plotCircle(LD()):
\end{eulerprompt}
\begin{euleroutput}
  Function angleBisector not found.
  Try list ... to find functions!
  Error in expression: circleWithCenter(lineIntersection(angleBisector([-3/2,-3/2],[3/2,3],[3,0]),angleBisector([3/2,3],[3,0],[-3/2,-3/2])),distance(lineIntersection(angleBisector([-3/2,-3/2],[3/2,3],[3,0]),angleBisector([3/2,3],[3,0],[-3/2,-3/2])),projectToLine(lineIntersection(angleBisector([-3/2,-3/2],[3/2,3],[3,0]),angleBisector([3/2,3],[3,0],[-3/2,-3/2])),lineThrough([-3/2,-3/2],[3,0]))))
  Error in:
  color(5); plotCircle(LD()): ...
                           ^
\end{euleroutput}
\eulersubheading{contoh lain dari materi trigonometri rasional}
\begin{eulerprompt}
>A&:=[2,3]; B&:=[5,4]; C&:=[0,5]; ...
>setPlotRange(-1,5,1,7); ...
>plotPoint(A,"A"); plotPoint(B,"B"); plotPoint(C,"C"); ...
>plotSegment(B,A,"c"); plotSegment(A,C,"b"); plotSegment(C,B,"a"); ...
>insimg;
\end{eulerprompt}
\begin{euleroutput}
  Function setPlotRange not found.
  Try list ... to find functions!
  Error in:
  ... ,3]; B&:=[5,4]; C&:=[0,5]; setPlotRange(-1,5,1,7); plotPoint(A ...
                                                       ^
\end{euleroutput}
\begin{eulerprompt}
>$distance(A,B)
>c &= quad(A,B); $c, b &= quad(A,C); $b, a &= quad(B,C); $a,
\end{eulerprompt}
\begin{eulercomment}
\end{eulercomment}
\begin{eulerprompt}
>wb &= computeAngle(A,B,C); $wb, $(wb/pi*180)()
\end{eulerprompt}
\begin{euleroutput}
  Function computeAngle not found.
  Try list ... to find functions!
  Error in expression: 180*computeAngle([2,3],[5,4],[0,5])/pi
  Error in:
  wb &= computeAngle(A,B,C); $wb, $(wb/pi*180)() ...
                                                ^
\end{euleroutput}
\begin{eulerprompt}
>$crosslaw(a,b,c,x), $solve(%,x), //(b+c-a)^=4b.c(1-x)
>sb &= spread(b,a,c); $sb
>$sin(computeAngle(A,B,C))^2
>ha &= c*sb; $ha
>$sqrt(ha)
>$sqrt(ha)*sqrt(a)/2
\end{eulerprompt}
\begin{eulercomment}
\end{eulercomment}
\begin{eulerprompt}
>$areaTriangle(B,A,C)
\end{eulerprompt}
\eulersubheading{Aturan penyebaran 3 kali lipat}
\begin{eulerprompt}
>setPlotRange(1); ...
>color(1); plotCircle(circleWithCenter([0,0],1)); ...
>A:=[cos(1),sin(1)]; B:=[cos(2),sin(2)]; C:=[cos(6),sin(6)]; ...
>plotPoint(A,"A"); plotPoint(B,"B"); plotPoint(C,"C"); ...
>color(3); plotSegment(A,B,"c"); plotSegment(A,C,"b"); plotSegment(C,B,"a"); ...
>color(1); O:=[0,0];  plotPoint(O,"0"); ...
>plotSegment(A,O); plotSegment(B,O); plotSegment(C,O,"r"); ...
>insimg;
\end{eulerprompt}
\begin{euleroutput}
  Function setPlotRange not found.
  Try list ... to find functions!
  Error in:
  setPlotRange(1); color(1); plotCircle(circleWithCenter([0,0],1 ...
                 ^
\end{euleroutput}
\begin{eulerprompt}
>&remvalue(a,b,c,r); // hapus nilai-nilai sebelumnya untuk perhitungan baru
>rabc &= rhs(solve(triplespread(spread(b,r,r),spread(a,r,r),spread(c,r,r)),r)[4]); $rabc
\end{eulerprompt}
\begin{euleroutput}
  Maxima said:
  part: invalid index of list or matrix.
   -- an error. To debug this try: debugmode(true);
  
  Error in:
  ... spread(b,r,r),spread(a,r,r),spread(c,r,r)),r)[4]); $rabc ...
                                                       ^
\end{euleroutput}
\begin{eulercomment}
\end{eulercomment}
\begin{eulerprompt}
>function periradius(a,b,c) &= rabc;
\end{eulerprompt}
\begin{eulercomment}
\end{eulercomment}
\begin{eulerprompt}
>a:=quadrance(B,C); b:=quadrance(A,C); c:=quadrance(A,B);
\end{eulerprompt}
\begin{euleroutput}
  Function quadrance not found.
  Try list ... to find functions!
  Error in:
  a:=quadrance(B,C); b:=quadrance(A,C); c:=quadrance(A,B); ...
                   ^
\end{euleroutput}
\begin{eulercomment}
\end{eulercomment}
\begin{eulerprompt}
>periradius(a,b,c)
\end{eulerprompt}
\begin{euleroutput}
  Variable rabc not found!
  Use global or local variables defined in function periradius.
  Try "trace errors" to inspect local variables after errors.
  periradius:
      useglobal; return rabc 
  Error in:
  periradius(a,b,c) ...
                   ^
\end{euleroutput}
\begin{eulerprompt}
>$spread(b,a,c)*rabc | ratsimp
>$doublespread(b/(4*r))-spread(b,r,r) | ratsimp
\end{eulerprompt}
\eulersubheading{Contoh 6: Jarak Minimal pada Bidang}
\begin{eulercomment}
\end{eulercomment}
\eulersubheading{Catatan awal}
\begin{eulercomment}
Fungsi yang, ke titik M di bidang, menetapkan jarak AM antara titik
tetap A dan M, memiliki garis level yang agak sederhana: lingkaran
berpusat di A.
\end{eulercomment}
\begin{eulerprompt}
>&remvalue();
>A=[-2,-2];
>function d1(x,y):=sqrt((x-A[1])^2+(y-A[2])^2)
>fcontour("d1",xmin=-2,xmax=0,ymin=-2,ymax=0,hue=1, ...
>title="If you see ellipses, please set your window square"):
\end{eulerprompt}
\begin{eulercomment}
dan grafiknya juga agak sederhana: bagian atas kerucut:
\end{eulercomment}
\begin{eulerprompt}
>plot3d("d1",xmin=-2,xmax=0,ymin=-2,ymax=0):
\end{eulerprompt}
\begin{eulercomment}
Ternyata setelah mencoba yang bisa hanya dengan memasukkan angka 1,
karena ketika memakai angka 2, plot tidak membentuk kerucut diatas.

\end{eulercomment}
\eulersubheading{Dua poin}
\begin{eulercomment}
\end{eulercomment}
\begin{eulerprompt}
>B=[2,-2];
>function d2(x,y):=d1(x,y)+sqrt((x-B[1])^2+(y-B[2])^2)
>fcontour("d2",xmin=-2,xmax=2,ymin=-3,ymax=1,hue=1):
\end{eulerprompt}
\begin{eulercomment}
Grafiknya lebih menarik:
\end{eulercomment}
\begin{eulerprompt}
>plot3d("d2",xmin=-2,xmax=2,ymin=-3,ymax=1):
\end{eulerprompt}
\begin{eulercomment}
Pembatasan garis (AB) lebih terkenal:
\end{eulercomment}
\begin{eulerprompt}
>plot2d("abs(x+1)+abs(x-1)",xmin=-3,xmax=3):
\end{eulerprompt}
\eulersubheading{Tiga poin}
\begin{eulercomment}
Contoh:
\end{eulercomment}
\begin{eulerprompt}
>C=[-3,2];
>function d3(x,y):=d2(x,y)+sqrt((x-C[1])^2+(y-C[2])^2)
>plot3d("d3",xmin=-5,xmax=3,ymin=-4,ymax=4);
>insimg;
>fcontour("d3",xmin=-4,xmax=1,ymin=-2,ymax=2,hue=1,title="The minimum is on A");
>P=(A_B_C_A)'; plot2d(P[1],P[2],add=1,color=12);
>insimg;
\end{eulerprompt}
\begin{eulercomment}
Tetapi jika semua sudut segitiga ABC kurang dari 120 °, minimumnya
adalah pada titik F di bagian dalam segitiga, yang merupakan
satu-satunya titik yang melihat sisi-sisi ABC dengan sudut yang sama
(maka masing-masing 120 ° ):
\end{eulercomment}
\begin{eulerprompt}
>C=[-1,2];
>plot3d("d3",xmin=-2,xmax=2,ymin=-2,ymax=2):
>fcontour("d3",xmin=-2,xmax=2,ymin=-2,ymax=2,hue=1,title="The Fermat point");
>P=(A_B_C_A)'; plot2d(P[1],P[2],add=1,color=12);
>insimg;
\end{eulerprompt}
\begin{eulercomment}
\end{eulercomment}
\eulersubheading{Empat poin}
\begin{eulercomment}
Langkah selanjutnya adalah menambahkan 4 titik D dan mencoba
meminimalkan MA+MB+MC+MD; katakan bahwa Anda adalah operator TV kabel
dan ingin mencari di bidang mana Anda harus meletakkan antena sehingga
Anda dapat memberi makan empat desa dan menggunakan panjang kabel
sesedikit mungkin!
\end{eulercomment}
\begin{eulerprompt}
>D=[2,21];
>function d4(x,y):=d3(x,y)+sqrt((x-D[1])^2+(y-D[2])^2)
>plot3d("d4",xmin=-1.5,xmax=1.5,ymin=-1.5,ymax=1.5):
>fcontour("d4",xmin=-1.5,xmax=1.5,ymin=-1.5,ymax=1.5,hue=1);
>P=(A_B_C_D)'; plot2d(P[1],P[2],points=1,add=1,color=12);
>insimg;
\end{eulerprompt}
\eulersubheading{Contoh 7: Bola Dandelin dengan Povray}
\begin{eulercomment}
\end{eulercomment}
\begin{eulerprompt}
>load geometry;
\end{eulerprompt}
\begin{eulercomment}
Pertama dua garis yang membentuk kerucut.
\end{eulercomment}
\begin{eulerprompt}
>g1 &= lineThrough([0,0],[2,a])
\end{eulerprompt}
\begin{euleroutput}
  
                               [- a, 2, 0]
  
\end{euleroutput}
\begin{eulerprompt}
>g2 &= lineThrough([0,0],[-2,a])
\end{eulerprompt}
\begin{euleroutput}
  
                              [- a, - 2, 0]
  
\end{euleroutput}
\begin{eulercomment}
\end{eulercomment}
\begin{eulerprompt}
>g &= lineThrough([-2,0],[2,2])
\end{eulerprompt}
\begin{euleroutput}
  
                               [- 2, 4, 4]
  
\end{euleroutput}
\begin{eulercomment}
\end{eulercomment}
\begin{eulerprompt}
>setPlotRange(-2,2,0,3);
>color(black); plotLine(g(),"")
>a:=2; color(blue); plotLine(g1(),""), plotLine(g2(),""):
\end{eulerprompt}
\begin{eulercomment}
Sekarang kita ambil titik umum pada sumbu y.
\end{eulercomment}
\begin{eulerprompt}
>P &= [0,u]
\end{eulerprompt}
\begin{euleroutput}
  
                                  [0, u]
  
\end{euleroutput}
\begin{eulercomment}
Hitung jarak ke g1.
\end{eulercomment}
\begin{eulerprompt}
>d1 &= distance(P,projectToLine(P,g1)); $d1
\end{eulerprompt}
\begin{eulercomment}
Hitung jarak ke g.
\end{eulercomment}
\begin{eulerprompt}
>d &= distance(P,projectToLine(P,g)); $d
\end{eulerprompt}
\begin{eulercomment}
Dan temukan pusat kedua lingkaran yang jaraknya sama.
\end{eulercomment}
\begin{eulerprompt}
>sol &= solve(d1^2=d^2,u); $sol
\end{eulerprompt}
\begin{eulercomment}
Ada dua solusi.

\end{eulercomment}
\begin{eulerprompt}
>u := sol()
\end{eulerprompt}
\begin{euleroutput}
  [0.558482,  4.77485]
\end{euleroutput}
\begin{eulerprompt}
>dd := d()
\end{eulerprompt}
\begin{euleroutput}
  [0.394906,  3.37633]
\end{euleroutput}
\begin{eulercomment}
Plot lingkaran ke dalam gambar.
\end{eulercomment}
\begin{eulerprompt}
>color(red);
>plotCircle(circleWithCenter([0,u[1]],dd[1]),"");
>plotCircle(circleWithCenter([0,u[2]],dd[2]),"");
>insimg;
\end{eulerprompt}
\eulersubheading{Latihan}
\begin{eulercomment}
1. Gambarlah segi-n beraturan jika diketahui titik pusat O, n, dan
jarak titik pusat ke titik-titik sudut segi-n tersebut (jari-jari
lingkaran luar segi-n), r.

Petunjuk:

- Besar sudut pusat yang menghadap masing-masing sisi segi-n adalah
(360/n).\\
- Titik-titik sudut segi-n merupakan perpotongan lingkaran luar segi-n
dan garis-garis yang melalui pusat dan saling membentuk sudut sebesar
kelipatan (360/n).\\
- Untuk n ganjil, pilih salah satu titik sudut adalah di atas.\\
- Untuk n genap, pilih 2 titik di kanan dan kiri lurus dengan titik
pusat.\\
- Anda dapat menggambar segi-3, 4, 5, 6, 7, dst beraturan.

Penyelesaian :
\end{eulercomment}
\begin{eulerprompt}
>load geometry
\end{eulerprompt}
\begin{euleroutput}
  Numerical and symbolic geometry.
\end{euleroutput}
\begin{eulerprompt}
>setPlotRange(-3.5,3.5,-3.5,3.5);
>A=[-2,-2]; plotPoint(A,"A");
>B=[2,-2]; plotPoint(B,"B");
>C=[0,3]; plotPoint(C,"C");
>plotSegment(A,B,"c");
>plotSegment(B,C,"a");
>plotSegment(A,C,"b");
>aspect(1):
>c=circleThrough(A,B,C);
>R=getCircleRadius(c);
>O=getCircleCenter(c);
>plotPoint(O,"O");
>l=angleBisector(A,C,B);
>color(2); plotLine(l); color(1);
>plotCircle(c,"Lingkaran luar segitiga ABC"):
\end{eulerprompt}
\begin{eulercomment}
2. Gambarlah suatu parabola yang melalui 3 titik yang diketahui.

Petunjuk:\\
- Misalkan persamaan parabolanya y= ax\textasciicircum{}2+bx+c.\\
- Substitusikan koordinat titik-titik yang diketahui ke persamaan
tersebut.\\
- Selesaikan SPL yang terbentuk untuk mendapatkan nilai-nilai a, b, c.

Penyelesaian :
\end{eulercomment}
\begin{eulerprompt}
>load geometry;
>setPlotRange(5); P=[2,0]; Q=[4,0]; R=[0,-4];
>plotPoint(P,"P"); plotPoint(Q,"Q"); plotPoint(R,"R"):
>sol &= solve([a+b=-c,16*a+4*b=-c,c=-4],[a,b,c])
\end{eulerprompt}
\begin{euleroutput}
  
                       [[a = - 1, b = 5, c = - 4]]
  
\end{euleroutput}
\begin{eulercomment}
Sehingga didapatkan nilai a = -1, b = 5 dan c = -4
\end{eulercomment}
\begin{eulerprompt}
>function y&=-x^2+5*x-4
\end{eulerprompt}
\begin{euleroutput}
  
                                 2
                              - x  + 5 x - 4
  
\end{euleroutput}
\begin{eulerprompt}
>plot2d("-x^2+5*x-4",-5,5,-5,5):
\end{eulerprompt}
\begin{eulercomment}
3. Gambarlah suatu segi-4 yang diketahui keempat titik sudutnya,
misalnya A, B, C, D.\\
\end{eulercomment}
\begin{eulerttcomment}
   - Tentukan apakah segi-4 tersebut merupakan segi-4 garis singgung
\end{eulerttcomment}
\begin{eulercomment}
(sisinya-sisintya merupakan garis singgung lingkaran yang sama yakni
lingkaran dalam segi-4 tersebut).\\
\end{eulercomment}
\begin{eulerttcomment}
   - Suatu segi-4 merupakan segi-4 garis singgung apabila keempat
\end{eulerttcomment}
\begin{eulercomment}
garis bagi sudutnya bertemu di satu titik.\\
\end{eulercomment}
\begin{eulerttcomment}
   - Jika segi-4 tersebut merupakan segi-4 garis singgung, gambar
\end{eulerttcomment}
\begin{eulercomment}
lingkaran dalamnya.\\
\end{eulercomment}
\begin{eulerttcomment}
   - Tunjukkan bahwa syarat suatu segi-4 merupakan segi-4 garis
\end{eulerttcomment}
\begin{eulercomment}
singgung apabila hasil kali panjang sisi-sisi yang berhadapan sama.

Penyelesaian :
\end{eulercomment}
\begin{eulerprompt}
>load geometry
\end{eulerprompt}
\begin{euleroutput}
  Numerical and symbolic geometry.
\end{euleroutput}
\begin{eulerprompt}
>setPlotRange(-4.5,4.5,-4.5,4.5);
>A=[-3,-3]; plotPoint(A,"A");
>B=[3,-3]; plotPoint(B,"B");
>C=[3,3]; plotPoint(C,"C");
>D=[-3,3]; plotPoint(D,"D");
>plotSegment(A,B,"");
>plotSegment(B,C,"");
>plotSegment(C,D,"");
>plotSegment(A,D,"");
>aspect(1):
>l=angleBisector(A,B,C);
>m=angleBisector(B,C,D);
>P=lineIntersection(l,m);
>color(5); plotLine(l); plotLine(m); color(1);
>plotPoint(P,"P"):
\end{eulerprompt}
\begin{eulercomment}
Dari gambar diatas terlihat bahwa keempat garis bagi sudutnya bertemu
di satu titik yaitu titik P.
\end{eulercomment}
\begin{eulerprompt}
>r=norm(P-projectToLine(P,lineThrough(A,B)));
>plotCircle(circleWithCenter(P,r),"Lingkaran dalam segiempat ABCD"):
\end{eulerprompt}
\begin{eulercomment}
Dari gambar diatas, terlihat bahwa sisi-sisinya merupakan garis
singgung lingkaran yang sama yaitu lingkaran dalam segiempat.\\
Akan ditunjukkan bahwa hasil kali panjang sisi-sisi yang berhadapan
sama.
\end{eulercomment}
\begin{eulerprompt}
>AB=norm(A-B) //panjang sisi AB
\end{eulerprompt}
\begin{euleroutput}
  6
\end{euleroutput}
\begin{eulerprompt}
>CD=norm(C-D) //panjang sisi CD
\end{eulerprompt}
\begin{euleroutput}
  6
\end{euleroutput}
\begin{eulerprompt}
>AD=norm(A-D) //panjang sisi AD
\end{eulerprompt}
\begin{euleroutput}
  6
\end{euleroutput}
\begin{eulerprompt}
>BC=norm(B-C) //panjang sisi BC
\end{eulerprompt}
\begin{euleroutput}
  6
\end{euleroutput}
\begin{eulerprompt}
>AB.CD
\end{eulerprompt}
\begin{euleroutput}
  36
\end{euleroutput}
\begin{eulerprompt}
>AD.BC
\end{eulerprompt}
\begin{euleroutput}
  36
\end{euleroutput}
\begin{eulercomment}
Terbukti bahwa hasil kali panjang sisi-sisi yang berhadapan sama yaitu
36. Jadi dapat dipastikan bahwa segiempat tersebut merupakan segiempat
garis singgung.


4. Gambarlah suatu ellips jika diketahui kedua titik fokusnya,
misalnya P dan Q. Ingat ellips dengan fokus P dan Q adalah tempat
kedudukan titik-titik yang jumlah jarak ke P dan ke Q selalu sama
(konstan).

Penyelesaian :\\
Diketahui kedua titik fokus P = [-1,-1] dan Q = [1,-1]
\end{eulercomment}
\begin{eulerprompt}
>P=[-1,-1]; Q=[1,-1];
>function d1(x,y):=sqrt((x-P[1])^2+(y-P[2])^2)
>Q=[1,-1]; function d2(x,y):=sqrt((x-P[1])^2+(y-P[2])^2)+sqrt((x-Q[1])^2+(y-Q[2])^2)
>fcontour("d2",xmin=-2,xmax=2,ymin=-3,ymax=1,hue=1):
\end{eulerprompt}
\begin{eulercomment}
Grafik yang lebih menarik
\end{eulercomment}
\begin{eulerprompt}
>plot3d("d2",xmin=-2,xmax=2,ymin=-3,ymax=1):
\end{eulerprompt}
\begin{eulercomment}
Batasan ke garis PQ
\end{eulercomment}
\begin{eulerprompt}
>plot2d("abs(x+1)+abs(x-1)",xmin=-3,xmax=3):
\end{eulerprompt}
\begin{eulercomment}
5. Gambarlah suatu hiperbola jika diketahui kedua titik fokusnya,
misalnya P dan Q. Ingat ellips dengan fokus P dan Q adalah tempat
kedudukan titik-titik yang selisih jarak ke P dan ke Q selalu sama
(konstan).

Penyelesaian :
\end{eulercomment}
\begin{eulerprompt}
>P=[-1,-1]; Q=[1,-1];
>function d1(x,y):=sqrt((x-p[1])^2+(y-p[2])^2)
>Q=[1,-1]; function d2(x,y):=sqrt((x-P[1])^2+(y-P[2])^2)+sqrt((x+Q[1])^2+(y+Q[2])^2)
>fcontour("d2",xmin=-2,xmax=2,ymin=-3,ymax=1,hue=1):
\end{eulerprompt}
\begin{eulercomment}
Grafik yang lebih menarik
\end{eulercomment}
\begin{eulerprompt}
>plot3d("d2",xmin=-2,xmax=2,ymin=-3,ymax=1):
>plot2d("abs(x+1)+abs(x-1)",xmin=-3,xmax=3):
\end{eulerprompt}
\end{eulernotebook}
\end{document}


\newpage
\chapter{KB Pekan 10; Menggunakan EMT untuk Statistika}
\documentclass[a4paper,10pt]{article}
\usepackage{eumat}

\begin{document}
\begin{eulernotebook}
\eulerheading{EMT untuk Statistika}
\begin{eulercomment}
Nama : Rasyid Shalahuddin\\
NIM  : 22305144016\\
Kelas: Matematika E 2022\\
\end{eulercomment}
\eulersubheading{}
\begin{eulercomment}
Di notebook ini, kami mendemonstrasikan plot statistik utama, tes dan
distribusi di Euler.

Mari kita mulai dengan beberapa statistik deskriptif. Ini bukan
pengantar statistik. Jadi, Anda mungkin memerlukan latar belakang
untuk memahami detailnya.

Asumsikan pengukuran berikut. Kami ingin menghitung nilai rata-rata
dan standar deviasi yang diukur.
\end{eulercomment}
\begin{eulerprompt}
>M=[1005,1030,997,980,1008,1000,978,1004,998,997]; ...
>mean(M), dev(M),
\end{eulerprompt}
\begin{euleroutput}
  999.7
  14.5682302746
\end{euleroutput}
\begin{eulercomment}
Kita dapat memplot plot box-and-whiskers untuk data tersebut. Dalam
kasus kami, tidak ada garis luar.
\end{eulercomment}
\begin{eulerprompt}
>boxplot(M):
\end{eulerprompt}
\eulerimg{29}{images/EMTStatistika_Rasyid Shalahuddin_22305144016-001.png}
\begin{eulercomment}
Kita menghitung probabilitas bahwa suatu nilai lebih besar dari 1005,
dengan asumsi nilai terukur dan distribusi normal.

Semua fungsi untuk distribusi di Euler diakhiri dengan ...dis dan
menghitung distribusi probabilitas kumulatif (CPF).


\end{eulercomment}
\begin{eulerformula}
\[
\text{normaldis(x,m,d)}=\int_{-\infty}^x \frac{1}{d\sqrt{2\pi}}e^{-\frac{1}{2}(\frac{t-m}{d})^2}\ dt.
\]
\end{eulerformula}
\begin{eulercomment}
Kami mencetak hasilnya dalam \% dengan akurasi 2 digit menggunakan
fungsi cetak.
\end{eulercomment}
\begin{eulerprompt}
>print((1-normaldis(1005,mean(M),dev(M)))*100,2,unit=" %")
\end{eulerprompt}
\begin{euleroutput}
       35.80 %
\end{euleroutput}
\begin{eulercomment}
Untuk contoh berikut, kita mengasumsikan jumlah pria berikut dalam
rentang ukuran tertentu.
\end{eulercomment}
\begin{eulerprompt}
>r=155.5:4:187.5; v=[20,70,135,170,138,71,32,8];
\end{eulerprompt}
\begin{eulercomment}
Berikut adalah plot distribusinya.
\end{eulercomment}
\begin{eulerprompt}
>plot2d(r,v,a=150,b=200,c=0,d=190,bar=1,style="\(\backslash\)/"):
\end{eulerprompt}
\eulerimg{29}{images/EMTStatistika_Rasyid Shalahuddin_22305144016-002.png}
\begin{eulercomment}
Kita bisa memasukkan data mentah tersebut ke dalam tabel.

Tabel adalah metode untuk menyimpan data statistik. Tabel kita harus
berisi tiga kolom: Mulai kisaran, akhir kisaran, jumlah laki-laki
dalam kisaran.

Tabel dapat dicetak dengan header. Kami menggunakan vektor string
untuk mengatur header.
\end{eulercomment}
\begin{eulerprompt}
>T:=r[1:8]' | r[2:9]' | v'; writetable(T,labc=["from","to","count"])
\end{eulerprompt}
\begin{euleroutput}
        from        to     count
       155.5     159.5        20
       159.5     163.5        70
       163.5     167.5       135
       167.5     171.5       170
       171.5     175.5       138
       175.5     179.5        71
       179.5     183.5        32
       183.5     187.5         8
\end{euleroutput}
\begin{eulercomment}
Jika kita membutuhkan nilai rata-rata dan statistik ukuran lainnya,
kita perlu menghitung titik tengah rentang. Kita dapat menggunakan dua
kolom pertama dari tabel kita untuk ini.

Simbol "\textbar{}" digunakan untuk memisahkan kolom, fungsi "writetable"
digunakan untuk menulis tabel, dengan pilihan "labc" adalah menentukan
header kolom.
\end{eulercomment}
\begin{eulerprompt}
>(T[,1]+T[,2])/2 // the midpoint of each interval
\end{eulerprompt}
\begin{euleroutput}
          157.5 
          161.5 
          165.5 
          169.5 
          173.5 
          177.5 
          181.5 
          185.5 
\end{euleroutput}
\begin{eulercomment}
Tapi lebih mudah, melipat rentang dengan vektor [1/2, 1/2].
\end{eulercomment}
\begin{eulerprompt}
>M=fold(r,[1,0.5])
\end{eulerprompt}
\begin{euleroutput}
  [235.25,  241.25,  247.25,  253.25,  259.25,  265.25,  271.25,  277.25]
\end{euleroutput}
\begin{eulercomment}
Sekarang kita dapat menghitung mean dan deviasi sampel dengan
frekuensi yang diberikan.
\end{eulercomment}
\begin{eulerprompt}
>\{m,d\}=meandev(M,v); m, d,
\end{eulerprompt}
\begin{euleroutput}
  253.930124224
  8.93123075067
\end{euleroutput}
\begin{eulercomment}
Mari kita tambahkan distribusi normal nilai ke plot batang di atas.
Rumus distribusi normal dengan mean m dan standar deviasi d adalah:

\end{eulercomment}
\begin{eulerformula}
\[
y=\frac{1}{d\sqrt{2\pi}}e^{\frac{-(x-m)^2}{2d^2}}.
\]
\end{eulerformula}
\begin{eulercomment}
Karena nilainya antara 0 dan 1, untuk memplotnya pada diagram batang
harus dikalikan dengan 4 kali jumlah data.
\end{eulercomment}
\begin{eulerprompt}
>plot2d("qnormal(x,m,d)*sum(v)*4", ...
>  xmin=min(r),xmax=max(r),thickness=3,add=1):
\end{eulerprompt}
\eulerimg{29}{images/EMTStatistika_Rasyid Shalahuddin_22305144016-003.png}
\eulerheading{Tabel}
\begin{eulercomment}
Dalam direktori buku catatan ini Anda menemukan file dengan tabel.
Data tersebut merupakan hasil survei. Berikut adalah empat baris
pertama file. Data tersebut berasal dari buku online Jerman
"Einführung in die Statistik mit R" oleh A. Handl.
\end{eulercomment}
\begin{eulerprompt}
>printfile("table.dat",4);
\end{eulerprompt}
\begin{euleroutput}
  Could not open the file
  table.dat
  for reading!
  Try "trace errors" to inspect local variables after errors.
  printfile:
      open(filename,"r");
\end{euleroutput}
\begin{eulercomment}
Tabel berisi 7 kolom angka atau token (string). Kami ingin membaca
tabel dari file. Pertama, kami menggunakan terjemahan kami sendiri
untuk token.

Untuk ini, kami mendefinisikan set token. Fungsi strtokens ()
mendapatkan vektor string token dari string tertentu.
\end{eulercomment}
\begin{eulerprompt}
>mf:=["m","f"]; yn:=["y","n"]; ev:=strtokens("g vg m b vb");
\end{eulerprompt}
\begin{eulercomment}
Sekarang kita membaca tabel dengan terjemahan ini.

Argumen tok2, tok4, dll. Adalah terjemahan dari kolom tabel. Argumen
ini tidak ada dalam daftar parameter readtable(), jadi Anda perlu
memberinya ":=".
\end{eulercomment}
\begin{eulerprompt}
>\{MT,hd\}=readtable("table.dat",tok2:=mf,tok4:=yn,tok5:=ev,tok7:=yn);
\end{eulerprompt}
\begin{euleroutput}
  Could not open the file
  table.dat
  for reading!
  Try "trace errors" to inspect local variables after errors.
  readtable:
      if filename!=none then open(filename,"r"); endif;
\end{euleroutput}
\begin{eulerprompt}
>load over statistics;
\end{eulerprompt}
\begin{eulercomment}
Untuk mencetak, kita perlu menentukan set token yang sama. Kita
mencetak empat baris pertama saja.
\end{eulercomment}
\begin{eulerprompt}
>writetable(MT[1:4],labc=hd,wc=5,tok2:=mf,tok4:=yn,tok5:=ev,tok7:=yn);
\end{eulerprompt}
\begin{euleroutput}
  MT is not a variable!
  Error in:
  writetable(MT[1:4],labc=hd,wc=5,tok2:=mf,tok4:=yn,tok5:=ev,tok ...
                    ^
\end{euleroutput}
\begin{eulercomment}
Titik "." mewakili nilai-nilai yang tidak tersedia.

Jika kita tidak ingin menentukan token untuk terjemahan terlebih
dahulu, kita hanya perlu menentukan, kolom mana yang berisi token dan
bukan angka.
\end{eulercomment}
\begin{eulerprompt}
>ctok=[2,4,5,7]; \{MT,hd,tok\}=readtable("table.dat",ctok=ctok);
\end{eulerprompt}
\begin{euleroutput}
  Could not open the file
  table.dat
  for reading!
  Try "trace errors" to inspect local variables after errors.
  readtable:
      if filename!=none then open(filename,"r"); endif;
\end{euleroutput}
\begin{eulercomment}
Fungsi readtable() sekarang mengembalikan satu set token.
\end{eulercomment}
\begin{eulerprompt}
>tok
\end{eulerprompt}
\begin{euleroutput}
  Variable tok not found!
  Error in:
  tok ...
     ^
\end{euleroutput}
\begin{eulercomment}
Tabel berisi entri dari file dengan token yang diterjemahkan menjadi
angka.

String khusus NA="." diartikan sebagai "Tidak Tersedia", dan
mendapatkan NAN (bukan angka) di tabel. Terjemahan ini dapat diubah
dengan parameter NA, dan NAval.
\end{eulercomment}
\begin{eulerprompt}
>MT[1]
\end{eulerprompt}
\begin{euleroutput}
  MT is not a variable!
  Error in:
  MT[1] ...
       ^
\end{euleroutput}
\begin{eulercomment}
Berikut adalah isi tabel dengan bilangan yang belum diterjemahkan.
\end{eulercomment}
\begin{eulerprompt}
>writetable(MT,wc=5)
\end{eulerprompt}
\begin{euleroutput}
  Variable or function MT not found.
  Error in:
  writetable(MT,wc=5) ...
               ^
\end{euleroutput}
\begin{eulercomment}
Untuk kenyamanan, Anda bisa memasukkan keluaran readtable() ke dalam
daftar.
\end{eulercomment}
\begin{eulerprompt}
>Table=\{\{readtable("table.dat",ctok=ctok)\}\};
\end{eulerprompt}
\begin{euleroutput}
  Could not open the file
  table.dat
  for reading!
  Try "trace errors" to inspect local variables after errors.
  readtable:
      if filename!=none then open(filename,"r"); endif;
\end{euleroutput}
\begin{eulercomment}
Dengan menggunakan kolom token yang sama dan token dibaca dari file,
kita dapat mencetak tabel. Kita dapat menentukan ctok, tok, dll. Atau
menggunakan Tabel daftar.
\end{eulercomment}
\begin{eulerprompt}
>writetable(Table,ctok=ctok,wc=5);
\end{eulerprompt}
\begin{euleroutput}
  Variable or function Table not found.
  Error in:
  writetable(Table,ctok=ctok,wc=5); ...
                  ^
\end{euleroutput}
\begin{eulercomment}
Fungsi tablecol() mengembalikan nilai kolom tabel, melewatkan baris
apa pun dengan nilai NAN ("." Dalam file), dan indeks kolom, yang
berisi nilai ini.
\end{eulercomment}
\begin{eulerprompt}
>\{c,i\}=tablecol(MT,[5,6]);
\end{eulerprompt}
\begin{euleroutput}
  Variable or function MT not found.
  Error in:
  \{c,i\}=tablecol(MT,[5,6]); ...
                   ^
\end{euleroutput}
\begin{eulercomment}
Kita dapat menggunakan ini untuk mengekstrak kolom dari tabel untuk
tabel baru
\end{eulercomment}
\begin{eulerprompt}
>j=[1,5,6]; writetable(MT[i,j],labc=hd[j],ctok=[2],tok=tok)
\end{eulerprompt}
\begin{euleroutput}
  Variable or function i not found.
  Error in:
  j=[1,5,6]; writetable(MT[i,j],labc=hd[j],ctok=[2],tok=tok) ...
                            ^
\end{euleroutput}
\begin{eulercomment}
Tentu saja, kita perlu mengekstrak tabel itu sendiri dari Daftar Tabel
dalam kasus ini.
\end{eulercomment}
\begin{eulerprompt}
>MT=Table[1];
\end{eulerprompt}
\begin{eulercomment}
Tentu saja, kami juga dapat menggunakannya untuk menentukan nilai
rata-rata kolom atau nilai statistik lainnya.
\end{eulercomment}
\begin{eulerprompt}
>mean(tablecol(MT,6))
\end{eulerprompt}
\begin{euleroutput}
  2.175
\end{euleroutput}
\begin{eulercomment}
Fungsi getstatistics() mengembalikan elemen dalam vektor, dan
jumlahnya. Kita menerapkannya ke nilai "m" dan "f" di kolom kedua
tabel kami.
\end{eulercomment}
\begin{eulerprompt}
>\{xu,count\}=getstatistics(tablecol(MT,2)); xu, count,
\end{eulerprompt}
\begin{euleroutput}
  [1,  3]
  [12,  13]
\end{euleroutput}
\begin{eulercomment}
Kita dapat mencetak hasilnya di tabel baru.
\end{eulercomment}
\begin{eulerprompt}
>writetable(count',labr=tok[xu])
\end{eulerprompt}
\begin{euleroutput}
           m        12
           f        13
\end{euleroutput}
\begin{eulercomment}
Fungsi selecttable() mengembalikan tabel baru dengan nilai dalam satu
kolom yang dipilih dari vektor indeks. Pertama kita mencari indeks
dari dua nilai kita di tabel token.
\end{eulercomment}
\begin{eulerprompt}
>v:=indexof(tok,["g","vg"])
\end{eulerprompt}
\begin{euleroutput}
  [5,  6]
\end{euleroutput}
\begin{eulercomment}
Sekarang kita dapat memilih baris tabel, yang memiliki salah satu
nilai dalam v di baris ke-5
\end{eulercomment}
\begin{eulerprompt}
>MT1:=MT[selectrows(MT,5,v)]; i:=sortedrows(MT1,5);
\end{eulerprompt}
\begin{eulercomment}
Sekarang kita dapat mencetak tabel, dengan nilai yang diekstrasi dan
diurutkan di kolom-5.
\end{eulercomment}
\begin{eulerprompt}
>writetable(MT1[i],labc=hd,ctok=ctok,tok=tok,wc=7);
\end{eulerprompt}
\begin{euleroutput}
   Person    Sex    Age Titanic Evaluation    Tip Problem
        2      f     23       y          g    1.8       n
        3      f     26       y          g    1.8       y
        6      m     28       y          g    2.8       y
       18      m     38       y          g      .       n
       16      m     26       y          g    2.8       n
       15      f     31       y          g    0.8       n
       12      m     32       y          g    1.8       n
       23      f     38       y          g    2.8       n
       14      f     25       y          g    1.8       y
        9      f     24       y         vg    1.8       y
        7      f     31       y         vg    2.8       n
       20      f     28       y         vg    1.8       n
       22      f     28       y         vg    1.8       y
       13      m     29       y         vg    1.8       y
       11      f     23       y         vg    1.8       y
\end{euleroutput}
\begin{eulercomment}
Untuk statistik berikutnya, kami ingin menghubungkan dua kolom dari
tabel. Jadi kami mengekstrak kolom 2 dan 4 dan mengurutkan tabel.
\end{eulercomment}
\begin{eulerprompt}
>i=sortedrows(,[2,4]);  ...
>  writetable(tablecol(MT[i],[2,4])',ctok=[1,2],tok=tok)
\end{eulerprompt}
\begin{euleroutput}
  Variable  not found!
  Error in:
  i=sortedrows(,[2,4]);    writetable(tablecol(MT[i],[2,4])',cto ...
                      ^
\end{euleroutput}
\begin{eulercomment}
Dengan getstatistics(), kita juga bisa menghubungkan hitungan dalam
dua kolom tabel satu sama lain.
\end{eulercomment}
\begin{eulerprompt}
>MT24=tablecol(MT,[2,4]); ...
>\{xu1,xu2,count\}=getstatistics(MT24[1],MT24[2]); ...
>writetable(count,labr=tok[xu1],labc=tok[xu2])
\end{eulerprompt}
\begin{euleroutput}
                     n         y
           m         7         5
           f         1        12
\end{euleroutput}
\begin{eulercomment}
Tabel dapat ditulis ke file.
\end{eulercomment}
\begin{eulerprompt}
>filename="test.dat"; ...
>writetable(count,labr=tok[xu1],labc=tok[xu2],file=filename);
\end{eulerprompt}
\begin{eulercomment}
Kemudian kita dapat membaca tabel dari file tersebut.
\end{eulercomment}
\begin{eulerprompt}
>\{MT2,hd,tok2,hdr\}=readtable(filename,>clabs,>rlabs); ...
>writetable(MT2,labr=hdr,labc=hd)
\end{eulerprompt}
\begin{euleroutput}
                     n         y
           m         7         5
           f         1        12
\end{euleroutput}
\begin{eulercomment}
Dan hapus file tersebut.
\end{eulercomment}
\begin{eulerprompt}
>fileremove(filename);
\end{eulerprompt}
\eulerheading{Distribusi}
\begin{eulercomment}
Dengan plot2d, terdapat metode yang sangat mudah untuk memplot sebaran
data eksperimen.
\end{eulercomment}
\begin{eulerprompt}
>p=normal(1,1000); //1000 random normal-distributed sample p
>plot2d(p,distribution=20,style="\(\backslash\)/"); // plot the random sample p
>plot2d("qnormal(x,0,1)",add=1): // add the standard normal distribution plot
\end{eulerprompt}
\eulerimg{29}{images/EMTStatistika_Rasyid Shalahuddin_22305144016-004.png}
\begin{eulercomment}
Harap perhatikan perbedaan antara plot batang (sampel) dan kurva
normal(distribusi nyata). Masukkan kembali tiga perintah untuk melihat
hasil pengambilan sampel lainnya.
\end{eulercomment}
\begin{eulercomment}
Berikut adalah perbandingan 10 simulasi dari 1000 nilai terdistribusi
normal menggunakan apa yang disebut box plot. Plot ini menunjukkan
median, kuartil 25\% dan 75\%, nilai minimal dan maksimal, dan outlier.
\end{eulercomment}
\begin{eulerprompt}
>p=normal(100,1000); boxplot(p):
\end{eulerprompt}
\eulerimg{29}{images/EMTStatistika_Rasyid Shalahuddin_22305144016-005.png}
\begin{eulercomment}
Untuk menghasilkan bilangan bulat acak, Euler memiliki intrandom. Mari
kita simulasikan lemparan dadu dan plot distribusinya.

Kami menggunakan fungsi getmultiplicities v, x), yang menghitung
seberapa sering elemen v muncul di x. Kemudian kita plot hasilnya
menggunakan columnplot().
\end{eulercomment}
\begin{eulerprompt}
>k=intrandom(1,6000,6);  ...
>columnsplot(getmultiplicities(1:6,k));  ...
>ygrid(1000,color=red):
\end{eulerprompt}
\eulerimg{29}{images/EMTStatistika_Rasyid Shalahuddin_22305144016-006.png}
\begin{eulercomment}
Sementara intrandom (n, m, k) mengembalikan bilangan bulat
terdistribusi seragam dari 1 ke k, dimungkinkan untuk menggunakan
distribusi bilangan bulat lain yang diberikan dengan randpint ().

Dalam contoh berikut, probabilitas 1,2,3 masing-masing adalah
0,4,0.1,0.5.
\end{eulercomment}
\begin{eulerprompt}
>randpint(1,1000,[0.4,0.1,0.5]); getmultiplicities(1:3,%)
\end{eulerprompt}
\begin{euleroutput}
  [378,  102,  520]
\end{euleroutput}
\begin{eulercomment}
Euler dapat menghasilkan nilai acak dari lebih banyak distribusi.
Simak referensinya.

Misalnya, kami mencoba distribusi eksponensial. Variabel acak kontinu
X dikatakan memiliki distribusi eksponensial, jika PDF-nya diberikan
oleh\\
\end{eulercomment}
\begin{eulerformula}
\[
f_X(x)=\lambda e^{-\lambda x},\quad x>0,\quad \lambda>0,
\]
\end{eulerformula}
\begin{eulercomment}
with parameter\\
\end{eulercomment}
\begin{eulerformula}
\[
\lambda=\frac{1}{\mu},\quad \mu \text{ is the mean, and denoted by } X \sim \text{Exponential}(\lambda).
\]
\end{eulerformula}
\begin{eulerprompt}
>plot2d(randexponential(1,1000,2),>distribution):
\end{eulerprompt}
\eulerimg{29}{images/EMTStatistika_Rasyid Shalahuddin_22305144016-007.png}
\begin{eulercomment}
Untuk banyak distribusi, Euler dapat menghitung fungsi distribusi dan
inversnya.
\end{eulercomment}
\begin{eulerprompt}
>plot2d("normaldis",-4,4): 
\end{eulerprompt}
\eulerimg{29}{images/EMTStatistika_Rasyid Shalahuddin_22305144016-008.png}
\begin{eulercomment}
Berikut ini adalah salah satu cara untuk memplot sebuah kuantil.
\end{eulercomment}
\begin{eulerprompt}
>plot2d("qnormal(x,1,1.5)",-4,6);  ...
>plot2d("qnormal(x,1,1.5)",a=2,b=5,>add,>filled):
\end{eulerprompt}
\eulerimg{29}{images/EMTStatistika_Rasyid Shalahuddin_22305144016-009.png}
\begin{eulerformula}
\[
\text{normaldis(x,m,d)}=\int_{-\infty}^x \frac{1}{d\sqrt{2\pi}}e^{-\frac{1}{2}(\frac{t-m}{d})^2}\ dt.
\]
\end{eulerformula}
\begin{eulercomment}
Kemungkinan berada di area hijau adalah sebagai berikut.
\end{eulercomment}
\begin{eulerprompt}
>normaldis(5,1,1.5)-normaldis(2,1,1.5)
\end{eulerprompt}
\begin{euleroutput}
  0.248662156979
\end{euleroutput}
\begin{eulercomment}
Ini dapat dihitung secara numerik dengan integral berikut.\\
\end{eulercomment}
\begin{eulerformula}
\[
\int_2^5 \frac{1}{1.5\sqrt{2\pi}}e^{-\frac{1}{2}(\frac{x-1}{1.5})^2}\ dx.
\]
\end{eulerformula}
\begin{eulerprompt}
>gauss("qnormal(x,1,1.5)",2,5)
\end{eulerprompt}
\begin{euleroutput}
  0.248662156979
\end{euleroutput}
\begin{eulercomment}
Mari kita bandingkan distribusi binomial dengan distribusi normal dari
mean dan deviasi yang sama. Fungsi invbindis () memecahkan interpolasi
linier antara nilai integer.
\end{eulercomment}
\begin{eulerprompt}
>invbindis(0.95,1000,0.5), invnormaldis(0.95,500,0.5*sqrt(1000))
\end{eulerprompt}
\begin{euleroutput}
  525.516721219
  526.007419394
\end{euleroutput}
\begin{eulercomment}
Fungsi qdis () adalah kepadatan dari distribusi chi-kuadrat. Seperti
biasa, Euler memetakan vektor ke fungsi ini. Jadi kita mendapatkan
plot dari semua distribusi chi-kuadrat dengan derajat 5 sampai 30
dengan mudah dengan cara berikut.
\end{eulercomment}
\begin{eulerprompt}
>plot2d("qchidis(x,(5:5:50)')",0,50):
\end{eulerprompt}
\eulerimg{29}{images/EMTStatistika_Rasyid Shalahuddin_22305144016-010.png}
\begin{eulercomment}
Euler memiliki fungsi yang akurat untuk mengevaluasi distribusi. Mari
kita periksa chidis () dengan integral.

Penamaan mencoba untuk konsisten. Misalnya.,

- distribusi chi-kuadrat adalah chidis (),\\
- fungsi kebalikannya adalah invchidis (),\\
- kepadatannya adalah qchidis ().

Pelengkap distribusi (ekor atas) adalah chicdis ().
\end{eulercomment}
\begin{eulerprompt}
>chidis(1.5,2), integrate("qchidis(x,2)",0,1.5)
\end{eulerprompt}
\begin{euleroutput}
  0.527633447259
  0.527633447259
\end{euleroutput}
\eulerheading{Distribusi Diskrit}
\begin{eulercomment}
Untuk menentukan distribusi diskrit Anda sendiri, Anda dapat
menggunakan metode berikut.

Pertama kita mengatur fungsi distribusi.
\end{eulercomment}
\begin{eulerprompt}
>wd = 0|((1:6)+[-0.01,0.01,0,0,0,0])/6
\end{eulerprompt}
\begin{euleroutput}
  [0,  0.165,  0.335,  0.5,  0.666667,  0.833333,  1]
\end{euleroutput}
\begin{eulercomment}
Artinya dengan probabilitas wd[i+1]-wd[i] kita menghasilkan nilai acak
i.

Ini hampir merupakan distribusi yang seragam. Mari kita tentukan
generator nomor acak untuk ini. Fungsi find (v, x) menemukan nilai x
pada vektor v. Fungsi ini juga berlaku untuk vektor x.
\end{eulercomment}
\begin{eulerprompt}
>function wrongdice (n,m) := find(wd,random(n,m))
\end{eulerprompt}
\begin{eulercomment}
Kesalahannya begitu halus sehingga kita hanya melihatnya dengan sangat
banyak iterasi.
\end{eulercomment}
\begin{eulerprompt}
>columnsplot(getmultiplicities(1:6,wrongdice(1,1000000))):
\end{eulerprompt}
\eulerimg{29}{images/EMTStatistika_Rasyid Shalahuddin_22305144016-011.png}
\begin{eulercomment}
Berikut adalah fungsi sederhana untuk memeriksa distribusi seragam
nilai 1 ... K dalam v. Kami menerima hasilnya, jika untuk semua
frekuensi

\end{eulercomment}
\begin{eulerformula}
\[
\left|f_i-\frac{1}{K}\right| < \frac{\delta}{\sqrt{n}}.
\]
\end{eulerformula}
\begin{eulerprompt}
>function checkrandom (v, delta=1) ...
\end{eulerprompt}
\begin{eulerudf}
    K=max(v); n=cols(v);
    fr=getfrequencies(v,1:K);
    return max(fr/n-1/K)<delta/sqrt(n);
    endfunction
\end{eulerudf}
\begin{eulercomment}
Memang fungsinya menolak distribusi seragam.
\end{eulercomment}
\begin{eulerprompt}
>checkrandom(wrongdice(1,1000000))
\end{eulerprompt}
\begin{euleroutput}
  0
\end{euleroutput}
\begin{eulercomment}
Dan itu menerima generator acak bawaan.
\end{eulercomment}
\begin{eulerprompt}
>checkrandom(intrandom(1,1000000,6))
\end{eulerprompt}
\begin{euleroutput}
  1
\end{euleroutput}
\begin{eulercomment}
Kami dapat menghitung distribusi binomial. Pertama ada binomialsum (),
yang mengembalikan probabilitas i atau kurang dari n percobaan.
\end{eulercomment}
\begin{eulerprompt}
>bindis(410,1000,0.4)
\end{eulerprompt}
\begin{euleroutput}
  0.751401349654
\end{euleroutput}
\begin{eulercomment}
Fungsi Beta terbalik digunakan untuk menghitung interval kepercayaan
Clopper-Pearson untuk parameter p. Tingkat defaultnya adalah alfa.

Arti dari interval ini adalah jika p berada di luar interval maka
hasil observasi 410 dalam 1000 jarang terjadi.
\end{eulercomment}
\begin{eulerprompt}
>clopperpearson(400,1000)
\end{eulerprompt}
\begin{euleroutput}
  [0.369469,  0.431122]
\end{euleroutput}
\begin{eulercomment}
Perintah berikut adalah cara langsung untuk mendapatkan hasil di atas.
Namun untuk n besar, penjumlahan langsung tidak akurat dan lambat.
\end{eulercomment}
\begin{eulerprompt}
>p=0.4; i=0:410; n=1000; sum(bin(n,i)*p^i*(1-p)^(n-i))
\end{eulerprompt}
\begin{euleroutput}
  0.751401349655
\end{euleroutput}
\begin{eulercomment}
Omong-omong, invbinsum() menghitung kebalikan dari binomialsum().
\end{eulercomment}
\begin{eulerprompt}
>invbindis(0.5,100,0.4)
\end{eulerprompt}
\begin{euleroutput}
  39.4669423584
\end{euleroutput}
\begin{eulercomment}
Di Bridge, kami mengasumsikan 5 kartu beredar (dari 52) di dua tangan
(26 kartu). Mari kita hitung probabilitas distribusi yang lebih buruk
dari 3:2 (misalnya 0:5, 1:4, 4:1 atau 5:0).
\end{eulercomment}
\begin{eulerprompt}
>2*hypergeomsum(1,5,13,26)
\end{eulerprompt}
\begin{euleroutput}
  0.321739130435
\end{euleroutput}
\begin{eulercomment}
Ada juga simulasi distribusi multinomial.
\end{eulercomment}
\begin{eulerprompt}
>randmultinomial(10,1000,[0.4,0.1,0.5])
\end{eulerprompt}
\begin{euleroutput}
            433            98           469 
            396           102           502 
            359           108           533 
            394           107           499 
            388           101           511 
            414           100           486 
            391            76           533 
            405           106           489 
            394           103           503 
            396           106           498 
\end{euleroutput}
\eulerheading{Merencanakan Data}
\begin{eulercomment}
Untuk memplot data, kita coba hasil pemilu Jerman sejak 1990, diukur
dalam kursi.
\end{eulercomment}
\begin{eulerprompt}
>BW := [ ...
>1990,662,319,239,79,8,17; ...
>1994,672,294,252,47,49,30; ...
>1998,669,245,298,43,47,36; ...
>2002,603,248,251,47,55,2; ...
>2005,614,226,222,61,51,54; ...
>2009,622,239,146,93,68,76; ...
>2013,631,311,193,0,63,64];
\end{eulerprompt}
\begin{eulercomment}
Untuk pesta, kita menggunakan serangkain nama.
\end{eulercomment}
\begin{eulerprompt}
>P:=["CDU/CSU","SPD","FDP","Gr","Li"];
\end{eulerprompt}
\begin{eulercomment}
Mari kita cetak persentase dengan baik.

Pertama kami mengekstrak kolom yang diperlukan. Kolom 3 sd 7 adalah
kursi masing-masing partai, dan kolom 2 adalah jumlah kursi. kolom
adalah tahun pemilihan.
\end{eulercomment}
\begin{eulerprompt}
>BT:=BW[,3:7]; BT:=BT/sum(BT); YT:=BW[,1]';
\end{eulerprompt}
\begin{eulercomment}
Kemudian kami mencetak statistik dalam bentuk tabel. Kami menggunakan
nama sebagai tajuk kolom, dan tahun sebagai tajuk untuk baris. Lebar
default untuk kolom adalah wc=10, tetapi kami lebih memilih keluaran
yang lebih padat. Kolom akan diperluas untuk label kolom, jika perlu.
\end{eulercomment}
\begin{eulerprompt}
>writetable(BT*100,wc=6,dc=0,>fixed,labc=P,labr=YT)
\end{eulerprompt}
\begin{euleroutput}
         CDU/CSU   SPD   FDP    Gr    Li
    1990      48    36    12     1     3
    1994      44    38     7     7     4
    1998      37    45     6     7     5
    2002      41    42     8     9     0
    2005      37    36    10     8     9
    2009      38    23    15    11    12
    2013      49    31     0    10    10
\end{euleroutput}
\begin{eulercomment}
Perkalian matriks berikut mengekstrak jumlah persentase dari dua
partai besar yang menunjukkan bahwa partai kecil telah mendapatkan
footage di parlemen hingga tahun 2009.
\end{eulercomment}
\begin{eulerprompt}
>BT1:=(BT.[1;1;0;0;0])'*100
\end{eulerprompt}
\begin{euleroutput}
  [84.29,  81.25,  81.1659,  82.7529,  72.9642,  61.8971,  79.8732]
\end{euleroutput}
\begin{eulercomment}
Ada juga plot statistik sederhana. Kami menggunakannya untuk
menampilkan garis dan titik secara bersamaan. Alternatifnya adalah
memanggil plot2d dua kali dengan\textgreater{} add.
\end{eulercomment}
\begin{eulerprompt}
>statplot(YT,BT1,"b"):
\end{eulerprompt}
\eulerimg{29}{images/EMTStatistika_Rasyid Shalahuddin_22305144016-012.png}
\begin{eulercomment}
Tentukan beberapa warna untuk setiap pesta.
\end{eulercomment}
\begin{eulerprompt}
>CP:=[rgb(0.5,0.5,0.5),red,yellow,green,rgb(0.8,0,0)];
\end{eulerprompt}
\begin{eulercomment}
Sekarang kita bisa memplot hasil Pemilu 2009 dan perubahannya menjadi
satu plot menggunakan gambar. Kita dapat menambahkan vektor kolom ke
setiap plot.
\end{eulercomment}
\begin{eulerprompt}
>figure(2,1);  ...
>figure(1); columnsplot(BW[6,3:7],P,color=CP); ...
>figure(2); columnsplot(BW[6,3:7]-BW[5,3:7],P,color=CP);  ...
>figure(0):
\end{eulerprompt}
\eulerimg{29}{images/EMTStatistika_Rasyid Shalahuddin_22305144016-013.png}
\begin{eulercomment}
Plot data menggabungkan deretan data statistik dalam satu plot.
\end{eulercomment}
\begin{eulerprompt}
>J:=BW[,1]'; DP:=BW[,3:7]'; ...
>dataplot(YT,BT',color=CP);  ...
>labelbox(P,colors=CP,styles="[]",>points,w=0.2,x=0.3,y=0.4):
\end{eulerprompt}
\eulerimg{29}{images/EMTStatistika_Rasyid Shalahuddin_22305144016-014.png}
\begin{eulercomment}
Plot kolom 3D menampilkan baris data statistik dalam bentuk kolom.
Kami memberikan label untuk baris dan kolom. sudut adalah sudut
pandang.
\end{eulercomment}
\begin{eulerprompt}
>columnsplot3d(BT,scols=P,srows=YT, ...
>  angle=30°,ccols=CP):
\end{eulerprompt}
\eulerimg{29}{images/EMTStatistika_Rasyid Shalahuddin_22305144016-015.png}
\begin{eulercomment}
Representasi lainnya adalah plot mosaik. Perhatikan bahwa kolom plot
mewakili kolom matriks di sini. Karena panjangnya label CDU / CSU,
kami mengambil jendela yang lebih kecil dari biasanya.
\end{eulercomment}
\begin{eulerprompt}
>shrinkwindow(>smaller);  ...
>mosaicplot(BT',srows=YT,scols=P,color=CP,style="#"); ...
>shrinkwindow():
\end{eulerprompt}
\eulerimg{29}{images/EMTStatistika_Rasyid Shalahuddin_22305144016-016.png}
\begin{eulercomment}
Kami juga bisa membuat diagram pie. Karena hitam dan kuning membentuk
koalisi, kami menyusun ulang elemen-elemennya.
\end{eulercomment}
\begin{eulerprompt}
>i=[1,3,5,4,2]; piechart(BW[6,3:7][i],color=CP[i],lab=P[i]):
\end{eulerprompt}
\begin{eulercomment}
Berikut ini jenis plot lainnya.
\end{eulercomment}
\begin{eulerprompt}
>starplot(normal(1,10)+4,lab=1:10,>rays):
\end{eulerprompt}
\begin{eulercomment}
Beberapa plot di plot2d bagus untuk statika. Berikut adalah plot
impuls data acak, didistribusikan secara seragam di [0,1].
\end{eulercomment}
\begin{eulerprompt}
>plot2d(makeimpulse(1:10,random(1,10)),>bar):
\end{eulerprompt}
\begin{eulercomment}
Tetapi untuk data yang terdistribusi secara eksponensial, kita mungkin
memerlukan plot logaritmik.
\end{eulercomment}
\begin{eulerprompt}
>logimpulseplot(1:10,-log(random(1,10))*10):
\end{eulerprompt}
\begin{eulercomment}
Fungsi columnplot() lebih mudah digunakan, karena hanya membutuhkan
vektor nilai. Selain itu, ia dapat mengatur labelnya menjadi apa pun
yang kami inginkan, kami telah menunjukkannya di tutorial ini.

Berikut adalah aplikasi lain, di mana kita menghitung karakter dalam
sebuah kalimat dan membuat plot statistik.
\end{eulercomment}
\begin{eulerprompt}
>v=strtochar("the quick brown fox jumps over the lazy dog"); ...
>w=ascii("a"):ascii("z"); x=getmultiplicities(w,v); ...
>cw=[]; for k=w; cw=cw|char(k); end; ...
>columnsplot(x,lab=cw,width=0.05):
\end{eulerprompt}
\begin{eulercomment}
Sumbu juga dapat diatur secara manual.
\end{eulercomment}
\begin{eulerprompt}
>n=10; p=0.4; i=0:n; x=bin(n,i)*p^i*(1-p)^(n-i); ...
>columnsplot(x,lab=i,width=0.05,<frame,<grid); ...
>yaxis(0,0:0.1:1,style="->",>left); xaxis(0,style="."); ...
>label("p",0,0.25), label("i",11,0); ...
>textbox(["Binomial distribution","with p=0.4"]):
\end{eulerprompt}
\begin{eulercomment}
Berikut ini adalah cara untuk memplot frekuensi bilangan dalam sebuah
vektor.

Kami membuat vektor bilangan bulat bilangan acak 1 hingga 6.
\end{eulercomment}
\begin{eulerprompt}
>v:=intrandom(1,10,10)
\end{eulerprompt}
\begin{euleroutput}
  [8,  5,  8,  8,  6,  8,  8,  3,  5,  5]
\end{euleroutput}
\begin{eulercomment}
Kemudian ekstrak nomor unik di v.
\end{eulercomment}
\begin{eulerprompt}
>vu:=unique(v)
\end{eulerprompt}
\begin{euleroutput}
  [3,  5,  6,  8]
\end{euleroutput}
\begin{eulercomment}
Dan plot frekuensi dalam plot kolom.
\end{eulercomment}
\begin{eulerprompt}
>columnsplot(getmultiplicities(vu,v),lab=vu,style="/"):
\end{eulerprompt}
\begin{eulercomment}
Kami ingin mendemonstrasikan fungsi untuk distribusi nilai empiris.
\end{eulercomment}
\begin{eulerprompt}
>x=normal(1,20);
\end{eulerprompt}
\begin{eulercomment}
Fungsi empdist (x, vs) membutuhkan array nilai yang diurutkan. Jadi
kita harus mengurutkan x sebelum dapat menggunakannya.
\end{eulercomment}
\begin{eulerprompt}
>xs=sort(x);
\end{eulerprompt}
\begin{eulercomment}
Kemudian kami memplot distribusi empiris dan beberapa batang kepadatan
ke dalam satu plot. Alih-alih plot batang untuk distribusi kami
menggunakan plot gigi gergaji kali ini.
\end{eulercomment}
\begin{eulerprompt}
>figure(2,1); ...
>figure(1); plot2d("empdist",-4,4;xs); ...
>figure(2); plot2d(histo(x,v=-4:0.2:4,<bar));  ...
>figure(0):
\end{eulerprompt}
\begin{eulercomment}
Plot pencar mudah dilakukan di Euler dengan plot titik biasa. Grafik
berikut menunjukkan bahwa X dan X+Y berkorelasi positif dengan jelas.
\end{eulercomment}
\begin{eulerprompt}
>x=normal(1,100); plot2d(x,x+rotright(x),>points,style=".."):
\end{eulerprompt}
\begin{eulercomment}
Seringkali, kami ingin membandingkan dua sampel dari distribusi yang
berbeda. Ini dapat dilakukan dengan plot-kuantil-kuantil.

Untuk pengujian, kami mencoba distribusi t siswa dan distribusi
eksponensial.
\end{eulercomment}
\begin{eulerprompt}
>x=randt(1,1000,5); y=randnormal(1,1000,mean(x),dev(x)); ...
>plot2d("x",r=6,style="--",yl="normal",xl="student-t",>vertical); ...
>plot2d(sort(x),sort(y),>points,color=red,style="x",>add):
\end{eulerprompt}
\begin{eulercomment}
Plot dengan jelas menunjukkan bahwa nilai terdistribusi normal
cenderung lebih kecil di ujung yang ekstrim.

Jika kita memiliki dua distribusi dengan ukuran berbeda, kita dapat
memperbesar yang lebih kecil atau mengecilkan yang lebih besar. Fungsi
berikut bagus untuk keduanya. Ini mengambil nilai median dengan
persentase antara 0 dan 1.
\end{eulercomment}
\begin{eulerprompt}
>function medianexpand (x,n) := median(x,p=linspace(0,1,n-1));
\end{eulerprompt}
\begin{eulercomment}
Mari kita bandingkan dua distribusi yang sama.
\end{eulercomment}
\begin{eulerprompt}
>x=random(1000); y=random(400); ...
>plot2d("x",0,1,style="--"); ...
>plot2d(sort(medianexpand(x,400)),sort(y),>points,color=red,style="x",>add):
\end{eulerprompt}
\eulerheading{Regresi dan Korelasi}
\begin{eulercomment}
Regresi linier dapat dilakukan dengan fungsi polyfit () atau berbagai
fungsi fit.

Sebagai permulaan kita menemukan garis regresi untuk data univariat
dengan polyfit(x, y, 1).
\end{eulercomment}
\begin{eulerprompt}
>x=1:10; y=[2,3,1,5,6,3,7,8,9,8]; writetable(x'|y',labc=["x","y"])
\end{eulerprompt}
\begin{euleroutput}
           x         y
           1         2
           2         3
           3         1
           4         5
           5         6
           6         3
           7         7
           8         8
           9         9
          10         8
\end{euleroutput}
\begin{eulercomment}
Kami ingin membandingkan ukuran yang tidak berbobot dan berbobot.
Pertama koefisien kesesuaian linier.
\end{eulercomment}
\begin{eulerprompt}
>p=polyfit(x,y,1)
\end{eulerprompt}
\begin{euleroutput}
  [0.733333,  0.812121]
\end{euleroutput}
\begin{eulercomment}
Sekarang koefisien dengan bobot yang menekankan nilai terakhir.
\end{eulercomment}
\begin{eulerprompt}
>w &= "exp(-(x-10)^2/10)"; pw=polyfit(x,y,1,w=w(x))
\end{eulerprompt}
\begin{euleroutput}
  [4.71566,  0.38319]
\end{euleroutput}
\begin{eulercomment}
Kami menempatkan semuanya ke dalam satu plot untuk titik dan garis
regresi, dan untuk bobot yang digunakan.
\end{eulercomment}
\begin{eulerprompt}
>figure(2,1);  ...
>figure(1); statplot(x,y,"b",xl="Regression"); ...
>  plot2d("evalpoly(x,p)",>add,color=blue,style="--"); ...
>  plot2d("evalpoly(x,pw)",5,10,>add,color=red,style="--"); ...
>figure(2); plot2d(w,1,10,>filled,style="/",fillcolor=red,xl=w); ...
>figure(0):
\end{eulerprompt}
\begin{eulercomment}
Untuk contoh lain kami membaca survei siswa, usia mereka, usia orang
tua mereka dan jumlah saudara kandung dari sebuah file.

Tabel ini berisi "m" dan "f" di kolom kedua. Kami menggunakan variabel
tok2 untuk menyetel terjemahan yang tepat alih-alih membiarkan
readtable () mengumpulkan terjemahan.
\end{eulercomment}
\begin{eulerprompt}
>\{MS,hd\}:=readtable("table1.dat",tok2:=["m","f"]);  ...
>writetable(MS,labc=hd,tok2:=["m","f"]);
\end{eulerprompt}
\begin{euleroutput}
      Person       Sex       Age    Mother    Father  Siblings
           1         m        29        58        61         1
           2         f        26        53        54         2
           3         m        24        49        55         1
           4         f        25        56        63         3
           5         f        25        49        53         0
           6         f        23        55        55         2
           7         m        23        48        54         2
           8         m        27        56        58         1
           9         m        25        57        59         1
          10         m        24        50        54         1
          11         f        26        61        65         1
          12         m        24        50        52         1
          13         m        29        54        56         1
          14         m        28        48        51         2
          15         f        23        52        52         1
          16         m        24        45        57         1
          17         f        24        59        63         0
          18         f        23        52        55         1
          19         m        24        54        61         2
          20         f        23        54        55         1
\end{euleroutput}
\begin{eulercomment}
How do the ages depend on each other? A first impression comes from a
pairwise scatterplot.
\end{eulercomment}
\begin{eulerprompt}
>scatterplots(tablecol(MS,3:5),hd[3:5]):
\end{eulerprompt}
\begin{eulercomment}
Jelas bahwa usia bapak dan ibu saling bergantung. Mari kita tentukan
dan plot garis regresi.
\end{eulercomment}
\begin{eulerprompt}
>cs:=MS[,4:5]'; ps:=polyfit(cs[1],cs[2],1)
\end{eulerprompt}
\begin{euleroutput}
  [17.3789,  0.740964]
\end{euleroutput}
\begin{eulercomment}
Ini jelas model yang salah. Garis regresinya adalah s = 17 + 0.74t,
dimana t adalah umur ibu dan s umur bapak. Perbedaan usia mungkin
sedikit bergantung pada usianya, tapi tidak terlalu banyak.

Sebaliknya, kami menduga fungsi seperti s = a + t. Maka a adalah mean
dari s-t. Ini adalah perbedaan usia rata-rata antara ayah dan ibu.
\end{eulercomment}
\begin{eulerprompt}
>da:=mean(cs[2]-cs[1])
\end{eulerprompt}
\begin{euleroutput}
  3.65
\end{euleroutput}
\begin{eulercomment}
Mari kita plot ini menjadi satu plot pencar.
\end{eulercomment}
\begin{eulerprompt}
>plot2d(cs[1],cs[2],>points);  ...
>plot2d("evalpoly(x,ps)",color=red,style=".",>add);  ...
>plot2d("x+da",color=blue,>add):
\end{eulerprompt}
\begin{eulercomment}
Berikut adalah plot kotak dari dua zaman. Ini hanya menunjukkan, bahwa
umurnya berbeda.
\end{eulercomment}
\begin{eulerprompt}
>boxplot(cs,["mothers","fathers"]):
\end{eulerprompt}
\begin{eulercomment}
Menariknya, perbedaan median tidak sebesar perbedaan rata rata.
\end{eulercomment}
\begin{eulerprompt}
>median(cs[2])-median(cs[1])
\end{eulerprompt}
\begin{euleroutput}
  1.5
\end{euleroutput}
\begin{eulercomment}
Koefisien korelasi menunjukkan korelasi positif.
\end{eulercomment}
\begin{eulerprompt}
>correl(cs[1],cs[2])
\end{eulerprompt}
\begin{euleroutput}
  0.7588307236
\end{euleroutput}
\begin{eulercomment}
Korelasi barisan adalah ukuran untuk urutan yang sama di kedua vektor.
Ini juga cukup positif.
\end{eulercomment}
\begin{eulerprompt}
>rankcorrel(cs[1],cs[2])
\end{eulerprompt}
\begin{euleroutput}
  0.758925292358
\end{euleroutput}
\eulerheading{Membuat Fungsi baru}
\begin{eulercomment}
Tentu saja, bahasa EMT dapat digunakan untuk memprogram fungsi baru.
Misalnya, kami mendefinisikan fungsi kemiringan.

\end{eulercomment}
\begin{eulerformula}
\[
\text{sk}(x) = \dfrac{\sqrt{n} \sum_i (x_i-m)^3}{\left(\sum_i (x_i-m)^2\right)^{3/2}}
\]
\end{eulerformula}
\begin{eulercomment}
dimana m adalah mean dari x.
\end{eulercomment}
\begin{eulerprompt}
>function skew (x:vector) ...
\end{eulerprompt}
\begin{eulerudf}
  m=mean(x);
  return sqrt(cols(x))*sum((x-m)^3)/(sum((x-m)^2))^(3/2);
  endfunction
\end{eulerudf}
\begin{eulercomment}
Seperti yang Anda lihat, kita dapat dengan mudah menggunakan bahasa
matriks untuk mendapatkan implementasi yang sangat singkat dan
efisien. Mari kita coba fungsi ini.
\end{eulercomment}
\begin{eulerprompt}
>data=normal(20); skew(normal(10))
\end{eulerprompt}
\begin{euleroutput}
  0.643769122478
\end{euleroutput}
\begin{eulercomment}
Berikut adalah fungsi lain, yang disebut koefisien kemiringan Pearson.
\end{eulercomment}
\begin{eulerprompt}
>function skew1 (x) := 3*(mean(x)-median(x))/dev(x)
>skew1(data)
\end{eulerprompt}
\begin{euleroutput}
  0.24951252184
\end{euleroutput}
\eulerheading{Simulasi Monte Carlo}
\begin{eulercomment}
Euler dapat digunakan untuk mensimulasikan peristiwa acak. Kami telah
melihat contoh sederhana di atas. Ini satu lagi, yang mensimulasikan
1000 kali lemparan 3 dadu, dan menanyakan distribusi jumlahnya.
\end{eulercomment}
\begin{eulerprompt}
>ds:=sum(intrandom(1000,3,6))';  fs=getmultiplicities(3:18,ds)
\end{eulerprompt}
\begin{euleroutput}
  [2,  13,  29,  53,  65,  105,  108,  117,  130,  115,  88,  74,  54,
  33,  12,  2]
\end{euleroutput}
\begin{eulercomment}
Kita bisa merencakannya sekarang.
\end{eulercomment}
\begin{eulerprompt}
>columnsplot(fs,lab=3:18):
\end{eulerprompt}
\begin{eulercomment}
Untuk menentukan distribusi yang diharapkan tidaklah mudah. Kami
menggunakan rekursi lanjutan untuk ini.

Fungsi berikut menghitung banyaknya cara bilangan k dapat
direpresentasikan sebagai jumlah dari n bilangan dalam rentang 1
hingga m. Ini bekerja secara rekursif dengan cara yang jelas.
\end{eulercomment}
\begin{eulerprompt}
>function map countways (k; n, m) ...
\end{eulerprompt}
\begin{eulerudf}
    if n==1 then return k>=1 && k<=m
    else
      sum=0; 
      loop 1 to m; sum=sum+countways(k-#,n-1,m); end;
      return sum;
    end;
  endfunction
\end{eulerudf}
\begin{eulercomment}
Ini adalah hasil dari tiga lemparan dadu.
\end{eulercomment}
\begin{eulerprompt}
>cw=countways(3:18,3,6)
\end{eulerprompt}
\begin{euleroutput}
  [1,  3,  6,  10,  15,  21,  25,  27,  27,  25,  21,  15,  10,  6,  3,
  1]
\end{euleroutput}
\begin{eulercomment}
Kami menambahkan nilai yang diharapkan ke plot.
\end{eulercomment}
\begin{eulerprompt}
>plot2d(cw/6^3*1000,>add); plot2d(cw/6^3*1000,>points,>add):
\end{eulerprompt}
\begin{eulercomment}
Untuk simulasi lain, deviasi nilai rata-rata n 0-1-variabel acak
terdistribusi normal adalah 1 / sqrt (n).
\end{eulercomment}
\begin{eulerprompt}
>longformat; 1/sqrt(10)
\end{eulerprompt}
\begin{euleroutput}
  0.316227766017
\end{euleroutput}
\begin{eulercomment}
Mari kita periksa dengan simulasi. Kami menghasilkan 10.000 kali 10
vektor acak.
\end{eulercomment}
\begin{eulerprompt}
>M=normal(10000,10); dev(mean(M)')
\end{eulerprompt}
\begin{euleroutput}
  0.319188944099
\end{euleroutput}
\begin{eulerprompt}
>plot2d(mean(M)',>distribution):
\end{eulerprompt}
\begin{eulercomment}
Median dari 10 bilangan acak terdistribusi normal 0-1 memiliki deviasi
yang lebih besar.
\end{eulercomment}
\begin{eulerprompt}
>dev(median(M)')
\end{eulerprompt}
\begin{euleroutput}
  0.373609827223
\end{euleroutput}
\begin{eulercomment}
Karena kami dapat dengan mudah membuat jalan acak, kami dapat
mensimulasikan proses Wiener. Kami mengambil 1000 langkah dari 1000
proses. Kami kemudian memplot deviasi standar dan mean dari langkah
ke-n dari proses ini bersama dengan nilai yang diharapkan berwarna
merah.
\end{eulercomment}
\begin{eulerprompt}
>n=1000; m=1000; M=cumsum(normal(n,m)/sqrt(m)); ...
>t=(1:n)/n; figure(2,1); ...
>figure(1); plot2d(t,mean(M')'); plot2d(t,0,color=red,>add); ...
>figure(2); plot2d(t,dev(M')'); plot2d(t,sqrt(t),color=red,>add); ...
>figure(0):
\end{eulerprompt}
\eulerheading{Tes}
\begin{eulercomment}
Tes adalah alat penting dalam statistik. Di Euler, banyak tes yang
diterapkan. Semua pengujian ini mengembalikan kesalahan yang kami
terima jika kami menolak hipotesis nol.

Sebagai contoh, kami menguji lemparan dadu untuk distribusi seragam.
Pada 600 lemparan, kami mendapatkan nilai berikut, yang kami masukkan
ke dalam uji chi-square.
\end{eulercomment}
\begin{eulerprompt}
>chitest([90,103,114,101,103,89],dup(100,6)')
\end{eulerprompt}
\begin{euleroutput}
  0.498830517952
\end{euleroutput}
\begin{eulercomment}
Uji chi-square juga memiliki mode, yang menggunakan simulasi Monte
Carlo untuk menguji statistik. Hasilnya harusnya hampir sama.
Parameter\textgreater{} p mengartikan vektor y sebagai vektor probabilitas.
\end{eulercomment}
\begin{eulerprompt}
>chitest([90,103,114,101,103,89],dup(1/6,6)',>p,>montecarlo)
\end{eulerprompt}
\begin{euleroutput}
  0.508
\end{euleroutput}
\begin{eulercomment}
Kesalahan ini terlalu besar. Jadi kita tidak bisa menolak distribusi
seragam. Ini tidak membuktikan bahwa dadu kami adil. Tapi kita tidak
bisa menolak hipotesis kita.

Selanjutnya kami menghasilkan 1000 lemparan dadu menggunakan generator
nomor acak, dan melakukan tes yang sama.
\end{eulercomment}
\begin{eulerprompt}
>n=1000; t=random([1,n*6]); chitest(count(t*6,6),dup(n,6)')
\end{eulerprompt}
\begin{euleroutput}
  0.243439570837
\end{euleroutput}
\begin{eulercomment}
Mari kita uji nilai rata-rata 100 dengan uji-t.
\end{eulercomment}
\begin{eulerprompt}
>s=200+normal([1,100])*10; ...
>ttest(mean(s),dev(s),100,200)
\end{eulerprompt}
\begin{euleroutput}
  0.426952606967
\end{euleroutput}
\begin{eulercomment}
Fungsi ttest() membutuhkan nilai mean, deviasi, jumlah data, dan nilai
mean untuk diuji.

Sekarang mari kita periksa dua pengukuran untuk mean yang sama. Kami
menolak hipotesis bahwa mereka memiliki mean yang sama, jika hasilnya
\textless{}0,05.
\end{eulercomment}
\begin{eulerprompt}
>tcomparedata(normal(1,10),normal(1,10))
\end{eulerprompt}
\begin{euleroutput}
  0.267190351647
\end{euleroutput}
\begin{eulercomment}
Jika kita menambahkan bias ke satu distribusi, kita mendapatkan lebih
banyak penolakan. Ulangi simulasi ini beberapa kali untuk melihat
efeknya.
\end{eulercomment}
\begin{eulerprompt}
>tcomparedata(normal(1,10),normal(1,10)+2)
\end{eulerprompt}
\begin{euleroutput}
  3.99849573895e-07
\end{euleroutput}
\begin{eulercomment}
Dalam contoh berikutnya, kami menghasilkan 20 lemparan dadu acak 100
kali dan menghitung yang ada di dalamnya. Harus ada rata-rata 20/6 =
3,3.
\end{eulercomment}
\begin{eulerprompt}
>R=random(100,20); R=sum(R*6<=1)'; mean(R)
\end{eulerprompt}
\begin{euleroutput}
  3.28
\end{euleroutput}
\begin{eulercomment}
Sekarang kami membandingkan jumlah satuan dengan distribusi binomial.
Pertama kami memplot distribusi satu.
\end{eulercomment}
\begin{eulerprompt}
>plot2d(R,distribution=max(R)+1,even=1,style="\(\backslash\)/"):
>t=count(R,21);
\end{eulerprompt}
\begin{eulercomment}
Kemudian kami menghitung nilai yang diharapkan.
\end{eulercomment}
\begin{eulerprompt}
>n=0:20; b=bin(20,n)*(1/6)^n*(5/6)^(20-n)*100;
\end{eulerprompt}
\begin{eulercomment}
Kita harus mengumpulkan beberapa nomor untuk mendapatkan kategori yang
cukup besar.
\end{eulercomment}
\begin{eulerprompt}
>t1=sum(t[1:2])|t[3:7]|sum(t[8:21]); ...
>b1=sum(b[1:2])|b[3:7]|sum(b[8:21]);
\end{eulerprompt}
\begin{eulercomment}
Uji chi-square menolak hipotesis bahwa distribusi kita adalah
distribusi binomial, jika hasilnya \textless{}0,05.
\end{eulercomment}
\begin{eulerprompt}
>chitest(t1,b1)
\end{eulerprompt}
\begin{euleroutput}
  0.185520705242
\end{euleroutput}
\begin{eulercomment}
Contoh berikut berisi hasil dari dua kelompok orang (misalnya
laki-laki dan perempuan) yang memberikan suara untuk satu dari enam
partai.
\end{eulercomment}
\begin{eulerprompt}
>A=[23,37,43,52,64,74;27,39,41,49,63,76];  ...
>  writetable(A,wc=6,labr=["m","f"],labc=1:6)
\end{eulerprompt}
\begin{euleroutput}
             1     2     3     4     5     6
       m    23    37    43    52    64    74
       f    27    39    41    49    63    76
\end{euleroutput}
\begin{eulercomment}
Kita ingin menguji independensi suara dari jenis kelamin. Tes tabel
chi\textasciicircum{}2 melakukan ini. Hasilnya adalah cara yang besar untuk menolak
kemerdekaan. Jadi kami tidak bisa mengatakan, apakah voting tergantung
jenis kelamin dari data ini.
\end{eulercomment}
\begin{eulerprompt}
>tabletest(A)
\end{eulerprompt}
\begin{euleroutput}
  0.990701632326
\end{euleroutput}
\begin{eulercomment}
Berikut adalah tabel yang diharapkan, jika kita mengasumsikan
frekuensi pemungutan suara yang diamati.
\end{eulercomment}
\begin{eulerprompt}
>writetable(expectedtable(A),wc=6,dc=1,labr=["m","f"],labc=1:6)
\end{eulerprompt}
\begin{euleroutput}
             1     2     3     4     5     6
       m  24.9  37.9  41.9  50.3  63.3  74.7
       f  25.1  38.1  42.1  50.7  63.7  75.3
\end{euleroutput}
\begin{eulercomment}
Kita dapat menghitung koefisien kontingensi yang dikoreksi. Karena
sangat mendekati 0, kami menyimpulkan bahwa pemungutan suara tidak
bergantung pada jenis kelamin.
\end{eulercomment}
\begin{eulerprompt}
>contingency(A)
\end{eulerprompt}
\begin{euleroutput}
  0.0427225484717
\end{euleroutput}
\eulerheading{Beberapa Tes Lagi}
\begin{eulercomment}
Selanjutnya kami menggunakan analisis varians (uji-F) untuk menguji
tiga sampel data terdistribusi normal untuk nilai rata-rata yang sama.
Metode tersebut dinamakan ANOVA (analysis of variance). Di Euler,
fungsi varanalysis() digunakan.
\end{eulercomment}
\begin{eulerprompt}
>x1=[109,111,98,119,91,118,109,99,115,109,94]; mean(x1),
\end{eulerprompt}
\begin{euleroutput}
  106.545454545
\end{euleroutput}
\begin{eulerprompt}
>x2=[120,124,115,139,114,110,113,120,117]; mean(x2),
\end{eulerprompt}
\begin{euleroutput}
  119.111111111
\end{euleroutput}
\begin{eulerprompt}
>x3=[120,112,115,110,105,134,105,130,121,111]; mean(x3)
\end{eulerprompt}
\begin{euleroutput}
  116.3
\end{euleroutput}
\begin{eulerprompt}
>varanalysis(x1,x2,x3)
\end{eulerprompt}
\begin{euleroutput}
  0.0138048221371
\end{euleroutput}
\begin{eulercomment}
Artinya, kami menolak hipotesis dengan nilai mean yang sama. Kami
melakukan ini dengan probabilitas kesalahan 1,3\%.

Ada juga uji median, yaitu menolak sampel data dengan distribusi
rata-rata yang berbeda menguji median dari sampel yang bersatu.
\end{eulercomment}
\begin{eulerprompt}
>a=[56,66,68,49,61,53,45,58,54];
>b=[72,81,51,73,69,78,59,67,65,71,68,71];
>mediantest(a,b)
\end{eulerprompt}
\begin{euleroutput}
  0.0241724220052
\end{euleroutput}
\begin{eulercomment}
Tes lain tentang kesetaraan adalah ujian peringkat. Ini jauh lebih
tajam daripada tes median.
\end{eulercomment}
\begin{eulerprompt}
>ranktest(a,b)
\end{eulerprompt}
\begin{euleroutput}
  0.00199969612469
\end{euleroutput}
\begin{eulercomment}
Dalam contoh berikut, kedua distribusi memiliki mean yang sama.
\end{eulercomment}
\begin{eulerprompt}
>ranktest(random(1,100),random(1,50)*3-1)
\end{eulerprompt}
\begin{euleroutput}
  0.468224146531
\end{euleroutput}
\begin{eulercomment}
Sekarang mari kita coba meniru dua perlakuan a dan b yang diterapkan
pada orang yang berbeda.
\end{eulercomment}
\begin{eulerprompt}
>a=[8.0,7.4,5.9,9.4,8.6,8.2,7.6,8.1,6.2,8.9];
>b=[6.8,7.1,6.8,8.3,7.9,7.2,7.4,6.8,6.8,8.1];
\end{eulerprompt}
\begin{eulercomment}
Tes signum memutuskan, jika a lebih baik dari b.
\end{eulercomment}
\begin{eulerprompt}
>signtest(a,b)
\end{eulerprompt}
\begin{euleroutput}
  0.0546875
\end{euleroutput}
\begin{eulercomment}
Ini terlalu banyak kesalahan. Kita tidak dapat menolak bahwa a sama
baiknya dengan b.

Tes Wilcoxon lebih tajam dari tes ini, tetapi bergantung pada nilai
kuantitatif perbedaannya.
\end{eulercomment}
\begin{eulerprompt}
>wilcoxon(a,b)
\end{eulerprompt}
\begin{euleroutput}
  0.0296680599405
\end{euleroutput}
\begin{eulercomment}
Mari kita coba dua tes lagi menggunakan seri yang dihasilkan.
\end{eulercomment}
\begin{eulerprompt}
>wilcoxon(normal(1,20),normal(1,20)-1)
\end{eulerprompt}
\begin{euleroutput}
  0.00202259852485
\end{euleroutput}
\begin{eulerprompt}
>wilcoxon(normal(1,20),normal(1,20))
\end{eulerprompt}
\begin{euleroutput}
  0.824671773049
\end{euleroutput}
\eulerheading{Angka Acak}
\begin{eulercomment}
Berikut ini adalah tes untuk generator bilangan acak. Euler
menggunakan generator yang sangat bagus, jadi kami tidak perlu
mengharapkan adanya masalah.

Pertama kami menghasilkan sepuluh juta bilangan acak di [0,1].
\end{eulercomment}
\begin{eulerprompt}
>n:=10000000; r:=random(1,n);
\end{eulerprompt}
\begin{eulercomment}
Selanjutnya kita menghitung jarak antara dua angka kurang dari 0,05.
\end{eulercomment}
\begin{eulerprompt}
>a:=0.05; d:=differences(nonzeros(r<a));
\end{eulerprompt}
\begin{eulercomment}
Akhirnya, kami memplot berapa kali, setiap jarak terjadi, dan
membandingkan dengan nilai yang diharapkan.
\end{eulercomment}
\begin{eulerprompt}
>m=getmultiplicities(1:100,d); plot2d(m); ...
>  plot2d("n*(1-a)^(x-1)*a^2",color=red,>add):
\end{eulerprompt}
\begin{eulercomment}
Hapus datanya.
\end{eulercomment}
\begin{eulerprompt}
>remvalue n;
\end{eulerprompt}
\eulerheading{Pengenalan untuk Pengguna Proyek R.}
\begin{eulercomment}
Jelas, EMT tidak bersaing dengan R sebagai paket statistik. Namun, ada
banyak prosedur dan fungsi statistik yang tersedia di EMT juga. Jadi
EMT dapat memenuhi kebutuhan dasar. Bagaimanapun, EMT hadir dengan
paket numerik dan sistem aljabar komputer.

Notebook ini untuk Anda jika Anda sudah familiar dengan R, tetapi
perlu mengetahui perbedaan sintaks EMT dan R. Kami mencoba memberikan
gambaran umum tentang hal-hal yang jelas dan kurang jelas yang perlu
Anda ketahui.

Selain itu, kami mencari cara untuk bertukar data antara kedua sistem.
\end{eulercomment}
\begin{eulercomment}
Perhatikan bahwa ini adalah pekerjaan yang sedang berjalan.
\end{eulercomment}
\eulerheading{Sintaks Dasar}
\begin{eulercomment}
Hal pertama yang Anda pelajari di R adalah membuat vektor. Dalam EMT,
perbedaan utamanya adalah: operator dapat mengambil ukuran langkah.
Selain itu memiliki daya ikat yang rendah.
\end{eulercomment}
\begin{eulerprompt}
>n=10; 0:n/20:n-1
\end{eulerprompt}
\begin{euleroutput}
  [0,  0.5,  1,  1.5,  2,  2.5,  3,  3.5,  4,  4.5,  5,  5.5,  6,  6.5,
  7,  7.5,  8,  8.5,  9]
\end{euleroutput}
\begin{eulercomment}
Fungsi c() tidak ada. Dimungkinkan untuk menggunakan vektor untuk
menggabungkan berbagai hal.

Contoh berikut, seperti banyak contoh lainnya, dari "Interoduction to
R" yang disertakan dengan proyek R. Jika Anda membaca PDF ini, Anda
akan menemukan bahwa saya mengikuti jalurnya dalam tutorial ini.
\end{eulercomment}
\begin{eulerprompt}
>x=[10.4, 5.6, 3.1, 6.4, 21.7]; [x,0,x]
\end{eulerprompt}
\begin{euleroutput}
  [10.4,  5.6,  3.1,  6.4,  21.7,  0,  10.4,  5.6,  3.1,  6.4,  21.7]
\end{euleroutput}
\begin{eulercomment}
Operator titik dua dengan ukuran langkah EMT diganti dengan fungsi
seq() di R. Kita bisa menulis fungsi ini di EMT.
\end{eulercomment}
\begin{eulerprompt}
>function seq(a,b,c) := a:b:c; ...
>seq(0,-0.1,-1)
\end{eulerprompt}
\begin{euleroutput}
  [0,  -0.1,  -0.2,  -0.3,  -0.4,  -0.5,  -0.6,  -0.7,  -0.8,  -0.9,  -1]
\end{euleroutput}
\begin{eulercomment}
Fungsi rep() dari R tidak ada di EMT. Untuk input vektor dapat
dituliskan sebagai berikut.
\end{eulercomment}
\begin{eulerprompt}
>function rep(x:vector,n:index) := flatten(dup(x,n)); ...
>rep(x,2)
\end{eulerprompt}
\begin{euleroutput}
  [10.4,  5.6,  3.1,  6.4,  21.7,  10.4,  5.6,  3.1,  6.4,  21.7]
\end{euleroutput}
\begin{eulercomment}
Perhatikan bahwa "=" atau ":=" digunakan untuk tugas. Operator "-\textgreater{}"
digunakan untuk unit di EMT.
\end{eulercomment}
\begin{eulerprompt}
>125km -> " miles"
\end{eulerprompt}
\begin{euleroutput}
  77.6713990297 miles
\end{euleroutput}
\begin{eulercomment}
Operator "\textless{}-" untuk penugasan menyesatkan, dan bukan ide yang baik
untuk R. Berikut ini akan membandingkan dan -4 di EMT.
\end{eulercomment}
\begin{eulerprompt}
>a=2; a<-4
\end{eulerprompt}
\begin{euleroutput}
  0
\end{euleroutput}
\begin{eulercomment}
Di R, "a \textless{}-4 \textless{}3" berfungsi, tetapi "a \textless{}-4 \textless{}-3" tidak. Saya juga
memiliki ambiguitas yang serupa dalam EMT, tetapi mencoba
menghilangkannya terus-menerus.

EMT dan R memiliki vektor tipe boolean. Tetapi di EMT, angka 0 dan 1
digunakan untuk mewakili salah dan benar. Di R, nilai benar dan salah
dapat digunakan dalam aritmatika biasa seperti di EMT.
\end{eulercomment}
\begin{eulerprompt}
>x<5, %*x
\end{eulerprompt}
\begin{euleroutput}
  [0,  0,  1,  0,  0]
  [0,  0,  3.1,  0,  0]
\end{euleroutput}
\begin{eulercomment}
EMT melempar kesalahan atau menghasilkan NAN tergantung pada bendera
"kesalahan".
\end{eulercomment}
\begin{eulerprompt}
>errors off; 0/0, isNAN(sqrt(-1)), errors on;
\end{eulerprompt}
\begin{euleroutput}
  NAN
  1
\end{euleroutput}
\begin{eulercomment}
String sama di R dan EMT. Keduanya ada di lokal saat ini, bukan di
Unicode.

Di R ada paket untuk Unicode. Di EMT, string bisa berupa string
Unicode. String unicode dapat diterjemahkan ke pengkodean lokal dan
sebaliknya. Selain itu, u "..." dapat berisi entitas HTML.
\end{eulercomment}
\begin{eulerprompt}
>u"&#169; Ren&eacut; Grothmann"
\end{eulerprompt}
\begin{euleroutput}
  © René Grothmann
\end{euleroutput}
\begin{eulercomment}
Berikut ini mungkin atau mungkin tidak ditampilkan dengan benar pada
sistem Anda sebagai A dengan titik dan tanda hubung di atasnya. Itu
tergantung pada font yang Anda gunakan.
\end{eulercomment}
\begin{eulerprompt}
>chartoutf([480])
\end{eulerprompt}
\begin{euleroutput}
  Ǡ
\end{euleroutput}
\begin{eulercomment}
Rangkaian string dilakukan dengan "+" atau "\textbar{}". Ini dapat menyertakan
angka, yang akan dicetak dalam format saat ini.
\end{eulercomment}
\begin{eulerprompt}
>"pi = "+pi
\end{eulerprompt}
\begin{euleroutput}
  pi = 3.14159265359
\end{euleroutput}
\eulerheading{Pengindeksan}
\begin{eulercomment}
Biasanya, ini akan berfungsi seperti di R.

Tetapi EMT akan menafsirkan indeks negatif dari belakang vektor,
sedangkan R menafsirkan x [n] sebagai x tanpa elemen ke-n.
\end{eulercomment}
\begin{eulerprompt}
>x, x[1:3], x[-2]
\end{eulerprompt}
\begin{euleroutput}
  [10.4,  5.6,  3.1,  6.4,  21.7]
  [10.4,  5.6,  3.1]
  6.4
\end{euleroutput}
\begin{eulercomment}
Perilaku R dapat dicapai di EMT dengan drop().
\end{eulercomment}
\begin{eulerprompt}
>drop(x,2)
\end{eulerprompt}
\begin{euleroutput}
  [10.4,  3.1,  6.4,  21.7]
\end{euleroutput}
\begin{eulercomment}
Vektor logika tidak diperlakukan secara berbeda sebagai indeks di EMT,
berbeda dengan R. Anda perlu mengekstrak elemen bukan nol terlebih
dahulu di EMT.
\end{eulercomment}
\begin{eulerprompt}
>x, x>5, x[nonzeros(x>5)]
\end{eulerprompt}
\begin{euleroutput}
  [10.4,  5.6,  3.1,  6.4,  21.7]
  [1,  1,  0,  1,  1]
  [10.4,  5.6,  6.4,  21.7]
\end{euleroutput}
\begin{eulercomment}
Sama seperti di R, vektor indeks dapat berisi pengulangan.
\end{eulercomment}
\begin{eulerprompt}
>x[[1,2,2,1]]
\end{eulerprompt}
\begin{euleroutput}
  [10.4,  5.6,  5.6,  10.4]
\end{euleroutput}
\begin{eulercomment}
Tetapi nama untuk indeks tidak dimungkinkan di EMT. Untuk paket
statistik, ini mungkin sering diperlukan untuk memudahkan akses ke
elemen vektor.

Untuk meniru perilaku ini, kita dapat mendefinisikan fungsi sebagai
berikut.
\end{eulercomment}
\begin{eulerprompt}
>function sel (v,i,s) := v[indexof(s,i)]; ...
>s=["first","second","third","fourth"]; sel(x,["first","third"],s)
\end{eulerprompt}
\begin{euleroutput}
  
  Trying to overwrite protected function sel!
  Error in:
  function sel (v,i,s) := v[indexof(s,i)]; ... ...
               ^
  [10.4,  3.1]
\end{euleroutput}
\eulerheading{Jenis Data}
\begin{eulercomment}
EMT memiliki lebih banyak tipe data tetap daripada R. Jelas, di R ada
vektor yang tumbuh. Anda dapat menyetel vektor numerik kosong v dan
menetapkan nilai ke elemen v [17]. Ini tidak mungkin dilakukan di EMT.

Berikut ini agak tidak efisien.
\end{eulercomment}
\begin{eulerprompt}
>v=[]; for i=1 to 10000; v=v|i; end;
\end{eulerprompt}
\begin{eulercomment}
EMT sekarang akan membangun vektor dengan v dan i ditambahkan pada
stack dan menyalin vektor itu kembali ke variabel global v.

Semakin efisien mendefinisikan vektor sebelumnya.
\end{eulercomment}
\begin{eulerprompt}
>v=zeros(10000); for i=1 to 10000; v[i]=i; end;
\end{eulerprompt}
\begin{eulercomment}
Untuk mengubah jenis data di EMT, Anda dapat menggunakan fungsi
seperti complex().
\end{eulercomment}
\begin{eulerprompt}
>complex(1:4)
\end{eulerprompt}
\begin{euleroutput}
  [ 1+0i ,  2+0i ,  3+0i ,  4+0i  ]
\end{euleroutput}
\begin{eulercomment}
Konversi ke string hanya dimungkinkan untuk tipe data dasar. Format
saat ini digunakan untuk penggabungan string sederhana. Tetapi ada
fungsi seperti print() atau frac().

Untuk vektor, Anda dapat dengan mudah menulis fungsi Anda sendiri.
\end{eulercomment}
\begin{eulerprompt}
>function tostr (v) ...
\end{eulerprompt}
\begin{eulerudf}
  s="[";
  loop 1 to length(v);
     s=s+print(v[#],2,0);
     if #<length(v) then s=s+","; endif;
  end;
  return s+"]";
  endfunction
\end{eulerudf}
\begin{eulerprompt}
>tostr(linspace(0,1,10))
\end{eulerprompt}
\begin{euleroutput}
  [0.00,0.10,0.20,0.30,0.40,0.50,0.60,0.70,0.80,0.90,1.00]
\end{euleroutput}
\begin{eulercomment}
Untuk komunikasi dengan Maxima, terdapat fungsi convertmxm (), yang
juga dapat digunakan untuk memformat vektor untuk keluaran.
\end{eulercomment}
\begin{eulerprompt}
>convertmxm(1:10)
\end{eulerprompt}
\begin{euleroutput}
  [1,2,3,4,5,6,7,8,9,10]
\end{euleroutput}
\begin{eulercomment}
Untuk Latex, perintah tex dapat digunakan untuk mendapatkan perintah
Latex.
\end{eulercomment}
\begin{eulerprompt}
>tex(&[1,2,3])
\end{eulerprompt}
\begin{euleroutput}
  \(\backslash\)left[ 1 , 2 , 3 \(\backslash\)right] 
\end{euleroutput}
\eulerheading{Faktor dan Tabel}
\begin{eulercomment}
Dalam pengantar R ada contoh dengan apa yang disebut faktor.

Berikut ini adalah daftar wilayah 30 negara bagian.
\end{eulercomment}
\begin{eulerprompt}
>austates = ["tas", "sa", "qld", "nsw", "nsw", "nt", "wa", "wa", ...
>"qld", "vic", "nsw", "vic", "qld", "qld", "sa", "tas", ...
>"sa", "nt", "wa", "vic", "qld", "nsw", "nsw", "wa", ...
>"sa", "act", "nsw", "vic", "vic", "act"];
\end{eulerprompt}
\begin{eulercomment}
Asumsikan, kami memiliki pendapatan yang sesuai di setiap negara
bagian.
\end{eulercomment}
\begin{eulerprompt}
>incomes = [60, 49, 40, 61, 64, 60, 59, 54, 62, 69, 70, 42, 56, ...
>61, 61, 61, 58, 51, 48, 65, 49, 49, 41, 48, 52, 46, ...
>59, 46, 58, 43];
\end{eulerprompt}
\begin{eulercomment}
Sekarang, kami ingin menghitung rata-rata pendapatan di wilayah
tersebut. Menjadi program statistik, R memiliki faktor () dan tappy ()
untuk ini.

EMT dapat melakukannya dengan mencari indeks teritori dalam daftar
teritori yang unik.
\end{eulercomment}
\begin{eulerprompt}
>auterr=sort(unique(austates)); f=indexofsorted(auterr,austates)
\end{eulerprompt}
\begin{euleroutput}
  [6,  5,  4,  2,  2,  3,  8,  8,  4,  7,  2,  7,  4,  4,  5,  6,  5,  3,
  8,  7,  4,  2,  2,  8,  5,  1,  2,  7,  7,  1]
\end{euleroutput}
\begin{eulercomment}
Pada titik itu, kita bisa menulis fungsi loop kita sendiri untuk
melakukan sesuatu hanya untuk satu faktor.

Atau kita bisa meniru fungsi tapply () dengan cara berikut.
\end{eulercomment}
\begin{eulerprompt}
>function map tappl (i; f$:call, cat, x) ...
\end{eulerprompt}
\begin{eulerudf}
  u=sort(unique(cat));
  f=indexof(u,cat);
  return f$(x[nonzeros(f==indexof(u,i))]);
  endfunction
\end{eulerudf}
\begin{eulercomment}
Ini sedikit tidak efisien, karena ini menghitung wilayah unik untuk
setiap i, tetapi berfungsi.
\end{eulercomment}
\begin{eulerprompt}
>tappl(auterr,"mean",austates,incomes)
\end{eulerprompt}
\begin{euleroutput}
  [44.5,  57.3333,  55.5,  53.6,  55,  60.5,  56,  52.25]
\end{euleroutput}
\begin{eulercomment}
Perhatikan bahwa ini berfungsi untuk setiap vektor wilayah.
\end{eulercomment}
\begin{eulerprompt}
>tappl(["act","nsw"],"mean",austates,incomes)
\end{eulerprompt}
\begin{euleroutput}
  [44.5,  57.3333]
\end{euleroutput}
\begin{eulercomment}
Sekarang, paket statistik EMT mendefinisikan tabel seperti di R.
Fungsi readtable() dan writetable() dapat digunakan untuk input dan
output.

Jadi kita bisa mencetak rata-rata pendapatan negara di wilayah dengan
cara yang bersahabat.
\end{eulercomment}
\begin{eulerprompt}
>writetable(tappl(auterr,"mean",austates,incomes),labc=auterr,wc=7)
\end{eulerprompt}
\begin{euleroutput}
      act    nsw     nt    qld     sa    tas    vic     wa
     44.5  57.33   55.5   53.6     55   60.5     56  52.25
\end{euleroutput}
\begin{eulercomment}
Kami juga dapat mencoba meniru perilaku R sepenuhnya.

Faktor-faktor tersebut harus disimpan dengan jelas dalam koleksi
dengan tipe dan kategori (negara bagian dan teritori dalam contoh
kita). Untuk EMT, kami menambahkan indeks yang telah dihitung
sebelumnya.
\end{eulercomment}
\begin{eulerprompt}
>function makef (t) ...
\end{eulerprompt}
\begin{eulerudf}
  ## Factor data
  ## Returns a collection with data t, unique data, indices.
  ## See: tapply
  u=sort(unique(t));
  return \{\{t,u,indexofsorted(u,t)\}\};
  endfunction
\end{eulerudf}
\begin{eulerprompt}
>statef=makef(austates);
\end{eulerprompt}
\begin{eulercomment}
Sekarang elemen ketiga dari koleksi akan berisi indeks.
\end{eulercomment}
\begin{eulerprompt}
>statef[3]
\end{eulerprompt}
\begin{euleroutput}
  [6,  5,  4,  2,  2,  3,  8,  8,  4,  7,  2,  7,  4,  4,  5,  6,  5,  3,  8,  7,  4,  2,  2,
  8,  5,  1,  2,  7,  7,  1]
\end{euleroutput}
\begin{eulercomment}
Sekarang kita bisa meniru tapply() dengan cara berikut. Ini akan
mengembalikan tabel sebagai kumpulan data tabel dan judul kolom.
\end{eulercomment}
\begin{eulerprompt}
>function tapply (t:vector,tf,f$:call) ...
\end{eulerprompt}
\begin{eulerudf}
  ## Makes a table of data and factors
  ## tf : output of makef()
  ## See: makef
  uf=tf[2]; f=tf[3]; x=zeros(length(uf));
  for i=1 to length(uf);
     ind=nonzeros(f==i);
     if length(ind)==0 then x[i]=NAN;
     else x[i]=f$(t[ind]);
     endif;
  end;
  return \{\{x,uf\}\};
  endfunction
\end{eulerudf}
\begin{eulercomment}
Kita tidak menambahkan banyak jenis pemeriksaan di sini. Tindakan
pencegahan hanya menyangkut kategori (faktor) tanpa data. Tetapi
seseorang harus memeriksa panjang yang benar dari t dan untuk
kebenaran dari koleksi tf.

Tabel ini dapat dicetak sebagai tabel dengan writetable ().
\end{eulercomment}
\begin{eulerprompt}
>writetable(tapply(incomes,statef,"mean"),wc=7)
\end{eulerprompt}
\begin{euleroutput}
      act    nsw     nt    qld     sa    tas    vic     wa
     44.5  57.33   55.5   53.6     55   60.5     56  52.25
\end{euleroutput}
\eulerheading{Array}
\begin{eulercomment}
EMT hanya memiliki dua dimensi untuk array. Tipe datanya disebut
matriks. Akan mudah untuk menulis fungsi untuk dimensi yang lebih
tinggi atau perpustakaan C untuk ini.

R memiliki lebih dari dua dimensi. Dalam R array adalah vektor dengan
bidang dimensi.

Dalam EMT, vektor adalah matriks dengan satu baris. Itu bisa dibuat
menjadi matriks dengan redim().
\end{eulercomment}
\begin{eulerprompt}
>shortformat; X=redim(1:20,4,5)
\end{eulerprompt}
\begin{euleroutput}
          1         2         3         4         5 
          6         7         8         9        10 
         11        12        13        14        15 
         16        17        18        19        20 
\end{euleroutput}
\begin{eulercomment}
Ekstraksi baris dan kolom, atau sub-matriks, sangat mirip dengan R.
\end{eulercomment}
\begin{eulerprompt}
>X[,2:3]
\end{eulerprompt}
\begin{euleroutput}
          2         3 
          7         8 
         12        13 
         17        18 
\end{euleroutput}
\begin{eulercomment}
Namun, di R dimungkinkan untuk mengatur daftar indeks tertentu dari
vektor ke nilai. Hal yang sama mungkin terjadi di EMT hanya dengan
satu loop.
\end{eulercomment}
\begin{eulerprompt}
>function setmatrixvalue (M, i, j, v) ...
\end{eulerprompt}
\begin{eulerudf}
  loop 1 to max(length(i),length(j),length(v))
     M[i\{#\},j\{#\}] = v\{#\};
  end;
  endfunction
\end{eulerudf}
\begin{eulercomment}
Kami mendemonstrasikan ini untuk menunjukkan bahwa matriks dilewatkan
melalui referensi di EMT. Jika Anda tidak ingin mengubah matriks M
asli, Anda perlu menyalinnya di fungsi.
\end{eulercomment}
\begin{eulerprompt}
>setmatrixvalue(X,1:3,3:-1:1,0); X,
\end{eulerprompt}
\begin{euleroutput}
          1         2         0         4         5 
          6         0         8         9        10 
          0        12        13        14        15 
         16        17        18        19        20 
\end{euleroutput}
\begin{eulercomment}
Produk luar di EMT hanya dapat dilakukan di antara vektor. Ini
otomatis karena bahasa matriks. Satu vektor harus menjadi vektor kolom
dan vektor lainnya adalah vektor baris.
\end{eulercomment}
\begin{eulerprompt}
>(1:5)*(1:5)'
\end{eulerprompt}
\begin{euleroutput}
          1         2         3         4         5 
          2         4         6         8        10 
          3         6         9        12        15 
          4         8        12        16        20 
          5        10        15        20        25 
\end{euleroutput}
\begin{eulercomment}
Dalam pengantar PDF untuk R ada contoh, yang menghitung distribusi
ab-cd untuk a, b, c, d yang dipilih dari 0 hingga n secara acak.
Solusi di R adalah membentuk matriks 4 dimensi dan menjalankan table()
di atasnya.

Tentu saja, ini bisa dicapai dengan satu putaran. Tapi loop tidak
efektif di EMT atau R. Di EMT, kita bisa menulis loop di C dan itu
akan menjadi solusi tercepat.

Tetapi kita ingin meniru perilaku R. Untuk ini, kita perlu meratakan
perkalian ab dan membuat matriks ab-cd.
\end{eulercomment}
\begin{eulerprompt}
>a=0:6; b=a'; p=flatten(a*b); q=flatten(p-p'); ...
>u=sort(unique(q)); f=getmultiplicities(u,q); ...
>statplot(u,f,"h"):
\end{eulerprompt}
\begin{eulercomment}
Selain perkalian yang tepat, EMT dapat menghitung frekuensi dalam
vektor.
\end{eulercomment}
\begin{eulerprompt}
>getfrequencies(q,-50:10:50)
\end{eulerprompt}
\begin{euleroutput}
  [0,  23,  132,  316,  602,  801,  333,  141,  53,  0]
\end{euleroutput}
\begin{eulercomment}
Cara paling mudah untuk memplotnya sebagai distribusi adalah sebagai
berikut.
\end{eulercomment}
\begin{eulerprompt}
>plot2d(q,distribution=11):
\end{eulerprompt}
\begin{eulercomment}
Tetapi dimungkinkan juga untuk menghitung sebelumnya dalam interval
yang dipilih sebelumnya. Tentu saja, berikut ini menggunakan
getfrequencies() secara internal.

Karena fungsi histo() mengembalikan frekuensi, kita perlu
menskalakannya sehingga integral di bawah grafik batang adalah 1.
\end{eulercomment}
\begin{eulerprompt}
>\{x,y\}=histo(q,v=-55:10:55); y=y/sum(y)/differences(x); ...
>plot2d(x,y,>bar,style="/"):
\end{eulerprompt}
\eulerheading{Daftar}
\begin{eulercomment}
EMT memiliki dua macam daftar. Salah satunya adalah daftar global yang
dapat berubah, dan yang lainnya adalah jenis daftar yang tidak dapat
diubah. Kami tidak peduli dengan daftar global di sini.

Jenis daftar yang tidak dapat diubah disebut koleksi di EMT. Ini
berperilaku seperti struktur di C, tetapi elemen hanya diberi nomor
dan tidak dinamai.
\end{eulercomment}
\begin{eulerprompt}
>L=\{\{"Fred","Flintstone",40,[1990,1992]\}\}
\end{eulerprompt}
\begin{euleroutput}
  Fred
  Flintstone
  40
  [1990,  1992]
\end{euleroutput}
\begin{eulercomment}
Saat ini elemen tidak memiliki nama, meskipun nama dapat diatur untuk
tujuan khusus. Mereka diakses dengan angka.
\end{eulercomment}
\begin{eulerprompt}
>(L[4])[2]
\end{eulerprompt}
\begin{euleroutput}
  1992
\end{euleroutput}
\eulerheading{Input dan Output File (Membaca dan Menulis Data)}
\begin{eulercomment}
Anda akan sering ingin mengimpor matriks data dari sumber lain ke EMT.
Tutorial ini memberi tahu Anda tentang banyak cara untuk mencapai ini.
Fungsi sederhana adalah writematrix() dan readmatrix().

Mari kita tunjukkan bagaimana membaca dan menulis vektor real ke file.
\end{eulercomment}
\begin{eulerprompt}
>a=random(1,100); mean(a), dev(a),
\end{eulerprompt}
\begin{euleroutput}
  0.47466
  0.27327
\end{euleroutput}
\begin{eulercomment}
Untuk menulis data ke file, kami menggunakan fungsi writematrix().

Karena pendahuluan ini kemungkinan besar ada di direktori, di mana
pengguna tidak memiliki akses tulis, kami menulis data ke direktori
home pengguna. Untuk buku catatan sendiri, ini tidak perlu, karena
file data akan ditulis ke direktori yang sama.
\end{eulercomment}
\begin{eulerprompt}
>filename="test.dat";
\end{eulerprompt}
\begin{eulercomment}
Sekarang kita menulis vektor kolom a 'ke file. Ini menghasilkan satu
nomor di setiap baris file.
\end{eulercomment}
\begin{eulerprompt}
>writematrix(a',filename);
\end{eulerprompt}
\begin{eulercomment}
Untuk membaca data, kami menggunakan readmatrix().
\end{eulercomment}
\begin{eulerprompt}
>a=readmatrix(filename)';
\end{eulerprompt}
\begin{eulercomment}
Dan hapus file tersebut.
\end{eulercomment}
\begin{eulerprompt}
>fileremove(filename);
>mean(a), dev(a),
\end{eulerprompt}
\begin{euleroutput}
  0.47466
  0.27327
\end{euleroutput}
\begin{eulercomment}
Fungsi writematrix() atau writetable() dapat dikonfigurasi untuk
bahasa lain.

Misalnya, jika Anda memiliki sistem Indonesia (titik desimal dengan
koma), Excel Anda memerlukan nilai dengan koma desimal yang dipisahkan
oleh titik koma dalam file csv (defaultnya adalah nilai yang
dipisahkan koma). File berikut "test.csv" akan muncul di folder
cuurent Anda.
\end{eulercomment}
\begin{eulerprompt}
>filename="test.csv"; ...
>writematrix(random(5,3),file=filename,separator=",");
\end{eulerprompt}
\begin{eulercomment}
Anda sekarang dapat membuka file ini dengan Excel Indonesia secara
langsung.
\end{eulercomment}
\begin{eulerprompt}
>fileremove(filename);
\end{eulerprompt}
\begin{eulercomment}
Terkadang kami memiliki string dengan token seperti berikut.
\end{eulercomment}
\begin{eulerprompt}
>s1:="f m m f m m m f f f m m f";  ...
>s2:="f f f m m f f";
\end{eulerprompt}
\begin{eulercomment}
Untuk membuat token ini, kita mendefinisikan vektor token.
\end{eulercomment}
\begin{eulerprompt}
>tok:=["f","m"]
\end{eulerprompt}
\begin{euleroutput}
  f
  m
\end{euleroutput}
\begin{eulercomment}
Kemudian kita dapat menghitung berapa kali setiap token muncul dalam
string, dan memasukkan hasilnya ke dalam tabel.
\end{eulercomment}
\begin{eulerprompt}
>M:=getmultiplicities(tok,strtokens(s1))_ ...
>  getmultiplicities(tok,strtokens(s2));
\end{eulerprompt}
\begin{eulercomment}
Tulis tabel dengan header token.
\end{eulercomment}
\begin{eulerprompt}
>writetable(M,labc=tok,labr=1:2,wc=8)
\end{eulerprompt}
\begin{euleroutput}
                 f       m
         1       6       7
         2       5       2
\end{euleroutput}
\begin{eulercomment}
Untuk statika, EMT dapat membaca dan menulis tabel.
\end{eulercomment}
\begin{eulerprompt}
>file="test.dat"; open(file,"w"); ...
>writeln("A,B,C"); writematrix(random(3,3)); ...
>close();
\end{eulerprompt}
\begin{eulercomment}
File tersebut terlihat seperti ini.
\end{eulercomment}
\begin{eulerprompt}
>printfile(file)
\end{eulerprompt}
\begin{euleroutput}
  A,B,C
  0.09108070085843074,0.8993342814237876,0.983635710835939
  0.211019679874495,0.5527768679829471,0.4941417784439638
  0.5797836292915086,0.8118750914043666,0.9838772074457084
  
\end{euleroutput}
\begin{eulercomment}
Fungsi readtable() dalam bentuknya yang paling sederhana bisa membaca
ini dan mengembalikan kumpulan nilai dan baris judul.
\end{eulercomment}
\begin{eulerprompt}
>L=readtable(file,>list);
\end{eulerprompt}
\begin{eulercomment}
Koleksi ini dapat dicetak dengan writetable() ke buku catatan, atau ke
file.
\end{eulercomment}
\begin{eulerprompt}
>writetable(L,wc=10,dc=5)
\end{eulerprompt}
\begin{euleroutput}
           A         B         C
     0.09108   0.89933   0.98364
     0.21102   0.55278   0.49414
     0.57978   0.81188   0.98388
\end{euleroutput}
\begin{eulercomment}
Matriks nilai adalah elemen pertama L. Perhatikan bahwa mean () dalam
EMT menghitung nilai mean dari baris-baris matriks.
\end{eulercomment}
\begin{eulerprompt}
>mean(L[1])
\end{eulerprompt}
\begin{euleroutput}
    0.65802 
    0.41931 
    0.79185 
\end{euleroutput}
\eulerheading{File CSV}
\begin{eulercomment}
Pertama, mari kita tulis matriks ke dalam file. Untuk hasilnya, kami
menghasilkan file di direktori kerja saat ini.
\end{eulercomment}
\begin{eulerprompt}
>file="test.csv";  ...
>M=random(3,3); writematrix(M,file);
\end{eulerprompt}
\begin{eulercomment}
Berikut isi dari file ini.
\end{eulercomment}
\begin{eulerprompt}
>printfile(file)
\end{eulerprompt}
\begin{euleroutput}
  0.2629555460769783,0.9938043902969794,0.9018322446643099
  0.2407959115075192,0.6359669287024015,0.1510861324343464
  0.2515041042947438,0.3339714884700144,0.7611390495192349
  
\end{euleroutput}
\begin{eulercomment}
CVS ini dapat dibuka pada sistem bahasa Inggris ke Excel dengan klik
dua kali. Jika Anda mendapatkan file seperti itu di sistem Jerman,
Anda perlu mengimpor data ke Excel dengan menggunakan titik desimal.

Tetapi titik desimal adalah format default untuk EMT juga. Anda bisa
membaca matriks dari file dengan readmatrix().
\end{eulercomment}
\begin{eulerprompt}
>readmatrix(file)
\end{eulerprompt}
\begin{euleroutput}
    0.26296    0.9938   0.90183 
     0.2408   0.63597   0.15109 
     0.2515   0.33397   0.76114 
\end{euleroutput}
\begin{eulercomment}
Dimungkinkan untuk menulis beberapa matriks ke satu file. Perintah
open() dapat membuka file untuk ditulis dengan parameter "w".
Standarnya adalah "r" untuk membaca.
\end{eulercomment}
\begin{eulerprompt}
>open(file,"w"); writematrix(M); writematrix(M'); close();
\end{eulerprompt}
\begin{eulercomment}
Matriks dipisahkan oleh garis kosong. Untuk membaca matriks, buka file
dan panggil readmatrix() beberapa kali.
\end{eulercomment}
\begin{eulerprompt}
>open(file); A=readmatrix(); B=readmatrix(); A==B, close();
\end{eulerprompt}
\begin{euleroutput}
          1         0         0 
          0         1         0 
          0         0         1 
\end{euleroutput}
\begin{eulercomment}
Di Excel atau spreadsheet serupa, Anda dapat mengekspor matriks
sebagai CSV (nilai dipisahkan koma). Di Excel 2007, gunakan "simpan
sebagai" dan "format lain", lalu pilih "CSV". Pastikan, tabel saat ini
hanya berisi data yang ingin Anda ekspor.

Berikut ini contohnya.
\end{eulercomment}
\begin{eulerprompt}
>printfile("excel-data.csv")
\end{eulerprompt}
\begin{euleroutput}
  0;1000;1000
  1;1051,271096;1072,508181
  2;1105,170918;1150,273799
  3;1161,834243;1233,67806
  4;1221,402758;1323,129812
  5;1284,025417;1419,067549
  6;1349,858808;1521,961556
  7;1419,067549;1632,31622
  8;1491,824698;1750,6725
  9;1568,312185;1877,610579
  10;1648,721271;2013,752707
\end{euleroutput}
\begin{eulercomment}
Seperti yang Anda lihat, sistem Jerman saya menggunakan titik koma
sebagai pemisah dan koma desimal. Anda dapat mengubahnya di pengaturan
sistem atau di Excel, tetapi tidak perlu membaca matriks ke EMT.

Cara termudah untuk membaca ini ke dalam Euler adalah readmatrix().
Semua koma diganti dengan titik dengan parameter\textgreater{} koma. Untuk CSV
bahasa Inggris, cukup abaikan parameter ini.
\end{eulercomment}
\begin{eulerprompt}
>M=readmatrix("excel-data.csv",>comma)
\end{eulerprompt}
\begin{euleroutput}
          0      1000      1000 
          1    1051.3    1072.5 
          2    1105.2    1150.3 
          3    1161.8    1233.7 
          4    1221.4    1323.1 
          5      1284    1419.1 
          6    1349.9      1522 
          7    1419.1    1632.3 
          8    1491.8    1750.7 
          9    1568.3    1877.6 
         10    1648.7    2013.8 
\end{euleroutput}
\begin{eulercomment}
Mari kita plot ini.
\end{eulercomment}
\begin{eulerprompt}
>plot2d(M'[1],M'[2:3],>points,color=[red,green]'):
\end{eulerprompt}
\begin{eulercomment}
Ada cara yang lebih mendasar untuk membaca data dari sebuah file. Anda
dapat membuka file dan membaca angka baris demi baris. Fungsi
getvectorline() akan membaca angka dari sebaris data. Secara default,
ini mengharapkan titik desimal. Tapi itu juga bisa menggunakan koma
desimal, jika Anda memanggil setdecimaldot (",") sebelum Anda
menggunakan fungsi ini.

Fungsi berikut adalah contoh untuk ini. Ini akan berhenti di akhir
file atau baris kosong.
\end{eulercomment}
\begin{eulerprompt}
>function myload (file) ...
\end{eulerprompt}
\begin{eulerudf}
  open(file);
  M=[];
  repeat
     until eof();
     v=getvectorline(3);
     if length(v)>0 then M=M_v; else break; endif;
  end;
  return M;
  close(file);
  endfunction
\end{eulerudf}
\begin{eulerprompt}
>myload(file)
\end{eulerprompt}
\begin{euleroutput}
    0.26296    0.9938   0.90183 
     0.2408   0.63597   0.15109 
     0.2515   0.33397   0.76114 
\end{euleroutput}
\begin{eulercomment}
Juga dimungkinkan untuk membaca semua angka dalam file itu dengan
getvector().
\end{eulercomment}
\begin{eulerprompt}
>open(file); v=getvector(10000); close(); redim(v[1:9],3,3)
\end{eulerprompt}
\begin{euleroutput}
    0.26296    0.9938   0.90183 
     0.2408   0.63597   0.15109 
     0.2515   0.33397   0.76114 
\end{euleroutput}
\begin{eulercomment}
Oleh karena itu, sangat mudah untuk menyimpan sebuah vektor nilai,
satu nilai di setiap baris dan membaca kembali vektor ini.
\end{eulercomment}
\begin{eulerprompt}
>v=random(1000); mean(v)
\end{eulerprompt}
\begin{euleroutput}
  0.50339
\end{euleroutput}
\begin{eulerprompt}
>writematrix(v',file); mean(readmatrix(file)')
\end{eulerprompt}
\begin{euleroutput}
  0.50339
\end{euleroutput}
\eulerheading{Menggunakan Tabel}
\begin{eulercomment}
Tabel dapat digunakan untuk membaca atau menulis data numerik. Sebagai
contoh, kami menulis tabel dengan judul baris dan kolom ke sebuah
file.
\end{eulercomment}
\begin{eulerprompt}
>file="test.tab"; M=random(3,3);  ...
>open(file,"w");  ...
>writetable(M,separator=",",labc=["one","two","three"]);  ...
>close(); ...
>printfile(file)
\end{eulerprompt}
\begin{euleroutput}
  one,two,three
        0.59,      0.81,      0.78
        0.03,      0.57,      0.22
        0.66,      0.54,      0.92
\end{euleroutput}
\begin{eulercomment}
Ini dapat diimpor ke Excel.

Untuk membaca file di EMT, kami menggunakan readtable().
\end{eulercomment}
\begin{eulerprompt}
>\{M,headings\}=readtable(file,>clabs); ...
>writetable(M,labc=headings)
\end{eulerprompt}
\begin{euleroutput}
         one       two     three
        0.59      0.81      0.78
        0.03      0.57      0.22
        0.66      0.54      0.92
\end{euleroutput}
\eulerheading{Menganalisis Garis}
\begin{eulercomment}
Anda bahkan dapat mengevaluasi setiap baris dengan tangan. Misalkan,
kita memiliki garis dengan format berikut.
\end{eulercomment}
\begin{eulerprompt}
>line="2020-11-03,Tue,1'114.05"
\end{eulerprompt}
\begin{euleroutput}
  2020-11-03,Tue,1'114.05
\end{euleroutput}
\begin{eulercomment}
Pertama kita bisa membuat token baris.
\end{eulercomment}
\begin{eulerprompt}
>vt=strtokens(line)
\end{eulerprompt}
\begin{euleroutput}
  2020-11-03
  Tue
  1'114.05
\end{euleroutput}
\begin{eulercomment}
Kemudian kita dapat mengevaluasi setiap elemen garis menggunakan
evaluasi yang sesuai.
\end{eulercomment}
\begin{eulerprompt}
>day(vt[1]),  ...
>indexof(["mon","tue","wed","thu","fri","sat","sun"],tolower(vt[2])),  ...
>strrepl(vt[3],"'","")()
\end{eulerprompt}
\begin{euleroutput}
  7.3816e+05
  2
  1114
\end{euleroutput}
\begin{eulercomment}
Menggunakan ekspresi reguler, dimungkinkan untuk mengekstrak hampir
semua informasi dari sebaris data.

Asumsikan kita memiliki baris berikut dokumen HTML.
\end{eulercomment}
\begin{eulerprompt}
>line="<tr><td>1145.45</td><td>5.6</td><td>-4.5</td><tr>"
\end{eulerprompt}
\begin{euleroutput}
  <tr><td>1145.45</td><td>5.6</td><td>-4.5</td><tr>
\end{euleroutput}
\begin{eulercomment}
Untuk mengekstrak ini, kami menggunakan ekspresi reguler, yang mencari

\end{eulercomment}
\begin{eulerttcomment}
 - braket penutup>,
 - string apapun yang tidak mengandung tanda kurung dengan
\end{eulerttcomment}
\begin{eulercomment}
sub-kecocokan "(...)",\\
\end{eulercomment}
\begin{eulerttcomment}
 - kurung buka dan tutup menggunakan solusi terpendek,
 - lagi string apapun yang tidak mengandung tanda kurung,
 - dan kurung buka <.
\end{eulerttcomment}
\begin{eulercomment}

Ekspresi reguler agak sulit dipelajari tetapi sangat kuat.
\end{eulercomment}
\begin{eulerprompt}
>\{pos,s,vt\}=strxfind(line,">([^<>]+)<.+?>([^<>]+)<");
\end{eulerprompt}
\begin{eulercomment}
Hasilnya adalah posisi pertandingan, string yang cocok, dan vektor
string untuk sub-pencocokan.
\end{eulercomment}
\begin{eulerprompt}
>for k=1:length(vt); vt[k](), end;
\end{eulerprompt}
\begin{euleroutput}
  1145.5
  5.6
\end{euleroutput}
\begin{eulercomment}
Ini adalah fungsi yang membaca semua item numerik antara \textless{}td\textgreater{} dan
\textless{}/td\textgreater{}.
\end{eulercomment}
\begin{eulerprompt}
>function readtd (line) ...
\end{eulerprompt}
\begin{eulerudf}
  v=[]; cp=0;
  repeat
     \{pos,s,vt\}=strxfind(line,"<td.*?>(.+?)</td>",cp);
     until pos==0;
     if length(vt)>0 then v=v|vt[1]; endif;
     cp=pos+strlen(s);
  end;
  return v;
  endfunction
\end{eulerudf}
\begin{eulerprompt}
>readtd(line+"<td>non-numerical</td>")
\end{eulerprompt}
\begin{euleroutput}
  1145.45
  5.6
  -4.5
  non-numerical
\end{euleroutput}
\eulerheading{Membaca dari Web}
\begin{eulercomment}
Situs web atau file dengan URL dapat dibuka di EMT dan dapat dibaca
baris demi baris.

Dalam contoh, kami membaca versi saat ini dari situs EMT. Kami
menggunakan ekspresi reguler untuk memindai "Versi ..." di sebuah
judul.
\end{eulercomment}
\begin{eulerprompt}
>function readversion () ...
\end{eulerprompt}
\begin{eulerudf}
  urlopen("http://www.euler-math-toolbox.de/Programs/Changes.html");
  repeat
    until urleof();
    s=urlgetline();
    k=strfind(s,"Version ",1);
    if k>0 then substring(s,k,strfind(s,"<",k)-1), break; endif;
  end;
  urlclose();
  endfunction
\end{eulerudf}
\begin{eulerprompt}
>readversion
\end{eulerprompt}
\begin{euleroutput}
  Version 2022-05-18
\end{euleroutput}
\eulerheading{Input dan Output Variabel}
\begin{eulercomment}
Anda dapat menulis variabel dalam bentuk definisi Euler ke file atau
ke baris perintah.
\end{eulercomment}
\begin{eulerprompt}
>writevar(pi,"mypi");
\end{eulerprompt}
\begin{euleroutput}
  mypi = 3.141592653589793;
\end{euleroutput}
\begin{eulercomment}
Untuk pengujian, kami menghasilkan file Euler di direktori kerja EMT.
\end{eulercomment}
\begin{eulerprompt}
>file="test.e"; ...
>writevar(random(2,2),"M",file); ...
>printfile(file,3)
\end{eulerprompt}
\begin{euleroutput}
  M = [ ..
  0.9607363360294088, 0.5090564281073382;
  0.67279466809125, 0.688176242399486];
\end{euleroutput}
\begin{eulercomment}
Kami sekarang dapat memuat file. Ini akan mendefinisikan matriks M.
\end{eulercomment}
\begin{eulerprompt}
>load(file); show M,
\end{eulerprompt}
\begin{euleroutput}
  M = 
    0.96074   0.50906 
    0.67279   0.68818 
\end{euleroutput}
\begin{eulercomment}
Ngomong-ngomong, jika writevar() digunakan pada variabel, itu akan
mencetak definisi variabel dengan nama variabel ini.
\end{eulercomment}
\begin{eulerprompt}
>writevar(M); writevar(inch$)
\end{eulerprompt}
\begin{euleroutput}
  M = [ ..
  0.9607363360294088, 0.5090564281073382;
  0.67279466809125, 0.688176242399486];
  inch$ = 0.0254;
\end{euleroutput}
\begin{eulercomment}
Kami juga dapat membuka file baru atau menambahkan ke file yang sudah
ada. Dalam contoh kami menambahkan file yang dibuat sebelumnya.
\end{eulercomment}
\begin{eulerprompt}
>open(file,"a"); ...
>writevar(random(2,2),"M1"); ...
>writevar(random(3,1),"M2"); ...
>close();
>load(file); show M1; show M2;
\end{eulerprompt}
\begin{euleroutput}
  M1 = 
    0.72639   0.62882 
    0.12019   0.03224 
  M2 = 
   0.099173 
    0.85504 
    0.92847 
\end{euleroutput}
\begin{eulercomment}
Untuk menghapus file apa pun, gunakan fileremove().
\end{eulercomment}
\begin{eulerprompt}
>fileremove(file);
\end{eulerprompt}
\begin{eulercomment}
Vektor baris dalam file tidak memerlukan koma, jika setiap nomor ada
di baris baru. Mari kita buat file seperti itu, tulis setiap baris
satu per satu dengan writeln().
\end{eulercomment}
\begin{eulerprompt}
>open(file,"w"); writeln("M = ["); ...
>for i=1 to 5; writeln(""+random()); end; ...
>writeln("];"); close(); ...
>printfile(file)
\end{eulerprompt}
\begin{euleroutput}
  M = [
  0.437022605979
  0.206717463476
  0.466146119229
  0.284765700905
  0.208351197517
  ];
\end{euleroutput}
\begin{eulerprompt}
>load(file); M
\end{eulerprompt}
\begin{euleroutput}
  [0.43702,  0.20672,  0.46615,  0.28477,  0.20835]
\end{euleroutput}
\eulersubheading{Contoh Soal}
\begin{eulercomment}
1.Tabulasi data penelitian antara dua varibel biaya promosi (X) dan
variabel penjualan rumah (Y)
\end{eulercomment}
\begin{eulerprompt}
>a=[10,28,29,35,48,55,71,73,80,88,91,111,131,144,160]
\end{eulerprompt}
\begin{euleroutput}
  [10,  28,  29,  35,  48,  55,  71,  73,  80,  88,  91,  111,  131,
  144,  160]
\end{euleroutput}
\begin{eulerprompt}
>b=[29,47,55,65,79,82,92,95,100,102,110,124,127,130,152]
\end{eulerprompt}
\begin{euleroutput}
  [29,  47,  55,  65,  79,  82,  92,  95,  100,  102,  110,  124,  127,
  130,  152]
\end{euleroutput}
\begin{eulerprompt}
>writetable(a'|b',labc=["a","b"])
\end{eulerprompt}
\begin{euleroutput}
           a         b
          10        29
          28        47
          29        55
          35        65
          48        79
          55        82
          71        92
          73        95
          80       100
          88       102
          91       110
         111       124
         131       127
         144       130
         160       152
\end{euleroutput}
\begin{eulerprompt}
>p=polyfit(a,b,1)
\end{eulerprompt}
\begin{euleroutput}
  [35.643,  0.74034]
\end{euleroutput}
\begin{eulercomment}
Nilai konstanta (a)=35.64 menunjukkan besarnya variabel rata-rata
penjualan rumah yang tidak dipengaruhi oleh biaya promosi atau dapat
diartikan pada saat nilai biaya promosi sebesar 0, maka rata-rata
penjualan rumah sebesar 35.64.

Nilai koefisien korelasi diperoleh sebesar 0,977.Hal ini berarti
adanya hubungan positif antara biaya yang dikeluarkan untuk promosi
dengan rata-rata penjualan rumah. Jika dilihat dari nilai korelasi
hubungan variabel termasuk kategori tinggi,dengan demikian berarti
biaya promosi memiliki hubungan yang tinggi terhadap kenaikan
rata-rata penjualan rumah.

Dari hasil pengukuran diperoleh data tinggi badan kesepuluh siswa
tersebut dalam ukuran sentimeter (cm) sebagai berikut.
\end{eulercomment}
\begin{eulerprompt}
>h=[172,167,180,170,169,160,175,165,173,170]
\end{eulerprompt}
\begin{euleroutput}
  [172,  167,  180,  170,  169,  160,  175,  165,  173,  170]
\end{euleroutput}
\begin{eulerprompt}
>mean(h)
\end{eulerprompt}
\begin{euleroutput}
  170.1
\end{euleroutput}
\begin{eulerprompt}
>dev(h)
\end{eulerprompt}
\begin{euleroutput}
  5.5066
\end{euleroutput}
\begin{eulerprompt}
>boxplot(h):
\end{eulerprompt}
\begin{eulercomment}
Suatu penelitian dilakukan untuk mengetahui apakah terdapat pengaruh
perbedaan kartu kredit terhadap penggunaannya. Data di bawah ini
adalah jumlah uang yang dibelanjakan ibu rumah tangga menggunakan
kartu kredit (dalam \textdollar{}). Empat jenis kartu kredit dibandingkan:
\end{eulercomment}
\begin{eulerprompt}
>Astra=[8,7,10,19,11]
\end{eulerprompt}
\begin{euleroutput}
  [8,  7,  10,  19,  11]
\end{euleroutput}
\begin{eulerprompt}
>BCA=[12,11,16,10,12]
\end{eulerprompt}
\begin{euleroutput}
  [12,  11,  16,  10,  12]
\end{euleroutput}
\begin{eulerprompt}
>CITI=[19,20,15,18,19]
\end{eulerprompt}
\begin{euleroutput}
  [19,  20,  15,  18,  19]
\end{euleroutput}
\begin{eulerprompt}
>AMEX=[13,12,14,15]
\end{eulerprompt}
\begin{euleroutput}
  [13,  12,  14,  15]
\end{euleroutput}
\begin{eulerprompt}
>varanalysis(Astra,BCA,CITI,AMEX)
\end{eulerprompt}
\begin{euleroutput}
  0.0082162
\end{euleroutput}
\begin{eulercomment}
Dari hasil tersebut dapat disimpulkan bahwa tidak ada kesamaan rata
rata dari data tersebut, atau Hipotesis kesamaan rata rata di tolak,
dengan probabilitas kesalahan sebesar 0.8\%
\end{eulercomment}
\end{eulernotebook}
\end{document}


\end{document}
