\documentclass{report}

\usepackage[utf8]{inputenc}
\usepackage{eumat}
\usepackage[Conny]{fncychap}
\usepackage[bahasa]{babel}

% Rename Contents
\addto\captionsenglish{\renewcommand{\contentsname}{\vspace{-0.5cm} \textbf{Daftar Isi} \vspace{-2cm}}}

\begin{document}

% Cover Page
\begin{titlepage}
    \begin{center}
        \vspace*{1cm}
        
        \vspace{0.5cm}
        
        \LARGE
        Tugas Latex Aplikasi Komputer  
        
        \vspace{1cm}
        
        \includegraphics[width=0.5\textwidth]{image.png}

        \vspace{1cm}
        
        \textbf{Rasyid Shalahuddin}\\
        22305144016\\
        Matematika E 2022
        
        \vspace{2cm}
        
        \Large
        \textbf{PRODI MATEMATIKA}\\
        \textbf{DEPARTEMEN PENDIDIKAN MATEMATIKA}\\
        \textbf{FAKULTAS MATEMATIKA DAN ILMU PENGETAHUAN ALAM}
        \textbf{UNIVERSITAS NEGERI YOGYAKARTA}\\
        \textbf{2022}
        
    \end{center}
\end{titlepage}

\newpage
\tableofcontents{}

\chapter{KB Pekan 2 (Belajar Menggunakan Software EMT)}
\input{Rasyid Shalahuddin_22305144016_EMT}

\newpage
\chapter{KB Pekan 3: Menggunakan EMT untuk menyelesaikan masalah-masalah Aljabar}
\input{Rasyid Shalahuddin_22305144016_EMT aljabar}

\newpage
\chapter{KB Pekan 4: Menggunakan EMT untuk mengambar grafik 2 dimensi (2D)}
\documentclass[a4paper,10pt]{article}
\usepackage{eumat}

\begin{document}
\begin{eulernotebook}
\eulerheading{EMT plot2D}
\begin{eulercomment}
Rasyid Salahuddin\\
22305144016\\
Matematika E



\begin{eulercomment}
\eulerheading{Sub Bab 1}
\begin{eulercomment}
Menggambar Grafik Fungsi Satu Variabel dalam Bentuk Ekspresi Langsung
Ekspresi tunggal

Di dalam program numerik EMT, ekspresi adalah string. Jika ditandai
sebagai simbolis, mereka akan mencetak melalui Maxima, jika tidak
melalui EMT. Ekspresi dalam string digunakan untuk membuat plot dan
banyak fungsi numerik. Untuk ini, variabel dalam ekspresi harus "x".

expresi dalam string
\end{eulercomment}
\begin{eulerprompt}
>expr := "x^5-x^2-3"
\end{eulerprompt}
\begin{euleroutput}
  x^5-x^2-3
\end{euleroutput}
\begin{eulercomment}
plot ekspresi
\end{eulercomment}
\begin{eulerprompt}
>plot2d(expr,-2,2) :
\end{eulerprompt}
\begin{eulercomment}
contoh 1
\end{eulercomment}
\begin{eulerprompt}
>expr := "sin (x-5)"
\end{eulerprompt}
\begin{euleroutput}
  sin (x-5)
\end{euleroutput}
\begin{eulerprompt}
>aspect (1) ; plot2d(expr,-2,2):
\end{eulerprompt}
\begin{eulercomment}
contoh 2 dan penggunaan grid
\end{eulercomment}
\begin{eulerprompt}
>aspect(1)plot2d("log(x) + 3",-0.1,2, grid=6):
\end{eulerprompt}
\begin{euleroutput}
  Commands must be separated by semicolon or comma!
  Found: plot2d("log(x) + 3",-0.1,2, grid=6): (character 112)
  You can disable this in the Options menu.
  Error in:
  aspect(1)plot2d("log(x) + 3",-0.1,2, grid=6): ...
           ^
\end{euleroutput}
\begin{eulercomment}
contoh 3 dan penggunaan parameter square (atau \textgreater{}square) untuk memilih
y-range secara otomatis 
\end{eulercomment}
\begin{eulerprompt}
>aspect(1,1) ; plot2d("x^4-2",-5,5, >square); insimg(15)
>aspect(2) ; plot2d("x^4-2", -5,5 ):
\end{eulerprompt}
\begin{eulercomment}
contoh 4 dan memberikan nama atau label pada garis sumbu
\end{eulercomment}
\begin{eulerprompt}
>plot2d("cos(x)", -4, 6, xl="x",yl="y") :
\end{eulerprompt}
\eulerheading{Sub Bab 2}
\begin{eulercomment}
Menggambar Grafik Fungsi Satu Variabel yang rumusnya Disimpan dalam
Variabel Ekspresi

\end{eulercomment}
\begin{eulerttcomment}
 ekspresi
\end{eulerttcomment}
\begin{eulerprompt}
>expr &= x^5-1
\end{eulerprompt}
\begin{euleroutput}
  
                                   5
                                  x  - 1
  
\end{euleroutput}
\begin{eulercomment}
plot dari ekspresi diatas 
\end{eulercomment}
\begin{eulerprompt}
>aspect(2); plot2d(expr,-1,1):
\end{eulerprompt}
\begin{eulercomment}
contoh 1
\end{eulercomment}
\begin{eulerprompt}
>expr := "x^10-x-5"
\end{eulerprompt}
\begin{euleroutput}
  x^10-x-5
\end{euleroutput}
\begin{eulerprompt}
>aspect(2) ; plot2d(expr,-1,1):
\end{eulerprompt}
\begin{eulercomment}
menggunakan variabel lokal
\end{eulercomment}
\begin{eulercomment}
Ekspresi dapat dievaluasi secara numerik. Variabel x,y,z ditetapkan
secara otomatis. Variabel lain dapat ditetapkan berdasarkan parameter
yang ditetapkan( variabel lokal ) atau melalui variabel global.
variabel global adalah variabel yang selalu bisa diakses kapan pun dan
di mana pun.
\end{eulercomment}
\begin{eulerprompt}
>expr &= a*x^5
\end{eulerprompt}
\begin{euleroutput}
  
                                      5
                                   a x
  
\end{euleroutput}
\begin{eulercomment}
menggunakan variabel global 
\end{eulercomment}
\begin{eulerprompt}
>a=6; expr(2.5)
\end{eulerprompt}
\begin{euleroutput}
  585.9375
\end{euleroutput}
\begin{eulercomment}
menggunakan variabel lokal
\end{eulercomment}
\begin{eulerprompt}
>expr(2.5,a=6)
\end{eulerprompt}
\begin{euleroutput}
  585.9375
\end{euleroutput}
\begin{eulercomment}
evaluasi langsung
\end{eulercomment}
\begin{eulerprompt}
>"a*x^5"(3,4)
\end{eulerprompt}
\begin{euleroutput}
  1458
\end{euleroutput}
\begin{eulercomment}
Oleh karena itu, banyak algoritma EMT yang dapat menggunakan ekspresi
dalam x, bukan fungsi. Namun jika parameter tambahan yang tidak
bersifat global dilibatkan, fungsi harus diutamakan.

menggunakan variabel  global "a"
\end{eulercomment}
\begin{eulerprompt}
>a=5; plot2d("a*x^3-x",0,1):
>function f(x,a) := a*x^3-x
\end{eulerprompt}
\begin{eulercomment}
gunakan "a=6" sebagai parameter
\end{eulercomment}
\begin{eulerprompt}
>plot2d("f",0,1;6):
\end{eulerprompt}
\begin{eulercomment}
alternatif lain
\end{eulercomment}
\begin{eulerprompt}
>plot2d(\{\{"f",6\}\},0,1):
\end{eulerprompt}
\begin{eulercomment}
alternatif lain 
\end{eulercomment}
\begin{eulerprompt}
>plot2d("f(x,6)",0,1):
\end{eulerprompt}
\eulerheading{Sub Bab 3}
\begin{eulercomment}
Menggambar Fungsi Simbolik

Fungsi Plot yang paling penting untuk plot planar adalah plot2d().
Fungsi ini diimplementasikan dalam bahasa Euler dalam file "plot.e",
yang dimuat diawal program.

plot2d() menerima ekspresi, fungsi, dan data.

Rentang plot diatur dengan parameter yang ditetapkan ssbagai berikut\\
- a,b: rentang x (default -2,2)\\
- -c,d: rentang y (default: skala dengan nilai)\\
- r: alternatifnya radius di sekitar pusat plot\\
- cx,cy: koordinat pusat plot (default 0,0)

Keterangan:(menggambar grafik fungsi satu variabel yang fungsinya
didefinisikan sebagai fungsi simbolik)\\
- \&: untuk menampilkan variabel pada teks

Berikut adalah beberapa contoh menggunakan fungsi. Seperti biasa di
EMT, fungsi yang berfungsi untuk fungsi atau ekspresi lain, jadi kita
dapat meneruskan parameter tambahan (selain x) yang bukan variabel
global ke fungsi dengan parameter titik koma atau dengan koleksi
panggilan.
\end{eulercomment}
\begin{eulerprompt}
>plot2d("f",0,1;0.4): // plot with a=0.4
>plot2d(\{\{"f",0.2\}\},0,1); 
>plot2d(\{\{"f(x,b)",b=0.1\}\},0,1):
>function f(x) := x^3-x;...
>plot2d("f",r=1):
>plot2d("exp(-a*x^2)/a"):
\end{eulerprompt}
\begin{eulercomment}
Berikut merupakan ringkasan dari fungsi yang diterima\\
- ekspresi atau ekspresi simbolik dalam x\\
- fungsi atau fungsi simbolis dengan nama sebagai "f"\\
- fungsi simbolis hanya dengan nama f\\
\end{eulercomment}
\begin{eulerttcomment}
 
\end{eulerttcomment}
\begin{eulercomment}
Fungsi plot2d() juga menerima fungsi simbolis. Untuk fungsi simbolis,
hanya nama saja yang berfungsi.
\end{eulercomment}
\begin{eulerprompt}
>function f(x) &= diff(x^x,x)
\end{eulerprompt}
\begin{euleroutput}
  
                              x
                             x  (log(x) + 1)
  
\end{euleroutput}
\begin{eulerprompt}
>plot2d(f,0,2):
>$&expr = sin (x)*exp(-x)
>plot2d(expr,0,3pi):
>plot2d("cos(x)","sin(3*x)"):
\end{eulerprompt}
\eulerheading{Sub Bab 4 }
\begin{eulercomment}
Menggambar Fungsi Numerik 

Fungsi Numerik adalah sebuah fungsi dengan himpunan bilangan cacah
sebagai domain dan himpunan mendasar yang melibatkan hubungan
matematis antara bilangan yang menjadi domain dan bilangan sebagai
kodomain.
\end{eulercomment}
\begin{eulerprompt}
> 
\end{eulerprompt}
\begin{eulercomment}
Fungsi numerik  memiliki  1  atau  lebih  variabel  independen, yang
sering dilambangkan sebagai "X". Variabel X adalah nilai atau
parameter yang dapat berubah, dan fungsi numerik menggambarkan
bagaimana variabel ini memengaruhi variabel dependen. Variabel
dependen adalah hasil perhitungan atau keluaran dari fungsi numerik
yang bergantung pada nilai atau perubahan dalam variabel independen.

\end{eulercomment}
\begin{eulercomment}
Dalam EMT cara mendefinisikan fungsi menggunakan syntak function.
untuk mendefinisikan fungsi numerik menggunakan tanda ":="

Fungsi  numerik  menjelaskan bagaimana bilangan  dalam  domain
berhubungan dengan bilangan sebagai kodomain, biasanya diberikan dalam
bentuk rumus matematik(persamaan) atau aturan yang memetakan setiap
domain kedalam kodomain yang sesuai. contoh:

f(x)=2x+3
\end{eulercomment}
\begin{eulerprompt}
> 
\end{eulerprompt}
\begin{eulercomment}
(x)(variabel dependen) adalah fungsi yang memetakan setiap nilai
x(variabel independen)kedalam nilai 2x+3. Terdapat berbagai jenis
fungsi yang termasuk ke dalam fungsi numerik, diantaranya:

Fungsi linier dengan bentuk umum\\
f (x) = ax + b
\end{eulercomment}
\begin{eulercomment}
Fungsi kuadrat dengan bentuk umum

f (x) = ax2 + bx + c
\end{eulercomment}
\begin{eulercomment}
Fungsi eksponensial dengan bentuk umum

f (x) = ax
\end{eulercomment}
\begin{eulercomment}
Fungsi logaritma dengan bentuk umum

f (x) = log a(x)

\end{eulercomment}
\begin{eulercomment}
Fungsi trigonometri dengan bentuk umum

f (x) = sin(x), f (x) = cos(x)

\end{eulercomment}
\begin{eulercomment}
Salah satu  cara  yang  umum  digunakan  untuk  memvisualisasikan
fungsi numerik adalah dengan menggambar grafiknya. Grafik ini
menggambarkan bagaimana variabel dependen berubah seiring perubahan
variabel independen dan membantu dalam memahami sifat-sifat fungsi,
seperti titik ekstrim
\end{eulercomment}
\eulersubheading{Contoh soal}
\begin{eulerprompt}
>function r(x):= abs(x-10)
>function s(x):= abs(sin(x))
>r(-5)
\end{eulerprompt}
\begin{euleroutput}
  15
\end{euleroutput}
\begin{eulerprompt}
>function t(x):=log(x*(2+sin(x/1000)))
>function u(x):=integrate("(sin(x)*exp(-x^2)"0,x)
>function v(x):=logbase((x^2),2)
>plot2d("v"):
>plot2d("s"):
>plot2d("t",-2,2):
>function P(x):=x*cos(x)
>plot2d("P",-2*pi,2*pi):
\end{eulerprompt}
\begin{eulercomment}
Fungsi plot2d() adalah fungsi serbaguna untuk membuat grafik dalam
bidang (grafik 2D). Fungsi ini dapat digunakan untuk membuat grafik
fungsi-fungsi satu variabel, grafik data,  kurva-kurva  dalam  bidang,
grafik batang (bar plots), grid dari bilangan kompleks, dan grafik
implisit dari fungsi dua variabel.

Parameter\\
x,y : persamaan, fungsi, atau vektor data a,b,c,d : area plot (default
a=-2, b=2)\\
r  :  jika  r  diatur,  maka  a=cx-r,  b=cx+r,  c=cy-r,  d=cy+r r bisa
berupa vektor [rx,ry] atau vektor [rx1,rx2,ry1,ry2]. xmin,xmax :
rentang parameter untuk kurva\\
auto : tentukan rentang y secara otomatis (default)\\
square : jika benar, mencoba menjaga rentang x-y tetap persegi n :
jumlah interval (default adalah adaptif)\\
grid : 0 = tanpa grid dan label, 1 = hanya sumbu,\\
2 = grid normal (lihat di bawah untuk jumlah garis grid) 3 = di dalam
sumbu\\
4 = tanpa grid\\
5 = grid penuh termasuk margin 6 = tanda di pinggiran\\
7 = hanya sumbu\\
8 = hanya sumbu, sub-ticks frame : 0 = tanpa bingkai\\
framecolor: warna bingkai dan grid\\
margin : angka antara 0 dan 0,4 untuk margin di sekitar plot color :
Warna kurva. Jika ini adalah vektor warna,akan digunakan untuk setiap
baris matriks plot. Dalam  hal grafik titik, harus berupa vektor
kolom. Jika vektor baris atau matriks penuh warna digunakan untuk
grafik titik, akan digunakan untuk setiap titik data.\\
thickness : ketebalan garis untuk kurva

Nilai ini dapat lebih kecil dari 1 untuk garis yang sangat tipis. \\
style: Gaya plot untuk garis, penanda, dan isian.

Untuk titik gunakan\\
"[]", "\textless{}\textgreater{}", ".", "..", "...", "*", "+", " ", "-", "o"\\
"[]", "\textless{}\textgreater{}", "o" (bentuk terisi)\\
"[]w", "\textless{}\textgreater{}w", "ow" (tidak transparan)

Untuk garis gunakan\\
"-", "-", "-.", ".", ".-.", "-.-", "-\textgreater{}"

Untuk poligon terisi atau plot batang gunakan\\
"", "O", "O", "/", "", "/","+", " ", "-", "t"

points : plot titik tunggal sebagai gantinya garis segmen addpoints :
jika benar, plot segmen garis dan titik\\
add : tambahkan plot ke plot yang ada\\
user : aktifkan interaksi pengguna untuk fungsi delta : ukuran langkah
untuk interaksi pengguna\\
bar : plot batang (x adalah batas interval, y adalah nilai interval)
histogram : plot frekuensi x dalam n subinterval\\
distribusi=n : plot distribusi x dengan n subinterval even : gunakan
nilai antar untuk histogram otomatis. steps : plot fungsi sebagai
fungsi langkah (steps=1,2)\\
adaptive : gunakan plot adaptif (n adalah jumlah minimal langkah)
level : plot garis level dari fungsi implisit dua variabel\\
outline : menggambar batas rentang level.
\end{eulercomment}
\begin{eulerprompt}
>function s(x):=(x-10)
>function r(x):=abs(sin(x))
>s(-5)
\end{eulerprompt}
\begin{euleroutput}
  -15
\end{euleroutput}
\begin{eulerprompt}
>function t(x):=log(x*(2+sin(x/1000)))
>function u(x):=integrate("(sin(x)*exp(-x^2)"),0,x)
>function v(x):=logbase((x^2),2)
>plot2d("v"):
>plot2d("s"):
>function P(x):=x*cos(x)
>plot2d("P", -2*pi,2*pi):
\end{eulerprompt}
\eulerheading{Sub Bab 5 }
\begin{eulercomment}
Menggambar Beberapa Kurva Sekaligus 


Dalam subtopik ini, kita akan membahas mengenai cara menggambar
beberapa kurva sekaligus. Dalam hal ini kita dapat menggambar beberapa
kurva dalam jendela grafik yang berbeda secara bersama-sama. Untuk
membuat ini kita dapat menggunakan perintah figure(). Berikut contoh
dari menggambar beberapa kurva sekaligus

Menggambar plot fungsi\\
\end{eulercomment}
\begin{eulerformula}
\[
x^n, 1 \leq n \leq 4
\]
\end{eulerformula}
\begin{eulerprompt}
>reset;
>figure(2,2);...
>for n=1 to 4; figure(n); plot2d("x^"+n); end;...
>figure(0):
\end{eulerprompt}
\begin{eulercomment}
Penjelasan sintaks dari plot fungsi

\end{eulercomment}
\begin{eulerformula}
\[
x^n,  1 \leq n \leq 4
\]
\end{eulerformula}
\begin{eulercomment}
- reset;\\
Perintah ini berguna untuk menghapus grafik yang telah ada sebelumnya,
sehingga kita dapat memulai dari awal untuk menggambar grafik\\
- figure(2x2);\\
Perintah figure() digunakan untuk membuat jendela grafik dengan ukuran\\
axb. Dalam kasus ini perintah figure(2,2) memiliki makna bahwa jendela
grafik yang dibuat berukuran 2x2. Artinya, akan ada empat jendela
grafik yang akan ditampilkan dengan tata letak 2 baris dan 2 kolom.\\
- for n=1 to 4;\\
Perintah ini digunakan untuk melakukan pengulangan (looping) perintah
sebanyak empat kali, yaitu dari 1 hingga 4.\\
- figure(n);\\
Perintah ini digunakan untuk beralih dari jendela grafik satu ke
jendela grafik lainnya (jendela grafik ke-n).\\
- plot2d("x\textasciicircum{}"+n);\\
Perintah plot2d() digunakan untuk membuat plot fungsi matematika.\\
Dalam hal ini fungsi yang diplot adalah x\textasciicircum{}n, di mana n adalah nilai
dari variabel yang sedang diulang. Dengan kata lain, ini akan membuat\\
plot dari x\textasciicircum{}1, x\textasciicircum{}2, x\textasciicircum{}3, dan x\textasciicircum{}4 dalam jendela grafik yang sesuai\\
- end;\\
Perintah ini menandakan akhir dari looping.\\
- figure(0);\\
Perintah ini digunakan untuk beralih kembali ke jendela grafik utama.
\end{eulercomment}
\begin{eulercomment}
Dari sini dapat kita perhatikan untuk membuat kurva fungsi x\textasciicircum{}n (x
pangkat n) perintahnya tidak ditulis dengan (x\textasciicircum{}n) melainkan ditulis
dengan ("x\textasciicircum{}"+n). Tanda petik dua ("...") digunakan untuk
mengidentifikasi bahwa teks tersebut merupakan ekspresi matematika.\\
Sedangkan tanda (+) digunakan untuk menggabungkan string dengan nilai
yang berubah-ubah atau variabel.

Contoh lain:\\
Menggambar plot fungsi\\
\end{eulercomment}
\begin{eulerformula}
\[
f(x)=x^3-x, -2<x<2
\]
\end{eulerformula}
\begin{eulerprompt}
>reset;
>figure(3,3);...
>for k=1:9; figure(k); plot2d("x^3-x",-2,2,grid=k); end;...
>figure(0):
\end{eulerprompt}
\begin{eulerttcomment}
 Penjelasan sintaks dari plot fungsi
\end{eulerttcomment}
\begin{eulerformula}
\[
f(x)=x^3-x, -2<x<2
\]
\end{eulerformula}
\begin{eulercomment}
- reset;\\
Perintah ini berguna untuk menghapus grafik yang telah ada sebelumnya,
sehingga kita dapat memulai dari awal untuk menggambar grafik\\
- figure (3,3);\\
Perintah ini digunakan untuk membuat jendela grafik dengan ukuran 3x3.
Artinya, akan ada empat jendela grafik yang akan ditampilkan dengan
tata letak 3 baris dan 3 kolom.\\
- for k=1:9;\\
Perintah ini digunakan untuk melakukan pengulangan (looping) perintah
sebanyak sembilan kali.\\
- figure(n);\\
Perintah ini digunakan untuk beralih dari jendela grafik satu ke\\
\end{eulercomment}
\begin{eulerttcomment}
 jendela grafik lainnya (jendela grafik ke-n).
\end{eulerttcomment}
\begin{eulercomment}
- plot2d("x\textasciicircum{}3-x",-2,2,grid=k);\\
Perintah plot2d() digunakan untuk membuat plot fungsi matematika.\\
Dalam hal ini fungsi yang diplot adalah x\textasciicircum{}3-x, dengan batas sumbu x
dari -2 hingga 2. Argumen grid=k digunakan untuk mengaktifkan grid
pada jendela grafik ke-k.\\
- end;\\
Perintah ini menandakan akhir dari looping.\\
- figure(0);\\
Perintah ini digunakan untuk beralih kembali ke jendela grafik utama.

Dari contoh diatas dapat kita perhatikan bahwa tampilan plot dari yang
ke-1 hingga ke-9 memiliki tampilan yang berbeda-beda. Dalam EMT
memiliki berbagai gaya plot 2D yang dapat dijalankan menggunakan
perintah grid=n dimana n adalah jumlah langkah minimal. Setiap nilai n
memiliki tampilan plot adaptif yang berbeda dalam plot 2D, diantaranya
yaitu:\\
0 : tidak ada grid (kisi), frame, sumbu, dan label, hanya kurva saja\\
1 : dengan sumbu, label-label sumbu di luar frame jendela grafik\\
2 : tampilan default\\
3 : dengan grid pada sumbu x dan y, label-label sumbu berada di dalam
jendela grafik\\
4 : tidak ada grid (kisi), sumbu x dan y, dan label berada di luar
frame jendela grafik\\
5 : tampilan default tanpa margin di sekitar plot\\
6 : hanya dengan sumbu x y dan label, tanpa grid\\
7 : hanya dengan sumbu x y dan tanda-tanda pada sumbu.\\
8 : hanya dengan sumbu dan tanda-tanda pada sumbu, dengan tanda-tanda
yang lebih halus pada sumbu.\\
9 : tampilan default dengan tanda-tanda kecil di dalam jendela\\
10: hanya dengan sumbu-sumbu, tanpa tanda

Contoh lain:\\
Menggambar plot fungsi\\
\end{eulercomment}
\begin{eulerformula}
\[
g(x)=2x^3-x
\]
\end{eulerformula}
\begin{eulerprompt}
>reset;
>aspect(1.2);
>figure(3,4); ...
> figure(2); plot2d("2x^3-x",grid=1); ... // x-y-axis
> figure(3); plot2d("2x^3-x",grid=2); ... // default ticks
>figure(4); plot2d("2x^3-x",grid=3); ... // x-y- axis with labels inside
> figure(5); plot2d("2x^3-x",grid=4); ... // no ticks, only labels
>figure(6); plot2d("2x^3-x",grid=5); ... // default, but no margin
>figure(7); plot2d("2x^3-x",grid=6); ... // axes only
>figure(8); plot2d("2x^3-x",grid=7); ... // axes only, ticks at axis
>figure(9); plot2d("2x^3-x",grid=8); ... // axes only, finer ticks at axis
>figure(10); plot2d("2x^3-x",grid=9); ... // default, small ticks inside
>figure(11); plot2d("2x^3-x",grid=10); ...// no ticks, axes only
>figure(0):
\end{eulerprompt}
\begin{eulercomment}
Penjelasan sintaks dari plot fungsi\\
\end{eulercomment}
\begin{eulerformula}
\[
g(x)=2x^3-x
\]
\end{eulerformula}
\begin{eulercomment}
- aspect(1.2);\\
Perintah aspect() digunakan untuk mengatur rasio aspek dari jendela
grafik. Hal ini berarti perintah aspect(1.2); akan menghasilkan plot
dengan perbandingan rasio panjang dan lebar 2:1.\\
- figure(3,4);\\
Perintah ini digunakan untuk membuat jendela grafik dengan ukuran 3x4.\\
Jadi, akan ada total 12 jendela grafik yang akan ditampilkan dalam
tata letak 3 baris dan 4 kolom.\\
- figure(1); plot2d("x\textasciicircum{}3-x",grid=0); ...\\
Adalah perintah untuk beralih ke jendela grafik pertama dan menggambar
plot dari fungsi x\textasciicircum{}3 - x tanpa grid, frame, atau sumbu.\\
- figure(2); plot2d("x\textasciicircum{}3-x",grid=1); ...\\
Adalah perintah untuk beralih ke jendela grafik kedua dan menggambar
plot dari fungsi x\textasciicircum{}3 - x dengan grid hanya pada sumbu x dan y.\\
- figure(3); plot2d("x\textasciicircum{}3-x",grid=2); ...\\
Adalah perintah untuk beralih ke jendela grafik ketiga dan menggambar
plot dari fungsi x\textasciicircum{}3 - x dengan tampilan default, termasuk tanda-tanda
default pada sumbu.\\
- figure(4); plot2d("x\textasciicircum{}3-x",grid=3); ...\\
Adalah perintah untuk beralih ke jendela grafik keempat dan menggambar
plot dari fungsi x\textasciicircum{}3 - x dengan grid pada sumbu x dan y, serta
label-label sumbu yang ada di dalam jendela.\\
- figure(5); plot2d("x\textasciicircum{}3-x",grid=4); ...\\
Adalah perintah untuk beralih ke jendela grafik kelima dan menggambar
plot dari fungsi x\textasciicircum{}3 - x tanpa tanda-tanda sumbu, hanya label-label
yang ada.\\
- figure(6); plot2d("x\textasciicircum{}3-x",grid=5); ...\\
Adalah perintah untuk beralih ke jendela grafik keenam dan menggambar
plot dari fungsi x\textasciicircum{}3 - x dengan tampilan default, tetapi tanpa margin
di sekitar plot.\\
- figure(7); plot2d("x\textasciicircum{}3-x",grid=6); ...\\
Adalah perintah untuk beralih ke jendela grafik ketujuh dan menggambar
plot dari fungsi x\textasciicircum{}3 - x hanya dengan sumbu-sumbu (tanpa grid atau
label).\\
- figure(8); plot2d("x\textasciicircum{}3-x",grid=7); ...\\
Adalah perintah untuk beralih ke jendela grafik kedelapan dan
menggambar plot dari fungsi x\textasciicircum{}3 - x hanya dengan sumbu-sumbu dan
tanda-tanda pada sumbu.\\
- figure(9); plot2d("x\textasciicircum{}3-x",grid=8); ...\\
Adalah perintah untuk beralih ke jendela grafik kesembilan dan
menggambar plot dari fungsi x\textasciicircum{}3 - x hanya dengan sumbu-sumbu dan
tanda-tanda pada sumbu, dengan tanda-tanda yang lebih halus pada
sumbu.\\
- figure(10); plot2d("x\textasciicircum{}3-x",grid=9); ...\\
Adalah perintah untuk beralih ke jendela grafik kesepuluh dan
menggambar plot dari fungsi x\textasciicircum{}3 - x dengan tanda-tanda default kecil
di dalam jendela.\\
- figure(11); plot2d("x\textasciicircum{}3-x",grid=10); ...\\
Adalah perintah untuk beralih ke jendela grafik kesebelas dan
menggambar plot dari fungsi x\textasciicircum{}3 - x hanya dengan sumbu-sumbu, tanpa
tanda-tanda.\\
- figure(0);\\
Adalah perintah untuk beralih kembali ke jendela grafik utama atau
jendela grafik dengan nomor 0 setelah semua perintah dalam urutan
selesai dieksekusi.

Dari ketiga contoh di atas, dapat kita katakan bahwa untuk menggambar
beberapa kurva sekaligus itu dapat dilakukan dengan satu baris
perintah ataupun dengan cara mendefinisikannya 1 per 1.

Terlihat beberapa jenis grid memiliki tampilan yang mirip atau sama,
seperti 1 dan 2, 2 dan 5, 4 dan 9, 7 dan 8, untuk dapat membedakannya
secara lebih jelas, ubah grid dari contoh di bawah ini.
\end{eulercomment}
\begin{eulerprompt}
>reset;
>aspect(1.3);
>figure(1,3);...
>figure (1); plot2d("x^2*exp(-x)",0,10);...
>figure (2); plot2d("2*exp(x)",-5,5);...
>figure (3); plot2d("exp(x^2)",-2,2);...
>figure (0):
\end{eulerprompt}
\begin{eulercomment}
Contoh lain:
\end{eulercomment}
\begin{eulerprompt}
>reset;
>aspect(3/4);
>figure(2,1);...
>for a=1:2; figure(a); plot2d("2*x*log(x^2)",0,3,grid=a); end;...
>figure(0):
\end{eulerprompt}
\eulerheading{Sub Bab 6 }
\begin{eulercomment}
Menggambar Beberapa Kurva pada bidang koordinat yang sama 

Plot lebih dari satu fungsi (multiple function) ke dalam satu jendela
dapat dilakukan dengan berbagai cara. Salah satu caranya adalah
menggunakan \textgreater{}add untuk beberapa panggilan ke plot2d secara
keseluruhan, kecuali panggilan pertama.

Berikut contohnya:\\
menggambar kurva\\
\end{eulercomment}
\begin{eulerformula}
\[
 f(x)=cos(x)
\]
\end{eulerformula}
\begin{eulerformula}
\[
f(x)= x^2
\]
\end{eulerformula}
\begin{eulerprompt}
>aspect(); plot2d("cos(x)",r=3); plot2d("x^2",style=".",>add):
\end{eulerprompt}
\begin{eulerformula}
\[
f(x)=cos(x)-1
\]
\end{eulerformula}
\begin{eulerformula}
\[
f(x)= sin(x)-1
\]
\end{eulerformula}
\begin{eulerprompt}
>aspect(2); plot2d("cos(x)-1",-1,6); plot2d("sin(x)-1",style="--",>add):
\end{eulerprompt}
\begin{eulercomment}
Selain menggunakan \textgreater{}add kita juga bisa menambahkannya secara langsung

Berikut contohnya:\\
Menggambar kurva\\
\end{eulercomment}
\begin{eulerformula}
\[
f(x)= 2x+1
\]
\end{eulerformula}
\begin{eulerformula}
\[
f(x)= -2x+1
\]
\end{eulerformula}
\begin{eulerprompt}
>plot2d(["2x+1","x"],0,8):
\end{eulerprompt}
\begin{eulerformula}
\[
f(x)=sin(2x)
\]
\end{eulerformula}
\begin{eulerformula}
\[
f(x)=cos(3x)
\]
\end{eulerformula}
\begin{eulerprompt}
>aspect(1.5); plot2d(["sin(2x)","cos(3x)"],0,8):
\end{eulerprompt}
\begin{eulercomment}
Kegunaan \textgreater{}add yang lain juga bisa untuk menambahkan titik pada kurva.

Berikut contohnya:\\
Menambahkan sebuah titik di\\
\end{eulercomment}
\begin{eulerformula}
\[
f(x)= x+4
\]
\end{eulerformula}
\begin{eulerprompt}
>aspect(); plot2d("x+4",-2,5,); plot2d(2,6,>points,>add):
\end{eulerprompt}
\begin{eulercomment}
Kita juga bisa mencari titik perpotongan dengan cara berikut:

\end{eulercomment}
\begin{eulerformula}
\[
sin(x)=2x
\]
\end{eulerformula}
\begin{eulerprompt}
>plot2d(["sin(x)","2x"],r=2,cx=1,cy=1, ...
>  color=[black,blue],style=["-","."], ...
>  grid=1);
>x0=solve("sin(x)-2x",1);  ...
>  plot2d(x0,x0,>points,>add);  ...
>  label("sin(x) = 2x",x0,x0,pos="cl",offset=20):
>function f(x,a) := x^2+a*x-x/a; ...
>plot2d("f",-10,10;1,title="a=1"):
> plot2d(\{\{"f",1\}\},-10,10); ...
>for a=1:10; plot2d(\{\{"f",a\}\},>add); end:
>function f(x,a) := x^2*exp(-x^2/a); ...
>plot2d("f",-10,10;5,thickness=2,title="a=5"):
>plot2d(\{\{"f",1\}\},-8,8); ...
>for a=2:5; plot2d(\{\{"f",a\}\},>add,thickness=2); end:
>aspect(2.1); &plot2d(1/x,[x,-1,1]):
>x=linspace(-1,1,50);...
>plot2d("1/x"):
\end{eulerprompt}
\eulerheading{Sub Bab 7 }
\begin{eulercomment}
Menuliskan Label koordinat,label kurva, dan keterangan 

kurva(legend) Dalam EMT, untuk menambahkan judul dapat dilakukan
dengan title="..."\\
untuk menambahkan sumbu x dan sumbu y dapat dilakukan dengan x1="...",
y1="..."\\
sebagai contoh:
\end{eulercomment}
\begin{eulerprompt}
>plot2d("x^2-4*x"):
\end{eulerprompt}
\begin{eulercomment}
untuk menambahkan judul dapat dilakukan dengan title="..."\\
untuk menambahkan sumbu x dan sumbu y dapat dilakukan dengan x1="...",
y1="..."
\end{eulercomment}
\begin{eulerprompt}
>plot2d("x^2-4*x",title="FUNGSI y=x^2-4*x",yl="Sumbu y",xl="Sumbu x"):
\end{eulerprompt}
\begin{eulercomment}
Selain itu juga dapat dengan cara lain seperti contoh berikut:
\end{eulercomment}
\begin{eulerprompt}
>expr := "x^3-x"; ...
>  plot2d(expr,title="y="+expr,xl="Sumbu x",yl="Sumbu y"); ...
>  label("(1,0)",1,0);  label("Max",E,expr(E),pos="lc"): 
\end{eulerprompt}
\eulerheading{Sub Bab 8 }
\begin{eulercomment}
Mengatur ukuran gambar,format(style),dan warna kurva 


Untuk mengubah ukuran, dapat dilakukan dengan menggunakan
aspect="...", semakin besar nilai aspect, maka ukuran kurva akan
semakin kecil, begitupun sebaliknya

untuk mengganti style, dapat dipilih dengan berbagai pilihan\\
style="...", dapat dipilih dari, misal : "-","\_',"-.",".-.","-.-".

untuk warna dapat dipilih sebagai salah satu warna default\\
color="...", warna default= red,green,blue,yellow, dll

sebagai contoh:
\end{eulercomment}
\begin{eulerprompt}
>aspect(1); plot2d("exp(x^2-3)"):
\end{eulerprompt}
\begin{eulercomment}
ukuran kurva dapat diganti dengan mengganti nilai aspect="...",
semakin besar nilai aspect, maka ukuran kurva akan semakin kecil Untuk
mengganti warna dapat ditambahkan dengan color="...", sedangkan untuk
mengganti format(style) dapat dilakukan dengan menambahkan style="..."
\end{eulercomment}
\begin{eulerprompt}
>aspect(2); plot2d("exp(x^2-3)", color=red, style="--"):
\end{eulerprompt}
\begin{eulercomment}
Berikut adalah tampilan warna EMT yang telah ditentukan
\end{eulercomment}
\begin{eulerprompt}
>aspect (1) ; columnsplot (ones(1,16),lab=0:15,grid=0, color=0:15) :
\end{eulerprompt}
\begin{eulercomment}
selain menggunakan warna default, untuk mengubah warna dapat juga
dengan menggunakan kode warna di atas\\
sebagai contoh:
\end{eulercomment}
\begin{eulerprompt}
>aspect(1); plot2d("exp(x^3+2*x)",r=3, color=1, style="--"):
\end{eulerprompt}
\eulerheading{Sub Bab 9 }
\begin{eulercomment}
Menggambar Sekumpulan Kurva dalam satu perintah plot2d. 


Dalam pembahasan sub-bab 9 kali ini akan membahas mengenai bagaimana
menggambar sekumpulan kurva dalam satu perintah plot2d. Menggambar
sekumpulan kurva dalam satu perintah plot2d adalah teknik yang
digunakan untuk memvisualisasikan beberapa fungsi dalam satu grafik.
Ini memudahkan perbandingan antara beberapa kurva.\\
\end{eulercomment}
\eulersubheading{}
\eulersubheading{Contoh}
\begin{eulerprompt}
>plot2d(["x^2","2*x"],-3,3):
\end{eulerprompt}
\begin{eulerttcomment}
 - Dalam contoh ini, merupakan gambar dua kurva sekaligus, yaitu x^2
\end{eulerttcomment}
\begin{eulercomment}
dan 2x, pada rentang -3 hingga 3.

\end{eulercomment}
\begin{eulerttcomment}
 - Hasilnya akan menunjukkan grafik dari kedua fungsi tersebut, dan
\end{eulerttcomment}
\begin{eulercomment}
titik-titik potongan antara keduanya adalah solusi dari persamaan
kuadrat.
\end{eulercomment}
\begin{eulerprompt}
>plot2d(["sin(x)","cos(x)"],0,2pi):
\end{eulerprompt}
\begin{eulercomment}
- Pada contoh ini, merupakan gambar dua fungsi trigonometri, sin(x)
dan cos(x), pada rentang 0 hingga 2p.

- Ini akan menghasilkan dua grafik yang memperlihatkan hubungan antara
sin(x) dan cos(x) dalam rentang tersebut.
\end{eulercomment}
\begin{eulerprompt}
>plot2d(["sin(x)","cos(x)"],0,2pi,color=red:green):
\end{eulerprompt}
\begin{eulercomment}
Sama seperti contoh kedua, gambar sin(x) dan cos(x) pada rentang 0
hingga 2phi, tetapi Anda juga memberikan warna yang berbeda pada kedua
grafik (sin(x) berwarna merah dan cos(x) berwana hijau.
\end{eulercomment}
\begin{eulerprompt}
>plot2d(["sin(x)","cos(x)"],xmin=0,xmax=2pi):
\end{eulerprompt}
\begin{eulercomment}
Dalam contoh ini, menggunakan parameter `xmin` dan `xmax` untuk
mengatur rentang tampilan grafik pada 0 hingga 2p.

\end{eulercomment}
\begin{eulerprompt}
> 
>plot2d(["cos(x)","sin(3*x)"],xmin=0,xmax=2pi):
\end{eulerprompt}
\begin{eulercomment}
- ini Merupakan gambar dua fungsi, yaitu cos(x) dan sin(3*x), dalam
rentang 0 hingga 2p.\\
\end{eulercomment}
\begin{eulerttcomment}
  
\end{eulerttcomment}
\begin{eulercomment}
- Penjelasan mencakup konsep bahwa grafik berulang dalam rentang
tertentu karena fungsi-fungsi ini memiliki frekuensi, periode, dan
amplitudo yang berbeda.
\end{eulercomment}
\begin{eulerprompt}
>plot2d("cos(x)","sin(3*x)",xmin=0,xmax=2pi):
\end{eulerprompt}
\begin{eulercomment}
Sintaks diatas lebih menjelaskan bahaimana hubungan periodik grafik
fungsi dari 2 fungsi yaitu cos x dan sin 3x dari rentang khusus dimana
xmin dari 0 sampai 2pi, hal tersebut dapat terjadi karena fungsi cos x
dan fungsi sin 3x memiliki frekuensi, periode, dan amplitudo yang
berbeda. grafik akan berulang pada rentang tertentu dan menghasilkan
sebuah pola.
\end{eulercomment}
\begin{eulerprompt}
>x=linspace(0,2pi,1000); plot2d(sin(5x),cos(7x)):
\end{eulerprompt}
\begin{eulercomment}
sintaks linspace digunakan untuk menghasilkan vektor x dari rentang
yang telah ditentukan yaitu o sampai 2pi yang berisi 1000 nilai yang
teratur
\end{eulercomment}
\begin{eulerprompt}
>a:=5.6; f &= exp(-a*x^2)/a;
>plot2d(f,r=1,thickness=2):
\end{eulerprompt}
\begin{eulercomment}
- Fungsi f(x) yang merupakan hasil dari ekspresi exp(-a*x\textasciicircum{}2)/a. Fungsi
ini memiliki parameter a yang bergantung pada nilai yang diterapkan
sebelumnya.\\
- Menggunakan perintah plot2d untuk menggambar grafik dari fungsi
f(x).\\
Parameter r digunakan untuk mengatur rentang plot dan parameter
thickness digunakan untuk mengatur ketebalan garis grafik.
\end{eulercomment}
\begin{eulerprompt}
>plot2d(&diff(f,x),>add,style="--",color=red):
\end{eulerprompt}
\begin{eulercomment}
ini adalah grafik fungsi f dan grafik turunan pertama dari fungsi f.
sintaks r=1 digunakan untuk mengatur rentang yang akan ditampilkan
pada plot, r=1 berarti rentang dari -1 sampai 1.\\
sintaks \textgreater{}add digunakan untuk menambahkan grafik kedalam jendela grafik
yang sudah ada sebelumnya.
\end{eulercomment}
\begin{eulerprompt}
>plot2d("x^2",0,1,steps=1,color=red,n=10):
>plot2d("x^2",>add,steps=2,color=blue,n=10):
\end{eulerprompt}
\begin{eulercomment}
sintaks steps digunakan untuk mengatur jumlah langkah atau titik-titik
yang digunakan dalam plot.\\
dan sintaks n digunakan untuk mengatur jumlah step yang akan
digunakan. semakin bayak n, maka bentuk grafik akan semakin mendekati
aslinya.
\end{eulercomment}
\begin{eulerprompt}
>function f(x) &= x^x;
>plot2d(f,r=1,cx=1,cy=1,color=blue,thickness=2);
>plot2d(&diff(f(x),x),>add,color=red,style="-.-"):
\end{eulerprompt}
\begin{eulercomment}
sintaks cx=1, cy=1 digunakan untuk mengatur pusat tampilan grafik,
maka plot akan diatur dengan titik pusat (1,1).
\end{eulercomment}
\begin{eulerprompt}
>plot2d("(1-x)^10",0,1);
>for i=1 to 10; plot2d("bin(10,i)*x^i*(1-x)^(10-i)",>add); end;
>insimg;
\end{eulerprompt}
\begin{eulercomment}
dalam contoh kita menggambar serangkaian plot yang menggambarkan
distribusi binomial dengan berbagai nilai i dari 1 hingga 10. kali ini
kita menggunakan sintaks untuk melakukan looping pada fungsi yang
berasosiasi dengan koefisien binomial dengan kombinasi 10 item. ini
memungkinkan untuk memahami bagaimana distribusu binomial berubah
dengan berbagai parameter.
\end{eulercomment}
\begin{eulerprompt}
>x=linspace(0,1,500);
>n=10; k=(0:n)';
>y=bin(n,k)*x^k*(1-x)^(n-k);
>plot2d(x,y):
\end{eulerprompt}
\begin{eulercomment}
n adalah vektor baris\\
k adalah vektor kolom\\
y adalah matrik dari vektor baris dan vektor kolom tersebut dengan
menggunakanfungsi binomial.
\end{eulercomment}
\begin{eulerprompt}
>x=linspace(0,1,200); y=x^(1:10)'; plot2d(x,y,color=1:10):
>n=(1:10)'; plot2d("x^n",0,1,color=1:10):
>  
>function f(x,a) := 1/a*exp(-x^2/a); ...
>plot2d("f",-10,10;5,thickness=2,title="a=5"):
>plot2d(\{\{"f",1\}\},-10,10); ...
>for a=2:10; plot2d(\{\{"f",a\}\},>add); end:
\end{eulerprompt}
\eulerheading{Sub Bab 10 }
\begin{eulercomment}
Membuat Gambar Kurva yang Bersifat Interaktif 


Kode ini, menggunakan `plot2d` untuk membuat plot dari fungsi
matematika `2*x\textasciicircum{}3-a*x` dengan parameter `a`. Flag `\textgreater{}user` memungkinkan
interaksi pengguna. Setelah plot ditampilkan, pengguna dapat melakukan
beberapa tindakan interaktif.\\
Saat plot ditampilkan dengan flag `\textgreater{}user`, pengguna dapat melakukan
beberapa tindakan interaktif sebagai berikut:

- Perbesar dengan + atau -: Pengguna dapat memperbesar atau
memperkecil plot dengan menggunakan tombol + atau - pada keyboard.

- Pindahkan Plot dengan Tombol Kursor: Pengguna dapat menggeser plot
dengan menggunakan tombol kursor (panning).\\
- Pilih Jendela Plot dengan Mouse: Pengguna dapat memilih area
tertentu dalam plot dengan menggunakan mouse.

- Atur Ulang Tampilan dengan Spasi: Jika pengguna menekan tombol
spasi, maka tampilan plot akan diatur ulang ke jendela plot.\\
- Keluar dengan Kembali: Jika pengguna menekan tombol kembali, maka
pengguna dapat keluar dari interaksi plot.
\end{eulercomment}
\begin{eulerprompt}
>plot2d(\{\{"2*x^3-a*x",a=1\}\},>user,title="Press any key!"); ...
>insimg;  
> plot2d("exp(x)*sin(x)",user=true, ...
>  title="+/- or cursor keys (return to exit)"):
\end{eulerprompt}
\begin{eulercomment}
Berikut ini menunjukkan cara interaksi pengguna tingkat lanjut

Ini adalah pemanggilan fungsi plot2d yang digunakan untuk membuat plot
dari fungsi matematika exp(x)*sin(x). Parameter user=true menunjukkan
bahwa ini adalah plot yang interaktif, yang berarti pengguna dapat
berinteraksi dengan plot ini.

title="+/- or cursor keys (return to exit)": Ini adalah judul yang
akan ditampilkan di atas plot. Pesan ini memberi petunjuk kepada
pengguna tentang bagaimana mereka dapat berinteraksi dengan plot ini.
Mereka dapat menggunakan tombol + atau - atau tombol kursor untuk
berinteraksi dengan plot, dan tombol return (Enter) untuk keluar dari
interaksi.

Berikut ini menunjukkan cara interaksi pengguna tingkat lanjut:\\
- mousedrag(): Ini adalah fungsi bawaan yang digunakan untuk menunggu
event mouse atau keyboard. Fungsi ini dapat mendeteksi kejadian
seperti klik mouse, pergerakan mouse, atau penekanan tombol.\\
- dragpoints(): Fungsi ini memanfaatkan mousedrag() untuk memungkinkan
pengguna menyeret titik-titik pada plot. Ini berarti pengguna dapat
mengklik dan menarik titik-titik dalam plot sesuai dengan preferensi
mereka.

Kita membutuhkan fungsi plot terlebih dahulu. Sebagai contoh, kita
interpolasi dalam 5 titik dengan polinomial. Fungsi harus diplot ke
area plot tetap.
\end{eulercomment}
\begin{eulerprompt}
>function plotf(xp,yp,select) ...
\end{eulerprompt}
\begin{eulerudf}
    d=interp(xp,yp);
    plot2d("interpval(xp,d,x)";d,xp,r=2);
    plot2d(xp,yp,>points,>add);
    if select>0 then
      plot2d(xp[select],yp[select],color=red,>points,>add);
    endif;
    title("Drag one point, or press space or return!");
  endfunction
\end{eulerudf}
\begin{eulercomment}
Perhatikan parameter titik koma di plot2d (d dan xp), yang diteruskan
ke evaluasi fungsi interp(). Tanpa ini, kita harus menulis fungsi
plotinterp() terlebih dahulu, mengakses nilai secara global.

Sekarang kita menghasilkan beberapa nilai acak, dan membiarkan
pengguna menyeret poin.

kode berikut digunakan untuk menghasilkan beberapa nilai acak t dan
membiarkan pengguna menyeret titik-titik pada plot dengan menggunakan
fungsi dragpoints():
\end{eulercomment}
\begin{eulerprompt}
>t=-1:0.5:1; dragpoints("plotf",t,random(size(t))-0.5):
\end{eulerprompt}
\begin{eulercomment}
Ada juga fungsi, yang memplot fungsi lain tergantung pada vektor
parameter, dan memungkinkan pengguna menyesuaikan parameter ini.

Pertama kita membutuhkan fungsi plot.
\end{eulercomment}
\begin{eulerprompt}
>function plotf([a,b]) := plot2d("exp(a*x)*cos(2pi*b*x)",0,2pi;a,b);
\end{eulerprompt}
\begin{eulercomment}
Kemudian kita membutuhkan nama untuk parameter, nilai awal dan matriks
rentang nx2, opsional baris judul.\\
Ada slider interaktif, yang dapat mengatur nilai oleh pengguna. Fungsi
dragvalues() menyediakan ini.
\end{eulercomment}
\begin{eulerprompt}
>dragvalues("plotf",["a","b"],[-1,2],[[-2,2];[1,10]], ...
>  heading="Drag these values:",hcolor=black):
\end{eulerprompt}
\begin{eulercomment}
Dimungkinkan untuk membatasi nilai yang diseret ke bilangan bulat.
Sebagai contoh, kita menulis fungsi plot, yang memplot polinomial
Taylor derajat n ke fungsi kosinus.
\end{eulercomment}
\begin{eulerprompt}
>function plotf(n) ...
\end{eulerprompt}
\begin{eulerudf}
  plot2d("cos(x)",0,2pi,>square,grid=6);
  plot2d(&"taylor(cos(x),x,0,@n)",color=blue,>add);
  textbox("Taylor polynomial of degree "+n,0.1,0.02,style="t",>left);
  endfunction
\end{eulerudf}
\begin{eulercomment}
Sekarang kami mengizinkan derajat n bervariasi dari 0 hingga 20 dalam
20 pemberhentian. Hasil dragvalues() digunakan untuk memplot sketsa
dengan n ini, dan untuk memasukkan plot ke dalam buku catatan.
\end{eulercomment}
\begin{eulerprompt}
>nd=dragvalues("plotf","degree",3,[0,10],10,y=0.8, ...
>   heading="Drag the value:"); ...
>plotf(nd):
\end{eulerprompt}
\begin{eulercomment}
Berikut ini adalah demonstrasi sederhana dari fungsi tersebut.
Pengguna dapat menggambar di atas jendela plot, meninggalkan jejak
poin.
\end{eulercomment}
\begin{eulerprompt}
>function dragtest ...
\end{eulerprompt}
\begin{eulerudf}
    plot2d(none,r=1,title="Drag with the mouse, or press any key!");
    start=0;
    repeat
      \{flag,m,time\}=mousedrag();
      if flag==0 then return; endif;
      if flag==2 then
        hold on; mark(m[1],m[2]); hold off;
      endif;
    end
  endfunction
\end{eulerudf}
\eulerheading{Sub Bab 11 }
\begin{eulercomment}
Menggambar Kurva Fungsi Parametrik 


Kita telah terbiasa dengan kurva yang didefinisikan oleh sebuah
persamaan yang menghubungkan koordinat x dan y Contohnya\\
\end{eulercomment}
\begin{eulerformula}
\[
y=x^2
\]
\end{eulerformula}
\begin{eulercomment}
Atau\\
\end{eulercomment}
\begin{eulerformula}
\[
x^2+y^2=13
\]
\end{eulerformula}
\begin{eulercomment}
dimana persamaan-persamaan ini tidak dikaitkan dengan panjang kurva s
, waktu t, dan besaran lainnya. Besaran besaran ini disebut parameter\\
persamaan parametrik adalah persamaan yang menyatakan hubungan
variabel x, y dituliskan dengan\\
\end{eulercomment}
\begin{eulerformula}
\[
x=f(t)
\]
\end{eulerformula}
\begin{eulerformula}
\[
y=g(t)
\]
\end{eulerformula}
\begin{eulercomment}
dengan a\textless{}=t\textless{}=b tiap nilai t menentukan titik(x,y) pada kurva. Jadi ,
dengan berubahnya nilai t. titik\\
\end{eulercomment}
\begin{eulerformula}
\[
(x,y) = (f(t),g(t))
\]
\end{eulerformula}
\begin{eulercomment}
bergerak sepanjang kurva yang disebut kurva parametrik


Dalam contoh berikut, kita memplot spiral

\end{eulercomment}
\begin{eulerformula}
\[
\gamma(t) = t \cdot (\cos(2\pi t),\sin(2\pi t))
\]
\end{eulerformula}
\begin{eulercomment}
Kita perlu menggunakan banyak titik untuk tampilan yang halus
\end{eulercomment}
\begin{eulerprompt}
>t=linspace(0,1,1000); ...
>plot2d(t*cos(2*pi*t),t*sin(2*pi*t),r=1):
\end{eulerprompt}
\begin{eulerttcomment}
 r digunakan untuk mengatur radius marker titik-titik yang akan
\end{eulerttcomment}
\begin{eulercomment}
digunakan dalam plot.



Sebagai alternatif, dimungkinkan untuk menggunakan dua ekspresi untuk
kurva. Berikut ini plot kurva yang sama seperti di atas.
\end{eulercomment}
\begin{eulerprompt}
>plot2d("x*cos(2*pi*x)","x*sin(2*pi*x)",xmin=0,xmax=1,r=1):
\end{eulerprompt}
\begin{eulercomment}
Perintah linspace digunakan untuk membuat array nilai yang
terdistribusi secara merata antara dua angka tertentu. Fungsi ini
sangat berguna untuk menentukan rentang nilai yang ingin digunakan
pada sumbu x atau y ketika membuat plot.

\end{eulercomment}
\begin{eulerttcomment}
    0  : Nilai awal dari rentang.
    1  : Nilai akhir dari rentang.
 
\end{eulerttcomment}
\begin{eulercomment}

Perintah linspace akan menghasilkan array dengan n elemen yang
terdistribusi merata antara start dan stop.
\end{eulercomment}
\begin{eulerprompt}
>t=linspace(0,1,1000); r=exp(-t); x=r*cos(2pi*t); y=r*sin(2pi*t);
>plot2d(x,y,r=1):
\end{eulerprompt}
\begin{eulercomment}
exp(-t) menghasilkan nilai yang semakin mendekati nol seiring dengan
pertambahan nilai t, karena eksponensial dari nilai negatif semakin
mendekati nol saat nilai t semakin besar.\\
Jadi, r = exp(-t) memberikan suatu fungsi yang menurun dengan nilai t.
Dalam konteks program ini, r digunakan untuk mengontrol jari-jari dari
kurva spiral dalam plot 2D. Jari-jari ini semakin kecil seiring dengan
pertambahan nilai t, menciptakan efek spiral yang semakin rapat ke
pusat pada bagian ujung kurva.





Pada contoh berikutnya, kita memplot kurvanya

\end{eulercomment}
\begin{eulerformula}
\[
\gamma(t) = (r(t) \cos(t), r(t) \sin(t))
\]
\end{eulerformula}
\begin{eulercomment}
dengan

\end{eulercomment}
\begin{eulerformula}
\[
r(t) = 1 + \dfrac{\sin(3t)}{2}.
\]
\end{eulerformula}
\begin{eulerprompt}
>t=linspace(0,2pi,1000); r=1+sin(3*t)/2; x=r*cos(t); y=r*sin(t); ...
>plot2d(x,y,>filled,fillcolor=red,style="/",r=1.5):
\end{eulerprompt}
\eulersubheading{Contoh lain}
\begin{eulerprompt}
>t=linspace(-3,3,1000); x=2*t+1; y=t^2-1;
>plot2d(x,y,r=8):
>t=linspace(0,2pi,1000); r=3; x=r*cos(t); y=r*sin(t);...
>plot2d(x,y,>filled,fillcolor=green,style="/",r=5):
>t=linspace(-1,1,1000); x=t^2; y=2*t;...
>plot2d(x,y):
>t=linspace(0,2pi,1000); x=3*cos(t); y=2*sin(t);...
>plot2d(x,y,>filled,fillcolor=green,style="/",r=3):
\end{eulerprompt}
\eulerheading{Sub Bab 12 }
\begin{eulercomment}
Menggambar Kurva Fungsi Implisit 


Fungsi implisit adalah fungsi yang memuat lebih dari satu variabel,
berjenis variabel bebas dan variabel terikat yang berada dalam satu
ruas sehingga tidak bisa dipisahkan pada ruas yang berbeda.

Untuk fungsi implisit, harus berupa fungsi atau ekspresi dari
parameter x dan y.

\end{eulercomment}
\begin{eulerformula}
\[
f(x,y)=c
\]
\end{eulerformula}
\begin{eulercomment}
Untuk menggambar himpunan f(x,y)=c untuk satu atau lebih konstanta c,
dapat menggunakan\\
plot2d().

Fungsi implisit juga dapat diisi dengan persamaan tingkat

\end{eulercomment}
\begin{eulerformula}
\[
a<=f(x,y)<=b
\]
\end{eulerformula}
\begin{eulercomment}
Untuk fungsi ini harus berupa matriks 2xn dimana baris pertama berisi
awal dan baris kedua adalah akhir dari setiap interval.

Plot implisit menunjukkan garis level yang menyelesaikan f(x,y)=level,
di mana "level" dapat berupa nilai tunggal atau vektor nilai. Jika
level="auto", akan ada garis level nc, yang akan menyebar antara
fungsi minimum dan maksimum secara merata. Warna yang lebih gelap atau
lebih terang dapat ditambahkan dengan \textgreater{}hue untuk menunjukkan nilai
fungsi. Untuk fungsi implisit, xv harus berupa fungsi atau ekspresi
dari parameter x dan y, atau, sebagai alternatif, xv dapat berupa
matriks nilai.

Euler dapat menandai garis level

\end{eulercomment}
\begin{eulerformula}
\[
f(x,y) = c
\]
\end{eulerformula}
\begin{eulercomment}
dari fungsi apapun.

Untuk menggambar himpunan f(x,y)=c untuk satu atau lebih konstanta c,
Anda dapat menggunakan plot2d() dengan plot implisitnya di dalam
bidang. Parameter untuk c adalah level=c, di mana c dapat berupa
vektor garis level. Selain itu, skema warna dapat digambar di latar
belakang untuk menunjukkan nilai fungsi untuk setiap titik dalam plot.
Parameter "n" menentukan kehalusan plot.

\end{eulercomment}
\eulersubheading{Contoh Soal}
\begin{eulercomment}
\end{eulercomment}
\begin{eulerformula}
\[
x^2+y^2-xy-x = 0
\]
\end{eulerformula}
\begin{eulerprompt}
>aspect(2)
>plot2d("x^2+y^2-x*y-x",r=1.5,level=0,contourcolor=green):
\end{eulerprompt}
\begin{eulerformula}
\[
2x^2+xy+3y^4+y = 0
\]
\end{eulerformula}
\begin{eulerprompt}
>expr := "2*x^2+x*y+3*y^4+y"; // define an expression f(x,y)
>plot2d(expr,level=0,contourcolor=green): // Solutions of f(x,y)=0
>plot2d(expr,level=0:0.5:20,>hue,contourcolor=white,n=200): // nice
\end{eulerprompt}
\begin{eulercomment}
Parameter \textgreater{}hue digunakan untuk memberikan warna pada kontur sesuai
dengan levelnya. Kontur dengan level yang lebih tinggi akan memiliki
warna yang berbeda.
\end{eulercomment}
\begin{eulerprompt}
>plot2d(expr,level=0:0.5:20,>hue,>spectral,n=200,grid=4): // nicer
\end{eulerprompt}
\begin{eulercomment}
\textgreater{}spectral digunakan untuk mengatur palet warna yang akan digunakan
pada kontur. Dalam hal ini, digunakan palet warna "spectral".
\end{eulercomment}
\begin{eulerprompt}
>x=-2:0.05:1; y=x'; z=expr(x,y);
>plot2d(z,level=0,a=-1,b=2,c=-2,d=1,>hue):
>plot2d("x^3-y^2",>contour,>hue,>spectral):
\end{eulerprompt}
\begin{eulercomment}
Perintah \textgreater{}contour adalah cara untuk menghasilkan plot kontur, yaitu
plot yang menunjukkan garis-garis kontur yang mewakili tingkat-tingkat
dari suatu fungsi. Jumlah dan posisi garis kontur akan secara otomatis
diatur oleh Euler Math Toolbox berdasarkan distribusi nilai-nilai
fungsi.

Penggunaan level memungkinkan Anda secara eksplisit menentukan tingkat
kontur yang ingin kita tampilkan pada plot. kita dapat mengatur level
kontur sesuai dengan preferensi kita, dan plot akan menampilkan garis
kontur pada tingkat-tingkat yang kita tentukan.
\end{eulercomment}
\begin{eulerprompt}
>plot2d("x^3-y^2",level=0,contourwidth=3,>add,contourcolor=red):
>z=z+normal(size(z))*0.2;
>plot2d(z,level=0.5,a=-1,b=2,c=-2,d=1):
\end{eulerprompt}
\begin{eulercomment}
normal(size(z)) menghasilkan matriks dengan ukuran yang sama dengan
matriks z, dan setiap elemennya diambil dari distribusi normal standar
(mean 0, deviasi standar 1). Kemudian, matriks z diubah dengan
menambahkan nilai-nilai acak ini, yang telah dikalikan dengan 0.2. Ini
menciptakan variasi acak dalam matriks z.

\end{eulercomment}
\begin{eulerprompt}
>plot2d(expr,level=[0:0.2:5;0.05:0.2:5.05],color=lightgray):
>plot2d("x^2+y^3+x*y",level=1,r=4,n=100):
>plot2d("x^2+2*y^2-x*y",level=0:0.1:10,n=100,contourcolor=white,>hue):
>plot2d(expr,level=[0;1],style="-",color=blue): // 0 <= f(x,y) <= 1
>plot2d("x^4+y^4",r=1.5,level=[0;1],color=blue,style="/"):
>plot2d("x^2+y^3+x*y",level=[0,2,4;1,3,5],style="/",r=2,n=100):
>plot2d("x^2+y^3+x*y",level=-10:20,r=2,style="-",dl=0.1,n=100):
\end{eulerprompt}
\begin{eulercomment}
dl Parameter ini mengatur tingkat penghalusan pada plot. Semakin kecil
nilai ini, semakin halus plotnya.
\end{eulercomment}
\begin{eulerprompt}
>plot2d("sin(x)*cos(y)",r=pi,>hue,>levels,n=100):
>plot2d("(x^2+y^2-1)^3-x^2*y^3",r=1.3, ...
>style="/",color=red,<outline, ...
>level=[-2;0],n=100):
\end{eulerprompt}
\begin{eulercomment}
\textless{}outline: Parameter ini mengatur plot agar hanya memiliki kontur saja
tanpa diisi.



Misal plot solusi dari persamaan

\end{eulercomment}
\begin{eulerformula}
\[
x^3-xy+x^2y^2=6
\]
\end{eulerformula}
\begin{eulerprompt}
>plot2d("x^3-x*y+x^2*y^2",r=6,level=6,n=100):
>plot2d("x^2+y^2-1",level=0):
>plot2d("x^2+y^2-1",r=3,level=0:1:10,n=200):
>plot2d("x^2+y^2-1",r=3,level=0:1:10,>hue,contourcolor=white):
>plot2d("x^2+y^2-1",r=3,level=0:1:20,>hue,>spectral,n=200,grid=4):
>plot2d("x^2+y^2-1",level=0,a=-2,b=2,c=-2,d=2,>hue):
>plot2d("x^2+y^2-1",>contour,>hue,>spectral):
>plot2d("x^2+y^2-1",level=[0:0.2:5;0.05:0.2:5.05],color=lightgray):
>plot2d("x^2+y^2-1",r=2,level=[0;1],style="-",color=blue): // 0 <= f(x,y) <= 1
>plot2d("x^2+y^2",r=1.5,level=[0;1],color=blue,style="/"):
>plot2d("x^2+y^2-1",level=[0,2,4;1,3,5],style="/",r=2,n=100):
>plot2d("x^2+y^2-1",level=-10:20,r=3,style="-",dl=0.1,n=100):
\end{eulerprompt}
\eulerheading{Sub bab 13 }
\begin{eulercomment}
Menggambar Grafik Bilangan Kompleks 


Bilangan kompleks secara visual dapat direpresentasikan sebagai
sepasang angka (a, b) membentuk vektor pada diagram yang disebut
diagram Argand, mewakili yang bidang kompleks. Sumbu-x adalah sumbu
nyata dan sumbu-y adalah sumbu imajiner.

Menggambar kurva fungsi kompleks sendiri adalah proses visualisasi
grafis dari fungsi matematika kompleks (yaitu fungsi yang melibatkan
bilangan kompleks, yaitu bilangan dengan bagian real dan imajiner)
berperilaku dalam koordinas kompleks. Hal tersebut memungkinkan untuk
melihat bagaimana pola, bentuk, dan sifat dari fungsi kompleks
tersebut.

Array bilangan kompleks juga dapat diplot. Kemudian titik-titik grid
akan terhubung. Jika sejumlah garis kisi ditentukan (atau vektor garis
kisi 1x2) dalam argumen cgrid, hanya garis kisi tersebut yang
terlihat.

Matriks bilangan kompleks akan secara otomatis diplot sebagai kisi di
bidang kompleks.

\textgreater{} Definisi fungsi kompleks, mendefinisikan fungsi kompleks yang
dianalisis atau digambarkan. Fungsi ini memiliki variabel kompleks z,
yang melibatkan bagian real dan imajiner.\\
\textgreater{} Selanjutnya kita dapat menggunakan fungsi linspace. Fungsi linspace
sendiri adalah salah satu fungsi yang umum digunakan dalam
pemrograman, terutama dalam konteks pemrograman numerik dan ilmu data.
Ini sering digunakan untuk menghasilkan urutan nilai dalam rentang
tertentu dengan jumlah titik yang sama di antara dua titik ujungnya.
Penggunaannya tidak terbatas pada pemrosesan sinyal atau
elektromagnetik, tetapi bisa digunakan dalam berbagai konteks di mana
Anda perlu membuat urutan nilai.\\
\textgreater{} Penentuan rentang, memilih rentang nilai z yang ingin ditampilkan di
dalam plot. Rentang ini mencakup wilayah kompleks tertentu yang ingin
diamati.\\
\textgreater{} Menggunakan sintaks plot2d.\\
\textgreater{} Penyesuaian plot, mengubah plot sesuai yang diinginkan (mengubah
warna, format (style), dan sebagainya).

Dalam contoh berikut, kami memplot gambar lingkaran satuan di bawah
fungsi eksponensial. Parameter cgrid menyembunyikan beberapa kurva
grid.

\begin{eulercomment}
\eulerheading{Contoh}
\begin{eulerprompt}
>aspect(); r=linspace(0,1,50); a=linspace(0,2pi,80)'; z=r*exp(I*a);...
>plot2d(z,a=-1.25,b=1.25,c=-1.25,d=1.25,cgrid=10):
\end{eulerprompt}
\begin{eulercomment}
Penjelasan sintaks

z       : sebuah ekspresi atau fungsi yang akan digambar dalam
koordinat kompleks.\\
a,b,c,d : parameter-parameter yang digunakan untuk mengatur jendela
tampilan (viewport) dalam koordinat kompleks. Parameter-parameter ini
akan menentukan rentang sumbu x dan sumbu y yang akan ditampilkan di
dalam plot.\\
cgrid   : parameter ini mengontrol tampilan grid pada plot. Jika
cgrid=n, maka grid akan ditampilkan, jika cgrid=0, maka grid akan
disembunyikan.

\begin{eulercomment}
\eulerheading{Bentuk lain}
\begin{eulerprompt}
>aspect(1.25); r=linspace(0,1,50); a=linspace(0,2pi,200)'; z=r*exp(I*a);
>plot2d(exp(z),cgrid=[40,10]):
\end{eulerprompt}
\begin{eulercomment}
Penjelasan :\\
Perintah tersebut merupakan perintah untuk menggambar kurva dari
fungsi kompleks eksponensial "exp(z)" dalam koordinat kompleks. Dalam
perintah tersebut juga menggunakan parameter cgrid dengan nilai
[40,10] untuk mengatur grid pada plot.\\
Dalam sintaks ini,\\
exp(z) : fungsi eksponensial kompleks yang akan digambar\\
cgrid=[40,10] : mengatur grid pada plot. cgrid tersebut adalah jumlah
garis grid yang akan digunakan pada sumbu x dan sumbu y. Nah di dalam
plot ini, akan ada 40 garis grid pada sumbu x dan 10 grid pada sumbu
y.

\begin{eulercomment}
\eulerheading{Bentuk lain}
\begin{eulerprompt}
>r=linspace(0,1,10); a=linspace(0,2pi,40)'; z=r*exp(I*a);
>plot2d(exp(z),>points,>add):
\end{eulerprompt}
\begin{eulercomment}
Sebuah vektor bilangan kompleks secara otomatis diplot sebagai kurva
pada bidang kompleks dengan bagian real dan bagian imajiner.

Penjelasan :\\
Perintah plot2d di atas adalah perintah untuk menggambar kurva fungsi
kompleks dalam koordinat kompleks, namun dengan opsi yang berbeda,\\
exp(z): fungsi kompleks yang akan digambar\\
\textgreater{}points : opsi ini mengubah cara plot untuk dilakukan. Dengan
menggunakan \textgreater{}points, plot ini akan menggunakan titik-titik diskrit
untuk merepresentasikan fungsi ke dalam bentuk titik,titik\\
\textgreater{}add    : sintaks ini menginstrusikan perintah untuk menambahkan plot
ini ke plot sebelumnya jika ada.

\begin{eulercomment}
\eulerheading{Contoh}
\begin{eulerprompt}
>t=linspace(0,2pi,1000); ...
>plot2d(exp(I*t)+exp(10*I*t),r=3):
\end{eulerprompt}
\begin{eulercomment}
Penjelasan :\\
Perintah plot2d di atas menggambarkan kurva dari fungsi kompleks yang
diberikan dalam koordinat kompleks dengan parameter-parameter
tertentu.\\
Sintaks yang digunakan yaitu,\\
exp(I*t)+exp(10*I*t) : fungsi kompleks yang akan digambar. Fungsi ini
terdiri dari dua bagian yang masing-masing merupakan fungsi kompleks
eksponensial. Dengan 10 adalah berapa kali putaran dalam gambar
tersebut.\\
r : parameter r digunakan untuk menentukan rentang nilai dari variabel
t. Dalam contoh ini, r=3, yaitu mengatur rentang nilai t dari 3 hingga
3.

\begin{eulercomment}
\eulerheading{Sub Bab 14 }
\begin{eulercomment}
Menggambar Daerah Yang Dibatasi Kurva 


Plot data benar-benar poligon. Kita juga dapat memplot kurva atau
kurva terisi.

Pada subtopik sebelumnya telah kita ketahui dan pelajari bersama bahwa
EMT dapat melakukan visualisasi plot mulai dari bentuk ekspresi
langsung hingga plot dari fungsi-fungsi. Pada subtopik ini merupakan
kelanjutan dari subtopik sebelumnya, bahwa kita dapat
membentuk/menggambar daerah dari perpotongan beberapa kurva yang telah
didefinisikan. Hal ini dapat bermanfaat untuk membantu dalam
menyelesaikan permasalahan dalam matematika, salah satu contohnya
seperti optimasi program linear, dimana disajikan beberapa
fungsi-fungsi kendala beserta dengan fungsi tujuannya dan perlu
divisualisasikan dalam bentuk grafik untuk melihat dimana letak daerah
layaknya untuk menentukan nilai optimum.

Dalam EMT ada beberapa perintah yang digunakan untuk menggambar daerah
yang dibatasi oleh beberapa kuva, diantaranya yaitu:

- plot2d\\
\end{eulercomment}
\begin{eulerttcomment}
  Digunakan untuk melakukan plotting.
\end{eulerttcomment}
\begin{eulercomment}

- filled=true\\
\end{eulercomment}
\begin{eulerttcomment}
  Digunakan untuk memberikan isian/arsiran pada daerah/area di bawah
\end{eulerttcomment}
\begin{eulercomment}
kurva saat plotting.

- style="..."\\
\end{eulercomment}
\begin{eulerttcomment}
  Digunakan untuk memilih gaya kurva yang akan digunakan saat
\end{eulerttcomment}
\begin{eulercomment}
plotting. Anda dapat memilih dari beberapa gaya, seperti "#", "/",
"\textbackslash{}", atau "-". Dan hal ini mempengaruhi tampilan daerah kurva yang
terbentuk.

- fillcolor\\
\end{eulercomment}
\begin{eulerttcomment}
  Digunakan untuk menentukan warna isian yang akan digunakan untuk
\end{eulerttcomment}
\begin{eulercomment}
mengiri area di bawah kurva.

\end{eulercomment}
\eulersubheading{Contoh}
\begin{eulerprompt}
>t=linspace(0,2pi,1000); // parameter for curve
>x=sin(t)*exp(t/pi); y=cos(t)*exp(t/pi); // x(t) and y(t)
>figure(1,2); aspect(16/9)
>figure(1); plot2d(x,y,r=10); // plot curve
>figure(2); plot2d(x,y,r=10,>filled,style="/",fillcolor=red); // fill curve
>figure(0):
\end{eulerprompt}
\begin{eulercomment}
Penjelasan:

- t=linspace(0,2pi,1000);\\
Pada langkah pertama yaitu mendefinisikan parameter t sebagai
serangkaian 1000 titik antara 0 dan 2pi. Parameter t ini akan
digunakan sebagai parameter untuk menggambar kurva.

- x=sin(t)*exp(t/pi); y=cos(t)*exp(t/pi); // x(t) and y(t)\\
Kemudian kita definisika dua vektor x dan y yang merupakan koordinat x
dan y dari kurva yang akan digambar. Fungsi\\
\end{eulercomment}
\begin{eulerformula}
\[
sin(t)*exp(t/pi)
\]
\end{eulerformula}
\begin{eulercomment}
digunakan untuk menghitung komponen x (x(t)), dan\\
\end{eulercomment}
\begin{eulerformula}
\[
cos(t)*exp(t/pi)
\]
\end{eulerformula}
\begin{eulercomment}
digunakan untuk menghitung komponen y (y(t)) dari kurva.

- figure(1,2); aspect(16/9)\\
Perintah ini digunakan untuk mengatur tampilan gambar. Perintah
figure(1,2) digunakan membuat dua gambar (1 dan 2) dalam satu jendela
gambar. Dan perintah aspect(16/9) mengatur rasio aspek gambar menjadi
16:9, yang mempengaruhi bentuk dan ukuran gambar yang akan digambar.

- figure(1); plot2d(x,y,r=10); // membuat plot kurva\\
Perintah ini memilih gambar pertama (1) dan menggunakan perintah
plot2d untuk menggambar kurva yang dihitung sebelumnya. Parameter r=10
mengatur lebar garis plot. Ini menghasilkan kurva tanpa adanya isi
atau arsiran di dalamnya.

- figure(2); plot2d(x,y,r=10,\textgreater{}filled,style="/",fillcolor=red); // fill
curve\\
Selanjutnya pada perintah ini beralih ke gambar kedua (2) dan
menggunakan perintah plot2d lagi untuk menggambar kurva yang sama
dengan pengisian area di bawahnya. Perintah \textgreater{}filled digunakan untuk
mengisi area di bawah kurva, style="/" digunakan untuk mengatur gaya
garis menjadi garis miring, dan fillcolor=red digunakan untuk mengatur
warna isian menjadi merah.

-figure(0):\\
Baris perintah ini digunakan untuk mengakhiri gambar dan kembali ke
tampilan biasa tanpa gambar. Ini berfungsi untuk menyelesaikan proses
penggambaran.

\end{eulercomment}
\eulersubheading{Contoh}
\begin{eulerprompt}
>x=linspace(0,2pi,1000); plot2d(cos(x),sin(x)*0.5,r=1,>filled,style="\(\backslash\)"):
\end{eulerprompt}
\begin{eulercomment}
Penjelasan:

- x=linspace(0,2pi,100);\\
Mendefinisikan vektor x dengan menggunakan perintah linspace. linspace
digunakan untuk membuat vektor dengan 100 titik yang secara merata
tersebar antara 0 dan 2phi. Dalam konteks ini, vektor x akan digunakan
sebagai parameter saat menggambar kurva.

- plot2d(cos(x),sin(x)*0.5,r=1,\textgreater{}filled,style="\textbackslash{}"):\\
Ini merupakan perintah utama yang digunakan untuk menggambar plot.
Perintah ini memiliki beberapa parameter sebagai berikut:\\
\textgreater{} cos(x) adalah komponen x dari kurva. Ini adalah hasil dari fungsi
kosinus yang dihitung pada vektor x.\\
\textgreater{} sin(x)*0.5 adalah komponen y dari kurva. Ini adalah hasil dari
fungsi sinus yang dihitung pada vektor x dan kemudian dikalikan dengan
0,5, yang mengubah amplitudonya.\\
\textgreater{} r=1 mengatur lebar garis plot menjadi 1.\\
\textgreater{} filled digunakan untuk mengisi area di bawah kurva, sehingga
menciptakan daerah yang terisi.\\
\textgreater{} style="\textbackslash{}" mengatur gaya garis kurva untuk membentuk garis miring
yang gunanya menutupi semua bagian kurva dengan garis miring.

\end{eulercomment}
\eulersubheading{Contoh}
\begin{eulerprompt}
>t=linspace(0,2pi,6); ...
>plot2d(cos(t),sin(t),>filled,style="/",fillcolor=red,r=1.5):
\end{eulerprompt}
\begin{eulercomment}
Penjelasan:

- t=linspace(0,2pi,6); ...\\
\end{eulercomment}
\begin{eulerttcomment}
  Pada perintah ini, kita definisikan vektor t dengan menggunakan
\end{eulerttcomment}
\begin{eulercomment}
perintah linspace. linspace digunakan untuk membuat vektor dengan 6
titik yang terletak secara merata antara 0 dan 2pi. Dalam konteks ini,
vektor t akan digunakan sebagai parameter saat menggambar kurva.

- plot2d(cos(t),sin(t),\textgreater{}filled,style="/",fillcolor=red,r=1.5):\\
\end{eulercomment}
\begin{eulerttcomment}
  Ini adalah perintah utama yang digunakan untuk menggambar plot.
\end{eulerttcomment}
\begin{eulercomment}
Perintah ini memiliki beberapa parameter sebagai berikut:\\
\textgreater{} cos(t) adalah komponen x dari kurva.\\
\textgreater{} sin(t) adalah komponen y dari kurva.\\
\textgreater{} filled digunakan untuk mengisi area di bawah kurva, sehingga
menciptakan bentuk yang terisi. Ini berarti daerah di bawah kurva akan
diwarnai.\\
\textgreater{} style="/" mengatur gaya garis kurva menjadi garis miring ("/").\\
\textgreater{} fillcolor=orange mengatur warna isian daerah di bawah kurva menjadi
oranye.\\
\textgreater{} r=1.5 mengatur lebar garis plot menjadi 1.5.

\end{eulercomment}
\eulersubheading{Contoh}
\begin{eulerprompt}
>t=linspace(0,2pi,6); plot2d(cos(t),sin(t),>filled,style="#"):
\end{eulerprompt}
\begin{eulercomment}
Penjelasan:\\
- t=linspace(0,2pi,6);\\
Pada perintah ini, kita definisikan vektor t dengan menggunakan
perintah linspace. linspace digunakan untuk membuat vektor dengan 6
titik yang terletak secara merata antara 0 dan 2phi. Dalam konteks
ini, vektor t akan digunakan sebagai parameter saat menggambar kurva.

- plot2d(cos(t),sin(t),\textgreater{}filled,style="#"):\\
Ini adalah perintah utama yang digunakan untuk menggambar plot.
Perintah ini memiliki beberapa parameter sebagai berikut:\\
\textgreater{} cos(t) adalah komponen x dari kurva.\\
\textgreater{} sin(t) adalah komponen y dari kurva.\\
\textgreater{} filled digunakan untuk mengisi area di bawah kurva, sehingga
menciptakan bentuk yang terisi. Ini berarti daerah di bawah kurva akan
diisi dengan warna atau pola tertentu.\\
\textgreater{} style="#" mengatur isian kurva menjadi warna solid dengan
menggunakan simbol tanda pagar ("#")

Pada contoh ini tidak ada perintah untuk mengatur warna, maka warna
yang dihasilkan pada plot ini akan mengikuti pada warna yang disetting
pada bagian sebelumnya.

\end{eulercomment}
\eulersubheading{Contoh}
\begin{eulercomment}
Contoh lainnya adalah segi empat, yang kita buat dengan 7 titik pada
lingkaran satuan.
\end{eulercomment}
\begin{eulerprompt}
>t=linspace(0,2pi,7);  ...
>plot2d(cos(t),sin(t),r=1,>filled,style="/",fillcolor=orange):
\end{eulerprompt}
\begin{eulercomment}
Penjelasan:

- t=linspace(0,2pi,7);:\\
Fungsi linspace digunakan untuk membuat array berisi sejumlah nilai
yang merata dalam rentang tertentu. Dalam hal ini, rentangnya adalah
dari 0 hingga 2p (dua kali nilai p) dan sebanyak 7 titik akan
dihasilkan. Ini akan digunakan sebagai sudut dalam koordinat polar
untuk menggambarkan data.

- plot2d(cos(t),sin(t),r=1,\textgreater{}filled,style="/",fillcolor=orange):\\
Ini adalah perintah untuk melakukan plotting data. Terdapat beberapa
argumen di sini:\\
\textgreater{} cos(t): Ini adalah nilai kosinus dari setiap elemen dalam array t.
Ini akan digunakan sebagai komponen sumbu Y dalam koordinat polar.\\
\textgreater{} sin(t): Ini adalah nilai sinus dari setiap elemen dalam array t. Ini
akan digunakan sebagai komponen sumbu X dalam koordinat polar.\\
\textgreater{} r=1: Ini adalah argumen opsional yang menentukan radius plot. Dalam
hal ini, radiusnya diatur menjadi 1.\\
\textgreater{} filled: Ini adalah argumen yang menginstruksikan untuk mengisi area
di dalam kurva plot.\\
\textgreater{} style="/": Ini adalah argumen yang menentukan gaya garis yang
digunakan untuk plot. Di sini, garisnya akan berbentuk garis miring
("/").\\
\textgreater{} fillcolor=orange: Ini adalah argumen yang menentukan warna pengisian
untuk area di dalam kurva plot. Dalam hal ini, warnanya diatur menjadi
oren.

\end{eulercomment}
\eulersubheading{Contoh}
\begin{eulerprompt}
>A=[2,1;1,2;-1,0;0,-1];
>function f(x,y) := max([x,y].A');
>plot2d("f",r=4,level=[0;3],color=red,n=111):
\end{eulerprompt}
\begin{eulercomment}
Penjelasan:

- A=[2,1;1,2;-1,0;0,-1];\\
Ini adalah perintah untuk membuat matriks A. Matriks ini memiliki
dimensi 4x2, yang berarti memiliki 4 baris dan 2 kolom. Isinya adalah:\\
\end{eulercomment}
\begin{eulerformula}
\[
\begin {bmatrix} 2 \hspace{10pt} 1 \\ 1 \hspace{10pt} 2 \\ \end{bmatrix}
\]
\end{eulerformula}
\begin{eulercomment}
- function f(x,y) := max([x,y].A');\\
Ini adalah perintah untuk mendefinisikan sebuah fungsi bernama f(x,
y). Fungsi ini mengambil dua argumen input, yaitu x dan y. Fungsi ini
melakukan operasi berikut:

\textgreater{} [x, y] menghasilkan vektor baris dengan elemen [x, y].\\
\textgreater{} [x, y].A' adalah perkalian dot (dot product) dari vektor baris [x,
y] dengan transpose dari matriks A.\\
\textgreater{} max([x, y].A') menghitung nilai maksimum dari hasil perkalian dot
tersebut.

Dengan kata lain, fungsi `f(x, y)` mengambil vektor `[x, y]` sebagai
input, mengalikannya dengan matriks `A`, dan mengembalikan nilai
maksimum dari hasil perkalian tersebut.

- plot2d("f",r=4,level=[0;3],color=red,n=111):\\
Ini adalah perintah untuk membuat plot 2D dari fungsi `f(x, y)` yang
telah didefinisikan. Rincian perintah ini adalah sebagai berikut:

\textgreater{} "f" adalah nama fungsi yang akan diplot.\\
\textgreater{} r=4 menentukan rentang plot, yang dalam hal ini adalah [-4, 4] untuk
kedua sumbu x dan y.\\
\textgreater{} level=[0;3] menentukan tingkat kontur (contour levels) yang akan
digunakan dalam plot. Ada dua tingkat kontur: 0 dan 3.\\
\textgreater{} color=green mengatur warna kontur plot menjadi merah.\\
\textgreater{} n=111 mengendalikan jumlah titik yang digunakan dalam plot.

Hasilnya akan menjadi sebuah grafik kontur 2D dari fungsi `f(x, y)`
dengan kontur berwarna merah pada tingkat 0 dan 3, yang mencakup
rentang -4 hingga 4 pada kedua sumbu x dan y.

\end{eulercomment}
\eulersubheading{Contoh}
\begin{eulerprompt}
>t=linspace(0,2pi,1000); x=cos(3*t); y=sin(4*t);
>plot2d(x,y,<grid,<frame,>filled):
\end{eulerprompt}
\begin{eulercomment}
Penjelasan:

- t = linspace(0, 2*pi, 1000);\\
Ini adalah perintah untuk membuat vektor t yang berisi 1000 nilai yang
merata terdistribusi antara 0 hingga 2pi. Vektor t ini akan digunakan
sebagai parameter waktu atau sudut dalam parameterisasi lingkaran.\\
linspace(0, 2*pi, 1000) membuat 1000 titik antara 0 hingga 2pi,
memberikan sudut-sudut yang merata di sepanjang satu putaran
lingkaran.

- x = cos(3*t); y = sin(4*t);\\
Ini adalah perintah untuk menghitung vektor x dan y yang menggambarkan
lintasan dalam koordinat polar.

\textgreater{} x = cos(3*t); menghitung nilai x sebagai hasil dari fungsi kosinus
dari 3 kali nilai t. Ini akan menghasilkan osilasi yang lebih cepat
pada sumbu x.\\
\textgreater{} y = sin(4*t); menghitung nilai y sebagai hasil dari fungsi sinus
dari 4 kali nilai t. Ini akan menghasilkan osilasi yang lebih cepat
pada sumbu y.

- plot2d(x, y, \textless{}grid, \textless{}frame, \textgreater{}filled);\\
Ini adalah perintah untuk membuat plot dari vektor x dan y. Berikut
adalah rincian perintah ini:

x adalah vektor yang digunakan sebagai data untuk sumbu x.\\
y adalah vektor yang digunakan sebagai data untuk sumbu y.\\
\textless{}grid mengaktifkan garis-garis koordinat (grid) di latar belakang
plot, membantu dalam visualisasi.\\
\textless{}frame mengaktifkan bingkai (frame) di sekitar plot.\\
\textgreater{}filled mengisi area di bawah kurva dengan warna, membuat plot menjadi
lebih berwarna.

\end{eulercomment}
\eulersubheading{Contoh}
\begin{eulercomment}
Sebuah vektor interval diplot terhadap nilai x sebagai daerah terisi\\
antara nilai interval bawah dan atas.

Ini dapat berguna untuk memplot kesalahan perhitungan. Tapi itu bisa\\
juga digunakan untuk memplot kesalahan statistik.
\end{eulercomment}
\begin{eulerprompt}
>t=0:0.1:1; ...
>plot2d(t,interval(t-random(size(t)),t+random(size(t))),style="|");  ...
>plot2d(t,t,add=true):
\end{eulerprompt}
\begin{eulercomment}
Penjelasan:

- t = 0:0.1:1;\\
Ini adalah perintah untuk membuat vektor t yang berisi nilai-nilai
dari 0 hingga 1 dengan interval 0.1. Hasilnya adalah vektor [0, 0.1,
0.2, 0.3, ..., 0.9, 1].

- plot2d(t, interval(t - random(size(t)), t + random(size(t))),
style="\textbar{}");\\
Ini adalah perintah untuk membuat plot pertama. Rincian perintah ini
adalah sebagai berikut:

\textgreater{} interval(t - random(size(t)), t + random(size(t))) adalah interval
yang digunakan untuk menggambar "garis" pada plot. Setiap titik pada
sumbu x (t) akan dihubungkan oleh dua garis vertikal yang dibuat
secara acak di sekitar titik tersebut menggunakan random(size(t)).
Hasilnya adalah plot dengan garis-garis vertikal yang mewakili
interval acak di sekitar setiap titik pada sumbu x.\\
\textgreater{} style="\textbar{}" mengatur gaya plot menjadi garis vertikal ("\textbar{}").

- plot2d(t, t, add=true);\\
Ini adalah perintah untuk membuat plot kedua dan menambahkannya ke
dalam plot yang sudah ada dari perintah sebelumnya. Rincian perintah
ini adalah sebagai berikut:

\textgreater{} t adalah sumbu x dan y plot ini, sehingga plot ini akan menjadi plot
garis diagonal dengan kemiringan 45 derajat.\\
\textgreater{} add=true digunakan untuk menambahkan plot ini ke dalam plot
sebelumnya, sehingga kedua plot akan ditampilkan dalam satu plot yang
sama.

\end{eulercomment}
\eulersubheading{Contoh}
\begin{eulercomment}
Jika x adalah vektor yang diurutkan, dan y adalah vektor interval,
maka plot2d akan memplot rentang interval yang terisi dalam bidang.
Gaya isian sama dengan gaya poligon.
\end{eulercomment}
\begin{eulerprompt}
>t=-1:0.01:1; x=~t-0.01,t+0.01~; y=x^3-x;
>plot2d(t,y):
\end{eulerprompt}
\begin{eulercomment}
Penjelasan:

- t = -1:0.01:1;\\
Ini adalah perintah untuk membuat vektor t yang berisi nilai-nilai
dari -1 hingga 1 dengan interval 0.01. Hasilnya adalah vektor t yang
berisi nilai-nilai seperti [-1, -0.99, -0.98, ..., 0.99, 1]. Vektor t
ini akan digunakan sebagai sumbu x pada plot.

- x = ~t - 0.01, t + 0.01~;\\
Ini adalah perintah yang menghitung vektor x. Tanda ~ digunakan di
sini untuk mendefinisikan dua interval, yaitu [~t - 0.01, t + 0.01~].
Ini menghasilkan vektor x yang memiliki dua interval, satu yang kurang
dari t - 0.01 dan satu yang lebih dari t + 0.01.

- y = x\textasciicircum{}3 - x;\\
Ini adalah perintah yang menghitung vektor y sebagai fungsi dari x.
Fungsi ini menghitung nilai y dengan memasukkan setiap nilai x ke
dalam rumus x\textasciicircum{}3 - x.

- plot2d(t, y);\\
Ini adalah perintah untuk membuat plot dari fungsi y sebagai fungsi
dari t. Rincian perintah ini adalah sebagai berikut:\\
\textgreater{} t adalah sumbu x pada plot, yang berisi vektor t yang telah
didefinisikan sebelumnya.\\
\textgreater{} y adalah sumbu y pada plot, yang berisi vektor y yang dihitung dari
rumus x\textasciicircum{}3 - x.

\end{eulercomment}
\eulersubheading{Contoh}
\begin{eulerprompt}
>expr := "2*x^2+x*y+3*y^4+y"; // define an expression f(x,y)
>plot2d(expr,level=[0;1],style="-",color=blue): // 0 <= f(x,y) <= 1
\end{eulerprompt}
\begin{eulercomment}
Penjelasan:

- expr := "2*x\textasciicircum{}2+x*y+3*y\textasciicircum{}4+y";\\
Ini adalah perintah untuk mendefinisikan ekspresi matematika yang
disimpan dalam variabel expr. Ekspresi ini merupakan suatu fungsi f(x,
y) yang tergantung pada dua variabel, yaitu x dan y. Ekspresi ini
memiliki bentuk matematika yang terdiri dari berbagai suku, seperti
kuadrat dari x, perkalian x*y, kuadrat dari y, dan lainnya.

- plot2d(expr, level=[0;1], style="-", color=blue);\\
Ini adalah perintah untuk membuat plot dari fungsi f(x, y) yang telah
didefinisikan sebelumnya. Berikut adalah rincian perintah ini:

\textgreater{} expr adalah ekspresi yang akan digunakan sebagai fungsi yang akan
diplotkan. Dalam hal ini, ekspresi 2*x\textasciicircum{}2+x*y+3*y\textasciicircum{}4+y adalah fungsi
f(x, y) yang telah didefinisikan sebelumnya.

\textgreater{} level=[0;1] mengatur tingkat kontur (contour levels) yang akan
digunakan dalam plot. Dalam hal ini, tingkat kontur adalah 0 hingga 1,
yang berarti plot akan menunjukkan wilayah di mana f(x, y) memiliki
nilai antara 0 hingga 1.

\textgreater{} style="-" mengatur gaya plot menjadi garis berjenis -, yang akan
menghasilkan plot kontur.

\textgreater{} color=blue mengatur warna garis plot menjadi biru.

\end{eulercomment}
\eulersubheading{Contoh}
\begin{eulerprompt}
>plot2d("(x^2+y^2)^2-x^2+y^2",r=1.2,level=[-1;0],style="/"):
\end{eulerprompt}
\begin{eulercomment}
Penjelasan:

plot2d("(x\textasciicircum{}2+y\textasciicircum{}2)\textasciicircum{}2-x\textasciicircum{}2+y\textasciicircum{}2", r=1.2, level=[-1;0], style="/");\\
Ini adalah perintah untuk membuat plot dari fungsi matematika yang
didefinisikan dalam bentuk string: "(x\textasciicircum{}2+y\textasciicircum{}2)\textasciicircum{}2-x\textasciicircum{}2+y\textasciicircum{}2". Fungsi ini
tergantung pada dua variabel, yaitu x dan y.

(x\textasciicircum{}2+y\textasciicircum{}2)\textasciicircum{}2-x\textasciicircum{}2+y\textasciicircum{}2 adalah rumus dari fungsi matematika yang akan
diplotkan.

r=1.2 mengatur rentang (range) plot untuk kedua sumbu x dan y. Dalam
hal ini, rentangnya adalah [-1.2, 1.2], yang berarti plot akan berada
dalam wilayah ini.

level=[-1;0] mengatur tingkat kontur (contour levels) yang akan
digunakan dalam plot. Dalam hal ini, ada dua tingkat kontur: -1 dan 0.
Ini akan menentukan wilayah kontur dalam plot.

style="/" mengatur gaya plot menjadi garis miring ("/"). Ini akan
menghasilkan plot dengan garis-garis miring yang menggambarkan kontur
fungsi.

\end{eulercomment}
\eulersubheading{Contoh}
\begin{eulerprompt}
>plot2d("cos(x)","sin(x)^3",xmin=0,xmax=2pi,>filled,style="/"):
\end{eulerprompt}
\begin{eulercomment}
Penjelasan:

plot2d("sin(x)\textasciicircum{}3", "cos(x)", xmin=0, xmax=2*pi, \textgreater{}filled, style="/");\\
Ini adalah perintah untuk membuat plot dari dua fungsi matematika,
yaitu sin(x)\textasciicircum{}3 dan cos(x), dalam satu plot yang sama. Berikut adalah
rincian perintah ini:

"sin(x)\textasciicircum{}3" adalah ekspresi pertama yang akan diplotkan. Ini adalah
fungsi trigonometri sin(x) yang dipangkatkan tiga. Fungsi ini
tergantung pada variabel x.

"cos(x)" adalah ekspresi kedua yang akan diplotkan. Ini adalah fungsi
trigonometri cos(x). Fungsi ini juga tergantung pada variabel x.

xmin=0 dan xmax=2*pi mengatur rentang (range) plot untuk sumbu x dari
0 hingga 2p. Ini adalah rentang yang akan ditampilkan dalam plot.

\textgreater{}filled mengisi area di bawah kurva fungsi dengan warna, sehingga area
di bawah kurva fungsi akan diisi dengan warna.

style="/" mengatur gaya plot menjadi garis miring ("\textbackslash{}"). Ini akan
menghasilkan plot dengan garis-garis miring.
\end{eulercomment}
\eulerheading{Sub Bab 15 }
\begin{eulercomment}
Menggambar Segi Banyak 


Data plot merupakan poligon atau segi banyak. Kita juga dapat membuat
kurva atau mengisi kurva.\\
Fungsi perintah yang digunakan untuk menggambar segi banyak atau
poligon.

Membentuk poligon dengan fungsi:\\
x=linspace(0,2pi,n); plot2d(cos(x),sin(x),r=1,\textgreater{}filled,style="..."):\\
atau\\
x=linspace(0,2pi,n);
plot2d(sin(x),cos(x),r=1,\textgreater{}filled,style="...",fillcolor=red):

Keterangan\\
- filled=true, mengisi plot.\\
- style="...": Pilih dari "#", "/", "\textbackslash{}", "\textbackslash{}/" dan gaya gaya lainnya.\\
- fillcolor: untuk memeberikan warna.

Warna isian ditentukan oleh argumen "fillcolor", dan pada \textless{}outline
opsional mencegah menggambar batas untuk semua gaya kecuali yang
default.

Poligon dalam EMT dapat digambar dengan fungsi maksimal. Dengan fungsi
maksimal ini, poligon yang dihasilkan dapat berupa poligon tak
beraturan.

A=[2,1;1,2;-1,0;0,-1];\\
function f(x,y) := max([x,y].A');\\
plot2d("f",r=4,level=[0;3],color=green,n=111):

Keterangan:\\
-A adalah titik koordinat dari poligon yang akan dibuat.\\
-"r" untuk menentukan ukuran bidang koordinat.

Berikut adalah himpunan nilai maksimal dari empat kondisi linear yang
kurang dari atau sama dengan 3. Ini merupakan A[k].v\textless{}=3 untuk semua
baris A. Untuk mendapatkan sudut yang bagus, kita menggunakan n yang
relatif besar.


1. Menggambar Segitiga
\end{eulercomment}
\begin{eulerprompt}
>x=linspace(0,2pi,3); ...
>plot2d(sin(x),cos(x),r=1):
\end{eulerprompt}
\begin{eulercomment}
Segitiga diatas digambar dari kurva tertutup dengan 3 titik.

Kita dapat membuat segitiga dengan gaya yang berbeda-beda. Seperti
pada contoh berikut ini.
\end{eulercomment}
\begin{eulerprompt}
>x=linspace(0,2pi,3); ...
>plot2d(sin(x),cos(x),>filled,style="/",fillcolor=red,r=1):
>x=linspace(0,2pi,3); ...
>plot2d(sin(x),cos(x),>filled,style="#",fillcolor=blue,r=2):
\end{eulerprompt}
\begin{eulercomment}
Dua gambar segitiga diatas memiliki gaya yang berbeda, dengan
menggunakan fungsi perintah "style=". Gambar segitiga juga dapat
dibuat dengan posisi yang berbeda, tergantung pada fungsi yang akan
diplot.


2. Menggambar Segiempat
\end{eulercomment}
\begin{eulerprompt}
>x=linspace(0,2pi,4); ...
>plot2d(cos(x),sin(x),r=1.5):
>x=linspace(0,2pi,4); 
>plot2d(cos(x),sin(x),r=2,>filled,outline=1):
\end{eulerprompt}
\begin{eulercomment}
Gambar diatas merupakan salah satu contoh segiempat yang dapat
digambar di EMT. Fungsi perintah yang digunakan masih sama seperti
fungsi perintah untuk menggambar segitiga. 

Selain fungsi perintah diatas, untuk menggambar segi banyak, dapat
menggunakan fungsi maksimum.
\end{eulercomment}
\begin{eulerprompt}
>A=[2,1;1,2;-1,0;0,-1];
>function f(x,y) := max([x,y].A');
>plot2d("f",r=4,level=[0;3],color=yellow,n=111):
>A=[1,1;-1,1;-1,-1;1,-1];
>function f(x,y) := max([x,y].A');
>plot2d("f",r=1,level=[0;1],color=gray,n=90):
\end{eulerprompt}
\begin{eulercomment}
Dengan fungsi maksimal ini, kita dapat menggambar segiempat atau segi
banyak sebarang.


3. Menggambar Segilima
\end{eulercomment}
\begin{eulerprompt}
>t=linspace(0,2pi,5); plot2d(sin(t),cos(t),r=1.5):
>t=linspace(0,2pi,5); ...
>plot2d(sin(t),cos(t),r=1.5,>filled,style="\(\backslash\)",fillcolor=orange):
>A=[0,5;3,2;1,-4;-1,-4;-3,2];
>function f(x,y) := max([x,y].A');
>plot2d("f",r=1,level=[0;2],color=cyan,n=111):
\end{eulerprompt}
\begin{eulercomment}
4. Menggambar Segienam
\end{eulercomment}
\begin{eulerprompt}
>t=linspace(0,2pi,6); ...
>plot2d(cos(t),sin(t),r=1.2):
>t=linspace(0,2pi,6); ...
>plot2d(cos(t),sin(t),>filled,style="/",fillcolor=olive,r=1.2):
\end{eulerprompt}
\begin{eulercomment}
5. Menggambar dekagon
\end{eulercomment}
\begin{eulerprompt}
>t=linspace(0,2pi,10); ...
>plot2d(cos(t),sin(t),r=1.2):
>t=linspace(0,2pi,10); ...
\end{eulerprompt}
\end{eulernotebook}
\end{document}


\newpage
\chapter{KB Pekan 5: Menggunakan EMT untuk mengambar grafik 3 dimensi (3D)}
\documentclass[a4paper,10pt]{article}
\usepackage{eumat}

\begin{document}
\begin{eulernotebook}
\eulerheading{Menggambar Plot 3D dengan EMT}
\begin{eulercomment}
Rasyid Shalahuddin\\
22305144016\\
Matematika E 2022 

This is an introduction to 3D plots in Euler. We need a 3D plot to
visualize a function of two variables.

Euler draws such functions using a sorting algorithm to hide parts in
the background. In general, Euler uses a central projection. The
default is from the positive x-y quadrant towards the origin x=y=z=0,
but angle=0° looks from into the direction of the y-axis. The view
angle and height can be changed.

Euler can plot

- surfaces with shading and level lines or level ranges,\\
- clouds of points,\\
- parametric curves,\\
- implicit surfaces.

A 3D plot of a function uses plot3d. The easiest way is to plot an
expression in x and y. The parameter r set the range of the plot
around (0,0).
\end{eulercomment}
\begin{eulerprompt}
>aspect(1.5); plot3d("x^2+sin(y)",-5,5,0,6*pi):
>plot3d("x^2+x*sin(y)",-5,5,0,6*pi):
\end{eulerprompt}
\begin{eulercomment}
Silakan lakukan modifikasi agar gambar "talang bergelombang" tersebut tidak lurus melainkan melengkung/melingkar, baik
melingkar secara mendatar maupun melingkar turun/naik (seperti papan peluncur pada kolam renang. Temukan rumusnya.
\end{eulercomment}
\eulerheading{Functions of two Variables}
\begin{eulercomment}
For the graph of a function, use

- a simple expression in x and y,\\
- the name of a function of two variablesl\\
- or data matrices.

The default is a filled wire grid with different colors on both sides. Note that the default number of grid intervals is
10, but the plot uses the default number of 40x40 rectangles to construct the surface. This can be changed.

- n=40, n=[40,40]: number of grid lines in each direction\\
- grid=10, grid=[10,10]: number of grid lines in each direction.

We use the default n=40 and grid=10.
\end{eulercomment}
\begin{eulerprompt}
>plot3d("x^2+y^2"):
\end{eulerprompt}
\begin{eulercomment}
User interaction is possible with the \textgreater{}user parameter. The user can press the following keys.

- left,right,up,down: turn the viewing angle\\
- +,-: zoom in or out\\
- a: produce an anaglyph (see below)\\
- l: toggle turning the light source (see below)\\
- space: reset to default\\
- return: end interaction
\end{eulercomment}
\begin{eulerprompt}
>plot3d("exp(-x^2+y^2)",>user, ...
>  title="Turn with the vector keys (press return to finish)"):
\end{eulerprompt}
\begin{eulercomment}
The plot range for functions can be specified with

- a,b: the x-range\\
- c,d: the y-range\\
- r: a symmetric square around (0,0).\\
- n: number of subintervals for the plot.

There are some parameters to scale the function or change the look of the graph.

fscale: scales to function values (default is \textless{}fscale).\\
scale: number or 1x2 vector to scale into x- and y-direction.\\
frame: type of frame (default 1).
\end{eulercomment}
\begin{eulerprompt}
>plot3d("exp(-(x^2+y^2)/5)",r=10,n=80,fscale=4,scale=1.2,frame=3,>user):
\end{eulerprompt}
\begin{eulercomment}
The view can be changed in many different ways.

- distance: the viewing distance to the plot.\\
- zoom: the zoom value.\\
- angle: the angle to the negative y-axis in radians.\\
- height: the height of the view in radians.

The default values can be inspected or changed with the function view(). It returns the parameters in the order above.
\end{eulercomment}
\begin{eulerprompt}
>view
\end{eulerprompt}
\begin{euleroutput}
  [5,  2.6,  2,  0.4]
\end{euleroutput}
\begin{eulercomment}
A closer distance needs less zoom. The effect is more like a wide
angle lens.

In the following example, angle=0 and height=0 look from the negative
y-axis. The axis labels for y are hidden in this case.
\end{eulercomment}
\begin{eulerprompt}
>plot3d("x^2+y",distance=3,zoom=1,angle=pi/2,height=0):
\end{eulerprompt}
\begin{eulercomment}
The plot looks always to the center of the plot cube. You can move the center with the center parameter.
\end{eulercomment}
\begin{eulerprompt}
>plot3d("x^4+y^2",a=0,b=1,c=-1,d=1,angle=-20°,height=20°, ...
>  center=[0.4,0,0],zoom=5):
\end{eulerprompt}
\begin{eulercomment}
The plot is scaled to fit into a unit cube for viewing. So there is no need to change the distance or zoom depending on
the size of the plot. The labels refer to the actual size, however.

If you turn this off with scale=false, you need to take care, that the plot still fits into the plotting window, by
changing the viewing distance or zoom, and moving the center.
\end{eulercomment}
\begin{eulerprompt}
>plot3d("5*exp(-x^2-y^2)",r=2,<fscale,<scale,distance=13,height=50°, ...
>  center=[0,0,-2],frame=3):
\end{eulerprompt}
\begin{eulercomment}
A polar plot is also available. The parameter polar=true draws a polar plot. The function must still be a function of x and y. The
parameter "fscale" scales the function with an own scale. Otherwise the function is scaled to fit into a cube.
\end{eulercomment}
\begin{eulerprompt}
>plot3d("1/(x^2+y^2+1)",r=5,>polar, ...
>fscale=2,>hue,n=100,zoom=4,>contour,color=blue):
>function f(r) := exp(-r/2)*cos(r); ...
>plot3d("f(x^2+y^2)",>polar,scale=[1,1,0.4],r=pi,frame=3,zoom=4):
\end{eulerprompt}
\begin{eulercomment}
The parameter rotate rotates a function in x around the x-axis.

- rotate=1: Uses the x-axis\\
- rotate=2: Uses the z-axis
\end{eulercomment}
\begin{eulerprompt}
>plot3d("x^2+1",a=-1,b=1,rotate=true,grid=5):
>plot3d("x^2+1",a=-1,b=1,rotate=2,grid=5):
>plot3d("sqrt(25-x^2)",a=0,b=5,rotate=1):
>plot3d("x*sin(x)",a=0,b=6pi,rotate=2):
\end{eulerprompt}
\begin{eulercomment}
Here is a plot with three functions.
\end{eulercomment}
\begin{eulerprompt}
>plot3d("x","x^2+y^2","y",r=2,zoom=3.5,frame=3):
\end{eulerprompt}
\eulerheading{Contour Plots}
\begin{eulercomment}
For the plot, Euler adds grid lines. Instead it is possible to use level lines and a
one-color hue or a spectral colored hue. Euler can draw the heights of functions on a
plot with shading. In all 3D plots Euler can produce red/cyan anaglyphs.

- \textgreater{}hue: Turns on light shading instead of wires.\\
- \textgreater{}contour: Plots automatic contour lines on a plot.\\
- level=... (or levels): A vector of values for the contour lines.

The default is level="auto", which computes some level lines automatically. As you
see in the plot, the levels are in fact ranges of levels.

The default style can be changed. For the following contour plot, we use a finer grid
fo 100x100 points, scale the function and the plot, and use different angle of view.
\end{eulercomment}
\begin{eulerprompt}
>plot3d("exp(-x^2-y^2)",r=2,n=100,level="thin", ...
> >contour,>spectral,fscale=1,scale=1.1,angle=45°,height=20°):
>plot3d("exp(x*y)",angle=100°,>contour,color=green):
\end{eulerprompt}
\begin{eulercomment}
The default shading uses a gray color. But a spectral range of colors is also available.

- \textgreater{}spectral: Used the default spectral scheme\\
- color=...: Uses special colors or spectral schemes

For the following plot, we use the default spectral scheme and increase the number of points to get a very smooth look.
\end{eulercomment}
\begin{eulerprompt}
>plot3d("x^2+y^2",>spectral,>contour,n=100):
\end{eulerprompt}
\begin{eulercomment}
Instead of automatic level lines, we can also set values of the level lines. This will produce thin level lines instead
of ranges of levels.
\end{eulercomment}
\begin{eulerprompt}
>plot3d("x^2-y^2",0,5,0,5,level=-1:0.1:1,color=redgreen):
\end{eulerprompt}
\begin{eulercomment}
In the following plot, we use two very broad level bands from -0.1 to 1, and from 0.9 to 1. This is entered as a matrix
with level bounds as columns.

Moreover, we overlay a grid with 10 intervals in each direction.
\end{eulercomment}
\begin{eulerprompt}
>plot3d("x^2+y^3",level=[-0.1,0.9;0,1], ...
>  >spectral,angle=30°,grid=10,contourcolor=gray):
\end{eulerprompt}
\begin{eulercomment}
In the following example, we plot the set, where

\end{eulercomment}
\begin{eulerformula}
\[
f(x,y) = x^y-y^x = 0
\]
\end{eulerformula}
\begin{eulercomment}
We use a single thin line for the level line.
\end{eulercomment}
\begin{eulerprompt}
>plot3d("x^y-y^x",level=0,a=0,b=6,c=0,d=6,contourcolor=red,n=100):
\end{eulerprompt}
\begin{eulercomment}
It is possible to show a contour plane below the plot. A color and
distance to the plot can be specified.
\end{eulercomment}
\begin{eulerprompt}
>plot3d("x^2+y^4",>cp,cpcolor=green,cpdelta=0.2):
\end{eulerprompt}
\begin{eulercomment}
Here are a few more styles. We always turn off the frame, and use
various color schemes for the plot and the grid.
\end{eulercomment}
\begin{eulerprompt}
>figure(2,2); ...
>expr="y^3-x^2"; ...
>figure(1);  ...
>  plot3d(expr,<frame,>cp,cpcolor=spectral); ...
>figure(2);  ...
>  plot3d(expr,<frame,>spectral,grid=10,cp=2); ...
>figure(3);  ...
>  plot3d(expr,<frame,>contour,color=gray,nc=5,cp=3,cpcolor=greenred); ...
>figure(4);  ...
>  plot3d(expr,<frame,>hue,grid=10,>transparent,>cp,cpcolor=gray); ...
>figure(0):
\end{eulerprompt}
\begin{eulercomment}
There are some other spectral schemes, numbered from 1 to 9. But you can also use the color=value, where value

- spectral: for a range from blue to red\\
- white: for a fainter range\\
- yellowblue,purplegreen,blueyellow,greenred\\
- blueyellow, greenpurple,yellowblue,redgreen
\end{eulercomment}
\begin{eulerprompt}
>figure(3,3); ...
>for i=1:9;  ...
>  figure(i); plot3d("x^2+y^2",spectral=i,>contour,>cp,<frame,zoom=4);  ...
>end; ...
>figure(0):
\end{eulerprompt}
\begin{eulercomment}
The light source can be changed with l and the cursor keys during the user interaction. It can also be set with
parameters.

- light: a direction for the light\\
- amb: ambient light between 0 and 1

Note that the program does not make a difference between the sides of the plot. There are no shadows. For this you would
need Povray.
\end{eulercomment}
\begin{eulerprompt}
>plot3d("-x^2-y^2", ...
>  hue=true,light=[0,1,1],amb=0,user=true, ...
>  title="Press l and cursor keys (return to exit)"):
\end{eulerprompt}
\begin{eulercomment}
The color parameter changes the color of the surface. The color of the level lines can also be changed.
\end{eulercomment}
\begin{eulerprompt}
>plot3d("-x^2-y^2",color=rgb(0.2,0.2,0),hue=true,frame=false, ...
>  zoom=3,contourcolor=red,level=-2:0.1:1,dl=0.01):
\end{eulerprompt}
\begin{eulercomment}
The color 0 gives a special rainbow effect.
\end{eulercomment}
\begin{eulerprompt}
>plot3d("x^2/(x^2+y^2+1)",color=0,hue=true,grid=10):
\end{eulerprompt}
\begin{eulercomment}
The surface can also be transparent.
\end{eulercomment}
\begin{eulerprompt}
>plot3d("x^2+y^2",>transparent,grid=10,wirecolor=red):
\end{eulerprompt}
\eulerheading{Implicit Plots}
\begin{eulercomment}
There are also implicit plots in three dimensions. Euler generates cuts through the objects. The features of plot3d
include implicit plots. These plots show the zero set of a function in three variables.\\
The solutions of

\end{eulercomment}
\begin{eulerformula}
\[
f(x,y,z) = 0
\]
\end{eulerformula}
\begin{eulercomment}
can be visualized in cuts parallel to the x-y-, the x-z- and the y-z-plane.

- implicit=1: cut parallel to the y-z-plane\\
- implicit=2: cut parallel to the x-z-plane\\
- implicit=4: cut parallel to the x-y-plane

Add these values, if you like. In the example we plot

\end{eulercomment}
\begin{eulerformula}
\[
M = \{ (x,y,z) : x^2+y^3+zy=1 \}
\]
\end{eulerformula}
\begin{eulerprompt}
>plot3d("x^2+y^3+z*y-1",r=5,implicit=3):
>c=1; d=1;
>plot3d("((x^2+y^2-c^2)^2+(z^2-1)^2)*((y^2+z^2-c^2)^2+(x^2-1)^2)*((z^2+x^2-c^2)^2+(y^2-1)^2)-d",r=2,<frame,>implicit,>user): 
\end{eulerprompt}
\begin{euleroutput}
  Cannot combine a 41x41 and a 1x81 matrix for +!
  Error in expression: ((x^2+y^2-c^2)^2+(z^2-1)^2)*((y^2+z^2-c^2)^2+(x^2-1)^2)*((z^2+x^2-c^2)^2+(y^2-1)^2)-d
  Try "trace errors" to inspect local variables after errors.
  pov3d:
      z=f(x,y;args());
\end{euleroutput}
\begin{eulerprompt}
>plot3d("x^2+y^2+4*x*z+z^3",>implicit,r=2,zoom=2.5):
\end{eulerprompt}
\eulerheading{Plotting 3D Data}
\begin{eulercomment}
Just as plot2d, plot3d accepts data. For 3D objects, you need to provide a matrix of x-, y- and z-values, or three
functions or expressions fx(x,y), fy(x,y), fz(x,y).

\end{eulercomment}
\begin{eulerformula}
\[
\gamma(t,s) = (x(t,s),y(t,s),z(t,s))
\]
\end{eulerformula}
\begin{eulercomment}
Since x,y,z are matrices, we assume that (t,s) run through a square grid. As a result, you can plot images of rectangles
in space.

You can use the Euler matrix language to produce the coordinates effectively.

In the following example, we use a vector of t values and a column vector of s values to parameterize the surface of the
ball. In the drawing we can mark regions, in our case the polar region.
\end{eulercomment}
\begin{eulerprompt}
>t=linspace(0,2pi,180); s=linspace(-pi/2,pi/2,90)'; ...
>x=cos(s)*cos(t); y=cos(s)*sin(t); z=sin(s); ...
>plot3d(x,y,z,>hue, ...
>color=blue,<frame,grid=[10,20], ...
>values=s,contourcolor=red,level=[90°-24°;90°-22°], ...
>scale=1.4,height=50°):
\end{eulerprompt}
\begin{eulercomment}
Here is an example, which is the graph of a function.
\end{eulercomment}
\begin{eulerprompt}
>t=-1:0.1:1; s=(-1:0.1:1)'; plot3d(t,s,t*s,grid=10):
\end{eulerprompt}
\begin{eulercomment}
However, we can make all sorts of surfaces. Here is the same surface
as a function

\end{eulercomment}
\begin{eulerformula}
\[
x = y \, z
\]
\end{eulerformula}
\begin{eulerprompt}
>plot3d(t*s,t,s,angle=180°,grid=10):
\end{eulerprompt}
\begin{eulercomment}
With more effort, we can produce many surfaces.

In the following example we make a shaded view of a distorted ball. The usual coordinates for the ball are

\end{eulercomment}
\begin{eulerformula}
\[
\gamma(t,s) = (\cos(t)\cos(s),\sin(t)\sin(s),\cos(s))
\]
\end{eulerformula}
\begin{eulercomment}
with

\end{eulercomment}
\begin{eulerformula}
\[
0 \le t \le 2\pi, \quad \frac{-\pi}{2} \le s \le \frac{\pi}{2}.
\]
\end{eulerformula}
\begin{eulercomment}
We distored this with a factor

\end{eulercomment}
\begin{eulerformula}
\[
d(t,s) = \frac{\cos(4t)+\cos(8s)}{4}.
\]
\end{eulerformula}
\begin{eulerprompt}
>t=linspace(0,2pi,320); s=linspace(-pi/2,pi/2,160)'; ...
>d=1+0.2*(cos(4*t)+cos(8*s)); ...
>plot3d(cos(t)*cos(s)*d,sin(t)*cos(s)*d,sin(s)*d,hue=1, ...
>  light=[1,0,1],frame=0,zoom=5):
\end{eulerprompt}
\begin{eulercomment}
Of course, a point cloud is also possible. To plot point data in the space, we need three vectors for the coordinates of
the points.

The styles are just as in plot2d with points=true;
\end{eulercomment}
\begin{eulerprompt}
>n=500;  ...
>  plot3d(normal(1,n),normal(1,n),normal(1,n),points=true,style="."):
\end{eulerprompt}
\begin{eulercomment}
It is also possible to plot a curve in 3D. In this case, it is easier to precompute
the points of the curve. For curves in the plane we use a sequence of coordinates and
the parameter wire=true.
\end{eulercomment}
\begin{eulerprompt}
>t=linspace(0,8pi,500); ...
>plot3d(sin(t),cos(t),t/10,>wire,zoom=3):
>t=linspace(0,4pi,1000); plot3d(cos(t),sin(t),t/2pi,>wire, ...
>linewidth=3,wirecolor=blue):
>X=cumsum(normal(3,100)); ...
> plot3d(X[1],X[2],X[3],>anaglyph,>wire):
\end{eulerprompt}
\begin{eulercomment}
EMT can also plot in anaglyph mode. To view such a plot, you need red/cyan glasses.
\end{eulercomment}
\begin{eulerprompt}
> plot3d("x^2+y^3",>anaglyph,>contour,angle=30°):
\end{eulerprompt}
\begin{eulercomment}
Often, a spectral color scheme is used for plots. This emphasizes the heights of the
function.
\end{eulercomment}
\begin{eulerprompt}
>plot3d("x^2*y^3-y",>spectral,>contour,zoom=3.2):
\end{eulerprompt}
\begin{eulercomment}
Euler can plot parameterized surfaces too, when the parameters are the x-, y-, and
z-values of an image of a rectangular grid in the space.

For the following demo, we setup u- and v- parameters, and generate space coordinates
from these.
\end{eulercomment}
\begin{eulerprompt}
>u=linspace(-1,1,10); v=linspace(0,2*pi,50)'; ...
>X=(3+u*cos(v/2))*cos(v); Y=(3+u*cos(v/2))*sin(v); Z=u*sin(v/2); ...
>plot3d(X,Y,Z,>anaglyph,<frame,>wire,scale=2.3):
\end{eulerprompt}
\begin{eulercomment}
Here is a more complicated example, which is majestic with red/cyan glasses.
\end{eulercomment}
\begin{eulerprompt}
>u:=linspace(-pi,pi,160); v:=linspace(-pi,pi,400)';  ...
>x:=(4*(1+.25*sin(3*v))+cos(u))*cos(2*v); ...
>y:=(4*(1+.25*sin(3*v))+cos(u))*sin(2*v); ...
> z=sin(u)+2*cos(3*v); ...
>plot3d(x,y,z,frame=0,scale=1.5,hue=1,light=[1,0,-1],zoom=2.8,>anaglyph):
\end{eulerprompt}
\eulerheading{Statistical Plots}
\begin{eulercomment}
Bar plots are possible too. For this, we have to provide

- x: row vector with n+1 elements\\
- y: column vector with n+1 elements\\
- z: nxn matrix of values.

z can be larger, but only nxn values will be used.

In the example, we first compute the values. Then we adjust x and y, so that the vectors center at the values used.
\end{eulercomment}
\begin{eulerprompt}
>x=-1:0.1:1; y=x'; z=x^2+y^2; ...
>xa=(x|1.1)-0.05; ya=(y_1.1)-0.05; ...
>plot3d(xa,ya,z,bar=true):
\end{eulerprompt}
\begin{eulercomment}
It is possible to split the plot of a surface in two or more parts.
\end{eulercomment}
\begin{eulerprompt}
>x=-1:0.1:1; y=x'; z=x+y; d=zeros(size(x)); ...
>plot3d(x,y,z,disconnect=2:2:20):
\end{eulerprompt}
\begin{eulercomment}
If load or generate a data matrix M from a file and need to plot it in
3D you can either scale the matrix to [-1,1] with scale(M), or scale
the matrix with \textgreater{}zscale. This can be combined with individual scaling
factors which are applied additionally.
\end{eulercomment}
\begin{eulerprompt}
>i=1:20; j=i'; ...
>plot3d(i*j^2+100*normal(20,20),>zscale,scale=[1,1,1.5],angle=-40°,zoom=1.8):
>Z=intrandom(5,100,6); v=zeros(5,6); ...
>loop 1 to 5; v[#]=getmultiplicities(1:6,Z[#]); end; ...
>columnsplot3d(v',scols=1:5,ccols=[1:5]):
\end{eulerprompt}
\eulerheading{Permukaan Benda Putar}
\begin{eulerprompt}
>plot2d("(x^2+y^2-1)^3-x^2*y^3",r=1.3, ...
>style="#",color=red,<outline, ...
>level=[-2;0],n=100):
>ekspresi &= (x^2+y^2-1)^3-x^2*y^3; $ekspresi
\end{eulerprompt}
\begin{eulercomment}
We wish to turn the heart curve around the y-axis. Here is the expression, which
defines the heart:

\end{eulercomment}
\begin{eulerformula}
\[
f(x,y)=(x^2+y^2-1)^3-x^2.y^3.
\]
\end{eulerformula}
\begin{eulercomment}
Next we set

\end{eulercomment}
\begin{eulerformula}
\[
x=r.cos(a),\quad y=r.sin(a).
\]
\end{eulerformula}
\begin{eulerprompt}
>function fr(r,a) &= ekspresi with [x=r*cos(a),y=r*sin(a)] | trigreduce; $fr(r,a)
\end{eulerprompt}
\begin{eulercomment}
This allows to define a numerical function, which solves for r, if a is given. With
that function we can plot the turned heart as a parametric surface.
\end{eulercomment}
\begin{eulerprompt}
>function map f(a) := bisect("fr",0,2;a); ...
>t=linspace(-pi/2,pi/2,100); r=f(t);  ...
>s=linspace(pi,2pi,100)'; ...
>plot3d(r*cos(t)*sin(s),r*cos(t)*cos(s),r*sin(t), ...
>>hue,<frame,color=red,zoom=4,amb=0,max=0.7,grid=12,height=50°):
\end{eulerprompt}
\begin{eulercomment}
The following is a 3D plot of the figure above rotated around the z-axis. We define
the function, which describes the object.
\end{eulercomment}
\begin{eulerprompt}
>function f(x,y,z) ...
\end{eulerprompt}
\begin{eulerudf}
  r=x^2+y^2;
  return (r+z^2-1)^3-r*z^3;
   endfunction
\end{eulerudf}
\begin{eulerprompt}
>plot3d("f(x,y,z)", ...
>xmin=0,xmax=1.2,ymin=-1.2,ymax=1.2,zmin=-1.2,zmax=1.4, ...
>implicit=1,angle=-30°,zoom=2.5,n=[10,100,60],>anaglyph):
\end{eulerprompt}
\eulerheading{Special 3D Plots}
\begin{eulercomment}
The plot3d function is nice to have, but it does not satisfy all needs. Besides more basic routines, it is possible to get a
framed plot of any object you like.

Though Euler is not a 3D program, it can combine some basic objects. We try to visualize a paraboloid and its tangent.
\end{eulercomment}
\begin{eulerprompt}
>function myplot ...
\end{eulerprompt}
\begin{eulerudf}
    y=-1:0.01:1; x=(-1:0.01:1)';
    plot3d(x,y,0.2*(x-0.1)/2,<scale,<frame,>hue, ..
      hues=0.5,>contour,color=orange);
    h=holding(1);
    plot3d(x,y,(x^2+y^2)/2,<scale,<frame,>contour,>hue);
    holding(h);
  endfunction
\end{eulerudf}
\begin{eulercomment}
Now framedplot() provides the frames, and sets the views.
\end{eulercomment}
\begin{eulerprompt}
>framedplot("myplot",[-1,1,-1,1,0,1],height=0,angle=-30°, ...
>  center=[0,0,-0.7],zoom=3):
\end{eulerprompt}
\begin{eulercomment}
In the same way, you can plot the contour plane manually. Note that plot3d() sets the window to fullwindow() by default, but
plotcontourplane() assumes that.
\end{eulercomment}
\begin{eulerprompt}
>x=-1:0.02:1.1; y=x'; z=x^2-y^4;
>function myplot (x,y,z) ...
\end{eulerprompt}
\begin{eulerudf}
    zoom(2);
    wi=fullwindow();
    plotcontourplane(x,y,z,level="auto",<scale);
    plot3d(x,y,z,>hue,<scale,>add,color=white,level="thin");
    window(wi);
    reset();
  endfunction
\end{eulerudf}
\begin{eulerprompt}
>myplot(x,y,z):
\end{eulerprompt}
\eulerheading{Animation}
\begin{eulercomment}
Euler can use frames to pre-compute the animation.

One function, which makes use of this technique is rotate. It can
change the angle of view and redraw a 3D plot. The function calls
addpage() for each new plot. Finally it animates the plots.

Please study the source of rotate to see more details.
\end{eulercomment}
\begin{eulerprompt}
>function testplot () := plot3d("x^2+y^3"); ...
>rotate("testplot"); testplot():
\end{eulerprompt}
\eulerheading{Menggambar Povray}
\begin{eulercomment}
With the help of the Euler file povray.e, Euler can generate Povray files. The results are very nice to look at.

You need to install Povray (32bit or 64bit) from http://www.povray.org/, and put the sub-directory "bin" of Povray into
the environment path, or set the variable "defaultpovray" with full path pointing to "pvengine.exe".

The Povray interface of Euler generates Povray files in the home directory of the user, and calls Povray to parse these
files. The default file name is current.pov, and the default directory is eulerhome(), usually c:\textbackslash{}Users\textbackslash{}Username\textbackslash{}Euler.
Povray generates a PNG file, which can be loaded by Euler into a notebook. To clean up these files, use povclear().

The pov3d function is in the same spirit as plot3d. It can generate the graph of a function f(x,y), or a surface with
coordinates X,Y,Z in matrices, including optional level lines. This function starts the raytracer automatically, and
loads the scene into the Euler notebook.

Besides pov3d(), there are many functions, which generate Povray objects. These functions return strings, containing the
Povray code for the objects. To use these functions, start the Povray file with povstart(). Then use writeln(...) to
write the objects to the scene file. Finally, end the file with povend(). By default, the raytracer will start, and the
PNG will be inserted into the Euler notebook.

The object functions have a parameter called "look", which needs a string with Povray code for the texture and the finish
of the object. The function povlook() can be used to produce this string. It has parameters for the color, the
transparency, Phong Shading etc.

Note that the Povray universe has another coordinate system. This interface translates all coordinates to the Povray
system. So you can keep thinking in the Euler coordinate system with z pointing vertically upwards,a nd x,y,z axes in
right hand sense.\\
You need to load the povray file.
\end{eulercomment}
\begin{eulerprompt}
>load povray;
\end{eulerprompt}
\begin{eulercomment}
Make sure, the Povray bin directory is in the path. If it is not edit the following variable so that it contains the path
to the povray executable.
\end{eulercomment}
\begin{eulerprompt}
>defaultpovray="C:\(\backslash\)Program Files\(\backslash\)POV-Ray\(\backslash\)v3.7\(\backslash\)bin\(\backslash\)pvengine.exe"
\end{eulerprompt}
\begin{euleroutput}
  C:\(\backslash\)Program Files\(\backslash\)POV-Ray\(\backslash\)v3.7\(\backslash\)bin\(\backslash\)pvengine.exe
\end{euleroutput}
\begin{eulercomment}
For a first impression, we plot a simple function. The following command generates a povray file in your user directory,
and runs Povray for ray tracing this file.

If you start the following command, the Povray GUI should open, run the file, and close automatically. Due to security
reasons, you will be asked, if you want to allow the exe file to run. You can press cancel to stop further questions. You
may have to press OK in the Povray window to acknowledge the start-up dialog of Povray.
\end{eulercomment}
\begin{eulerprompt}
>plot3d("x^2+y^2",zoom=2):
>pov3d("x^2+y^2",zoom=3);
\end{eulerprompt}
\begin{eulercomment}
We can make the function transparent and add another finish. We can
also add level lines to the function plot.
\end{eulercomment}
\begin{eulerprompt}
>pov3d("x^2+y^3",axiscolor=red,angle=-45°,>anaglyph, ...
>  look=povlook(cyan,0.2),level=-1:0.5:1,zoom=3.8);
\end{eulerprompt}
\begin{eulercomment}
Sometimes it is necessary to prevent the scaling of the function, and scale the function by hand.

We plot the set of points in the complex plane, where the product of the distances to 1 and -1 is equal to 1.
\end{eulercomment}
\begin{eulerprompt}
>pov3d("((x-1)^2+y^2)*((x+1)^2+y^2)/40",r=2, ...
>  angle=-120°,level=1/40,dlevel=0.005,light=[-1,1,1],height=10°,n=50, ...
>  <fscale,zoom=3.8);
\end{eulerprompt}
\eulerheading{Plotting with Coordinates}
\begin{eulercomment}
Instead of functions, we can plot with coordinates. As in plot3d, we need three matrices to define the object.

In the example we turn a function around the z-axis.
\end{eulercomment}
\begin{eulerprompt}
>function f(x) := x^3-x+1; ...
>x=-1:0.01:1; t=linspace(0,2pi,50)'; ...
>Z=x; X=cos(t)*f(x); Y=sin(t)*f(x); ...
>pov3d(X,Y,Z,angle=40°,look=povlook(red,0.1),height=50°,axis=0,zoom=4,light=[10,5,15]);
\end{eulerprompt}
\begin{eulercomment}
In the following example, we plot a damped wave. We generate the wave with the matrix language of Euler.

We also show, how an additional object can be added to a pov3d scene. For the generation of objects, see the following
examples. Note that plot3d scales the plot, so that it fits into the unit cube.
\end{eulercomment}
\begin{eulerprompt}
>r=linspace(0,1,80); phi=linspace(0,2pi,80)'; ...
>x=r*cos(phi); y=r*sin(phi); z=exp(-5*r)*cos(8*pi*r)/3;  ...
>pov3d(x,y,z,zoom=6,axis=0,height=30°,add=povsphere([0.5,0,0.25],0.15,povlook(red)), ...
>  w=500,h=300);
\end{eulerprompt}
\begin{eulercomment}
With the advanced shading method of Povray, very few points can
produce very smooth surfaces. Only at the boundaries and in shadows
the trick might become obvious.

For this, we need to add normal vectors in each matrix point.
\end{eulercomment}
\begin{eulerprompt}
>Z &= x^2*y^3
\end{eulerprompt}
\begin{euleroutput}
  
                                                             2  3
                                                            x  y
  
\end{euleroutput}
\begin{eulercomment}
The equation of the surface is [x,y,Z]. We compute the two derivatives
to x and y of this and take the cross product as the normal.
\end{eulercomment}
\begin{eulerprompt}
>dx &= diff([x,y,Z],x); dy &= diff([x,y,Z],y);
\end{eulerprompt}
\begin{eulercomment}
We define the normal as the cross product of these derivatives, and
define coordinate functions.
\end{eulercomment}
\begin{eulerprompt}
>N &= crossproduct(dx,dy); NX &= N[1]; NY &= N[2]; NZ &= N[3]; N,
\end{eulerprompt}
\begin{euleroutput}
  
                                                          3       2  2
                                                  [- 2 x y , - 3 x  y , 1]
  
\end{euleroutput}
\begin{eulercomment}
We use only 25 points.
\end{eulercomment}
\begin{eulerprompt}
>x=-1:0.5:1; y=x';
>pov3d(x,y,Z(x,y),angle=10°, ...
>  xv=NX(x,y),yv=NY(x,y),zv=NZ(x,y),<shadow);
\end{eulerprompt}
\begin{eulercomment}
The following is the Trefoil knot done by A. Busser in Povray. There
is an improved version of this in the examples.

See: Examples\textbackslash{}Trefoil Knot \textbar{} Trefoil Knot

For a good look with not too many points, we add normal vectors here.
We use Maxima to compute the normals for us. First, the three
functions for the coordinates as symbolic expressions.
\end{eulercomment}
\begin{eulerprompt}
>X &= ((4+sin(3*y))+cos(x))*cos(2*y); ...
>Y &= ((4+sin(3*y))+cos(x))*sin(2*y); ...
>Z &= sin(x)+2*cos(3*y);
\end{eulerprompt}
\begin{eulercomment}
Then the two derivative vectors to x and y.
\end{eulercomment}
\begin{eulerprompt}
>dx &= diff([X,Y,Z],x); dy &= diff([X,Y,Z],y);
\end{eulerprompt}
\begin{eulercomment}
Now the normal, which is the cross product of the two derivatives.
\end{eulercomment}
\begin{eulerprompt}
>dn &= crossproduct(dx,dy);
\end{eulerprompt}
\begin{eulercomment}
We now evaluate all this numerically.
\end{eulercomment}
\begin{eulerprompt}
>x:=linspace(-%pi,%pi,40); y:=linspace(-%pi,%pi,100)';
\end{eulerprompt}
\begin{eulercomment}
The normal vectors are evaluations of the symbolic expressions dn[i]
for i=1,2,3. The syntax for this is \&"expression"(parameters). This is
an alternative to the method in the previous example, where we defined
symbolic expressions NX, NY, NZ first.
\end{eulercomment}
\begin{eulerprompt}
>pov3d(X(x,y),Y(x,y),Z(x,y),>anaglyph,axis=0,zoom=5,w=450,h=350, ...
>  <shadow,look=povlook(blue), ...
>  xv=&"dn[1]"(x,y), yv=&"dn[2]"(x,y), zv=&"dn[3]"(x,y));
\end{eulerprompt}
\begin{eulercomment}
We can also generate a grid in 3D.
\end{eulercomment}
\begin{eulerprompt}
>povstart(zoom=4); ...
>x=-1:0.5:1; r=1-(x+1)^2/6; ...
>t=(0°:30°:360°)'; y=r*cos(t); z=r*sin(t); ...
>writeln(povgrid(x,y,z,d=0.02,dballs=0.05)); ...
>povend();
\end{eulerprompt}
\begin{eulercomment}
With povgrid(), curves are possible.
\end{eulercomment}
\begin{eulerprompt}
>povstart(center=[0,0,1],zoom=3.6); ...
>t=linspace(0,2,1000); r=exp(-t); ...
>x=cos(2*pi*10*t)*r; y=sin(2*pi*10*t)*r; z=t; ...
>writeln(povgrid(x,y,z,povlook(red))); ...
>writeAxis(0,2,axis=3); ...
>povend();
\end{eulerprompt}
\eulerheading{Povray Objects}
\begin{eulercomment}
Above, we used pov3d to plot surfaces. The povray interface in Euler can also generate Povray objects. These objects are
stored as strings in Euler, and need to be written to a Povray file.

We start the output with povstart().
\end{eulercomment}
\begin{eulerprompt}
>povstart(zoom=4);
\end{eulerprompt}
\begin{eulercomment}
First we define the three cylinders, and store them in strings in Euler.

The functions povx() etc. simply returns the vector [1,0,0], which could be used instead.
\end{eulercomment}
\begin{eulerprompt}
>c1=povcylinder(-povx,povx,1,povlook(red)); ...
>c2=povcylinder(-povy,povy,1,povlook(yellow)); ...
>c3=povcylinder(-povz,povz,1,povlook(blue)); ...
\end{eulerprompt}
\begin{eulercomment}
The strings contain Povray code, which we need not understand at that
point.
\end{eulercomment}
\begin{eulerprompt}
>c2
\end{eulerprompt}
\begin{euleroutput}
  cylinder \{ <0,0,-1>, <0,0,1>, 1
   texture \{ pigment \{ color rgb <0.941176,0.941176,0.392157> \}  \} 
   finish \{ ambient 0.2 \} 
   \}
\end{euleroutput}
\begin{eulercomment}
As you see, we added texture to the objects in three different colors.

That is done by povlook(), which returns a string with the relevant
Povray code. We can use the default Euler colors, or define our own
color. We can also add transparency, or change the ambient light.
\end{eulercomment}
\begin{eulerprompt}
>povlook(rgb(0.1,0.2,0.3),0.1,0.5)
\end{eulerprompt}
\begin{euleroutput}
   texture \{ pigment \{ color rgbf <0.101961,0.2,0.301961,0.1> \}  \} 
   finish \{ ambient 0.5 \} 
  
\end{euleroutput}
\begin{eulercomment}
Now we define an intersection object, and write the result to the
file.
\end{eulercomment}
\begin{eulerprompt}
>writeln(povintersection([c1,c2,c3]));
\end{eulerprompt}
\begin{eulercomment}
The intersection of three cylinders is hard to visualize, if you never
saw it before.
\end{eulercomment}
\begin{eulerprompt}
>povend;
\end{eulerprompt}
\begin{eulercomment}
The following functions generate a fractal recursively.

The first function shows, how Euler handles simple Povray objects. The
function povbox() returns a string, containing the box coordinates,
the texture and the finish.
\end{eulercomment}
\begin{eulerprompt}
>function onebox(x,y,z,d) := povbox([x,y,z],[x+d,y+d,z+d],povlook());
>function fractal (x,y,z,h,n) ...
\end{eulerprompt}
\begin{eulerudf}
   if n==1 then writeln(onebox(x,y,z,h));
   else
     h=h/3;
     fractal(x,y,z,h,n-1);
     fractal(x+2*h,y,z,h,n-1);
     fractal(x,y+2*h,z,h,n-1);
     fractal(x,y,z+2*h,h,n-1);
     fractal(x+2*h,y+2*h,z,h,n-1);
     fractal(x+2*h,y,z+2*h,h,n-1);
     fractal(x,y+2*h,z+2*h,h,n-1);
     fractal(x+2*h,y+2*h,z+2*h,h,n-1);
     fractal(x+h,y+h,z+h,h,n-1);
   endif;
  endfunction
\end{eulerudf}
\begin{eulerprompt}
>povstart(fade=10,<shadow);
>fractal(-1,-1,-1,2,4);
>povend();
\end{eulerprompt}
\begin{eulercomment}
Differences allow cutting off one object from another. Like
intersections, there are part of the CSG objects of Povray.
\end{eulercomment}
\begin{eulerprompt}
>povstart(light=[5,-5,5],fade=10);
\end{eulerprompt}
\begin{eulercomment}
For this demonstration, we define an object in Povray, instead of
using a string in Euler. Definitions are written to the file
immediately.

A box coordinate of -1 just means [-1,-1,-1].
\end{eulercomment}
\begin{eulerprompt}
>povdefine("mycube",povbox(-1,1));
\end{eulerprompt}
\begin{eulercomment}
We can use this object in povobject(), which returns a string as
usual.
\end{eulercomment}
\begin{eulerprompt}
>c1=povobject("mycube",povlook(red));
\end{eulerprompt}
\begin{eulercomment}
We generate a second cube, and rotate and scale it a bit.
\end{eulercomment}
\begin{eulerprompt}
>c2=povobject("mycube",povlook(yellow),translate=[1,1,1], ...
>  rotate=xrotate(10°)+yrotate(10°), scale=1.2);
\end{eulerprompt}
\begin{eulercomment}
Then we take the difference of the two objects.
\end{eulercomment}
\begin{eulerprompt}
>writeln(povdifference(c1,c2));
\end{eulerprompt}
\begin{eulercomment}
Now add three axes.
\end{eulercomment}
\begin{eulerprompt}
>writeAxis(-1.2,1.2,axis=1); ...
>writeAxis(-1.2,1.2,axis=2); ...
>writeAxis(-1.2,1.2,axis=4); ...
>povend();
\end{eulerprompt}
\eulerheading{Implicit Functions}
\begin{eulercomment}
Povray can plot the set where f(x,y,z)=0, just like the implicit parameter in plot3d. The results looks much better,
however.

The syntax for the functions is a bit different. You cannot use the output of Maxima or Euler expressions.

\end{eulercomment}
\begin{eulerformula}
\[
((x^2+y^2-c^2)^2+(z^2-1)^2)*((y^2+z^2-c^2)^2+(x^2-1)^2)*((z^2+x^2-c^2)^2+(y^2-1)^2)=d
\]
\end{eulerformula}
\begin{eulerprompt}
>povstart(angle=70°,height=50°,zoom=4);
>c=0.1; d=0.1; ...
>writeln(povsurface("(pow(pow(x,2)+pow(y,2)-pow(c,2),2)+pow(pow(z,2)-1,2))*(pow(pow(y,2)+pow(z,2)-pow(c,2),2)+pow(pow(x,2)-1,2))*(pow(pow(z,2)+pow(x,2)-pow(c,2),2)+pow(pow(y,2)-1,2))-d",povlook(red))); ...
>povend();
\end{eulerprompt}
\begin{euleroutput}
  Error : Povray error!
  
  Error generated by error() command
  
  povray:
      error("Povray error!");
  Try "trace errors" to inspect local variables after errors.
  povend:
      povray(file,w,h,aspect,exit); 
\end{euleroutput}
\begin{eulerprompt}
>povstart(angle=25°,height=10°); 
>writeln(povsurface("pow(x,2)+pow(y,2)*pow(z,2)-1",povlook(blue),povbox(-2,2,"")));
>povend();
>povstart(angle=70°,height=50°,zoom=4);
\end{eulerprompt}
\begin{eulercomment}
Create the implicit surface. Note the different syntax in the
expression.
\end{eulercomment}
\begin{eulerprompt}
>writeln(povsurface("pow(x,2)*y-pow(y,3)-pow(z,2)",povlook(green))); ...
>writeAxes(); ...
>povend();
\end{eulerprompt}
\eulerheading{Mesh Object}
\begin{eulercomment}
In this example, we show how to create a mesh object, and draw it with additional information.

We like to maximize xy under the condition x+y=1 and demonstrate the tangential touching of the level lines.
\end{eulercomment}
\begin{eulerprompt}
>povstart(angle=-10°,center=[0.5,0.5,0.5],zoom=7);
\end{eulerprompt}
\begin{eulercomment}
We cannot store the object in a string as before, since is too large. So we define the object in a Povray file using
#declare. The function povtriangle() does this automatically. It can accept normal vectors just like pov3d().

The following defines the mesh object, and writes it immediately into the file.
\end{eulercomment}
\begin{eulerprompt}
>x=0:0.02:1; y=x'; z=x*y; vx=-y; vy=-x; vz=1;
>mesh=povtriangles(x,y,z,"",vx,vy,vz);
\end{eulerprompt}
\begin{eulercomment}
Now we define two discs, which will be intersected with the surface.
\end{eulercomment}
\begin{eulerprompt}
>cl=povdisc([0.5,0.5,0],[1,1,0],2); ...
>ll=povdisc([0,0,1/4],[0,0,1],2);
\end{eulerprompt}
\begin{eulercomment}
Write the surface minus the two discs.
\end{eulercomment}
\begin{eulerprompt}
>writeln(povdifference(mesh,povunion([cl,ll]),povlook(green)));
\end{eulerprompt}
\begin{eulercomment}
Write the two intersections.
\end{eulercomment}
\begin{eulerprompt}
>writeln(povintersection([mesh,cl],povlook(red))); ...
>writeln(povintersection([mesh,ll],povlook(gray)));
\end{eulerprompt}
\begin{eulercomment}
Write a point at the maximum.
\end{eulercomment}
\begin{eulerprompt}
>writeln(povpoint([1/2,1/2,1/4],povlook(gray),size=2*defaultpointsize));
\end{eulerprompt}
\begin{eulercomment}
Add axes and finish.
\end{eulercomment}
\begin{eulerprompt}
>writeAxes(0,1,0,1,0,1,d=0.015); ...
>povend();
\end{eulerprompt}
\eulerheading{Anaglyphs in Povray}
\begin{eulercomment}
To generate an anaglyph for a red/cyan glasses, Povray must run twice
from different camera positions. It generates two Povray files and two
PNG files, which are loaded with the function loadanaglyph().

Of course, you need red/cyan glasses to view the following examples
properly.

The function pov3d() has a simple switch to generate anaglyphs.
\end{eulercomment}
\begin{eulerprompt}
>pov3d("-exp(-x^2-y^2)/2",r=2,height=45°,>anaglyph, ...
>  center=[0,0,0.5],zoom=3.5);
\end{eulerprompt}
\begin{eulercomment}
If you create a scene with objects, you need to put the generation of
the scene into a function, and run it twice with different values for
the anaglyph parameter.
\end{eulercomment}
\begin{eulerprompt}
>function myscene ...
\end{eulerprompt}
\begin{eulerudf}
    s=povsphere(povc,1);
    cl=povcylinder(-povz,povz,0.5);
    clx=povobject(cl,rotate=xrotate(90°));
    cly=povobject(cl,rotate=yrotate(90°));
    c=povbox([-1,-1,0],1);
    un=povunion([cl,clx,cly,c]);
    obj=povdifference(s,un,povlook(red));
    writeln(obj);
    writeAxes();
  endfunction
\end{eulerudf}
\begin{eulercomment}
The function povanaglyph() does all this. The parameters are like in
povstart() and povend() combined.
\end{eulercomment}
\begin{eulerprompt}
>povanaglyph("myscene",zoom=4.5);
\end{eulerprompt}
\eulerheading{Defining own Objects}
\begin{eulercomment}
The povray interface of Euler contains a lot of objects. But you are
not restricted to these. You can create own objects, which combine
other objects, or are completely new objects.

We demonstrate a torus. The Povray command for this is "torus". So we
return a string with this command and its parameters. Note that the
torus is always centered at the origin.
\end{eulercomment}
\begin{eulerprompt}
>function povdonat (r1,r2,look="") ...
\end{eulerprompt}
\begin{eulerudf}
    return "torus \{"+r1+","+r2+look+"\}";
  endfunction
\end{eulerudf}
\begin{eulercomment}
Here is our first torus.
\end{eulercomment}
\begin{eulerprompt}
>t1=povdonat(0.8,0.2)
\end{eulerprompt}
\begin{euleroutput}
  torus \{0.8,0.2\}
\end{euleroutput}
\begin{eulercomment}
Let us use this object to create a second torus, translated and
rotated.
\end{eulercomment}
\begin{eulerprompt}
>t2=povobject(t1,rotate=xrotate(90°),translate=[0.8,0,0])
\end{eulerprompt}
\begin{euleroutput}
  object \{ torus \{0.8,0.2\}
   rotate 90 *x 
   translate <0.8,0,0>
   \}
\end{euleroutput}
\begin{eulercomment}
Now we place these objects into a scene. For the look, we use Phong
Shading.
\end{eulercomment}
\begin{eulerprompt}
>povstart(center=[0.4,0,0],angle=0°,zoom=3.8,aspect=1.5); ...
>writeln(povobject(t1,povlook(green,phong=1))); ...
>writeln(povobject(t2,povlook(green,phong=1))); ...
\end{eulerprompt}
\begin{eulerttcomment}
 >povend();
\end{eulerttcomment}
\begin{eulercomment}
calls the Povray program. However, in case of errors, it does not
display the error. You should therefore use

\end{eulercomment}
\begin{eulerttcomment}
 >povend(<exit);
\end{eulerttcomment}
\begin{eulercomment}

if anything did not work. This will leave the Povray window open.
\end{eulercomment}
\begin{eulerprompt}
>povend(h=320,w=480);
\end{eulerprompt}
\begin{euleroutput}
  Function povstart not found.
  Try list ... to find functions!
  Error in:
  povstart(center=[0.4,0,0],angle=0°,zoom=3.8,aspect=1.5); writeln(povobject(t1,povlook(green,phong=1))); writeln(povobj ...
                                                         ^
\end{euleroutput}
\begin{eulercomment}
Here is a more elaborate example. We solve

\end{eulercomment}
\begin{eulerformula}
\[
Ax \le b, \quad x \ge 0, \quad c.x \to \text{Max.}
\]
\end{eulerformula}
\begin{eulercomment}
and show the feasible points and the optimum in a 3D plot.
\end{eulercomment}
\begin{eulerprompt}
>A=[10,8,4;5,6,8;6,3,2;9,5,6];
>b=[10,10,10,10]';
>c=[1,1,1];
\end{eulerprompt}
\begin{eulercomment}
First, let us check, if this example has a solution at all.
\end{eulercomment}
\begin{eulerprompt}
>x=simplex(A,b,c,>max,>check)'
\end{eulerprompt}
\begin{euleroutput}
  [0,  1,  0.5]
\end{euleroutput}
\begin{eulercomment}
Yes, it has.

Next we define two objects. The first is the plane

\end{eulercomment}
\begin{eulerformula}
\[
a \cdot x \le b
\]
\end{eulerformula}
\begin{eulerprompt}
>function oneplane (a,b,look="") ...
\end{eulerprompt}
\begin{eulerudf}
    return povplane(a,b,look)
  endfunction
\end{eulerudf}
\begin{eulercomment}
Then we define the intersection of all half spaces and a cube.
\end{eulercomment}
\begin{eulerprompt}
>function adm (A, b, r, look="") ...
\end{eulerprompt}
\begin{eulerudf}
    ol=[];
    loop 1 to rows(A); ol=ol|oneplane(A[#],b[#]); end;
    ol=ol|povbox([0,0,0],[r,r,r]);
    return povintersection(ol,look);
  endfunction
\end{eulerudf}
\begin{eulercomment}
We can now plot the scene.
\end{eulercomment}
\begin{eulerprompt}
>povstart(angle=120°,center=[0.5,0.5,0.5],zoom=3.5); ...
>writeln(adm(A,b,2,povlook(green,0.4))); ...
>writeAxes(0,1.3,0,1.6,0,1.5); ...
\end{eulerprompt}
\begin{eulercomment}
The following is a circle around the optimum.
\end{eulercomment}
\begin{eulerprompt}
>writeln(povintersection([povsphere(x,0.5),povplane(c,c.x')], ...
>  povlook(red,0.9)));
\end{eulerprompt}
\begin{eulercomment}
And an error in the direction of the optimum.
\end{eulercomment}
\begin{eulerprompt}
>writeln(povarrow(x,c*0.5,povlook(red)));
\end{eulerprompt}
\begin{eulercomment}
We add text to the screen. Text is just a 3D object. We need to place
and turn it according to our view.
\end{eulercomment}
\begin{eulerprompt}
>writeln(povtext("Linear Problem",[0,0.2,1.3],size=0.05,rotate=5°)); ...
>povend();
\end{eulerprompt}
\eulerheading{More Examples}
\begin{eulercomment}
You can find some more examples for Povray in Euler in the following
files.

See: Examples/Dandelin Spheres\\
See: Examples/Donat Math\\
See: Examples/Trefoil Knot\\
See: Examples/Optimization by Affine Scaling
\end{eulercomment}
\end{eulernotebook}
\end{document}


\newpage
\chapter{KB Pekan 6-7: Menggunakan EMT untuk kalkulus}
\input{EMTKalkulus_Rasyid Shalahuddin_22305144016}

\newpage
\chapter{KB Pekan 8: Menggunakan EMT untuk Geometri}
\input{Rasyid Shalahuddin_22305144016_EMT geometri}

\newpage
\chapter{KB Pekan 10; Menggunakan EMT untuk Statistika}
\documentclass[a4paper,10pt]{article}
\usepackage{eumat}

\begin{document}
\begin{eulernotebook}
\eulerheading{EMT untuk Statistika}
\begin{eulercomment}
Nama : Rasyid Shalahuddin\\
NIM  : 22305144016\\
Kelas: Matematika E 2022\\
\end{eulercomment}
\eulersubheading{}
\begin{eulercomment}
Di notebook ini, kami mendemonstrasikan plot statistik utama, tes dan
distribusi di Euler.

Mari kita mulai dengan beberapa statistik deskriptif. Ini bukan
pengantar statistik. Jadi, Anda mungkin memerlukan latar belakang
untuk memahami detailnya.

Asumsikan pengukuran berikut. Kami ingin menghitung nilai rata-rata
dan standar deviasi yang diukur.
\end{eulercomment}
\begin{eulerprompt}
>M=[1005,1030,997,980,1008,1000,978,1004,998,997]; ...
>mean(M), dev(M),
\end{eulerprompt}
\begin{euleroutput}
  999.7
  14.5682302746
\end{euleroutput}
\begin{eulercomment}
Kita dapat memplot plot box-and-whiskers untuk data tersebut. Dalam
kasus kami, tidak ada garis luar.
\end{eulercomment}
\begin{eulerprompt}
>boxplot(M):
\end{eulerprompt}
\eulerimg{29}{images/EMTStatistika_Rasyid Shalahuddin_22305144016-001.png}
\begin{eulercomment}
Kita menghitung probabilitas bahwa suatu nilai lebih besar dari 1005,
dengan asumsi nilai terukur dan distribusi normal.

Semua fungsi untuk distribusi di Euler diakhiri dengan ...dis dan
menghitung distribusi probabilitas kumulatif (CPF).


\end{eulercomment}
\begin{eulerformula}
\[
\text{normaldis(x,m,d)}=\int_{-\infty}^x \frac{1}{d\sqrt{2\pi}}e^{-\frac{1}{2}(\frac{t-m}{d})^2}\ dt.
\]
\end{eulerformula}
\begin{eulercomment}
Kami mencetak hasilnya dalam \% dengan akurasi 2 digit menggunakan
fungsi cetak.
\end{eulercomment}
\begin{eulerprompt}
>print((1-normaldis(1005,mean(M),dev(M)))*100,2,unit=" %")
\end{eulerprompt}
\begin{euleroutput}
       35.80 %
\end{euleroutput}
\begin{eulercomment}
Untuk contoh berikut, kita mengasumsikan jumlah pria berikut dalam
rentang ukuran tertentu.
\end{eulercomment}
\begin{eulerprompt}
>r=155.5:4:187.5; v=[20,70,135,170,138,71,32,8];
\end{eulerprompt}
\begin{eulercomment}
Berikut adalah plot distribusinya.
\end{eulercomment}
\begin{eulerprompt}
>plot2d(r,v,a=150,b=200,c=0,d=190,bar=1,style="\(\backslash\)/"):
\end{eulerprompt}
\eulerimg{29}{images/EMTStatistika_Rasyid Shalahuddin_22305144016-002.png}
\begin{eulercomment}
Kita bisa memasukkan data mentah tersebut ke dalam tabel.

Tabel adalah metode untuk menyimpan data statistik. Tabel kita harus
berisi tiga kolom: Mulai kisaran, akhir kisaran, jumlah laki-laki
dalam kisaran.

Tabel dapat dicetak dengan header. Kami menggunakan vektor string
untuk mengatur header.
\end{eulercomment}
\begin{eulerprompt}
>T:=r[1:8]' | r[2:9]' | v'; writetable(T,labc=["from","to","count"])
\end{eulerprompt}
\begin{euleroutput}
        from        to     count
       155.5     159.5        20
       159.5     163.5        70
       163.5     167.5       135
       167.5     171.5       170
       171.5     175.5       138
       175.5     179.5        71
       179.5     183.5        32
       183.5     187.5         8
\end{euleroutput}
\begin{eulercomment}
Jika kita membutuhkan nilai rata-rata dan statistik ukuran lainnya,
kita perlu menghitung titik tengah rentang. Kita dapat menggunakan dua
kolom pertama dari tabel kita untuk ini.

Simbol "\textbar{}" digunakan untuk memisahkan kolom, fungsi "writetable"
digunakan untuk menulis tabel, dengan pilihan "labc" adalah menentukan
header kolom.
\end{eulercomment}
\begin{eulerprompt}
>(T[,1]+T[,2])/2 // the midpoint of each interval
\end{eulerprompt}
\begin{euleroutput}
          157.5 
          161.5 
          165.5 
          169.5 
          173.5 
          177.5 
          181.5 
          185.5 
\end{euleroutput}
\begin{eulercomment}
Tapi lebih mudah, melipat rentang dengan vektor [1/2, 1/2].
\end{eulercomment}
\begin{eulerprompt}
>M=fold(r,[1,0.5])
\end{eulerprompt}
\begin{euleroutput}
  [235.25,  241.25,  247.25,  253.25,  259.25,  265.25,  271.25,  277.25]
\end{euleroutput}
\begin{eulercomment}
Sekarang kita dapat menghitung mean dan deviasi sampel dengan
frekuensi yang diberikan.
\end{eulercomment}
\begin{eulerprompt}
>\{m,d\}=meandev(M,v); m, d,
\end{eulerprompt}
\begin{euleroutput}
  253.930124224
  8.93123075067
\end{euleroutput}
\begin{eulercomment}
Mari kita tambahkan distribusi normal nilai ke plot batang di atas.
Rumus distribusi normal dengan mean m dan standar deviasi d adalah:

\end{eulercomment}
\begin{eulerformula}
\[
y=\frac{1}{d\sqrt{2\pi}}e^{\frac{-(x-m)^2}{2d^2}}.
\]
\end{eulerformula}
\begin{eulercomment}
Karena nilainya antara 0 dan 1, untuk memplotnya pada diagram batang
harus dikalikan dengan 4 kali jumlah data.
\end{eulercomment}
\begin{eulerprompt}
>plot2d("qnormal(x,m,d)*sum(v)*4", ...
>  xmin=min(r),xmax=max(r),thickness=3,add=1):
\end{eulerprompt}
\eulerimg{29}{images/EMTStatistika_Rasyid Shalahuddin_22305144016-003.png}
\eulerheading{Tabel}
\begin{eulercomment}
Dalam direktori buku catatan ini Anda menemukan file dengan tabel.
Data tersebut merupakan hasil survei. Berikut adalah empat baris
pertama file. Data tersebut berasal dari buku online Jerman
"Einführung in die Statistik mit R" oleh A. Handl.
\end{eulercomment}
\begin{eulerprompt}
>printfile("table.dat",4);
\end{eulerprompt}
\begin{euleroutput}
  Could not open the file
  table.dat
  for reading!
  Try "trace errors" to inspect local variables after errors.
  printfile:
      open(filename,"r");
\end{euleroutput}
\begin{eulercomment}
Tabel berisi 7 kolom angka atau token (string). Kami ingin membaca
tabel dari file. Pertama, kami menggunakan terjemahan kami sendiri
untuk token.

Untuk ini, kami mendefinisikan set token. Fungsi strtokens ()
mendapatkan vektor string token dari string tertentu.
\end{eulercomment}
\begin{eulerprompt}
>mf:=["m","f"]; yn:=["y","n"]; ev:=strtokens("g vg m b vb");
\end{eulerprompt}
\begin{eulercomment}
Sekarang kita membaca tabel dengan terjemahan ini.

Argumen tok2, tok4, dll. Adalah terjemahan dari kolom tabel. Argumen
ini tidak ada dalam daftar parameter readtable(), jadi Anda perlu
memberinya ":=".
\end{eulercomment}
\begin{eulerprompt}
>\{MT,hd\}=readtable("table.dat",tok2:=mf,tok4:=yn,tok5:=ev,tok7:=yn);
\end{eulerprompt}
\begin{euleroutput}
  Could not open the file
  table.dat
  for reading!
  Try "trace errors" to inspect local variables after errors.
  readtable:
      if filename!=none then open(filename,"r"); endif;
\end{euleroutput}
\begin{eulerprompt}
>load over statistics;
\end{eulerprompt}
\begin{eulercomment}
Untuk mencetak, kita perlu menentukan set token yang sama. Kita
mencetak empat baris pertama saja.
\end{eulercomment}
\begin{eulerprompt}
>writetable(MT[1:4],labc=hd,wc=5,tok2:=mf,tok4:=yn,tok5:=ev,tok7:=yn);
\end{eulerprompt}
\begin{euleroutput}
  MT is not a variable!
  Error in:
  writetable(MT[1:4],labc=hd,wc=5,tok2:=mf,tok4:=yn,tok5:=ev,tok ...
                    ^
\end{euleroutput}
\begin{eulercomment}
Titik "." mewakili nilai-nilai yang tidak tersedia.

Jika kita tidak ingin menentukan token untuk terjemahan terlebih
dahulu, kita hanya perlu menentukan, kolom mana yang berisi token dan
bukan angka.
\end{eulercomment}
\begin{eulerprompt}
>ctok=[2,4,5,7]; \{MT,hd,tok\}=readtable("table.dat",ctok=ctok);
\end{eulerprompt}
\begin{euleroutput}
  Could not open the file
  table.dat
  for reading!
  Try "trace errors" to inspect local variables after errors.
  readtable:
      if filename!=none then open(filename,"r"); endif;
\end{euleroutput}
\begin{eulercomment}
Fungsi readtable() sekarang mengembalikan satu set token.
\end{eulercomment}
\begin{eulerprompt}
>tok
\end{eulerprompt}
\begin{euleroutput}
  Variable tok not found!
  Error in:
  tok ...
     ^
\end{euleroutput}
\begin{eulercomment}
Tabel berisi entri dari file dengan token yang diterjemahkan menjadi
angka.

String khusus NA="." diartikan sebagai "Tidak Tersedia", dan
mendapatkan NAN (bukan angka) di tabel. Terjemahan ini dapat diubah
dengan parameter NA, dan NAval.
\end{eulercomment}
\begin{eulerprompt}
>MT[1]
\end{eulerprompt}
\begin{euleroutput}
  MT is not a variable!
  Error in:
  MT[1] ...
       ^
\end{euleroutput}
\begin{eulercomment}
Berikut adalah isi tabel dengan bilangan yang belum diterjemahkan.
\end{eulercomment}
\begin{eulerprompt}
>writetable(MT,wc=5)
\end{eulerprompt}
\begin{euleroutput}
  Variable or function MT not found.
  Error in:
  writetable(MT,wc=5) ...
               ^
\end{euleroutput}
\begin{eulercomment}
Untuk kenyamanan, Anda bisa memasukkan keluaran readtable() ke dalam
daftar.
\end{eulercomment}
\begin{eulerprompt}
>Table=\{\{readtable("table.dat",ctok=ctok)\}\};
\end{eulerprompt}
\begin{euleroutput}
  Could not open the file
  table.dat
  for reading!
  Try "trace errors" to inspect local variables after errors.
  readtable:
      if filename!=none then open(filename,"r"); endif;
\end{euleroutput}
\begin{eulercomment}
Dengan menggunakan kolom token yang sama dan token dibaca dari file,
kita dapat mencetak tabel. Kita dapat menentukan ctok, tok, dll. Atau
menggunakan Tabel daftar.
\end{eulercomment}
\begin{eulerprompt}
>writetable(Table,ctok=ctok,wc=5);
\end{eulerprompt}
\begin{euleroutput}
  Variable or function Table not found.
  Error in:
  writetable(Table,ctok=ctok,wc=5); ...
                  ^
\end{euleroutput}
\begin{eulercomment}
Fungsi tablecol() mengembalikan nilai kolom tabel, melewatkan baris
apa pun dengan nilai NAN ("." Dalam file), dan indeks kolom, yang
berisi nilai ini.
\end{eulercomment}
\begin{eulerprompt}
>\{c,i\}=tablecol(MT,[5,6]);
\end{eulerprompt}
\begin{euleroutput}
  Variable or function MT not found.
  Error in:
  \{c,i\}=tablecol(MT,[5,6]); ...
                   ^
\end{euleroutput}
\begin{eulercomment}
Kita dapat menggunakan ini untuk mengekstrak kolom dari tabel untuk
tabel baru
\end{eulercomment}
\begin{eulerprompt}
>j=[1,5,6]; writetable(MT[i,j],labc=hd[j],ctok=[2],tok=tok)
\end{eulerprompt}
\begin{euleroutput}
  Variable or function i not found.
  Error in:
  j=[1,5,6]; writetable(MT[i,j],labc=hd[j],ctok=[2],tok=tok) ...
                            ^
\end{euleroutput}
\begin{eulercomment}
Tentu saja, kita perlu mengekstrak tabel itu sendiri dari Daftar Tabel
dalam kasus ini.
\end{eulercomment}
\begin{eulerprompt}
>MT=Table[1];
\end{eulerprompt}
\begin{eulercomment}
Tentu saja, kami juga dapat menggunakannya untuk menentukan nilai
rata-rata kolom atau nilai statistik lainnya.
\end{eulercomment}
\begin{eulerprompt}
>mean(tablecol(MT,6))
\end{eulerprompt}
\begin{euleroutput}
  2.175
\end{euleroutput}
\begin{eulercomment}
Fungsi getstatistics() mengembalikan elemen dalam vektor, dan
jumlahnya. Kita menerapkannya ke nilai "m" dan "f" di kolom kedua
tabel kami.
\end{eulercomment}
\begin{eulerprompt}
>\{xu,count\}=getstatistics(tablecol(MT,2)); xu, count,
\end{eulerprompt}
\begin{euleroutput}
  [1,  3]
  [12,  13]
\end{euleroutput}
\begin{eulercomment}
Kita dapat mencetak hasilnya di tabel baru.
\end{eulercomment}
\begin{eulerprompt}
>writetable(count',labr=tok[xu])
\end{eulerprompt}
\begin{euleroutput}
           m        12
           f        13
\end{euleroutput}
\begin{eulercomment}
Fungsi selecttable() mengembalikan tabel baru dengan nilai dalam satu
kolom yang dipilih dari vektor indeks. Pertama kita mencari indeks
dari dua nilai kita di tabel token.
\end{eulercomment}
\begin{eulerprompt}
>v:=indexof(tok,["g","vg"])
\end{eulerprompt}
\begin{euleroutput}
  [5,  6]
\end{euleroutput}
\begin{eulercomment}
Sekarang kita dapat memilih baris tabel, yang memiliki salah satu
nilai dalam v di baris ke-5
\end{eulercomment}
\begin{eulerprompt}
>MT1:=MT[selectrows(MT,5,v)]; i:=sortedrows(MT1,5);
\end{eulerprompt}
\begin{eulercomment}
Sekarang kita dapat mencetak tabel, dengan nilai yang diekstrasi dan
diurutkan di kolom-5.
\end{eulercomment}
\begin{eulerprompt}
>writetable(MT1[i],labc=hd,ctok=ctok,tok=tok,wc=7);
\end{eulerprompt}
\begin{euleroutput}
   Person    Sex    Age Titanic Evaluation    Tip Problem
        2      f     23       y          g    1.8       n
        3      f     26       y          g    1.8       y
        6      m     28       y          g    2.8       y
       18      m     38       y          g      .       n
       16      m     26       y          g    2.8       n
       15      f     31       y          g    0.8       n
       12      m     32       y          g    1.8       n
       23      f     38       y          g    2.8       n
       14      f     25       y          g    1.8       y
        9      f     24       y         vg    1.8       y
        7      f     31       y         vg    2.8       n
       20      f     28       y         vg    1.8       n
       22      f     28       y         vg    1.8       y
       13      m     29       y         vg    1.8       y
       11      f     23       y         vg    1.8       y
\end{euleroutput}
\begin{eulercomment}
Untuk statistik berikutnya, kami ingin menghubungkan dua kolom dari
tabel. Jadi kami mengekstrak kolom 2 dan 4 dan mengurutkan tabel.
\end{eulercomment}
\begin{eulerprompt}
>i=sortedrows(,[2,4]);  ...
>  writetable(tablecol(MT[i],[2,4])',ctok=[1,2],tok=tok)
\end{eulerprompt}
\begin{euleroutput}
  Variable  not found!
  Error in:
  i=sortedrows(,[2,4]);    writetable(tablecol(MT[i],[2,4])',cto ...
                      ^
\end{euleroutput}
\begin{eulercomment}
Dengan getstatistics(), kita juga bisa menghubungkan hitungan dalam
dua kolom tabel satu sama lain.
\end{eulercomment}
\begin{eulerprompt}
>MT24=tablecol(MT,[2,4]); ...
>\{xu1,xu2,count\}=getstatistics(MT24[1],MT24[2]); ...
>writetable(count,labr=tok[xu1],labc=tok[xu2])
\end{eulerprompt}
\begin{euleroutput}
                     n         y
           m         7         5
           f         1        12
\end{euleroutput}
\begin{eulercomment}
Tabel dapat ditulis ke file.
\end{eulercomment}
\begin{eulerprompt}
>filename="test.dat"; ...
>writetable(count,labr=tok[xu1],labc=tok[xu2],file=filename);
\end{eulerprompt}
\begin{eulercomment}
Kemudian kita dapat membaca tabel dari file tersebut.
\end{eulercomment}
\begin{eulerprompt}
>\{MT2,hd,tok2,hdr\}=readtable(filename,>clabs,>rlabs); ...
>writetable(MT2,labr=hdr,labc=hd)
\end{eulerprompt}
\begin{euleroutput}
                     n         y
           m         7         5
           f         1        12
\end{euleroutput}
\begin{eulercomment}
Dan hapus file tersebut.
\end{eulercomment}
\begin{eulerprompt}
>fileremove(filename);
\end{eulerprompt}
\eulerheading{Distribusi}
\begin{eulercomment}
Dengan plot2d, terdapat metode yang sangat mudah untuk memplot sebaran
data eksperimen.
\end{eulercomment}
\begin{eulerprompt}
>p=normal(1,1000); //1000 random normal-distributed sample p
>plot2d(p,distribution=20,style="\(\backslash\)/"); // plot the random sample p
>plot2d("qnormal(x,0,1)",add=1): // add the standard normal distribution plot
\end{eulerprompt}
\eulerimg{29}{images/EMTStatistika_Rasyid Shalahuddin_22305144016-004.png}
\begin{eulercomment}
Harap perhatikan perbedaan antara plot batang (sampel) dan kurva
normal(distribusi nyata). Masukkan kembali tiga perintah untuk melihat
hasil pengambilan sampel lainnya.
\end{eulercomment}
\begin{eulercomment}
Berikut adalah perbandingan 10 simulasi dari 1000 nilai terdistribusi
normal menggunakan apa yang disebut box plot. Plot ini menunjukkan
median, kuartil 25\% dan 75\%, nilai minimal dan maksimal, dan outlier.
\end{eulercomment}
\begin{eulerprompt}
>p=normal(100,1000); boxplot(p):
\end{eulerprompt}
\eulerimg{29}{images/EMTStatistika_Rasyid Shalahuddin_22305144016-005.png}
\begin{eulercomment}
Untuk menghasilkan bilangan bulat acak, Euler memiliki intrandom. Mari
kita simulasikan lemparan dadu dan plot distribusinya.

Kami menggunakan fungsi getmultiplicities v, x), yang menghitung
seberapa sering elemen v muncul di x. Kemudian kita plot hasilnya
menggunakan columnplot().
\end{eulercomment}
\begin{eulerprompt}
>k=intrandom(1,6000,6);  ...
>columnsplot(getmultiplicities(1:6,k));  ...
>ygrid(1000,color=red):
\end{eulerprompt}
\eulerimg{29}{images/EMTStatistika_Rasyid Shalahuddin_22305144016-006.png}
\begin{eulercomment}
Sementara intrandom (n, m, k) mengembalikan bilangan bulat
terdistribusi seragam dari 1 ke k, dimungkinkan untuk menggunakan
distribusi bilangan bulat lain yang diberikan dengan randpint ().

Dalam contoh berikut, probabilitas 1,2,3 masing-masing adalah
0,4,0.1,0.5.
\end{eulercomment}
\begin{eulerprompt}
>randpint(1,1000,[0.4,0.1,0.5]); getmultiplicities(1:3,%)
\end{eulerprompt}
\begin{euleroutput}
  [378,  102,  520]
\end{euleroutput}
\begin{eulercomment}
Euler dapat menghasilkan nilai acak dari lebih banyak distribusi.
Simak referensinya.

Misalnya, kami mencoba distribusi eksponensial. Variabel acak kontinu
X dikatakan memiliki distribusi eksponensial, jika PDF-nya diberikan
oleh\\
\end{eulercomment}
\begin{eulerformula}
\[
f_X(x)=\lambda e^{-\lambda x},\quad x>0,\quad \lambda>0,
\]
\end{eulerformula}
\begin{eulercomment}
with parameter\\
\end{eulercomment}
\begin{eulerformula}
\[
\lambda=\frac{1}{\mu},\quad \mu \text{ is the mean, and denoted by } X \sim \text{Exponential}(\lambda).
\]
\end{eulerformula}
\begin{eulerprompt}
>plot2d(randexponential(1,1000,2),>distribution):
\end{eulerprompt}
\eulerimg{29}{images/EMTStatistika_Rasyid Shalahuddin_22305144016-007.png}
\begin{eulercomment}
Untuk banyak distribusi, Euler dapat menghitung fungsi distribusi dan
inversnya.
\end{eulercomment}
\begin{eulerprompt}
>plot2d("normaldis",-4,4): 
\end{eulerprompt}
\eulerimg{29}{images/EMTStatistika_Rasyid Shalahuddin_22305144016-008.png}
\begin{eulercomment}
Berikut ini adalah salah satu cara untuk memplot sebuah kuantil.
\end{eulercomment}
\begin{eulerprompt}
>plot2d("qnormal(x,1,1.5)",-4,6);  ...
>plot2d("qnormal(x,1,1.5)",a=2,b=5,>add,>filled):
\end{eulerprompt}
\eulerimg{29}{images/EMTStatistika_Rasyid Shalahuddin_22305144016-009.png}
\begin{eulerformula}
\[
\text{normaldis(x,m,d)}=\int_{-\infty}^x \frac{1}{d\sqrt{2\pi}}e^{-\frac{1}{2}(\frac{t-m}{d})^2}\ dt.
\]
\end{eulerformula}
\begin{eulercomment}
Kemungkinan berada di area hijau adalah sebagai berikut.
\end{eulercomment}
\begin{eulerprompt}
>normaldis(5,1,1.5)-normaldis(2,1,1.5)
\end{eulerprompt}
\begin{euleroutput}
  0.248662156979
\end{euleroutput}
\begin{eulercomment}
Ini dapat dihitung secara numerik dengan integral berikut.\\
\end{eulercomment}
\begin{eulerformula}
\[
\int_2^5 \frac{1}{1.5\sqrt{2\pi}}e^{-\frac{1}{2}(\frac{x-1}{1.5})^2}\ dx.
\]
\end{eulerformula}
\begin{eulerprompt}
>gauss("qnormal(x,1,1.5)",2,5)
\end{eulerprompt}
\begin{euleroutput}
  0.248662156979
\end{euleroutput}
\begin{eulercomment}
Mari kita bandingkan distribusi binomial dengan distribusi normal dari
mean dan deviasi yang sama. Fungsi invbindis () memecahkan interpolasi
linier antara nilai integer.
\end{eulercomment}
\begin{eulerprompt}
>invbindis(0.95,1000,0.5), invnormaldis(0.95,500,0.5*sqrt(1000))
\end{eulerprompt}
\begin{euleroutput}
  525.516721219
  526.007419394
\end{euleroutput}
\begin{eulercomment}
Fungsi qdis () adalah kepadatan dari distribusi chi-kuadrat. Seperti
biasa, Euler memetakan vektor ke fungsi ini. Jadi kita mendapatkan
plot dari semua distribusi chi-kuadrat dengan derajat 5 sampai 30
dengan mudah dengan cara berikut.
\end{eulercomment}
\begin{eulerprompt}
>plot2d("qchidis(x,(5:5:50)')",0,50):
\end{eulerprompt}
\eulerimg{29}{images/EMTStatistika_Rasyid Shalahuddin_22305144016-010.png}
\begin{eulercomment}
Euler memiliki fungsi yang akurat untuk mengevaluasi distribusi. Mari
kita periksa chidis () dengan integral.

Penamaan mencoba untuk konsisten. Misalnya.,

- distribusi chi-kuadrat adalah chidis (),\\
- fungsi kebalikannya adalah invchidis (),\\
- kepadatannya adalah qchidis ().

Pelengkap distribusi (ekor atas) adalah chicdis ().
\end{eulercomment}
\begin{eulerprompt}
>chidis(1.5,2), integrate("qchidis(x,2)",0,1.5)
\end{eulerprompt}
\begin{euleroutput}
  0.527633447259
  0.527633447259
\end{euleroutput}
\eulerheading{Distribusi Diskrit}
\begin{eulercomment}
Untuk menentukan distribusi diskrit Anda sendiri, Anda dapat
menggunakan metode berikut.

Pertama kita mengatur fungsi distribusi.
\end{eulercomment}
\begin{eulerprompt}
>wd = 0|((1:6)+[-0.01,0.01,0,0,0,0])/6
\end{eulerprompt}
\begin{euleroutput}
  [0,  0.165,  0.335,  0.5,  0.666667,  0.833333,  1]
\end{euleroutput}
\begin{eulercomment}
Artinya dengan probabilitas wd[i+1]-wd[i] kita menghasilkan nilai acak
i.

Ini hampir merupakan distribusi yang seragam. Mari kita tentukan
generator nomor acak untuk ini. Fungsi find (v, x) menemukan nilai x
pada vektor v. Fungsi ini juga berlaku untuk vektor x.
\end{eulercomment}
\begin{eulerprompt}
>function wrongdice (n,m) := find(wd,random(n,m))
\end{eulerprompt}
\begin{eulercomment}
Kesalahannya begitu halus sehingga kita hanya melihatnya dengan sangat
banyak iterasi.
\end{eulercomment}
\begin{eulerprompt}
>columnsplot(getmultiplicities(1:6,wrongdice(1,1000000))):
\end{eulerprompt}
\eulerimg{29}{images/EMTStatistika_Rasyid Shalahuddin_22305144016-011.png}
\begin{eulercomment}
Berikut adalah fungsi sederhana untuk memeriksa distribusi seragam
nilai 1 ... K dalam v. Kami menerima hasilnya, jika untuk semua
frekuensi

\end{eulercomment}
\begin{eulerformula}
\[
\left|f_i-\frac{1}{K}\right| < \frac{\delta}{\sqrt{n}}.
\]
\end{eulerformula}
\begin{eulerprompt}
>function checkrandom (v, delta=1) ...
\end{eulerprompt}
\begin{eulerudf}
    K=max(v); n=cols(v);
    fr=getfrequencies(v,1:K);
    return max(fr/n-1/K)<delta/sqrt(n);
    endfunction
\end{eulerudf}
\begin{eulercomment}
Memang fungsinya menolak distribusi seragam.
\end{eulercomment}
\begin{eulerprompt}
>checkrandom(wrongdice(1,1000000))
\end{eulerprompt}
\begin{euleroutput}
  0
\end{euleroutput}
\begin{eulercomment}
Dan itu menerima generator acak bawaan.
\end{eulercomment}
\begin{eulerprompt}
>checkrandom(intrandom(1,1000000,6))
\end{eulerprompt}
\begin{euleroutput}
  1
\end{euleroutput}
\begin{eulercomment}
Kami dapat menghitung distribusi binomial. Pertama ada binomialsum (),
yang mengembalikan probabilitas i atau kurang dari n percobaan.
\end{eulercomment}
\begin{eulerprompt}
>bindis(410,1000,0.4)
\end{eulerprompt}
\begin{euleroutput}
  0.751401349654
\end{euleroutput}
\begin{eulercomment}
Fungsi Beta terbalik digunakan untuk menghitung interval kepercayaan
Clopper-Pearson untuk parameter p. Tingkat defaultnya adalah alfa.

Arti dari interval ini adalah jika p berada di luar interval maka
hasil observasi 410 dalam 1000 jarang terjadi.
\end{eulercomment}
\begin{eulerprompt}
>clopperpearson(400,1000)
\end{eulerprompt}
\begin{euleroutput}
  [0.369469,  0.431122]
\end{euleroutput}
\begin{eulercomment}
Perintah berikut adalah cara langsung untuk mendapatkan hasil di atas.
Namun untuk n besar, penjumlahan langsung tidak akurat dan lambat.
\end{eulercomment}
\begin{eulerprompt}
>p=0.4; i=0:410; n=1000; sum(bin(n,i)*p^i*(1-p)^(n-i))
\end{eulerprompt}
\begin{euleroutput}
  0.751401349655
\end{euleroutput}
\begin{eulercomment}
Omong-omong, invbinsum() menghitung kebalikan dari binomialsum().
\end{eulercomment}
\begin{eulerprompt}
>invbindis(0.5,100,0.4)
\end{eulerprompt}
\begin{euleroutput}
  39.4669423584
\end{euleroutput}
\begin{eulercomment}
Di Bridge, kami mengasumsikan 5 kartu beredar (dari 52) di dua tangan
(26 kartu). Mari kita hitung probabilitas distribusi yang lebih buruk
dari 3:2 (misalnya 0:5, 1:4, 4:1 atau 5:0).
\end{eulercomment}
\begin{eulerprompt}
>2*hypergeomsum(1,5,13,26)
\end{eulerprompt}
\begin{euleroutput}
  0.321739130435
\end{euleroutput}
\begin{eulercomment}
Ada juga simulasi distribusi multinomial.
\end{eulercomment}
\begin{eulerprompt}
>randmultinomial(10,1000,[0.4,0.1,0.5])
\end{eulerprompt}
\begin{euleroutput}
            433            98           469 
            396           102           502 
            359           108           533 
            394           107           499 
            388           101           511 
            414           100           486 
            391            76           533 
            405           106           489 
            394           103           503 
            396           106           498 
\end{euleroutput}
\eulerheading{Merencanakan Data}
\begin{eulercomment}
Untuk memplot data, kita coba hasil pemilu Jerman sejak 1990, diukur
dalam kursi.
\end{eulercomment}
\begin{eulerprompt}
>BW := [ ...
>1990,662,319,239,79,8,17; ...
>1994,672,294,252,47,49,30; ...
>1998,669,245,298,43,47,36; ...
>2002,603,248,251,47,55,2; ...
>2005,614,226,222,61,51,54; ...
>2009,622,239,146,93,68,76; ...
>2013,631,311,193,0,63,64];
\end{eulerprompt}
\begin{eulercomment}
Untuk pesta, kita menggunakan serangkain nama.
\end{eulercomment}
\begin{eulerprompt}
>P:=["CDU/CSU","SPD","FDP","Gr","Li"];
\end{eulerprompt}
\begin{eulercomment}
Mari kita cetak persentase dengan baik.

Pertama kami mengekstrak kolom yang diperlukan. Kolom 3 sd 7 adalah
kursi masing-masing partai, dan kolom 2 adalah jumlah kursi. kolom
adalah tahun pemilihan.
\end{eulercomment}
\begin{eulerprompt}
>BT:=BW[,3:7]; BT:=BT/sum(BT); YT:=BW[,1]';
\end{eulerprompt}
\begin{eulercomment}
Kemudian kami mencetak statistik dalam bentuk tabel. Kami menggunakan
nama sebagai tajuk kolom, dan tahun sebagai tajuk untuk baris. Lebar
default untuk kolom adalah wc=10, tetapi kami lebih memilih keluaran
yang lebih padat. Kolom akan diperluas untuk label kolom, jika perlu.
\end{eulercomment}
\begin{eulerprompt}
>writetable(BT*100,wc=6,dc=0,>fixed,labc=P,labr=YT)
\end{eulerprompt}
\begin{euleroutput}
         CDU/CSU   SPD   FDP    Gr    Li
    1990      48    36    12     1     3
    1994      44    38     7     7     4
    1998      37    45     6     7     5
    2002      41    42     8     9     0
    2005      37    36    10     8     9
    2009      38    23    15    11    12
    2013      49    31     0    10    10
\end{euleroutput}
\begin{eulercomment}
Perkalian matriks berikut mengekstrak jumlah persentase dari dua
partai besar yang menunjukkan bahwa partai kecil telah mendapatkan
footage di parlemen hingga tahun 2009.
\end{eulercomment}
\begin{eulerprompt}
>BT1:=(BT.[1;1;0;0;0])'*100
\end{eulerprompt}
\begin{euleroutput}
  [84.29,  81.25,  81.1659,  82.7529,  72.9642,  61.8971,  79.8732]
\end{euleroutput}
\begin{eulercomment}
Ada juga plot statistik sederhana. Kami menggunakannya untuk
menampilkan garis dan titik secara bersamaan. Alternatifnya adalah
memanggil plot2d dua kali dengan\textgreater{} add.
\end{eulercomment}
\begin{eulerprompt}
>statplot(YT,BT1,"b"):
\end{eulerprompt}
\eulerimg{29}{images/EMTStatistika_Rasyid Shalahuddin_22305144016-012.png}
\begin{eulercomment}
Tentukan beberapa warna untuk setiap pesta.
\end{eulercomment}
\begin{eulerprompt}
>CP:=[rgb(0.5,0.5,0.5),red,yellow,green,rgb(0.8,0,0)];
\end{eulerprompt}
\begin{eulercomment}
Sekarang kita bisa memplot hasil Pemilu 2009 dan perubahannya menjadi
satu plot menggunakan gambar. Kita dapat menambahkan vektor kolom ke
setiap plot.
\end{eulercomment}
\begin{eulerprompt}
>figure(2,1);  ...
>figure(1); columnsplot(BW[6,3:7],P,color=CP); ...
>figure(2); columnsplot(BW[6,3:7]-BW[5,3:7],P,color=CP);  ...
>figure(0):
\end{eulerprompt}
\eulerimg{29}{images/EMTStatistika_Rasyid Shalahuddin_22305144016-013.png}
\begin{eulercomment}
Plot data menggabungkan deretan data statistik dalam satu plot.
\end{eulercomment}
\begin{eulerprompt}
>J:=BW[,1]'; DP:=BW[,3:7]'; ...
>dataplot(YT,BT',color=CP);  ...
>labelbox(P,colors=CP,styles="[]",>points,w=0.2,x=0.3,y=0.4):
\end{eulerprompt}
\eulerimg{29}{images/EMTStatistika_Rasyid Shalahuddin_22305144016-014.png}
\begin{eulercomment}
Plot kolom 3D menampilkan baris data statistik dalam bentuk kolom.
Kami memberikan label untuk baris dan kolom. sudut adalah sudut
pandang.
\end{eulercomment}
\begin{eulerprompt}
>columnsplot3d(BT,scols=P,srows=YT, ...
>  angle=30°,ccols=CP):
\end{eulerprompt}
\eulerimg{29}{images/EMTStatistika_Rasyid Shalahuddin_22305144016-015.png}
\begin{eulercomment}
Representasi lainnya adalah plot mosaik. Perhatikan bahwa kolom plot
mewakili kolom matriks di sini. Karena panjangnya label CDU / CSU,
kami mengambil jendela yang lebih kecil dari biasanya.
\end{eulercomment}
\begin{eulerprompt}
>shrinkwindow(>smaller);  ...
>mosaicplot(BT',srows=YT,scols=P,color=CP,style="#"); ...
>shrinkwindow():
\end{eulerprompt}
\eulerimg{29}{images/EMTStatistika_Rasyid Shalahuddin_22305144016-016.png}
\begin{eulercomment}
Kami juga bisa membuat diagram pie. Karena hitam dan kuning membentuk
koalisi, kami menyusun ulang elemen-elemennya.
\end{eulercomment}
\begin{eulerprompt}
>i=[1,3,5,4,2]; piechart(BW[6,3:7][i],color=CP[i],lab=P[i]):
\end{eulerprompt}
\begin{eulercomment}
Berikut ini jenis plot lainnya.
\end{eulercomment}
\begin{eulerprompt}
>starplot(normal(1,10)+4,lab=1:10,>rays):
\end{eulerprompt}
\begin{eulercomment}
Beberapa plot di plot2d bagus untuk statika. Berikut adalah plot
impuls data acak, didistribusikan secara seragam di [0,1].
\end{eulercomment}
\begin{eulerprompt}
>plot2d(makeimpulse(1:10,random(1,10)),>bar):
\end{eulerprompt}
\begin{eulercomment}
Tetapi untuk data yang terdistribusi secara eksponensial, kita mungkin
memerlukan plot logaritmik.
\end{eulercomment}
\begin{eulerprompt}
>logimpulseplot(1:10,-log(random(1,10))*10):
\end{eulerprompt}
\begin{eulercomment}
Fungsi columnplot() lebih mudah digunakan, karena hanya membutuhkan
vektor nilai. Selain itu, ia dapat mengatur labelnya menjadi apa pun
yang kami inginkan, kami telah menunjukkannya di tutorial ini.

Berikut adalah aplikasi lain, di mana kita menghitung karakter dalam
sebuah kalimat dan membuat plot statistik.
\end{eulercomment}
\begin{eulerprompt}
>v=strtochar("the quick brown fox jumps over the lazy dog"); ...
>w=ascii("a"):ascii("z"); x=getmultiplicities(w,v); ...
>cw=[]; for k=w; cw=cw|char(k); end; ...
>columnsplot(x,lab=cw,width=0.05):
\end{eulerprompt}
\begin{eulercomment}
Sumbu juga dapat diatur secara manual.
\end{eulercomment}
\begin{eulerprompt}
>n=10; p=0.4; i=0:n; x=bin(n,i)*p^i*(1-p)^(n-i); ...
>columnsplot(x,lab=i,width=0.05,<frame,<grid); ...
>yaxis(0,0:0.1:1,style="->",>left); xaxis(0,style="."); ...
>label("p",0,0.25), label("i",11,0); ...
>textbox(["Binomial distribution","with p=0.4"]):
\end{eulerprompt}
\begin{eulercomment}
Berikut ini adalah cara untuk memplot frekuensi bilangan dalam sebuah
vektor.

Kami membuat vektor bilangan bulat bilangan acak 1 hingga 6.
\end{eulercomment}
\begin{eulerprompt}
>v:=intrandom(1,10,10)
\end{eulerprompt}
\begin{euleroutput}
  [8,  5,  8,  8,  6,  8,  8,  3,  5,  5]
\end{euleroutput}
\begin{eulercomment}
Kemudian ekstrak nomor unik di v.
\end{eulercomment}
\begin{eulerprompt}
>vu:=unique(v)
\end{eulerprompt}
\begin{euleroutput}
  [3,  5,  6,  8]
\end{euleroutput}
\begin{eulercomment}
Dan plot frekuensi dalam plot kolom.
\end{eulercomment}
\begin{eulerprompt}
>columnsplot(getmultiplicities(vu,v),lab=vu,style="/"):
\end{eulerprompt}
\begin{eulercomment}
Kami ingin mendemonstrasikan fungsi untuk distribusi nilai empiris.
\end{eulercomment}
\begin{eulerprompt}
>x=normal(1,20);
\end{eulerprompt}
\begin{eulercomment}
Fungsi empdist (x, vs) membutuhkan array nilai yang diurutkan. Jadi
kita harus mengurutkan x sebelum dapat menggunakannya.
\end{eulercomment}
\begin{eulerprompt}
>xs=sort(x);
\end{eulerprompt}
\begin{eulercomment}
Kemudian kami memplot distribusi empiris dan beberapa batang kepadatan
ke dalam satu plot. Alih-alih plot batang untuk distribusi kami
menggunakan plot gigi gergaji kali ini.
\end{eulercomment}
\begin{eulerprompt}
>figure(2,1); ...
>figure(1); plot2d("empdist",-4,4;xs); ...
>figure(2); plot2d(histo(x,v=-4:0.2:4,<bar));  ...
>figure(0):
\end{eulerprompt}
\begin{eulercomment}
Plot pencar mudah dilakukan di Euler dengan plot titik biasa. Grafik
berikut menunjukkan bahwa X dan X+Y berkorelasi positif dengan jelas.
\end{eulercomment}
\begin{eulerprompt}
>x=normal(1,100); plot2d(x,x+rotright(x),>points,style=".."):
\end{eulerprompt}
\begin{eulercomment}
Seringkali, kami ingin membandingkan dua sampel dari distribusi yang
berbeda. Ini dapat dilakukan dengan plot-kuantil-kuantil.

Untuk pengujian, kami mencoba distribusi t siswa dan distribusi
eksponensial.
\end{eulercomment}
\begin{eulerprompt}
>x=randt(1,1000,5); y=randnormal(1,1000,mean(x),dev(x)); ...
>plot2d("x",r=6,style="--",yl="normal",xl="student-t",>vertical); ...
>plot2d(sort(x),sort(y),>points,color=red,style="x",>add):
\end{eulerprompt}
\begin{eulercomment}
Plot dengan jelas menunjukkan bahwa nilai terdistribusi normal
cenderung lebih kecil di ujung yang ekstrim.

Jika kita memiliki dua distribusi dengan ukuran berbeda, kita dapat
memperbesar yang lebih kecil atau mengecilkan yang lebih besar. Fungsi
berikut bagus untuk keduanya. Ini mengambil nilai median dengan
persentase antara 0 dan 1.
\end{eulercomment}
\begin{eulerprompt}
>function medianexpand (x,n) := median(x,p=linspace(0,1,n-1));
\end{eulerprompt}
\begin{eulercomment}
Mari kita bandingkan dua distribusi yang sama.
\end{eulercomment}
\begin{eulerprompt}
>x=random(1000); y=random(400); ...
>plot2d("x",0,1,style="--"); ...
>plot2d(sort(medianexpand(x,400)),sort(y),>points,color=red,style="x",>add):
\end{eulerprompt}
\eulerheading{Regresi dan Korelasi}
\begin{eulercomment}
Regresi linier dapat dilakukan dengan fungsi polyfit () atau berbagai
fungsi fit.

Sebagai permulaan kita menemukan garis regresi untuk data univariat
dengan polyfit(x, y, 1).
\end{eulercomment}
\begin{eulerprompt}
>x=1:10; y=[2,3,1,5,6,3,7,8,9,8]; writetable(x'|y',labc=["x","y"])
\end{eulerprompt}
\begin{euleroutput}
           x         y
           1         2
           2         3
           3         1
           4         5
           5         6
           6         3
           7         7
           8         8
           9         9
          10         8
\end{euleroutput}
\begin{eulercomment}
Kami ingin membandingkan ukuran yang tidak berbobot dan berbobot.
Pertama koefisien kesesuaian linier.
\end{eulercomment}
\begin{eulerprompt}
>p=polyfit(x,y,1)
\end{eulerprompt}
\begin{euleroutput}
  [0.733333,  0.812121]
\end{euleroutput}
\begin{eulercomment}
Sekarang koefisien dengan bobot yang menekankan nilai terakhir.
\end{eulercomment}
\begin{eulerprompt}
>w &= "exp(-(x-10)^2/10)"; pw=polyfit(x,y,1,w=w(x))
\end{eulerprompt}
\begin{euleroutput}
  [4.71566,  0.38319]
\end{euleroutput}
\begin{eulercomment}
Kami menempatkan semuanya ke dalam satu plot untuk titik dan garis
regresi, dan untuk bobot yang digunakan.
\end{eulercomment}
\begin{eulerprompt}
>figure(2,1);  ...
>figure(1); statplot(x,y,"b",xl="Regression"); ...
>  plot2d("evalpoly(x,p)",>add,color=blue,style="--"); ...
>  plot2d("evalpoly(x,pw)",5,10,>add,color=red,style="--"); ...
>figure(2); plot2d(w,1,10,>filled,style="/",fillcolor=red,xl=w); ...
>figure(0):
\end{eulerprompt}
\begin{eulercomment}
Untuk contoh lain kami membaca survei siswa, usia mereka, usia orang
tua mereka dan jumlah saudara kandung dari sebuah file.

Tabel ini berisi "m" dan "f" di kolom kedua. Kami menggunakan variabel
tok2 untuk menyetel terjemahan yang tepat alih-alih membiarkan
readtable () mengumpulkan terjemahan.
\end{eulercomment}
\begin{eulerprompt}
>\{MS,hd\}:=readtable("table1.dat",tok2:=["m","f"]);  ...
>writetable(MS,labc=hd,tok2:=["m","f"]);
\end{eulerprompt}
\begin{euleroutput}
      Person       Sex       Age    Mother    Father  Siblings
           1         m        29        58        61         1
           2         f        26        53        54         2
           3         m        24        49        55         1
           4         f        25        56        63         3
           5         f        25        49        53         0
           6         f        23        55        55         2
           7         m        23        48        54         2
           8         m        27        56        58         1
           9         m        25        57        59         1
          10         m        24        50        54         1
          11         f        26        61        65         1
          12         m        24        50        52         1
          13         m        29        54        56         1
          14         m        28        48        51         2
          15         f        23        52        52         1
          16         m        24        45        57         1
          17         f        24        59        63         0
          18         f        23        52        55         1
          19         m        24        54        61         2
          20         f        23        54        55         1
\end{euleroutput}
\begin{eulercomment}
How do the ages depend on each other? A first impression comes from a
pairwise scatterplot.
\end{eulercomment}
\begin{eulerprompt}
>scatterplots(tablecol(MS,3:5),hd[3:5]):
\end{eulerprompt}
\begin{eulercomment}
Jelas bahwa usia bapak dan ibu saling bergantung. Mari kita tentukan
dan plot garis regresi.
\end{eulercomment}
\begin{eulerprompt}
>cs:=MS[,4:5]'; ps:=polyfit(cs[1],cs[2],1)
\end{eulerprompt}
\begin{euleroutput}
  [17.3789,  0.740964]
\end{euleroutput}
\begin{eulercomment}
Ini jelas model yang salah. Garis regresinya adalah s = 17 + 0.74t,
dimana t adalah umur ibu dan s umur bapak. Perbedaan usia mungkin
sedikit bergantung pada usianya, tapi tidak terlalu banyak.

Sebaliknya, kami menduga fungsi seperti s = a + t. Maka a adalah mean
dari s-t. Ini adalah perbedaan usia rata-rata antara ayah dan ibu.
\end{eulercomment}
\begin{eulerprompt}
>da:=mean(cs[2]-cs[1])
\end{eulerprompt}
\begin{euleroutput}
  3.65
\end{euleroutput}
\begin{eulercomment}
Mari kita plot ini menjadi satu plot pencar.
\end{eulercomment}
\begin{eulerprompt}
>plot2d(cs[1],cs[2],>points);  ...
>plot2d("evalpoly(x,ps)",color=red,style=".",>add);  ...
>plot2d("x+da",color=blue,>add):
\end{eulerprompt}
\begin{eulercomment}
Berikut adalah plot kotak dari dua zaman. Ini hanya menunjukkan, bahwa
umurnya berbeda.
\end{eulercomment}
\begin{eulerprompt}
>boxplot(cs,["mothers","fathers"]):
\end{eulerprompt}
\begin{eulercomment}
Menariknya, perbedaan median tidak sebesar perbedaan rata rata.
\end{eulercomment}
\begin{eulerprompt}
>median(cs[2])-median(cs[1])
\end{eulerprompt}
\begin{euleroutput}
  1.5
\end{euleroutput}
\begin{eulercomment}
Koefisien korelasi menunjukkan korelasi positif.
\end{eulercomment}
\begin{eulerprompt}
>correl(cs[1],cs[2])
\end{eulerprompt}
\begin{euleroutput}
  0.7588307236
\end{euleroutput}
\begin{eulercomment}
Korelasi barisan adalah ukuran untuk urutan yang sama di kedua vektor.
Ini juga cukup positif.
\end{eulercomment}
\begin{eulerprompt}
>rankcorrel(cs[1],cs[2])
\end{eulerprompt}
\begin{euleroutput}
  0.758925292358
\end{euleroutput}
\eulerheading{Membuat Fungsi baru}
\begin{eulercomment}
Tentu saja, bahasa EMT dapat digunakan untuk memprogram fungsi baru.
Misalnya, kami mendefinisikan fungsi kemiringan.

\end{eulercomment}
\begin{eulerformula}
\[
\text{sk}(x) = \dfrac{\sqrt{n} \sum_i (x_i-m)^3}{\left(\sum_i (x_i-m)^2\right)^{3/2}}
\]
\end{eulerformula}
\begin{eulercomment}
dimana m adalah mean dari x.
\end{eulercomment}
\begin{eulerprompt}
>function skew (x:vector) ...
\end{eulerprompt}
\begin{eulerudf}
  m=mean(x);
  return sqrt(cols(x))*sum((x-m)^3)/(sum((x-m)^2))^(3/2);
  endfunction
\end{eulerudf}
\begin{eulercomment}
Seperti yang Anda lihat, kita dapat dengan mudah menggunakan bahasa
matriks untuk mendapatkan implementasi yang sangat singkat dan
efisien. Mari kita coba fungsi ini.
\end{eulercomment}
\begin{eulerprompt}
>data=normal(20); skew(normal(10))
\end{eulerprompt}
\begin{euleroutput}
  0.643769122478
\end{euleroutput}
\begin{eulercomment}
Berikut adalah fungsi lain, yang disebut koefisien kemiringan Pearson.
\end{eulercomment}
\begin{eulerprompt}
>function skew1 (x) := 3*(mean(x)-median(x))/dev(x)
>skew1(data)
\end{eulerprompt}
\begin{euleroutput}
  0.24951252184
\end{euleroutput}
\eulerheading{Simulasi Monte Carlo}
\begin{eulercomment}
Euler dapat digunakan untuk mensimulasikan peristiwa acak. Kami telah
melihat contoh sederhana di atas. Ini satu lagi, yang mensimulasikan
1000 kali lemparan 3 dadu, dan menanyakan distribusi jumlahnya.
\end{eulercomment}
\begin{eulerprompt}
>ds:=sum(intrandom(1000,3,6))';  fs=getmultiplicities(3:18,ds)
\end{eulerprompt}
\begin{euleroutput}
  [2,  13,  29,  53,  65,  105,  108,  117,  130,  115,  88,  74,  54,
  33,  12,  2]
\end{euleroutput}
\begin{eulercomment}
Kita bisa merencakannya sekarang.
\end{eulercomment}
\begin{eulerprompt}
>columnsplot(fs,lab=3:18):
\end{eulerprompt}
\begin{eulercomment}
Untuk menentukan distribusi yang diharapkan tidaklah mudah. Kami
menggunakan rekursi lanjutan untuk ini.

Fungsi berikut menghitung banyaknya cara bilangan k dapat
direpresentasikan sebagai jumlah dari n bilangan dalam rentang 1
hingga m. Ini bekerja secara rekursif dengan cara yang jelas.
\end{eulercomment}
\begin{eulerprompt}
>function map countways (k; n, m) ...
\end{eulerprompt}
\begin{eulerudf}
    if n==1 then return k>=1 && k<=m
    else
      sum=0; 
      loop 1 to m; sum=sum+countways(k-#,n-1,m); end;
      return sum;
    end;
  endfunction
\end{eulerudf}
\begin{eulercomment}
Ini adalah hasil dari tiga lemparan dadu.
\end{eulercomment}
\begin{eulerprompt}
>cw=countways(3:18,3,6)
\end{eulerprompt}
\begin{euleroutput}
  [1,  3,  6,  10,  15,  21,  25,  27,  27,  25,  21,  15,  10,  6,  3,
  1]
\end{euleroutput}
\begin{eulercomment}
Kami menambahkan nilai yang diharapkan ke plot.
\end{eulercomment}
\begin{eulerprompt}
>plot2d(cw/6^3*1000,>add); plot2d(cw/6^3*1000,>points,>add):
\end{eulerprompt}
\begin{eulercomment}
Untuk simulasi lain, deviasi nilai rata-rata n 0-1-variabel acak
terdistribusi normal adalah 1 / sqrt (n).
\end{eulercomment}
\begin{eulerprompt}
>longformat; 1/sqrt(10)
\end{eulerprompt}
\begin{euleroutput}
  0.316227766017
\end{euleroutput}
\begin{eulercomment}
Mari kita periksa dengan simulasi. Kami menghasilkan 10.000 kali 10
vektor acak.
\end{eulercomment}
\begin{eulerprompt}
>M=normal(10000,10); dev(mean(M)')
\end{eulerprompt}
\begin{euleroutput}
  0.319188944099
\end{euleroutput}
\begin{eulerprompt}
>plot2d(mean(M)',>distribution):
\end{eulerprompt}
\begin{eulercomment}
Median dari 10 bilangan acak terdistribusi normal 0-1 memiliki deviasi
yang lebih besar.
\end{eulercomment}
\begin{eulerprompt}
>dev(median(M)')
\end{eulerprompt}
\begin{euleroutput}
  0.373609827223
\end{euleroutput}
\begin{eulercomment}
Karena kami dapat dengan mudah membuat jalan acak, kami dapat
mensimulasikan proses Wiener. Kami mengambil 1000 langkah dari 1000
proses. Kami kemudian memplot deviasi standar dan mean dari langkah
ke-n dari proses ini bersama dengan nilai yang diharapkan berwarna
merah.
\end{eulercomment}
\begin{eulerprompt}
>n=1000; m=1000; M=cumsum(normal(n,m)/sqrt(m)); ...
>t=(1:n)/n; figure(2,1); ...
>figure(1); plot2d(t,mean(M')'); plot2d(t,0,color=red,>add); ...
>figure(2); plot2d(t,dev(M')'); plot2d(t,sqrt(t),color=red,>add); ...
>figure(0):
\end{eulerprompt}
\eulerheading{Tes}
\begin{eulercomment}
Tes adalah alat penting dalam statistik. Di Euler, banyak tes yang
diterapkan. Semua pengujian ini mengembalikan kesalahan yang kami
terima jika kami menolak hipotesis nol.

Sebagai contoh, kami menguji lemparan dadu untuk distribusi seragam.
Pada 600 lemparan, kami mendapatkan nilai berikut, yang kami masukkan
ke dalam uji chi-square.
\end{eulercomment}
\begin{eulerprompt}
>chitest([90,103,114,101,103,89],dup(100,6)')
\end{eulerprompt}
\begin{euleroutput}
  0.498830517952
\end{euleroutput}
\begin{eulercomment}
Uji chi-square juga memiliki mode, yang menggunakan simulasi Monte
Carlo untuk menguji statistik. Hasilnya harusnya hampir sama.
Parameter\textgreater{} p mengartikan vektor y sebagai vektor probabilitas.
\end{eulercomment}
\begin{eulerprompt}
>chitest([90,103,114,101,103,89],dup(1/6,6)',>p,>montecarlo)
\end{eulerprompt}
\begin{euleroutput}
  0.508
\end{euleroutput}
\begin{eulercomment}
Kesalahan ini terlalu besar. Jadi kita tidak bisa menolak distribusi
seragam. Ini tidak membuktikan bahwa dadu kami adil. Tapi kita tidak
bisa menolak hipotesis kita.

Selanjutnya kami menghasilkan 1000 lemparan dadu menggunakan generator
nomor acak, dan melakukan tes yang sama.
\end{eulercomment}
\begin{eulerprompt}
>n=1000; t=random([1,n*6]); chitest(count(t*6,6),dup(n,6)')
\end{eulerprompt}
\begin{euleroutput}
  0.243439570837
\end{euleroutput}
\begin{eulercomment}
Mari kita uji nilai rata-rata 100 dengan uji-t.
\end{eulercomment}
\begin{eulerprompt}
>s=200+normal([1,100])*10; ...
>ttest(mean(s),dev(s),100,200)
\end{eulerprompt}
\begin{euleroutput}
  0.426952606967
\end{euleroutput}
\begin{eulercomment}
Fungsi ttest() membutuhkan nilai mean, deviasi, jumlah data, dan nilai
mean untuk diuji.

Sekarang mari kita periksa dua pengukuran untuk mean yang sama. Kami
menolak hipotesis bahwa mereka memiliki mean yang sama, jika hasilnya
\textless{}0,05.
\end{eulercomment}
\begin{eulerprompt}
>tcomparedata(normal(1,10),normal(1,10))
\end{eulerprompt}
\begin{euleroutput}
  0.267190351647
\end{euleroutput}
\begin{eulercomment}
Jika kita menambahkan bias ke satu distribusi, kita mendapatkan lebih
banyak penolakan. Ulangi simulasi ini beberapa kali untuk melihat
efeknya.
\end{eulercomment}
\begin{eulerprompt}
>tcomparedata(normal(1,10),normal(1,10)+2)
\end{eulerprompt}
\begin{euleroutput}
  3.99849573895e-07
\end{euleroutput}
\begin{eulercomment}
Dalam contoh berikutnya, kami menghasilkan 20 lemparan dadu acak 100
kali dan menghitung yang ada di dalamnya. Harus ada rata-rata 20/6 =
3,3.
\end{eulercomment}
\begin{eulerprompt}
>R=random(100,20); R=sum(R*6<=1)'; mean(R)
\end{eulerprompt}
\begin{euleroutput}
  3.28
\end{euleroutput}
\begin{eulercomment}
Sekarang kami membandingkan jumlah satuan dengan distribusi binomial.
Pertama kami memplot distribusi satu.
\end{eulercomment}
\begin{eulerprompt}
>plot2d(R,distribution=max(R)+1,even=1,style="\(\backslash\)/"):
>t=count(R,21);
\end{eulerprompt}
\begin{eulercomment}
Kemudian kami menghitung nilai yang diharapkan.
\end{eulercomment}
\begin{eulerprompt}
>n=0:20; b=bin(20,n)*(1/6)^n*(5/6)^(20-n)*100;
\end{eulerprompt}
\begin{eulercomment}
Kita harus mengumpulkan beberapa nomor untuk mendapatkan kategori yang
cukup besar.
\end{eulercomment}
\begin{eulerprompt}
>t1=sum(t[1:2])|t[3:7]|sum(t[8:21]); ...
>b1=sum(b[1:2])|b[3:7]|sum(b[8:21]);
\end{eulerprompt}
\begin{eulercomment}
Uji chi-square menolak hipotesis bahwa distribusi kita adalah
distribusi binomial, jika hasilnya \textless{}0,05.
\end{eulercomment}
\begin{eulerprompt}
>chitest(t1,b1)
\end{eulerprompt}
\begin{euleroutput}
  0.185520705242
\end{euleroutput}
\begin{eulercomment}
Contoh berikut berisi hasil dari dua kelompok orang (misalnya
laki-laki dan perempuan) yang memberikan suara untuk satu dari enam
partai.
\end{eulercomment}
\begin{eulerprompt}
>A=[23,37,43,52,64,74;27,39,41,49,63,76];  ...
>  writetable(A,wc=6,labr=["m","f"],labc=1:6)
\end{eulerprompt}
\begin{euleroutput}
             1     2     3     4     5     6
       m    23    37    43    52    64    74
       f    27    39    41    49    63    76
\end{euleroutput}
\begin{eulercomment}
Kita ingin menguji independensi suara dari jenis kelamin. Tes tabel
chi\textasciicircum{}2 melakukan ini. Hasilnya adalah cara yang besar untuk menolak
kemerdekaan. Jadi kami tidak bisa mengatakan, apakah voting tergantung
jenis kelamin dari data ini.
\end{eulercomment}
\begin{eulerprompt}
>tabletest(A)
\end{eulerprompt}
\begin{euleroutput}
  0.990701632326
\end{euleroutput}
\begin{eulercomment}
Berikut adalah tabel yang diharapkan, jika kita mengasumsikan
frekuensi pemungutan suara yang diamati.
\end{eulercomment}
\begin{eulerprompt}
>writetable(expectedtable(A),wc=6,dc=1,labr=["m","f"],labc=1:6)
\end{eulerprompt}
\begin{euleroutput}
             1     2     3     4     5     6
       m  24.9  37.9  41.9  50.3  63.3  74.7
       f  25.1  38.1  42.1  50.7  63.7  75.3
\end{euleroutput}
\begin{eulercomment}
Kita dapat menghitung koefisien kontingensi yang dikoreksi. Karena
sangat mendekati 0, kami menyimpulkan bahwa pemungutan suara tidak
bergantung pada jenis kelamin.
\end{eulercomment}
\begin{eulerprompt}
>contingency(A)
\end{eulerprompt}
\begin{euleroutput}
  0.0427225484717
\end{euleroutput}
\eulerheading{Beberapa Tes Lagi}
\begin{eulercomment}
Selanjutnya kami menggunakan analisis varians (uji-F) untuk menguji
tiga sampel data terdistribusi normal untuk nilai rata-rata yang sama.
Metode tersebut dinamakan ANOVA (analysis of variance). Di Euler,
fungsi varanalysis() digunakan.
\end{eulercomment}
\begin{eulerprompt}
>x1=[109,111,98,119,91,118,109,99,115,109,94]; mean(x1),
\end{eulerprompt}
\begin{euleroutput}
  106.545454545
\end{euleroutput}
\begin{eulerprompt}
>x2=[120,124,115,139,114,110,113,120,117]; mean(x2),
\end{eulerprompt}
\begin{euleroutput}
  119.111111111
\end{euleroutput}
\begin{eulerprompt}
>x3=[120,112,115,110,105,134,105,130,121,111]; mean(x3)
\end{eulerprompt}
\begin{euleroutput}
  116.3
\end{euleroutput}
\begin{eulerprompt}
>varanalysis(x1,x2,x3)
\end{eulerprompt}
\begin{euleroutput}
  0.0138048221371
\end{euleroutput}
\begin{eulercomment}
Artinya, kami menolak hipotesis dengan nilai mean yang sama. Kami
melakukan ini dengan probabilitas kesalahan 1,3\%.

Ada juga uji median, yaitu menolak sampel data dengan distribusi
rata-rata yang berbeda menguji median dari sampel yang bersatu.
\end{eulercomment}
\begin{eulerprompt}
>a=[56,66,68,49,61,53,45,58,54];
>b=[72,81,51,73,69,78,59,67,65,71,68,71];
>mediantest(a,b)
\end{eulerprompt}
\begin{euleroutput}
  0.0241724220052
\end{euleroutput}
\begin{eulercomment}
Tes lain tentang kesetaraan adalah ujian peringkat. Ini jauh lebih
tajam daripada tes median.
\end{eulercomment}
\begin{eulerprompt}
>ranktest(a,b)
\end{eulerprompt}
\begin{euleroutput}
  0.00199969612469
\end{euleroutput}
\begin{eulercomment}
Dalam contoh berikut, kedua distribusi memiliki mean yang sama.
\end{eulercomment}
\begin{eulerprompt}
>ranktest(random(1,100),random(1,50)*3-1)
\end{eulerprompt}
\begin{euleroutput}
  0.468224146531
\end{euleroutput}
\begin{eulercomment}
Sekarang mari kita coba meniru dua perlakuan a dan b yang diterapkan
pada orang yang berbeda.
\end{eulercomment}
\begin{eulerprompt}
>a=[8.0,7.4,5.9,9.4,8.6,8.2,7.6,8.1,6.2,8.9];
>b=[6.8,7.1,6.8,8.3,7.9,7.2,7.4,6.8,6.8,8.1];
\end{eulerprompt}
\begin{eulercomment}
Tes signum memutuskan, jika a lebih baik dari b.
\end{eulercomment}
\begin{eulerprompt}
>signtest(a,b)
\end{eulerprompt}
\begin{euleroutput}
  0.0546875
\end{euleroutput}
\begin{eulercomment}
Ini terlalu banyak kesalahan. Kita tidak dapat menolak bahwa a sama
baiknya dengan b.

Tes Wilcoxon lebih tajam dari tes ini, tetapi bergantung pada nilai
kuantitatif perbedaannya.
\end{eulercomment}
\begin{eulerprompt}
>wilcoxon(a,b)
\end{eulerprompt}
\begin{euleroutput}
  0.0296680599405
\end{euleroutput}
\begin{eulercomment}
Mari kita coba dua tes lagi menggunakan seri yang dihasilkan.
\end{eulercomment}
\begin{eulerprompt}
>wilcoxon(normal(1,20),normal(1,20)-1)
\end{eulerprompt}
\begin{euleroutput}
  0.00202259852485
\end{euleroutput}
\begin{eulerprompt}
>wilcoxon(normal(1,20),normal(1,20))
\end{eulerprompt}
\begin{euleroutput}
  0.824671773049
\end{euleroutput}
\eulerheading{Angka Acak}
\begin{eulercomment}
Berikut ini adalah tes untuk generator bilangan acak. Euler
menggunakan generator yang sangat bagus, jadi kami tidak perlu
mengharapkan adanya masalah.

Pertama kami menghasilkan sepuluh juta bilangan acak di [0,1].
\end{eulercomment}
\begin{eulerprompt}
>n:=10000000; r:=random(1,n);
\end{eulerprompt}
\begin{eulercomment}
Selanjutnya kita menghitung jarak antara dua angka kurang dari 0,05.
\end{eulercomment}
\begin{eulerprompt}
>a:=0.05; d:=differences(nonzeros(r<a));
\end{eulerprompt}
\begin{eulercomment}
Akhirnya, kami memplot berapa kali, setiap jarak terjadi, dan
membandingkan dengan nilai yang diharapkan.
\end{eulercomment}
\begin{eulerprompt}
>m=getmultiplicities(1:100,d); plot2d(m); ...
>  plot2d("n*(1-a)^(x-1)*a^2",color=red,>add):
\end{eulerprompt}
\begin{eulercomment}
Hapus datanya.
\end{eulercomment}
\begin{eulerprompt}
>remvalue n;
\end{eulerprompt}
\eulerheading{Pengenalan untuk Pengguna Proyek R.}
\begin{eulercomment}
Jelas, EMT tidak bersaing dengan R sebagai paket statistik. Namun, ada
banyak prosedur dan fungsi statistik yang tersedia di EMT juga. Jadi
EMT dapat memenuhi kebutuhan dasar. Bagaimanapun, EMT hadir dengan
paket numerik dan sistem aljabar komputer.

Notebook ini untuk Anda jika Anda sudah familiar dengan R, tetapi
perlu mengetahui perbedaan sintaks EMT dan R. Kami mencoba memberikan
gambaran umum tentang hal-hal yang jelas dan kurang jelas yang perlu
Anda ketahui.

Selain itu, kami mencari cara untuk bertukar data antara kedua sistem.
\end{eulercomment}
\begin{eulercomment}
Perhatikan bahwa ini adalah pekerjaan yang sedang berjalan.
\end{eulercomment}
\eulerheading{Sintaks Dasar}
\begin{eulercomment}
Hal pertama yang Anda pelajari di R adalah membuat vektor. Dalam EMT,
perbedaan utamanya adalah: operator dapat mengambil ukuran langkah.
Selain itu memiliki daya ikat yang rendah.
\end{eulercomment}
\begin{eulerprompt}
>n=10; 0:n/20:n-1
\end{eulerprompt}
\begin{euleroutput}
  [0,  0.5,  1,  1.5,  2,  2.5,  3,  3.5,  4,  4.5,  5,  5.5,  6,  6.5,
  7,  7.5,  8,  8.5,  9]
\end{euleroutput}
\begin{eulercomment}
Fungsi c() tidak ada. Dimungkinkan untuk menggunakan vektor untuk
menggabungkan berbagai hal.

Contoh berikut, seperti banyak contoh lainnya, dari "Interoduction to
R" yang disertakan dengan proyek R. Jika Anda membaca PDF ini, Anda
akan menemukan bahwa saya mengikuti jalurnya dalam tutorial ini.
\end{eulercomment}
\begin{eulerprompt}
>x=[10.4, 5.6, 3.1, 6.4, 21.7]; [x,0,x]
\end{eulerprompt}
\begin{euleroutput}
  [10.4,  5.6,  3.1,  6.4,  21.7,  0,  10.4,  5.6,  3.1,  6.4,  21.7]
\end{euleroutput}
\begin{eulercomment}
Operator titik dua dengan ukuran langkah EMT diganti dengan fungsi
seq() di R. Kita bisa menulis fungsi ini di EMT.
\end{eulercomment}
\begin{eulerprompt}
>function seq(a,b,c) := a:b:c; ...
>seq(0,-0.1,-1)
\end{eulerprompt}
\begin{euleroutput}
  [0,  -0.1,  -0.2,  -0.3,  -0.4,  -0.5,  -0.6,  -0.7,  -0.8,  -0.9,  -1]
\end{euleroutput}
\begin{eulercomment}
Fungsi rep() dari R tidak ada di EMT. Untuk input vektor dapat
dituliskan sebagai berikut.
\end{eulercomment}
\begin{eulerprompt}
>function rep(x:vector,n:index) := flatten(dup(x,n)); ...
>rep(x,2)
\end{eulerprompt}
\begin{euleroutput}
  [10.4,  5.6,  3.1,  6.4,  21.7,  10.4,  5.6,  3.1,  6.4,  21.7]
\end{euleroutput}
\begin{eulercomment}
Perhatikan bahwa "=" atau ":=" digunakan untuk tugas. Operator "-\textgreater{}"
digunakan untuk unit di EMT.
\end{eulercomment}
\begin{eulerprompt}
>125km -> " miles"
\end{eulerprompt}
\begin{euleroutput}
  77.6713990297 miles
\end{euleroutput}
\begin{eulercomment}
Operator "\textless{}-" untuk penugasan menyesatkan, dan bukan ide yang baik
untuk R. Berikut ini akan membandingkan dan -4 di EMT.
\end{eulercomment}
\begin{eulerprompt}
>a=2; a<-4
\end{eulerprompt}
\begin{euleroutput}
  0
\end{euleroutput}
\begin{eulercomment}
Di R, "a \textless{}-4 \textless{}3" berfungsi, tetapi "a \textless{}-4 \textless{}-3" tidak. Saya juga
memiliki ambiguitas yang serupa dalam EMT, tetapi mencoba
menghilangkannya terus-menerus.

EMT dan R memiliki vektor tipe boolean. Tetapi di EMT, angka 0 dan 1
digunakan untuk mewakili salah dan benar. Di R, nilai benar dan salah
dapat digunakan dalam aritmatika biasa seperti di EMT.
\end{eulercomment}
\begin{eulerprompt}
>x<5, %*x
\end{eulerprompt}
\begin{euleroutput}
  [0,  0,  1,  0,  0]
  [0,  0,  3.1,  0,  0]
\end{euleroutput}
\begin{eulercomment}
EMT melempar kesalahan atau menghasilkan NAN tergantung pada bendera
"kesalahan".
\end{eulercomment}
\begin{eulerprompt}
>errors off; 0/0, isNAN(sqrt(-1)), errors on;
\end{eulerprompt}
\begin{euleroutput}
  NAN
  1
\end{euleroutput}
\begin{eulercomment}
String sama di R dan EMT. Keduanya ada di lokal saat ini, bukan di
Unicode.

Di R ada paket untuk Unicode. Di EMT, string bisa berupa string
Unicode. String unicode dapat diterjemahkan ke pengkodean lokal dan
sebaliknya. Selain itu, u "..." dapat berisi entitas HTML.
\end{eulercomment}
\begin{eulerprompt}
>u"&#169; Ren&eacut; Grothmann"
\end{eulerprompt}
\begin{euleroutput}
  © René Grothmann
\end{euleroutput}
\begin{eulercomment}
Berikut ini mungkin atau mungkin tidak ditampilkan dengan benar pada
sistem Anda sebagai A dengan titik dan tanda hubung di atasnya. Itu
tergantung pada font yang Anda gunakan.
\end{eulercomment}
\begin{eulerprompt}
>chartoutf([480])
\end{eulerprompt}
\begin{euleroutput}
  Ǡ
\end{euleroutput}
\begin{eulercomment}
Rangkaian string dilakukan dengan "+" atau "\textbar{}". Ini dapat menyertakan
angka, yang akan dicetak dalam format saat ini.
\end{eulercomment}
\begin{eulerprompt}
>"pi = "+pi
\end{eulerprompt}
\begin{euleroutput}
  pi = 3.14159265359
\end{euleroutput}
\eulerheading{Pengindeksan}
\begin{eulercomment}
Biasanya, ini akan berfungsi seperti di R.

Tetapi EMT akan menafsirkan indeks negatif dari belakang vektor,
sedangkan R menafsirkan x [n] sebagai x tanpa elemen ke-n.
\end{eulercomment}
\begin{eulerprompt}
>x, x[1:3], x[-2]
\end{eulerprompt}
\begin{euleroutput}
  [10.4,  5.6,  3.1,  6.4,  21.7]
  [10.4,  5.6,  3.1]
  6.4
\end{euleroutput}
\begin{eulercomment}
Perilaku R dapat dicapai di EMT dengan drop().
\end{eulercomment}
\begin{eulerprompt}
>drop(x,2)
\end{eulerprompt}
\begin{euleroutput}
  [10.4,  3.1,  6.4,  21.7]
\end{euleroutput}
\begin{eulercomment}
Vektor logika tidak diperlakukan secara berbeda sebagai indeks di EMT,
berbeda dengan R. Anda perlu mengekstrak elemen bukan nol terlebih
dahulu di EMT.
\end{eulercomment}
\begin{eulerprompt}
>x, x>5, x[nonzeros(x>5)]
\end{eulerprompt}
\begin{euleroutput}
  [10.4,  5.6,  3.1,  6.4,  21.7]
  [1,  1,  0,  1,  1]
  [10.4,  5.6,  6.4,  21.7]
\end{euleroutput}
\begin{eulercomment}
Sama seperti di R, vektor indeks dapat berisi pengulangan.
\end{eulercomment}
\begin{eulerprompt}
>x[[1,2,2,1]]
\end{eulerprompt}
\begin{euleroutput}
  [10.4,  5.6,  5.6,  10.4]
\end{euleroutput}
\begin{eulercomment}
Tetapi nama untuk indeks tidak dimungkinkan di EMT. Untuk paket
statistik, ini mungkin sering diperlukan untuk memudahkan akses ke
elemen vektor.

Untuk meniru perilaku ini, kita dapat mendefinisikan fungsi sebagai
berikut.
\end{eulercomment}
\begin{eulerprompt}
>function sel (v,i,s) := v[indexof(s,i)]; ...
>s=["first","second","third","fourth"]; sel(x,["first","third"],s)
\end{eulerprompt}
\begin{euleroutput}
  
  Trying to overwrite protected function sel!
  Error in:
  function sel (v,i,s) := v[indexof(s,i)]; ... ...
               ^
  [10.4,  3.1]
\end{euleroutput}
\eulerheading{Jenis Data}
\begin{eulercomment}
EMT memiliki lebih banyak tipe data tetap daripada R. Jelas, di R ada
vektor yang tumbuh. Anda dapat menyetel vektor numerik kosong v dan
menetapkan nilai ke elemen v [17]. Ini tidak mungkin dilakukan di EMT.

Berikut ini agak tidak efisien.
\end{eulercomment}
\begin{eulerprompt}
>v=[]; for i=1 to 10000; v=v|i; end;
\end{eulerprompt}
\begin{eulercomment}
EMT sekarang akan membangun vektor dengan v dan i ditambahkan pada
stack dan menyalin vektor itu kembali ke variabel global v.

Semakin efisien mendefinisikan vektor sebelumnya.
\end{eulercomment}
\begin{eulerprompt}
>v=zeros(10000); for i=1 to 10000; v[i]=i; end;
\end{eulerprompt}
\begin{eulercomment}
Untuk mengubah jenis data di EMT, Anda dapat menggunakan fungsi
seperti complex().
\end{eulercomment}
\begin{eulerprompt}
>complex(1:4)
\end{eulerprompt}
\begin{euleroutput}
  [ 1+0i ,  2+0i ,  3+0i ,  4+0i  ]
\end{euleroutput}
\begin{eulercomment}
Konversi ke string hanya dimungkinkan untuk tipe data dasar. Format
saat ini digunakan untuk penggabungan string sederhana. Tetapi ada
fungsi seperti print() atau frac().

Untuk vektor, Anda dapat dengan mudah menulis fungsi Anda sendiri.
\end{eulercomment}
\begin{eulerprompt}
>function tostr (v) ...
\end{eulerprompt}
\begin{eulerudf}
  s="[";
  loop 1 to length(v);
     s=s+print(v[#],2,0);
     if #<length(v) then s=s+","; endif;
  end;
  return s+"]";
  endfunction
\end{eulerudf}
\begin{eulerprompt}
>tostr(linspace(0,1,10))
\end{eulerprompt}
\begin{euleroutput}
  [0.00,0.10,0.20,0.30,0.40,0.50,0.60,0.70,0.80,0.90,1.00]
\end{euleroutput}
\begin{eulercomment}
Untuk komunikasi dengan Maxima, terdapat fungsi convertmxm (), yang
juga dapat digunakan untuk memformat vektor untuk keluaran.
\end{eulercomment}
\begin{eulerprompt}
>convertmxm(1:10)
\end{eulerprompt}
\begin{euleroutput}
  [1,2,3,4,5,6,7,8,9,10]
\end{euleroutput}
\begin{eulercomment}
Untuk Latex, perintah tex dapat digunakan untuk mendapatkan perintah
Latex.
\end{eulercomment}
\begin{eulerprompt}
>tex(&[1,2,3])
\end{eulerprompt}
\begin{euleroutput}
  \(\backslash\)left[ 1 , 2 , 3 \(\backslash\)right] 
\end{euleroutput}
\eulerheading{Faktor dan Tabel}
\begin{eulercomment}
Dalam pengantar R ada contoh dengan apa yang disebut faktor.

Berikut ini adalah daftar wilayah 30 negara bagian.
\end{eulercomment}
\begin{eulerprompt}
>austates = ["tas", "sa", "qld", "nsw", "nsw", "nt", "wa", "wa", ...
>"qld", "vic", "nsw", "vic", "qld", "qld", "sa", "tas", ...
>"sa", "nt", "wa", "vic", "qld", "nsw", "nsw", "wa", ...
>"sa", "act", "nsw", "vic", "vic", "act"];
\end{eulerprompt}
\begin{eulercomment}
Asumsikan, kami memiliki pendapatan yang sesuai di setiap negara
bagian.
\end{eulercomment}
\begin{eulerprompt}
>incomes = [60, 49, 40, 61, 64, 60, 59, 54, 62, 69, 70, 42, 56, ...
>61, 61, 61, 58, 51, 48, 65, 49, 49, 41, 48, 52, 46, ...
>59, 46, 58, 43];
\end{eulerprompt}
\begin{eulercomment}
Sekarang, kami ingin menghitung rata-rata pendapatan di wilayah
tersebut. Menjadi program statistik, R memiliki faktor () dan tappy ()
untuk ini.

EMT dapat melakukannya dengan mencari indeks teritori dalam daftar
teritori yang unik.
\end{eulercomment}
\begin{eulerprompt}
>auterr=sort(unique(austates)); f=indexofsorted(auterr,austates)
\end{eulerprompt}
\begin{euleroutput}
  [6,  5,  4,  2,  2,  3,  8,  8,  4,  7,  2,  7,  4,  4,  5,  6,  5,  3,
  8,  7,  4,  2,  2,  8,  5,  1,  2,  7,  7,  1]
\end{euleroutput}
\begin{eulercomment}
Pada titik itu, kita bisa menulis fungsi loop kita sendiri untuk
melakukan sesuatu hanya untuk satu faktor.

Atau kita bisa meniru fungsi tapply () dengan cara berikut.
\end{eulercomment}
\begin{eulerprompt}
>function map tappl (i; f$:call, cat, x) ...
\end{eulerprompt}
\begin{eulerudf}
  u=sort(unique(cat));
  f=indexof(u,cat);
  return f$(x[nonzeros(f==indexof(u,i))]);
  endfunction
\end{eulerudf}
\begin{eulercomment}
Ini sedikit tidak efisien, karena ini menghitung wilayah unik untuk
setiap i, tetapi berfungsi.
\end{eulercomment}
\begin{eulerprompt}
>tappl(auterr,"mean",austates,incomes)
\end{eulerprompt}
\begin{euleroutput}
  [44.5,  57.3333,  55.5,  53.6,  55,  60.5,  56,  52.25]
\end{euleroutput}
\begin{eulercomment}
Perhatikan bahwa ini berfungsi untuk setiap vektor wilayah.
\end{eulercomment}
\begin{eulerprompt}
>tappl(["act","nsw"],"mean",austates,incomes)
\end{eulerprompt}
\begin{euleroutput}
  [44.5,  57.3333]
\end{euleroutput}
\begin{eulercomment}
Sekarang, paket statistik EMT mendefinisikan tabel seperti di R.
Fungsi readtable() dan writetable() dapat digunakan untuk input dan
output.

Jadi kita bisa mencetak rata-rata pendapatan negara di wilayah dengan
cara yang bersahabat.
\end{eulercomment}
\begin{eulerprompt}
>writetable(tappl(auterr,"mean",austates,incomes),labc=auterr,wc=7)
\end{eulerprompt}
\begin{euleroutput}
      act    nsw     nt    qld     sa    tas    vic     wa
     44.5  57.33   55.5   53.6     55   60.5     56  52.25
\end{euleroutput}
\begin{eulercomment}
Kami juga dapat mencoba meniru perilaku R sepenuhnya.

Faktor-faktor tersebut harus disimpan dengan jelas dalam koleksi
dengan tipe dan kategori (negara bagian dan teritori dalam contoh
kita). Untuk EMT, kami menambahkan indeks yang telah dihitung
sebelumnya.
\end{eulercomment}
\begin{eulerprompt}
>function makef (t) ...
\end{eulerprompt}
\begin{eulerudf}
  ## Factor data
  ## Returns a collection with data t, unique data, indices.
  ## See: tapply
  u=sort(unique(t));
  return \{\{t,u,indexofsorted(u,t)\}\};
  endfunction
\end{eulerudf}
\begin{eulerprompt}
>statef=makef(austates);
\end{eulerprompt}
\begin{eulercomment}
Sekarang elemen ketiga dari koleksi akan berisi indeks.
\end{eulercomment}
\begin{eulerprompt}
>statef[3]
\end{eulerprompt}
\begin{euleroutput}
  [6,  5,  4,  2,  2,  3,  8,  8,  4,  7,  2,  7,  4,  4,  5,  6,  5,  3,  8,  7,  4,  2,  2,
  8,  5,  1,  2,  7,  7,  1]
\end{euleroutput}
\begin{eulercomment}
Sekarang kita bisa meniru tapply() dengan cara berikut. Ini akan
mengembalikan tabel sebagai kumpulan data tabel dan judul kolom.
\end{eulercomment}
\begin{eulerprompt}
>function tapply (t:vector,tf,f$:call) ...
\end{eulerprompt}
\begin{eulerudf}
  ## Makes a table of data and factors
  ## tf : output of makef()
  ## See: makef
  uf=tf[2]; f=tf[3]; x=zeros(length(uf));
  for i=1 to length(uf);
     ind=nonzeros(f==i);
     if length(ind)==0 then x[i]=NAN;
     else x[i]=f$(t[ind]);
     endif;
  end;
  return \{\{x,uf\}\};
  endfunction
\end{eulerudf}
\begin{eulercomment}
Kita tidak menambahkan banyak jenis pemeriksaan di sini. Tindakan
pencegahan hanya menyangkut kategori (faktor) tanpa data. Tetapi
seseorang harus memeriksa panjang yang benar dari t dan untuk
kebenaran dari koleksi tf.

Tabel ini dapat dicetak sebagai tabel dengan writetable ().
\end{eulercomment}
\begin{eulerprompt}
>writetable(tapply(incomes,statef,"mean"),wc=7)
\end{eulerprompt}
\begin{euleroutput}
      act    nsw     nt    qld     sa    tas    vic     wa
     44.5  57.33   55.5   53.6     55   60.5     56  52.25
\end{euleroutput}
\eulerheading{Array}
\begin{eulercomment}
EMT hanya memiliki dua dimensi untuk array. Tipe datanya disebut
matriks. Akan mudah untuk menulis fungsi untuk dimensi yang lebih
tinggi atau perpustakaan C untuk ini.

R memiliki lebih dari dua dimensi. Dalam R array adalah vektor dengan
bidang dimensi.

Dalam EMT, vektor adalah matriks dengan satu baris. Itu bisa dibuat
menjadi matriks dengan redim().
\end{eulercomment}
\begin{eulerprompt}
>shortformat; X=redim(1:20,4,5)
\end{eulerprompt}
\begin{euleroutput}
          1         2         3         4         5 
          6         7         8         9        10 
         11        12        13        14        15 
         16        17        18        19        20 
\end{euleroutput}
\begin{eulercomment}
Ekstraksi baris dan kolom, atau sub-matriks, sangat mirip dengan R.
\end{eulercomment}
\begin{eulerprompt}
>X[,2:3]
\end{eulerprompt}
\begin{euleroutput}
          2         3 
          7         8 
         12        13 
         17        18 
\end{euleroutput}
\begin{eulercomment}
Namun, di R dimungkinkan untuk mengatur daftar indeks tertentu dari
vektor ke nilai. Hal yang sama mungkin terjadi di EMT hanya dengan
satu loop.
\end{eulercomment}
\begin{eulerprompt}
>function setmatrixvalue (M, i, j, v) ...
\end{eulerprompt}
\begin{eulerudf}
  loop 1 to max(length(i),length(j),length(v))
     M[i\{#\},j\{#\}] = v\{#\};
  end;
  endfunction
\end{eulerudf}
\begin{eulercomment}
Kami mendemonstrasikan ini untuk menunjukkan bahwa matriks dilewatkan
melalui referensi di EMT. Jika Anda tidak ingin mengubah matriks M
asli, Anda perlu menyalinnya di fungsi.
\end{eulercomment}
\begin{eulerprompt}
>setmatrixvalue(X,1:3,3:-1:1,0); X,
\end{eulerprompt}
\begin{euleroutput}
          1         2         0         4         5 
          6         0         8         9        10 
          0        12        13        14        15 
         16        17        18        19        20 
\end{euleroutput}
\begin{eulercomment}
Produk luar di EMT hanya dapat dilakukan di antara vektor. Ini
otomatis karena bahasa matriks. Satu vektor harus menjadi vektor kolom
dan vektor lainnya adalah vektor baris.
\end{eulercomment}
\begin{eulerprompt}
>(1:5)*(1:5)'
\end{eulerprompt}
\begin{euleroutput}
          1         2         3         4         5 
          2         4         6         8        10 
          3         6         9        12        15 
          4         8        12        16        20 
          5        10        15        20        25 
\end{euleroutput}
\begin{eulercomment}
Dalam pengantar PDF untuk R ada contoh, yang menghitung distribusi
ab-cd untuk a, b, c, d yang dipilih dari 0 hingga n secara acak.
Solusi di R adalah membentuk matriks 4 dimensi dan menjalankan table()
di atasnya.

Tentu saja, ini bisa dicapai dengan satu putaran. Tapi loop tidak
efektif di EMT atau R. Di EMT, kita bisa menulis loop di C dan itu
akan menjadi solusi tercepat.

Tetapi kita ingin meniru perilaku R. Untuk ini, kita perlu meratakan
perkalian ab dan membuat matriks ab-cd.
\end{eulercomment}
\begin{eulerprompt}
>a=0:6; b=a'; p=flatten(a*b); q=flatten(p-p'); ...
>u=sort(unique(q)); f=getmultiplicities(u,q); ...
>statplot(u,f,"h"):
\end{eulerprompt}
\begin{eulercomment}
Selain perkalian yang tepat, EMT dapat menghitung frekuensi dalam
vektor.
\end{eulercomment}
\begin{eulerprompt}
>getfrequencies(q,-50:10:50)
\end{eulerprompt}
\begin{euleroutput}
  [0,  23,  132,  316,  602,  801,  333,  141,  53,  0]
\end{euleroutput}
\begin{eulercomment}
Cara paling mudah untuk memplotnya sebagai distribusi adalah sebagai
berikut.
\end{eulercomment}
\begin{eulerprompt}
>plot2d(q,distribution=11):
\end{eulerprompt}
\begin{eulercomment}
Tetapi dimungkinkan juga untuk menghitung sebelumnya dalam interval
yang dipilih sebelumnya. Tentu saja, berikut ini menggunakan
getfrequencies() secara internal.

Karena fungsi histo() mengembalikan frekuensi, kita perlu
menskalakannya sehingga integral di bawah grafik batang adalah 1.
\end{eulercomment}
\begin{eulerprompt}
>\{x,y\}=histo(q,v=-55:10:55); y=y/sum(y)/differences(x); ...
>plot2d(x,y,>bar,style="/"):
\end{eulerprompt}
\eulerheading{Daftar}
\begin{eulercomment}
EMT memiliki dua macam daftar. Salah satunya adalah daftar global yang
dapat berubah, dan yang lainnya adalah jenis daftar yang tidak dapat
diubah. Kami tidak peduli dengan daftar global di sini.

Jenis daftar yang tidak dapat diubah disebut koleksi di EMT. Ini
berperilaku seperti struktur di C, tetapi elemen hanya diberi nomor
dan tidak dinamai.
\end{eulercomment}
\begin{eulerprompt}
>L=\{\{"Fred","Flintstone",40,[1990,1992]\}\}
\end{eulerprompt}
\begin{euleroutput}
  Fred
  Flintstone
  40
  [1990,  1992]
\end{euleroutput}
\begin{eulercomment}
Saat ini elemen tidak memiliki nama, meskipun nama dapat diatur untuk
tujuan khusus. Mereka diakses dengan angka.
\end{eulercomment}
\begin{eulerprompt}
>(L[4])[2]
\end{eulerprompt}
\begin{euleroutput}
  1992
\end{euleroutput}
\eulerheading{Input dan Output File (Membaca dan Menulis Data)}
\begin{eulercomment}
Anda akan sering ingin mengimpor matriks data dari sumber lain ke EMT.
Tutorial ini memberi tahu Anda tentang banyak cara untuk mencapai ini.
Fungsi sederhana adalah writematrix() dan readmatrix().

Mari kita tunjukkan bagaimana membaca dan menulis vektor real ke file.
\end{eulercomment}
\begin{eulerprompt}
>a=random(1,100); mean(a), dev(a),
\end{eulerprompt}
\begin{euleroutput}
  0.47466
  0.27327
\end{euleroutput}
\begin{eulercomment}
Untuk menulis data ke file, kami menggunakan fungsi writematrix().

Karena pendahuluan ini kemungkinan besar ada di direktori, di mana
pengguna tidak memiliki akses tulis, kami menulis data ke direktori
home pengguna. Untuk buku catatan sendiri, ini tidak perlu, karena
file data akan ditulis ke direktori yang sama.
\end{eulercomment}
\begin{eulerprompt}
>filename="test.dat";
\end{eulerprompt}
\begin{eulercomment}
Sekarang kita menulis vektor kolom a 'ke file. Ini menghasilkan satu
nomor di setiap baris file.
\end{eulercomment}
\begin{eulerprompt}
>writematrix(a',filename);
\end{eulerprompt}
\begin{eulercomment}
Untuk membaca data, kami menggunakan readmatrix().
\end{eulercomment}
\begin{eulerprompt}
>a=readmatrix(filename)';
\end{eulerprompt}
\begin{eulercomment}
Dan hapus file tersebut.
\end{eulercomment}
\begin{eulerprompt}
>fileremove(filename);
>mean(a), dev(a),
\end{eulerprompt}
\begin{euleroutput}
  0.47466
  0.27327
\end{euleroutput}
\begin{eulercomment}
Fungsi writematrix() atau writetable() dapat dikonfigurasi untuk
bahasa lain.

Misalnya, jika Anda memiliki sistem Indonesia (titik desimal dengan
koma), Excel Anda memerlukan nilai dengan koma desimal yang dipisahkan
oleh titik koma dalam file csv (defaultnya adalah nilai yang
dipisahkan koma). File berikut "test.csv" akan muncul di folder
cuurent Anda.
\end{eulercomment}
\begin{eulerprompt}
>filename="test.csv"; ...
>writematrix(random(5,3),file=filename,separator=",");
\end{eulerprompt}
\begin{eulercomment}
Anda sekarang dapat membuka file ini dengan Excel Indonesia secara
langsung.
\end{eulercomment}
\begin{eulerprompt}
>fileremove(filename);
\end{eulerprompt}
\begin{eulercomment}
Terkadang kami memiliki string dengan token seperti berikut.
\end{eulercomment}
\begin{eulerprompt}
>s1:="f m m f m m m f f f m m f";  ...
>s2:="f f f m m f f";
\end{eulerprompt}
\begin{eulercomment}
Untuk membuat token ini, kita mendefinisikan vektor token.
\end{eulercomment}
\begin{eulerprompt}
>tok:=["f","m"]
\end{eulerprompt}
\begin{euleroutput}
  f
  m
\end{euleroutput}
\begin{eulercomment}
Kemudian kita dapat menghitung berapa kali setiap token muncul dalam
string, dan memasukkan hasilnya ke dalam tabel.
\end{eulercomment}
\begin{eulerprompt}
>M:=getmultiplicities(tok,strtokens(s1))_ ...
>  getmultiplicities(tok,strtokens(s2));
\end{eulerprompt}
\begin{eulercomment}
Tulis tabel dengan header token.
\end{eulercomment}
\begin{eulerprompt}
>writetable(M,labc=tok,labr=1:2,wc=8)
\end{eulerprompt}
\begin{euleroutput}
                 f       m
         1       6       7
         2       5       2
\end{euleroutput}
\begin{eulercomment}
Untuk statika, EMT dapat membaca dan menulis tabel.
\end{eulercomment}
\begin{eulerprompt}
>file="test.dat"; open(file,"w"); ...
>writeln("A,B,C"); writematrix(random(3,3)); ...
>close();
\end{eulerprompt}
\begin{eulercomment}
File tersebut terlihat seperti ini.
\end{eulercomment}
\begin{eulerprompt}
>printfile(file)
\end{eulerprompt}
\begin{euleroutput}
  A,B,C
  0.09108070085843074,0.8993342814237876,0.983635710835939
  0.211019679874495,0.5527768679829471,0.4941417784439638
  0.5797836292915086,0.8118750914043666,0.9838772074457084
  
\end{euleroutput}
\begin{eulercomment}
Fungsi readtable() dalam bentuknya yang paling sederhana bisa membaca
ini dan mengembalikan kumpulan nilai dan baris judul.
\end{eulercomment}
\begin{eulerprompt}
>L=readtable(file,>list);
\end{eulerprompt}
\begin{eulercomment}
Koleksi ini dapat dicetak dengan writetable() ke buku catatan, atau ke
file.
\end{eulercomment}
\begin{eulerprompt}
>writetable(L,wc=10,dc=5)
\end{eulerprompt}
\begin{euleroutput}
           A         B         C
     0.09108   0.89933   0.98364
     0.21102   0.55278   0.49414
     0.57978   0.81188   0.98388
\end{euleroutput}
\begin{eulercomment}
Matriks nilai adalah elemen pertama L. Perhatikan bahwa mean () dalam
EMT menghitung nilai mean dari baris-baris matriks.
\end{eulercomment}
\begin{eulerprompt}
>mean(L[1])
\end{eulerprompt}
\begin{euleroutput}
    0.65802 
    0.41931 
    0.79185 
\end{euleroutput}
\eulerheading{File CSV}
\begin{eulercomment}
Pertama, mari kita tulis matriks ke dalam file. Untuk hasilnya, kami
menghasilkan file di direktori kerja saat ini.
\end{eulercomment}
\begin{eulerprompt}
>file="test.csv";  ...
>M=random(3,3); writematrix(M,file);
\end{eulerprompt}
\begin{eulercomment}
Berikut isi dari file ini.
\end{eulercomment}
\begin{eulerprompt}
>printfile(file)
\end{eulerprompt}
\begin{euleroutput}
  0.2629555460769783,0.9938043902969794,0.9018322446643099
  0.2407959115075192,0.6359669287024015,0.1510861324343464
  0.2515041042947438,0.3339714884700144,0.7611390495192349
  
\end{euleroutput}
\begin{eulercomment}
CVS ini dapat dibuka pada sistem bahasa Inggris ke Excel dengan klik
dua kali. Jika Anda mendapatkan file seperti itu di sistem Jerman,
Anda perlu mengimpor data ke Excel dengan menggunakan titik desimal.

Tetapi titik desimal adalah format default untuk EMT juga. Anda bisa
membaca matriks dari file dengan readmatrix().
\end{eulercomment}
\begin{eulerprompt}
>readmatrix(file)
\end{eulerprompt}
\begin{euleroutput}
    0.26296    0.9938   0.90183 
     0.2408   0.63597   0.15109 
     0.2515   0.33397   0.76114 
\end{euleroutput}
\begin{eulercomment}
Dimungkinkan untuk menulis beberapa matriks ke satu file. Perintah
open() dapat membuka file untuk ditulis dengan parameter "w".
Standarnya adalah "r" untuk membaca.
\end{eulercomment}
\begin{eulerprompt}
>open(file,"w"); writematrix(M); writematrix(M'); close();
\end{eulerprompt}
\begin{eulercomment}
Matriks dipisahkan oleh garis kosong. Untuk membaca matriks, buka file
dan panggil readmatrix() beberapa kali.
\end{eulercomment}
\begin{eulerprompt}
>open(file); A=readmatrix(); B=readmatrix(); A==B, close();
\end{eulerprompt}
\begin{euleroutput}
          1         0         0 
          0         1         0 
          0         0         1 
\end{euleroutput}
\begin{eulercomment}
Di Excel atau spreadsheet serupa, Anda dapat mengekspor matriks
sebagai CSV (nilai dipisahkan koma). Di Excel 2007, gunakan "simpan
sebagai" dan "format lain", lalu pilih "CSV". Pastikan, tabel saat ini
hanya berisi data yang ingin Anda ekspor.

Berikut ini contohnya.
\end{eulercomment}
\begin{eulerprompt}
>printfile("excel-data.csv")
\end{eulerprompt}
\begin{euleroutput}
  0;1000;1000
  1;1051,271096;1072,508181
  2;1105,170918;1150,273799
  3;1161,834243;1233,67806
  4;1221,402758;1323,129812
  5;1284,025417;1419,067549
  6;1349,858808;1521,961556
  7;1419,067549;1632,31622
  8;1491,824698;1750,6725
  9;1568,312185;1877,610579
  10;1648,721271;2013,752707
\end{euleroutput}
\begin{eulercomment}
Seperti yang Anda lihat, sistem Jerman saya menggunakan titik koma
sebagai pemisah dan koma desimal. Anda dapat mengubahnya di pengaturan
sistem atau di Excel, tetapi tidak perlu membaca matriks ke EMT.

Cara termudah untuk membaca ini ke dalam Euler adalah readmatrix().
Semua koma diganti dengan titik dengan parameter\textgreater{} koma. Untuk CSV
bahasa Inggris, cukup abaikan parameter ini.
\end{eulercomment}
\begin{eulerprompt}
>M=readmatrix("excel-data.csv",>comma)
\end{eulerprompt}
\begin{euleroutput}
          0      1000      1000 
          1    1051.3    1072.5 
          2    1105.2    1150.3 
          3    1161.8    1233.7 
          4    1221.4    1323.1 
          5      1284    1419.1 
          6    1349.9      1522 
          7    1419.1    1632.3 
          8    1491.8    1750.7 
          9    1568.3    1877.6 
         10    1648.7    2013.8 
\end{euleroutput}
\begin{eulercomment}
Mari kita plot ini.
\end{eulercomment}
\begin{eulerprompt}
>plot2d(M'[1],M'[2:3],>points,color=[red,green]'):
\end{eulerprompt}
\begin{eulercomment}
Ada cara yang lebih mendasar untuk membaca data dari sebuah file. Anda
dapat membuka file dan membaca angka baris demi baris. Fungsi
getvectorline() akan membaca angka dari sebaris data. Secara default,
ini mengharapkan titik desimal. Tapi itu juga bisa menggunakan koma
desimal, jika Anda memanggil setdecimaldot (",") sebelum Anda
menggunakan fungsi ini.

Fungsi berikut adalah contoh untuk ini. Ini akan berhenti di akhir
file atau baris kosong.
\end{eulercomment}
\begin{eulerprompt}
>function myload (file) ...
\end{eulerprompt}
\begin{eulerudf}
  open(file);
  M=[];
  repeat
     until eof();
     v=getvectorline(3);
     if length(v)>0 then M=M_v; else break; endif;
  end;
  return M;
  close(file);
  endfunction
\end{eulerudf}
\begin{eulerprompt}
>myload(file)
\end{eulerprompt}
\begin{euleroutput}
    0.26296    0.9938   0.90183 
     0.2408   0.63597   0.15109 
     0.2515   0.33397   0.76114 
\end{euleroutput}
\begin{eulercomment}
Juga dimungkinkan untuk membaca semua angka dalam file itu dengan
getvector().
\end{eulercomment}
\begin{eulerprompt}
>open(file); v=getvector(10000); close(); redim(v[1:9],3,3)
\end{eulerprompt}
\begin{euleroutput}
    0.26296    0.9938   0.90183 
     0.2408   0.63597   0.15109 
     0.2515   0.33397   0.76114 
\end{euleroutput}
\begin{eulercomment}
Oleh karena itu, sangat mudah untuk menyimpan sebuah vektor nilai,
satu nilai di setiap baris dan membaca kembali vektor ini.
\end{eulercomment}
\begin{eulerprompt}
>v=random(1000); mean(v)
\end{eulerprompt}
\begin{euleroutput}
  0.50339
\end{euleroutput}
\begin{eulerprompt}
>writematrix(v',file); mean(readmatrix(file)')
\end{eulerprompt}
\begin{euleroutput}
  0.50339
\end{euleroutput}
\eulerheading{Menggunakan Tabel}
\begin{eulercomment}
Tabel dapat digunakan untuk membaca atau menulis data numerik. Sebagai
contoh, kami menulis tabel dengan judul baris dan kolom ke sebuah
file.
\end{eulercomment}
\begin{eulerprompt}
>file="test.tab"; M=random(3,3);  ...
>open(file,"w");  ...
>writetable(M,separator=",",labc=["one","two","three"]);  ...
>close(); ...
>printfile(file)
\end{eulerprompt}
\begin{euleroutput}
  one,two,three
        0.59,      0.81,      0.78
        0.03,      0.57,      0.22
        0.66,      0.54,      0.92
\end{euleroutput}
\begin{eulercomment}
Ini dapat diimpor ke Excel.

Untuk membaca file di EMT, kami menggunakan readtable().
\end{eulercomment}
\begin{eulerprompt}
>\{M,headings\}=readtable(file,>clabs); ...
>writetable(M,labc=headings)
\end{eulerprompt}
\begin{euleroutput}
         one       two     three
        0.59      0.81      0.78
        0.03      0.57      0.22
        0.66      0.54      0.92
\end{euleroutput}
\eulerheading{Menganalisis Garis}
\begin{eulercomment}
Anda bahkan dapat mengevaluasi setiap baris dengan tangan. Misalkan,
kita memiliki garis dengan format berikut.
\end{eulercomment}
\begin{eulerprompt}
>line="2020-11-03,Tue,1'114.05"
\end{eulerprompt}
\begin{euleroutput}
  2020-11-03,Tue,1'114.05
\end{euleroutput}
\begin{eulercomment}
Pertama kita bisa membuat token baris.
\end{eulercomment}
\begin{eulerprompt}
>vt=strtokens(line)
\end{eulerprompt}
\begin{euleroutput}
  2020-11-03
  Tue
  1'114.05
\end{euleroutput}
\begin{eulercomment}
Kemudian kita dapat mengevaluasi setiap elemen garis menggunakan
evaluasi yang sesuai.
\end{eulercomment}
\begin{eulerprompt}
>day(vt[1]),  ...
>indexof(["mon","tue","wed","thu","fri","sat","sun"],tolower(vt[2])),  ...
>strrepl(vt[3],"'","")()
\end{eulerprompt}
\begin{euleroutput}
  7.3816e+05
  2
  1114
\end{euleroutput}
\begin{eulercomment}
Menggunakan ekspresi reguler, dimungkinkan untuk mengekstrak hampir
semua informasi dari sebaris data.

Asumsikan kita memiliki baris berikut dokumen HTML.
\end{eulercomment}
\begin{eulerprompt}
>line="<tr><td>1145.45</td><td>5.6</td><td>-4.5</td><tr>"
\end{eulerprompt}
\begin{euleroutput}
  <tr><td>1145.45</td><td>5.6</td><td>-4.5</td><tr>
\end{euleroutput}
\begin{eulercomment}
Untuk mengekstrak ini, kami menggunakan ekspresi reguler, yang mencari

\end{eulercomment}
\begin{eulerttcomment}
 - braket penutup>,
 - string apapun yang tidak mengandung tanda kurung dengan
\end{eulerttcomment}
\begin{eulercomment}
sub-kecocokan "(...)",\\
\end{eulercomment}
\begin{eulerttcomment}
 - kurung buka dan tutup menggunakan solusi terpendek,
 - lagi string apapun yang tidak mengandung tanda kurung,
 - dan kurung buka <.
\end{eulerttcomment}
\begin{eulercomment}

Ekspresi reguler agak sulit dipelajari tetapi sangat kuat.
\end{eulercomment}
\begin{eulerprompt}
>\{pos,s,vt\}=strxfind(line,">([^<>]+)<.+?>([^<>]+)<");
\end{eulerprompt}
\begin{eulercomment}
Hasilnya adalah posisi pertandingan, string yang cocok, dan vektor
string untuk sub-pencocokan.
\end{eulercomment}
\begin{eulerprompt}
>for k=1:length(vt); vt[k](), end;
\end{eulerprompt}
\begin{euleroutput}
  1145.5
  5.6
\end{euleroutput}
\begin{eulercomment}
Ini adalah fungsi yang membaca semua item numerik antara \textless{}td\textgreater{} dan
\textless{}/td\textgreater{}.
\end{eulercomment}
\begin{eulerprompt}
>function readtd (line) ...
\end{eulerprompt}
\begin{eulerudf}
  v=[]; cp=0;
  repeat
     \{pos,s,vt\}=strxfind(line,"<td.*?>(.+?)</td>",cp);
     until pos==0;
     if length(vt)>0 then v=v|vt[1]; endif;
     cp=pos+strlen(s);
  end;
  return v;
  endfunction
\end{eulerudf}
\begin{eulerprompt}
>readtd(line+"<td>non-numerical</td>")
\end{eulerprompt}
\begin{euleroutput}
  1145.45
  5.6
  -4.5
  non-numerical
\end{euleroutput}
\eulerheading{Membaca dari Web}
\begin{eulercomment}
Situs web atau file dengan URL dapat dibuka di EMT dan dapat dibaca
baris demi baris.

Dalam contoh, kami membaca versi saat ini dari situs EMT. Kami
menggunakan ekspresi reguler untuk memindai "Versi ..." di sebuah
judul.
\end{eulercomment}
\begin{eulerprompt}
>function readversion () ...
\end{eulerprompt}
\begin{eulerudf}
  urlopen("http://www.euler-math-toolbox.de/Programs/Changes.html");
  repeat
    until urleof();
    s=urlgetline();
    k=strfind(s,"Version ",1);
    if k>0 then substring(s,k,strfind(s,"<",k)-1), break; endif;
  end;
  urlclose();
  endfunction
\end{eulerudf}
\begin{eulerprompt}
>readversion
\end{eulerprompt}
\begin{euleroutput}
  Version 2022-05-18
\end{euleroutput}
\eulerheading{Input dan Output Variabel}
\begin{eulercomment}
Anda dapat menulis variabel dalam bentuk definisi Euler ke file atau
ke baris perintah.
\end{eulercomment}
\begin{eulerprompt}
>writevar(pi,"mypi");
\end{eulerprompt}
\begin{euleroutput}
  mypi = 3.141592653589793;
\end{euleroutput}
\begin{eulercomment}
Untuk pengujian, kami menghasilkan file Euler di direktori kerja EMT.
\end{eulercomment}
\begin{eulerprompt}
>file="test.e"; ...
>writevar(random(2,2),"M",file); ...
>printfile(file,3)
\end{eulerprompt}
\begin{euleroutput}
  M = [ ..
  0.9607363360294088, 0.5090564281073382;
  0.67279466809125, 0.688176242399486];
\end{euleroutput}
\begin{eulercomment}
Kami sekarang dapat memuat file. Ini akan mendefinisikan matriks M.
\end{eulercomment}
\begin{eulerprompt}
>load(file); show M,
\end{eulerprompt}
\begin{euleroutput}
  M = 
    0.96074   0.50906 
    0.67279   0.68818 
\end{euleroutput}
\begin{eulercomment}
Ngomong-ngomong, jika writevar() digunakan pada variabel, itu akan
mencetak definisi variabel dengan nama variabel ini.
\end{eulercomment}
\begin{eulerprompt}
>writevar(M); writevar(inch$)
\end{eulerprompt}
\begin{euleroutput}
  M = [ ..
  0.9607363360294088, 0.5090564281073382;
  0.67279466809125, 0.688176242399486];
  inch$ = 0.0254;
\end{euleroutput}
\begin{eulercomment}
Kami juga dapat membuka file baru atau menambahkan ke file yang sudah
ada. Dalam contoh kami menambahkan file yang dibuat sebelumnya.
\end{eulercomment}
\begin{eulerprompt}
>open(file,"a"); ...
>writevar(random(2,2),"M1"); ...
>writevar(random(3,1),"M2"); ...
>close();
>load(file); show M1; show M2;
\end{eulerprompt}
\begin{euleroutput}
  M1 = 
    0.72639   0.62882 
    0.12019   0.03224 
  M2 = 
   0.099173 
    0.85504 
    0.92847 
\end{euleroutput}
\begin{eulercomment}
Untuk menghapus file apa pun, gunakan fileremove().
\end{eulercomment}
\begin{eulerprompt}
>fileremove(file);
\end{eulerprompt}
\begin{eulercomment}
Vektor baris dalam file tidak memerlukan koma, jika setiap nomor ada
di baris baru. Mari kita buat file seperti itu, tulis setiap baris
satu per satu dengan writeln().
\end{eulercomment}
\begin{eulerprompt}
>open(file,"w"); writeln("M = ["); ...
>for i=1 to 5; writeln(""+random()); end; ...
>writeln("];"); close(); ...
>printfile(file)
\end{eulerprompt}
\begin{euleroutput}
  M = [
  0.437022605979
  0.206717463476
  0.466146119229
  0.284765700905
  0.208351197517
  ];
\end{euleroutput}
\begin{eulerprompt}
>load(file); M
\end{eulerprompt}
\begin{euleroutput}
  [0.43702,  0.20672,  0.46615,  0.28477,  0.20835]
\end{euleroutput}
\eulersubheading{Contoh Soal}
\begin{eulercomment}
1.Tabulasi data penelitian antara dua varibel biaya promosi (X) dan
variabel penjualan rumah (Y)
\end{eulercomment}
\begin{eulerprompt}
>a=[10,28,29,35,48,55,71,73,80,88,91,111,131,144,160]
\end{eulerprompt}
\begin{euleroutput}
  [10,  28,  29,  35,  48,  55,  71,  73,  80,  88,  91,  111,  131,
  144,  160]
\end{euleroutput}
\begin{eulerprompt}
>b=[29,47,55,65,79,82,92,95,100,102,110,124,127,130,152]
\end{eulerprompt}
\begin{euleroutput}
  [29,  47,  55,  65,  79,  82,  92,  95,  100,  102,  110,  124,  127,
  130,  152]
\end{euleroutput}
\begin{eulerprompt}
>writetable(a'|b',labc=["a","b"])
\end{eulerprompt}
\begin{euleroutput}
           a         b
          10        29
          28        47
          29        55
          35        65
          48        79
          55        82
          71        92
          73        95
          80       100
          88       102
          91       110
         111       124
         131       127
         144       130
         160       152
\end{euleroutput}
\begin{eulerprompt}
>p=polyfit(a,b,1)
\end{eulerprompt}
\begin{euleroutput}
  [35.643,  0.74034]
\end{euleroutput}
\begin{eulercomment}
Nilai konstanta (a)=35.64 menunjukkan besarnya variabel rata-rata
penjualan rumah yang tidak dipengaruhi oleh biaya promosi atau dapat
diartikan pada saat nilai biaya promosi sebesar 0, maka rata-rata
penjualan rumah sebesar 35.64.

Nilai koefisien korelasi diperoleh sebesar 0,977.Hal ini berarti
adanya hubungan positif antara biaya yang dikeluarkan untuk promosi
dengan rata-rata penjualan rumah. Jika dilihat dari nilai korelasi
hubungan variabel termasuk kategori tinggi,dengan demikian berarti
biaya promosi memiliki hubungan yang tinggi terhadap kenaikan
rata-rata penjualan rumah.

Dari hasil pengukuran diperoleh data tinggi badan kesepuluh siswa
tersebut dalam ukuran sentimeter (cm) sebagai berikut.
\end{eulercomment}
\begin{eulerprompt}
>h=[172,167,180,170,169,160,175,165,173,170]
\end{eulerprompt}
\begin{euleroutput}
  [172,  167,  180,  170,  169,  160,  175,  165,  173,  170]
\end{euleroutput}
\begin{eulerprompt}
>mean(h)
\end{eulerprompt}
\begin{euleroutput}
  170.1
\end{euleroutput}
\begin{eulerprompt}
>dev(h)
\end{eulerprompt}
\begin{euleroutput}
  5.5066
\end{euleroutput}
\begin{eulerprompt}
>boxplot(h):
\end{eulerprompt}
\begin{eulercomment}
Suatu penelitian dilakukan untuk mengetahui apakah terdapat pengaruh
perbedaan kartu kredit terhadap penggunaannya. Data di bawah ini
adalah jumlah uang yang dibelanjakan ibu rumah tangga menggunakan
kartu kredit (dalam \textdollar{}). Empat jenis kartu kredit dibandingkan:
\end{eulercomment}
\begin{eulerprompt}
>Astra=[8,7,10,19,11]
\end{eulerprompt}
\begin{euleroutput}
  [8,  7,  10,  19,  11]
\end{euleroutput}
\begin{eulerprompt}
>BCA=[12,11,16,10,12]
\end{eulerprompt}
\begin{euleroutput}
  [12,  11,  16,  10,  12]
\end{euleroutput}
\begin{eulerprompt}
>CITI=[19,20,15,18,19]
\end{eulerprompt}
\begin{euleroutput}
  [19,  20,  15,  18,  19]
\end{euleroutput}
\begin{eulerprompt}
>AMEX=[13,12,14,15]
\end{eulerprompt}
\begin{euleroutput}
  [13,  12,  14,  15]
\end{euleroutput}
\begin{eulerprompt}
>varanalysis(Astra,BCA,CITI,AMEX)
\end{eulerprompt}
\begin{euleroutput}
  0.0082162
\end{euleroutput}
\begin{eulercomment}
Dari hasil tersebut dapat disimpulkan bahwa tidak ada kesamaan rata
rata dari data tersebut, atau Hipotesis kesamaan rata rata di tolak,
dengan probabilitas kesalahan sebesar 0.8\%
\end{eulercomment}
\end{eulernotebook}
\end{document}


\end{document}
