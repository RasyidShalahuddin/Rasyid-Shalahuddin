\documentclass[a4paper,10pt]{article}
\usepackage{eumat}

\begin{document}
\begin{eulernotebook}
\eulerheading{EMT plot2D}
\begin{eulercomment}
Rasyid Salahuddin\\
22305144016\\
Matematika E



\begin{eulercomment}
\eulerheading{Sub Bab 1}
\begin{eulercomment}
Menggambar Grafik Fungsi Satu Variabel dalam Bentuk Ekspresi Langsung
Ekspresi tunggal

Di dalam program numerik EMT, ekspresi adalah string. Jika ditandai
sebagai simbolis, mereka akan mencetak melalui Maxima, jika tidak
melalui EMT. Ekspresi dalam string digunakan untuk membuat plot dan
banyak fungsi numerik. Untuk ini, variabel dalam ekspresi harus "x".

expresi dalam string
\end{eulercomment}
\begin{eulerprompt}
>expr := "x^5-x^2-3"
\end{eulerprompt}
\begin{euleroutput}
  x^5-x^2-3
\end{euleroutput}
\begin{eulercomment}
plot ekspresi
\end{eulercomment}
\begin{eulerprompt}
>plot2d(expr,-2,2) :
\end{eulerprompt}
\begin{eulercomment}
contoh 1
\end{eulercomment}
\begin{eulerprompt}
>expr := "sin (x-5)"
\end{eulerprompt}
\begin{euleroutput}
  sin (x-5)
\end{euleroutput}
\begin{eulerprompt}
>aspect (1) ; plot2d(expr,-2,2):
\end{eulerprompt}
\begin{eulercomment}
contoh 2 dan penggunaan grid
\end{eulercomment}
\begin{eulerprompt}
>aspect(1)plot2d("log(x) + 3",-0.1,2, grid=6):
\end{eulerprompt}
\begin{euleroutput}
  Commands must be separated by semicolon or comma!
  Found: plot2d("log(x) + 3",-0.1,2, grid=6): (character 112)
  You can disable this in the Options menu.
  Error in:
  aspect(1)plot2d("log(x) + 3",-0.1,2, grid=6): ...
           ^
\end{euleroutput}
\begin{eulercomment}
contoh 3 dan penggunaan parameter square (atau \textgreater{}square) untuk memilih
y-range secara otomatis 
\end{eulercomment}
\begin{eulerprompt}
>aspect(1,1) ; plot2d("x^4-2",-5,5, >square); insimg(15)
>aspect(2) ; plot2d("x^4-2", -5,5 ):
\end{eulerprompt}
\begin{eulercomment}
contoh 4 dan memberikan nama atau label pada garis sumbu
\end{eulercomment}
\begin{eulerprompt}
>plot2d("cos(x)", -4, 6, xl="x",yl="y") :
\end{eulerprompt}
\eulerheading{Sub Bab 2}
\begin{eulercomment}
Menggambar Grafik Fungsi Satu Variabel yang rumusnya Disimpan dalam
Variabel Ekspresi

\end{eulercomment}
\begin{eulerttcomment}
 ekspresi
\end{eulerttcomment}
\begin{eulerprompt}
>expr &= x^5-1
\end{eulerprompt}
\begin{euleroutput}
  
                                   5
                                  x  - 1
  
\end{euleroutput}
\begin{eulercomment}
plot dari ekspresi diatas 
\end{eulercomment}
\begin{eulerprompt}
>aspect(2); plot2d(expr,-1,1):
\end{eulerprompt}
\begin{eulercomment}
contoh 1
\end{eulercomment}
\begin{eulerprompt}
>expr := "x^10-x-5"
\end{eulerprompt}
\begin{euleroutput}
  x^10-x-5
\end{euleroutput}
\begin{eulerprompt}
>aspect(2) ; plot2d(expr,-1,1):
\end{eulerprompt}
\begin{eulercomment}
menggunakan variabel lokal
\end{eulercomment}
\begin{eulercomment}
Ekspresi dapat dievaluasi secara numerik. Variabel x,y,z ditetapkan
secara otomatis. Variabel lain dapat ditetapkan berdasarkan parameter
yang ditetapkan( variabel lokal ) atau melalui variabel global.
variabel global adalah variabel yang selalu bisa diakses kapan pun dan
di mana pun.
\end{eulercomment}
\begin{eulerprompt}
>expr &= a*x^5
\end{eulerprompt}
\begin{euleroutput}
  
                                      5
                                   a x
  
\end{euleroutput}
\begin{eulercomment}
menggunakan variabel global 
\end{eulercomment}
\begin{eulerprompt}
>a=6; expr(2.5)
\end{eulerprompt}
\begin{euleroutput}
  585.9375
\end{euleroutput}
\begin{eulercomment}
menggunakan variabel lokal
\end{eulercomment}
\begin{eulerprompt}
>expr(2.5,a=6)
\end{eulerprompt}
\begin{euleroutput}
  585.9375
\end{euleroutput}
\begin{eulercomment}
evaluasi langsung
\end{eulercomment}
\begin{eulerprompt}
>"a*x^5"(3,4)
\end{eulerprompt}
\begin{euleroutput}
  1458
\end{euleroutput}
\begin{eulercomment}
Oleh karena itu, banyak algoritma EMT yang dapat menggunakan ekspresi
dalam x, bukan fungsi. Namun jika parameter tambahan yang tidak
bersifat global dilibatkan, fungsi harus diutamakan.

menggunakan variabel  global "a"
\end{eulercomment}
\begin{eulerprompt}
>a=5; plot2d("a*x^3-x",0,1):
>function f(x,a) := a*x^3-x
\end{eulerprompt}
\begin{eulercomment}
gunakan "a=6" sebagai parameter
\end{eulercomment}
\begin{eulerprompt}
>plot2d("f",0,1;6):
\end{eulerprompt}
\begin{eulercomment}
alternatif lain
\end{eulercomment}
\begin{eulerprompt}
>plot2d(\{\{"f",6\}\},0,1):
\end{eulerprompt}
\begin{eulercomment}
alternatif lain 
\end{eulercomment}
\begin{eulerprompt}
>plot2d("f(x,6)",0,1):
\end{eulerprompt}
\eulerheading{Sub Bab 3}
\begin{eulercomment}
Menggambar Fungsi Simbolik

Fungsi Plot yang paling penting untuk plot planar adalah plot2d().
Fungsi ini diimplementasikan dalam bahasa Euler dalam file "plot.e",
yang dimuat diawal program.

plot2d() menerima ekspresi, fungsi, dan data.

Rentang plot diatur dengan parameter yang ditetapkan ssbagai berikut\\
- a,b: rentang x (default -2,2)\\
- -c,d: rentang y (default: skala dengan nilai)\\
- r: alternatifnya radius di sekitar pusat plot\\
- cx,cy: koordinat pusat plot (default 0,0)

Keterangan:(menggambar grafik fungsi satu variabel yang fungsinya
didefinisikan sebagai fungsi simbolik)\\
- \&: untuk menampilkan variabel pada teks

Berikut adalah beberapa contoh menggunakan fungsi. Seperti biasa di
EMT, fungsi yang berfungsi untuk fungsi atau ekspresi lain, jadi kita
dapat meneruskan parameter tambahan (selain x) yang bukan variabel
global ke fungsi dengan parameter titik koma atau dengan koleksi
panggilan.
\end{eulercomment}
\begin{eulerprompt}
>plot2d("f",0,1;0.4): // plot with a=0.4
>plot2d(\{\{"f",0.2\}\},0,1); 
>plot2d(\{\{"f(x,b)",b=0.1\}\},0,1):
>function f(x) := x^3-x;...
>plot2d("f",r=1):
>plot2d("exp(-a*x^2)/a"):
\end{eulerprompt}
\begin{eulercomment}
Berikut merupakan ringkasan dari fungsi yang diterima\\
- ekspresi atau ekspresi simbolik dalam x\\
- fungsi atau fungsi simbolis dengan nama sebagai "f"\\
- fungsi simbolis hanya dengan nama f\\
\end{eulercomment}
\begin{eulerttcomment}
 
\end{eulerttcomment}
\begin{eulercomment}
Fungsi plot2d() juga menerima fungsi simbolis. Untuk fungsi simbolis,
hanya nama saja yang berfungsi.
\end{eulercomment}
\begin{eulerprompt}
>function f(x) &= diff(x^x,x)
\end{eulerprompt}
\begin{euleroutput}
  
                              x
                             x  (log(x) + 1)
  
\end{euleroutput}
\begin{eulerprompt}
>plot2d(f,0,2):
>$&expr = sin (x)*exp(-x)
>plot2d(expr,0,3pi):
>plot2d("cos(x)","sin(3*x)"):
\end{eulerprompt}
\eulerheading{Sub Bab 4 }
\begin{eulercomment}
Menggambar Fungsi Numerik 

Fungsi Numerik adalah sebuah fungsi dengan himpunan bilangan cacah
sebagai domain dan himpunan mendasar yang melibatkan hubungan
matematis antara bilangan yang menjadi domain dan bilangan sebagai
kodomain.
\end{eulercomment}
\begin{eulerprompt}
> 
\end{eulerprompt}
\begin{eulercomment}
Fungsi numerik  memiliki  1  atau  lebih  variabel  independen, yang
sering dilambangkan sebagai "X". Variabel X adalah nilai atau
parameter yang dapat berubah, dan fungsi numerik menggambarkan
bagaimana variabel ini memengaruhi variabel dependen. Variabel
dependen adalah hasil perhitungan atau keluaran dari fungsi numerik
yang bergantung pada nilai atau perubahan dalam variabel independen.

\end{eulercomment}
\begin{eulercomment}
Dalam EMT cara mendefinisikan fungsi menggunakan syntak function.
untuk mendefinisikan fungsi numerik menggunakan tanda ":="

Fungsi  numerik  menjelaskan bagaimana bilangan  dalam  domain
berhubungan dengan bilangan sebagai kodomain, biasanya diberikan dalam
bentuk rumus matematik(persamaan) atau aturan yang memetakan setiap
domain kedalam kodomain yang sesuai. contoh:

f(x)=2x+3
\end{eulercomment}
\begin{eulerprompt}
> 
\end{eulerprompt}
\begin{eulercomment}
(x)(variabel dependen) adalah fungsi yang memetakan setiap nilai
x(variabel independen)kedalam nilai 2x+3. Terdapat berbagai jenis
fungsi yang termasuk ke dalam fungsi numerik, diantaranya:

Fungsi linier dengan bentuk umum\\
f (x) = ax + b
\end{eulercomment}
\begin{eulercomment}
Fungsi kuadrat dengan bentuk umum

f (x) = ax2 + bx + c
\end{eulercomment}
\begin{eulercomment}
Fungsi eksponensial dengan bentuk umum

f (x) = ax
\end{eulercomment}
\begin{eulercomment}
Fungsi logaritma dengan bentuk umum

f (x) = log a(x)

\end{eulercomment}
\begin{eulercomment}
Fungsi trigonometri dengan bentuk umum

f (x) = sin(x), f (x) = cos(x)

\end{eulercomment}
\begin{eulercomment}
Salah satu  cara  yang  umum  digunakan  untuk  memvisualisasikan
fungsi numerik adalah dengan menggambar grafiknya. Grafik ini
menggambarkan bagaimana variabel dependen berubah seiring perubahan
variabel independen dan membantu dalam memahami sifat-sifat fungsi,
seperti titik ekstrim
\end{eulercomment}
\eulersubheading{Contoh soal}
\begin{eulerprompt}
>function r(x):= abs(x-10)
>function s(x):= abs(sin(x))
>r(-5)
\end{eulerprompt}
\begin{euleroutput}
  15
\end{euleroutput}
\begin{eulerprompt}
>function t(x):=log(x*(2+sin(x/1000)))
>function u(x):=integrate("(sin(x)*exp(-x^2)"0,x)
>function v(x):=logbase((x^2),2)
>plot2d("v"):
>plot2d("s"):
>plot2d("t",-2,2):
>function P(x):=x*cos(x)
>plot2d("P",-2*pi,2*pi):
\end{eulerprompt}
\begin{eulercomment}
Fungsi plot2d() adalah fungsi serbaguna untuk membuat grafik dalam
bidang (grafik 2D). Fungsi ini dapat digunakan untuk membuat grafik
fungsi-fungsi satu variabel, grafik data,  kurva-kurva  dalam  bidang,
grafik batang (bar plots), grid dari bilangan kompleks, dan grafik
implisit dari fungsi dua variabel.

Parameter\\
x,y : persamaan, fungsi, atau vektor data a,b,c,d : area plot (default
a=-2, b=2)\\
r  :  jika  r  diatur,  maka  a=cx-r,  b=cx+r,  c=cy-r,  d=cy+r r bisa
berupa vektor [rx,ry] atau vektor [rx1,rx2,ry1,ry2]. xmin,xmax :
rentang parameter untuk kurva\\
auto : tentukan rentang y secara otomatis (default)\\
square : jika benar, mencoba menjaga rentang x-y tetap persegi n :
jumlah interval (default adalah adaptif)\\
grid : 0 = tanpa grid dan label, 1 = hanya sumbu,\\
2 = grid normal (lihat di bawah untuk jumlah garis grid) 3 = di dalam
sumbu\\
4 = tanpa grid\\
5 = grid penuh termasuk margin 6 = tanda di pinggiran\\
7 = hanya sumbu\\
8 = hanya sumbu, sub-ticks frame : 0 = tanpa bingkai\\
framecolor: warna bingkai dan grid\\
margin : angka antara 0 dan 0,4 untuk margin di sekitar plot color :
Warna kurva. Jika ini adalah vektor warna,akan digunakan untuk setiap
baris matriks plot. Dalam  hal grafik titik, harus berupa vektor
kolom. Jika vektor baris atau matriks penuh warna digunakan untuk
grafik titik, akan digunakan untuk setiap titik data.\\
thickness : ketebalan garis untuk kurva

Nilai ini dapat lebih kecil dari 1 untuk garis yang sangat tipis. \\
style: Gaya plot untuk garis, penanda, dan isian.

Untuk titik gunakan\\
"[]", "\textless{}\textgreater{}", ".", "..", "...", "*", "+", " ", "-", "o"\\
"[]", "\textless{}\textgreater{}", "o" (bentuk terisi)\\
"[]w", "\textless{}\textgreater{}w", "ow" (tidak transparan)

Untuk garis gunakan\\
"-", "-", "-.", ".", ".-.", "-.-", "-\textgreater{}"

Untuk poligon terisi atau plot batang gunakan\\
"", "O", "O", "/", "", "/","+", " ", "-", "t"

points : plot titik tunggal sebagai gantinya garis segmen addpoints :
jika benar, plot segmen garis dan titik\\
add : tambahkan plot ke plot yang ada\\
user : aktifkan interaksi pengguna untuk fungsi delta : ukuran langkah
untuk interaksi pengguna\\
bar : plot batang (x adalah batas interval, y adalah nilai interval)
histogram : plot frekuensi x dalam n subinterval\\
distribusi=n : plot distribusi x dengan n subinterval even : gunakan
nilai antar untuk histogram otomatis. steps : plot fungsi sebagai
fungsi langkah (steps=1,2)\\
adaptive : gunakan plot adaptif (n adalah jumlah minimal langkah)
level : plot garis level dari fungsi implisit dua variabel\\
outline : menggambar batas rentang level.
\end{eulercomment}
\begin{eulerprompt}
>function s(x):=(x-10)
>function r(x):=abs(sin(x))
>s(-5)
\end{eulerprompt}
\begin{euleroutput}
  -15
\end{euleroutput}
\begin{eulerprompt}
>function t(x):=log(x*(2+sin(x/1000)))
>function u(x):=integrate("(sin(x)*exp(-x^2)"),0,x)
>function v(x):=logbase((x^2),2)
>plot2d("v"):
>plot2d("s"):
>function P(x):=x*cos(x)
>plot2d("P", -2*pi,2*pi):
\end{eulerprompt}
\eulerheading{Sub Bab 5 }
\begin{eulercomment}
Menggambar Beberapa Kurva Sekaligus 


Dalam subtopik ini, kita akan membahas mengenai cara menggambar
beberapa kurva sekaligus. Dalam hal ini kita dapat menggambar beberapa
kurva dalam jendela grafik yang berbeda secara bersama-sama. Untuk
membuat ini kita dapat menggunakan perintah figure(). Berikut contoh
dari menggambar beberapa kurva sekaligus

Menggambar plot fungsi\\
\end{eulercomment}
\begin{eulerformula}
\[
x^n, 1 \leq n \leq 4
\]
\end{eulerformula}
\begin{eulerprompt}
>reset;
>figure(2,2);...
>for n=1 to 4; figure(n); plot2d("x^"+n); end;...
>figure(0):
\end{eulerprompt}
\begin{eulercomment}
Penjelasan sintaks dari plot fungsi

\end{eulercomment}
\begin{eulerformula}
\[
x^n,  1 \leq n \leq 4
\]
\end{eulerformula}
\begin{eulercomment}
- reset;\\
Perintah ini berguna untuk menghapus grafik yang telah ada sebelumnya,
sehingga kita dapat memulai dari awal untuk menggambar grafik\\
- figure(2x2);\\
Perintah figure() digunakan untuk membuat jendela grafik dengan ukuran\\
axb. Dalam kasus ini perintah figure(2,2) memiliki makna bahwa jendela
grafik yang dibuat berukuran 2x2. Artinya, akan ada empat jendela
grafik yang akan ditampilkan dengan tata letak 2 baris dan 2 kolom.\\
- for n=1 to 4;\\
Perintah ini digunakan untuk melakukan pengulangan (looping) perintah
sebanyak empat kali, yaitu dari 1 hingga 4.\\
- figure(n);\\
Perintah ini digunakan untuk beralih dari jendela grafik satu ke
jendela grafik lainnya (jendela grafik ke-n).\\
- plot2d("x\textasciicircum{}"+n);\\
Perintah plot2d() digunakan untuk membuat plot fungsi matematika.\\
Dalam hal ini fungsi yang diplot adalah x\textasciicircum{}n, di mana n adalah nilai
dari variabel yang sedang diulang. Dengan kata lain, ini akan membuat\\
plot dari x\textasciicircum{}1, x\textasciicircum{}2, x\textasciicircum{}3, dan x\textasciicircum{}4 dalam jendela grafik yang sesuai\\
- end;\\
Perintah ini menandakan akhir dari looping.\\
- figure(0);\\
Perintah ini digunakan untuk beralih kembali ke jendela grafik utama.
\end{eulercomment}
\begin{eulercomment}
Dari sini dapat kita perhatikan untuk membuat kurva fungsi x\textasciicircum{}n (x
pangkat n) perintahnya tidak ditulis dengan (x\textasciicircum{}n) melainkan ditulis
dengan ("x\textasciicircum{}"+n). Tanda petik dua ("...") digunakan untuk
mengidentifikasi bahwa teks tersebut merupakan ekspresi matematika.\\
Sedangkan tanda (+) digunakan untuk menggabungkan string dengan nilai
yang berubah-ubah atau variabel.

Contoh lain:\\
Menggambar plot fungsi\\
\end{eulercomment}
\begin{eulerformula}
\[
f(x)=x^3-x, -2<x<2
\]
\end{eulerformula}
\begin{eulerprompt}
>reset;
>figure(3,3);...
>for k=1:9; figure(k); plot2d("x^3-x",-2,2,grid=k); end;...
>figure(0):
\end{eulerprompt}
\begin{eulerttcomment}
 Penjelasan sintaks dari plot fungsi
\end{eulerttcomment}
\begin{eulerformula}
\[
f(x)=x^3-x, -2<x<2
\]
\end{eulerformula}
\begin{eulercomment}
- reset;\\
Perintah ini berguna untuk menghapus grafik yang telah ada sebelumnya,
sehingga kita dapat memulai dari awal untuk menggambar grafik\\
- figure (3,3);\\
Perintah ini digunakan untuk membuat jendela grafik dengan ukuran 3x3.
Artinya, akan ada empat jendela grafik yang akan ditampilkan dengan
tata letak 3 baris dan 3 kolom.\\
- for k=1:9;\\
Perintah ini digunakan untuk melakukan pengulangan (looping) perintah
sebanyak sembilan kali.\\
- figure(n);\\
Perintah ini digunakan untuk beralih dari jendela grafik satu ke\\
\end{eulercomment}
\begin{eulerttcomment}
 jendela grafik lainnya (jendela grafik ke-n).
\end{eulerttcomment}
\begin{eulercomment}
- plot2d("x\textasciicircum{}3-x",-2,2,grid=k);\\
Perintah plot2d() digunakan untuk membuat plot fungsi matematika.\\
Dalam hal ini fungsi yang diplot adalah x\textasciicircum{}3-x, dengan batas sumbu x
dari -2 hingga 2. Argumen grid=k digunakan untuk mengaktifkan grid
pada jendela grafik ke-k.\\
- end;\\
Perintah ini menandakan akhir dari looping.\\
- figure(0);\\
Perintah ini digunakan untuk beralih kembali ke jendela grafik utama.

Dari contoh diatas dapat kita perhatikan bahwa tampilan plot dari yang
ke-1 hingga ke-9 memiliki tampilan yang berbeda-beda. Dalam EMT
memiliki berbagai gaya plot 2D yang dapat dijalankan menggunakan
perintah grid=n dimana n adalah jumlah langkah minimal. Setiap nilai n
memiliki tampilan plot adaptif yang berbeda dalam plot 2D, diantaranya
yaitu:\\
0 : tidak ada grid (kisi), frame, sumbu, dan label, hanya kurva saja\\
1 : dengan sumbu, label-label sumbu di luar frame jendela grafik\\
2 : tampilan default\\
3 : dengan grid pada sumbu x dan y, label-label sumbu berada di dalam
jendela grafik\\
4 : tidak ada grid (kisi), sumbu x dan y, dan label berada di luar
frame jendela grafik\\
5 : tampilan default tanpa margin di sekitar plot\\
6 : hanya dengan sumbu x y dan label, tanpa grid\\
7 : hanya dengan sumbu x y dan tanda-tanda pada sumbu.\\
8 : hanya dengan sumbu dan tanda-tanda pada sumbu, dengan tanda-tanda
yang lebih halus pada sumbu.\\
9 : tampilan default dengan tanda-tanda kecil di dalam jendela\\
10: hanya dengan sumbu-sumbu, tanpa tanda

Contoh lain:\\
Menggambar plot fungsi\\
\end{eulercomment}
\begin{eulerformula}
\[
g(x)=2x^3-x
\]
\end{eulerformula}
\begin{eulerprompt}
>reset;
>aspect(1.2);
>figure(3,4); ...
> figure(2); plot2d("2x^3-x",grid=1); ... // x-y-axis
> figure(3); plot2d("2x^3-x",grid=2); ... // default ticks
>figure(4); plot2d("2x^3-x",grid=3); ... // x-y- axis with labels inside
> figure(5); plot2d("2x^3-x",grid=4); ... // no ticks, only labels
>figure(6); plot2d("2x^3-x",grid=5); ... // default, but no margin
>figure(7); plot2d("2x^3-x",grid=6); ... // axes only
>figure(8); plot2d("2x^3-x",grid=7); ... // axes only, ticks at axis
>figure(9); plot2d("2x^3-x",grid=8); ... // axes only, finer ticks at axis
>figure(10); plot2d("2x^3-x",grid=9); ... // default, small ticks inside
>figure(11); plot2d("2x^3-x",grid=10); ...// no ticks, axes only
>figure(0):
\end{eulerprompt}
\begin{eulercomment}
Penjelasan sintaks dari plot fungsi\\
\end{eulercomment}
\begin{eulerformula}
\[
g(x)=2x^3-x
\]
\end{eulerformula}
\begin{eulercomment}
- aspect(1.2);\\
Perintah aspect() digunakan untuk mengatur rasio aspek dari jendela
grafik. Hal ini berarti perintah aspect(1.2); akan menghasilkan plot
dengan perbandingan rasio panjang dan lebar 2:1.\\
- figure(3,4);\\
Perintah ini digunakan untuk membuat jendela grafik dengan ukuran 3x4.\\
Jadi, akan ada total 12 jendela grafik yang akan ditampilkan dalam
tata letak 3 baris dan 4 kolom.\\
- figure(1); plot2d("x\textasciicircum{}3-x",grid=0); ...\\
Adalah perintah untuk beralih ke jendela grafik pertama dan menggambar
plot dari fungsi x\textasciicircum{}3 - x tanpa grid, frame, atau sumbu.\\
- figure(2); plot2d("x\textasciicircum{}3-x",grid=1); ...\\
Adalah perintah untuk beralih ke jendela grafik kedua dan menggambar
plot dari fungsi x\textasciicircum{}3 - x dengan grid hanya pada sumbu x dan y.\\
- figure(3); plot2d("x\textasciicircum{}3-x",grid=2); ...\\
Adalah perintah untuk beralih ke jendela grafik ketiga dan menggambar
plot dari fungsi x\textasciicircum{}3 - x dengan tampilan default, termasuk tanda-tanda
default pada sumbu.\\
- figure(4); plot2d("x\textasciicircum{}3-x",grid=3); ...\\
Adalah perintah untuk beralih ke jendela grafik keempat dan menggambar
plot dari fungsi x\textasciicircum{}3 - x dengan grid pada sumbu x dan y, serta
label-label sumbu yang ada di dalam jendela.\\
- figure(5); plot2d("x\textasciicircum{}3-x",grid=4); ...\\
Adalah perintah untuk beralih ke jendela grafik kelima dan menggambar
plot dari fungsi x\textasciicircum{}3 - x tanpa tanda-tanda sumbu, hanya label-label
yang ada.\\
- figure(6); plot2d("x\textasciicircum{}3-x",grid=5); ...\\
Adalah perintah untuk beralih ke jendela grafik keenam dan menggambar
plot dari fungsi x\textasciicircum{}3 - x dengan tampilan default, tetapi tanpa margin
di sekitar plot.\\
- figure(7); plot2d("x\textasciicircum{}3-x",grid=6); ...\\
Adalah perintah untuk beralih ke jendela grafik ketujuh dan menggambar
plot dari fungsi x\textasciicircum{}3 - x hanya dengan sumbu-sumbu (tanpa grid atau
label).\\
- figure(8); plot2d("x\textasciicircum{}3-x",grid=7); ...\\
Adalah perintah untuk beralih ke jendela grafik kedelapan dan
menggambar plot dari fungsi x\textasciicircum{}3 - x hanya dengan sumbu-sumbu dan
tanda-tanda pada sumbu.\\
- figure(9); plot2d("x\textasciicircum{}3-x",grid=8); ...\\
Adalah perintah untuk beralih ke jendela grafik kesembilan dan
menggambar plot dari fungsi x\textasciicircum{}3 - x hanya dengan sumbu-sumbu dan
tanda-tanda pada sumbu, dengan tanda-tanda yang lebih halus pada
sumbu.\\
- figure(10); plot2d("x\textasciicircum{}3-x",grid=9); ...\\
Adalah perintah untuk beralih ke jendela grafik kesepuluh dan
menggambar plot dari fungsi x\textasciicircum{}3 - x dengan tanda-tanda default kecil
di dalam jendela.\\
- figure(11); plot2d("x\textasciicircum{}3-x",grid=10); ...\\
Adalah perintah untuk beralih ke jendela grafik kesebelas dan
menggambar plot dari fungsi x\textasciicircum{}3 - x hanya dengan sumbu-sumbu, tanpa
tanda-tanda.\\
- figure(0);\\
Adalah perintah untuk beralih kembali ke jendela grafik utama atau
jendela grafik dengan nomor 0 setelah semua perintah dalam urutan
selesai dieksekusi.

Dari ketiga contoh di atas, dapat kita katakan bahwa untuk menggambar
beberapa kurva sekaligus itu dapat dilakukan dengan satu baris
perintah ataupun dengan cara mendefinisikannya 1 per 1.

Terlihat beberapa jenis grid memiliki tampilan yang mirip atau sama,
seperti 1 dan 2, 2 dan 5, 4 dan 9, 7 dan 8, untuk dapat membedakannya
secara lebih jelas, ubah grid dari contoh di bawah ini.
\end{eulercomment}
\begin{eulerprompt}
>reset;
>aspect(1.3);
>figure(1,3);...
>figure (1); plot2d("x^2*exp(-x)",0,10);...
>figure (2); plot2d("2*exp(x)",-5,5);...
>figure (3); plot2d("exp(x^2)",-2,2);...
>figure (0):
\end{eulerprompt}
\begin{eulercomment}
Contoh lain:
\end{eulercomment}
\begin{eulerprompt}
>reset;
>aspect(3/4);
>figure(2,1);...
>for a=1:2; figure(a); plot2d("2*x*log(x^2)",0,3,grid=a); end;...
>figure(0):
\end{eulerprompt}
\eulerheading{Sub Bab 6 }
\begin{eulercomment}
Menggambar Beberapa Kurva pada bidang koordinat yang sama 

Plot lebih dari satu fungsi (multiple function) ke dalam satu jendela
dapat dilakukan dengan berbagai cara. Salah satu caranya adalah
menggunakan \textgreater{}add untuk beberapa panggilan ke plot2d secara
keseluruhan, kecuali panggilan pertama.

Berikut contohnya:\\
menggambar kurva\\
\end{eulercomment}
\begin{eulerformula}
\[
 f(x)=cos(x)
\]
\end{eulerformula}
\begin{eulerformula}
\[
f(x)= x^2
\]
\end{eulerformula}
\begin{eulerprompt}
>aspect(); plot2d("cos(x)",r=3); plot2d("x^2",style=".",>add):
\end{eulerprompt}
\begin{eulerformula}
\[
f(x)=cos(x)-1
\]
\end{eulerformula}
\begin{eulerformula}
\[
f(x)= sin(x)-1
\]
\end{eulerformula}
\begin{eulerprompt}
>aspect(2); plot2d("cos(x)-1",-1,6); plot2d("sin(x)-1",style="--",>add):
\end{eulerprompt}
\begin{eulercomment}
Selain menggunakan \textgreater{}add kita juga bisa menambahkannya secara langsung

Berikut contohnya:\\
Menggambar kurva\\
\end{eulercomment}
\begin{eulerformula}
\[
f(x)= 2x+1
\]
\end{eulerformula}
\begin{eulerformula}
\[
f(x)= -2x+1
\]
\end{eulerformula}
\begin{eulerprompt}
>plot2d(["2x+1","x"],0,8):
\end{eulerprompt}
\begin{eulerformula}
\[
f(x)=sin(2x)
\]
\end{eulerformula}
\begin{eulerformula}
\[
f(x)=cos(3x)
\]
\end{eulerformula}
\begin{eulerprompt}
>aspect(1.5); plot2d(["sin(2x)","cos(3x)"],0,8):
\end{eulerprompt}
\begin{eulercomment}
Kegunaan \textgreater{}add yang lain juga bisa untuk menambahkan titik pada kurva.

Berikut contohnya:\\
Menambahkan sebuah titik di\\
\end{eulercomment}
\begin{eulerformula}
\[
f(x)= x+4
\]
\end{eulerformula}
\begin{eulerprompt}
>aspect(); plot2d("x+4",-2,5,); plot2d(2,6,>points,>add):
\end{eulerprompt}
\begin{eulercomment}
Kita juga bisa mencari titik perpotongan dengan cara berikut:

\end{eulercomment}
\begin{eulerformula}
\[
sin(x)=2x
\]
\end{eulerformula}
\begin{eulerprompt}
>plot2d(["sin(x)","2x"],r=2,cx=1,cy=1, ...
>  color=[black,blue],style=["-","."], ...
>  grid=1);
>x0=solve("sin(x)-2x",1);  ...
>  plot2d(x0,x0,>points,>add);  ...
>  label("sin(x) = 2x",x0,x0,pos="cl",offset=20):
>function f(x,a) := x^2+a*x-x/a; ...
>plot2d("f",-10,10;1,title="a=1"):
> plot2d(\{\{"f",1\}\},-10,10); ...
>for a=1:10; plot2d(\{\{"f",a\}\},>add); end:
>function f(x,a) := x^2*exp(-x^2/a); ...
>plot2d("f",-10,10;5,thickness=2,title="a=5"):
>plot2d(\{\{"f",1\}\},-8,8); ...
>for a=2:5; plot2d(\{\{"f",a\}\},>add,thickness=2); end:
>aspect(2.1); &plot2d(1/x,[x,-1,1]):
>x=linspace(-1,1,50);...
>plot2d("1/x"):
\end{eulerprompt}
\eulerheading{Sub Bab 7 }
\begin{eulercomment}
Menuliskan Label koordinat,label kurva, dan keterangan 

kurva(legend) Dalam EMT, untuk menambahkan judul dapat dilakukan
dengan title="..."\\
untuk menambahkan sumbu x dan sumbu y dapat dilakukan dengan x1="...",
y1="..."\\
sebagai contoh:
\end{eulercomment}
\begin{eulerprompt}
>plot2d("x^2-4*x"):
\end{eulerprompt}
\begin{eulercomment}
untuk menambahkan judul dapat dilakukan dengan title="..."\\
untuk menambahkan sumbu x dan sumbu y dapat dilakukan dengan x1="...",
y1="..."
\end{eulercomment}
\begin{eulerprompt}
>plot2d("x^2-4*x",title="FUNGSI y=x^2-4*x",yl="Sumbu y",xl="Sumbu x"):
\end{eulerprompt}
\begin{eulercomment}
Selain itu juga dapat dengan cara lain seperti contoh berikut:
\end{eulercomment}
\begin{eulerprompt}
>expr := "x^3-x"; ...
>  plot2d(expr,title="y="+expr,xl="Sumbu x",yl="Sumbu y"); ...
>  label("(1,0)",1,0);  label("Max",E,expr(E),pos="lc"): 
\end{eulerprompt}
\eulerheading{Sub Bab 8 }
\begin{eulercomment}
Mengatur ukuran gambar,format(style),dan warna kurva 


Untuk mengubah ukuran, dapat dilakukan dengan menggunakan
aspect="...", semakin besar nilai aspect, maka ukuran kurva akan
semakin kecil, begitupun sebaliknya

untuk mengganti style, dapat dipilih dengan berbagai pilihan\\
style="...", dapat dipilih dari, misal : "-","\_',"-.",".-.","-.-".

untuk warna dapat dipilih sebagai salah satu warna default\\
color="...", warna default= red,green,blue,yellow, dll

sebagai contoh:
\end{eulercomment}
\begin{eulerprompt}
>aspect(1); plot2d("exp(x^2-3)"):
\end{eulerprompt}
\begin{eulercomment}
ukuran kurva dapat diganti dengan mengganti nilai aspect="...",
semakin besar nilai aspect, maka ukuran kurva akan semakin kecil Untuk
mengganti warna dapat ditambahkan dengan color="...", sedangkan untuk
mengganti format(style) dapat dilakukan dengan menambahkan style="..."
\end{eulercomment}
\begin{eulerprompt}
>aspect(2); plot2d("exp(x^2-3)", color=red, style="--"):
\end{eulerprompt}
\begin{eulercomment}
Berikut adalah tampilan warna EMT yang telah ditentukan
\end{eulercomment}
\begin{eulerprompt}
>aspect (1) ; columnsplot (ones(1,16),lab=0:15,grid=0, color=0:15) :
\end{eulerprompt}
\begin{eulercomment}
selain menggunakan warna default, untuk mengubah warna dapat juga
dengan menggunakan kode warna di atas\\
sebagai contoh:
\end{eulercomment}
\begin{eulerprompt}
>aspect(1); plot2d("exp(x^3+2*x)",r=3, color=1, style="--"):
\end{eulerprompt}
\eulerheading{Sub Bab 9 }
\begin{eulercomment}
Menggambar Sekumpulan Kurva dalam satu perintah plot2d. 


Dalam pembahasan sub-bab 9 kali ini akan membahas mengenai bagaimana
menggambar sekumpulan kurva dalam satu perintah plot2d. Menggambar
sekumpulan kurva dalam satu perintah plot2d adalah teknik yang
digunakan untuk memvisualisasikan beberapa fungsi dalam satu grafik.
Ini memudahkan perbandingan antara beberapa kurva.\\
\end{eulercomment}
\eulersubheading{}
\eulersubheading{Contoh}
\begin{eulerprompt}
>plot2d(["x^2","2*x"],-3,3):
\end{eulerprompt}
\begin{eulerttcomment}
 - Dalam contoh ini, merupakan gambar dua kurva sekaligus, yaitu x^2
\end{eulerttcomment}
\begin{eulercomment}
dan 2x, pada rentang -3 hingga 3.

\end{eulercomment}
\begin{eulerttcomment}
 - Hasilnya akan menunjukkan grafik dari kedua fungsi tersebut, dan
\end{eulerttcomment}
\begin{eulercomment}
titik-titik potongan antara keduanya adalah solusi dari persamaan
kuadrat.
\end{eulercomment}
\begin{eulerprompt}
>plot2d(["sin(x)","cos(x)"],0,2pi):
\end{eulerprompt}
\begin{eulercomment}
- Pada contoh ini, merupakan gambar dua fungsi trigonometri, sin(x)
dan cos(x), pada rentang 0 hingga 2p.

- Ini akan menghasilkan dua grafik yang memperlihatkan hubungan antara
sin(x) dan cos(x) dalam rentang tersebut.
\end{eulercomment}
\begin{eulerprompt}
>plot2d(["sin(x)","cos(x)"],0,2pi,color=red:green):
\end{eulerprompt}
\begin{eulercomment}
Sama seperti contoh kedua, gambar sin(x) dan cos(x) pada rentang 0
hingga 2phi, tetapi Anda juga memberikan warna yang berbeda pada kedua
grafik (sin(x) berwarna merah dan cos(x) berwana hijau.
\end{eulercomment}
\begin{eulerprompt}
>plot2d(["sin(x)","cos(x)"],xmin=0,xmax=2pi):
\end{eulerprompt}
\begin{eulercomment}
Dalam contoh ini, menggunakan parameter `xmin` dan `xmax` untuk
mengatur rentang tampilan grafik pada 0 hingga 2p.

\end{eulercomment}
\begin{eulerprompt}
> 
>plot2d(["cos(x)","sin(3*x)"],xmin=0,xmax=2pi):
\end{eulerprompt}
\begin{eulercomment}
- ini Merupakan gambar dua fungsi, yaitu cos(x) dan sin(3*x), dalam
rentang 0 hingga 2p.\\
\end{eulercomment}
\begin{eulerttcomment}
  
\end{eulerttcomment}
\begin{eulercomment}
- Penjelasan mencakup konsep bahwa grafik berulang dalam rentang
tertentu karena fungsi-fungsi ini memiliki frekuensi, periode, dan
amplitudo yang berbeda.
\end{eulercomment}
\begin{eulerprompt}
>plot2d("cos(x)","sin(3*x)",xmin=0,xmax=2pi):
\end{eulerprompt}
\begin{eulercomment}
Sintaks diatas lebih menjelaskan bahaimana hubungan periodik grafik
fungsi dari 2 fungsi yaitu cos x dan sin 3x dari rentang khusus dimana
xmin dari 0 sampai 2pi, hal tersebut dapat terjadi karena fungsi cos x
dan fungsi sin 3x memiliki frekuensi, periode, dan amplitudo yang
berbeda. grafik akan berulang pada rentang tertentu dan menghasilkan
sebuah pola.
\end{eulercomment}
\begin{eulerprompt}
>x=linspace(0,2pi,1000); plot2d(sin(5x),cos(7x)):
\end{eulerprompt}
\begin{eulercomment}
sintaks linspace digunakan untuk menghasilkan vektor x dari rentang
yang telah ditentukan yaitu o sampai 2pi yang berisi 1000 nilai yang
teratur
\end{eulercomment}
\begin{eulerprompt}
>a:=5.6; f &= exp(-a*x^2)/a;
>plot2d(f,r=1,thickness=2):
\end{eulerprompt}
\begin{eulercomment}
- Fungsi f(x) yang merupakan hasil dari ekspresi exp(-a*x\textasciicircum{}2)/a. Fungsi
ini memiliki parameter a yang bergantung pada nilai yang diterapkan
sebelumnya.\\
- Menggunakan perintah plot2d untuk menggambar grafik dari fungsi
f(x).\\
Parameter r digunakan untuk mengatur rentang plot dan parameter
thickness digunakan untuk mengatur ketebalan garis grafik.
\end{eulercomment}
\begin{eulerprompt}
>plot2d(&diff(f,x),>add,style="--",color=red):
\end{eulerprompt}
\begin{eulercomment}
ini adalah grafik fungsi f dan grafik turunan pertama dari fungsi f.
sintaks r=1 digunakan untuk mengatur rentang yang akan ditampilkan
pada plot, r=1 berarti rentang dari -1 sampai 1.\\
sintaks \textgreater{}add digunakan untuk menambahkan grafik kedalam jendela grafik
yang sudah ada sebelumnya.
\end{eulercomment}
\begin{eulerprompt}
>plot2d("x^2",0,1,steps=1,color=red,n=10):
>plot2d("x^2",>add,steps=2,color=blue,n=10):
\end{eulerprompt}
\begin{eulercomment}
sintaks steps digunakan untuk mengatur jumlah langkah atau titik-titik
yang digunakan dalam plot.\\
dan sintaks n digunakan untuk mengatur jumlah step yang akan
digunakan. semakin bayak n, maka bentuk grafik akan semakin mendekati
aslinya.
\end{eulercomment}
\begin{eulerprompt}
>function f(x) &= x^x;
>plot2d(f,r=1,cx=1,cy=1,color=blue,thickness=2);
>plot2d(&diff(f(x),x),>add,color=red,style="-.-"):
\end{eulerprompt}
\begin{eulercomment}
sintaks cx=1, cy=1 digunakan untuk mengatur pusat tampilan grafik,
maka plot akan diatur dengan titik pusat (1,1).
\end{eulercomment}
\begin{eulerprompt}
>plot2d("(1-x)^10",0,1);
>for i=1 to 10; plot2d("bin(10,i)*x^i*(1-x)^(10-i)",>add); end;
>insimg;
\end{eulerprompt}
\begin{eulercomment}
dalam contoh kita menggambar serangkaian plot yang menggambarkan
distribusi binomial dengan berbagai nilai i dari 1 hingga 10. kali ini
kita menggunakan sintaks untuk melakukan looping pada fungsi yang
berasosiasi dengan koefisien binomial dengan kombinasi 10 item. ini
memungkinkan untuk memahami bagaimana distribusu binomial berubah
dengan berbagai parameter.
\end{eulercomment}
\begin{eulerprompt}
>x=linspace(0,1,500);
>n=10; k=(0:n)';
>y=bin(n,k)*x^k*(1-x)^(n-k);
>plot2d(x,y):
\end{eulerprompt}
\begin{eulercomment}
n adalah vektor baris\\
k adalah vektor kolom\\
y adalah matrik dari vektor baris dan vektor kolom tersebut dengan
menggunakanfungsi binomial.
\end{eulercomment}
\begin{eulerprompt}
>x=linspace(0,1,200); y=x^(1:10)'; plot2d(x,y,color=1:10):
>n=(1:10)'; plot2d("x^n",0,1,color=1:10):
>  
>function f(x,a) := 1/a*exp(-x^2/a); ...
>plot2d("f",-10,10;5,thickness=2,title="a=5"):
>plot2d(\{\{"f",1\}\},-10,10); ...
>for a=2:10; plot2d(\{\{"f",a\}\},>add); end:
\end{eulerprompt}
\eulerheading{Sub Bab 10 }
\begin{eulercomment}
Membuat Gambar Kurva yang Bersifat Interaktif 


Kode ini, menggunakan `plot2d` untuk membuat plot dari fungsi
matematika `2*x\textasciicircum{}3-a*x` dengan parameter `a`. Flag `\textgreater{}user` memungkinkan
interaksi pengguna. Setelah plot ditampilkan, pengguna dapat melakukan
beberapa tindakan interaktif.\\
Saat plot ditampilkan dengan flag `\textgreater{}user`, pengguna dapat melakukan
beberapa tindakan interaktif sebagai berikut:

- Perbesar dengan + atau -: Pengguna dapat memperbesar atau
memperkecil plot dengan menggunakan tombol + atau - pada keyboard.

- Pindahkan Plot dengan Tombol Kursor: Pengguna dapat menggeser plot
dengan menggunakan tombol kursor (panning).\\
- Pilih Jendela Plot dengan Mouse: Pengguna dapat memilih area
tertentu dalam plot dengan menggunakan mouse.

- Atur Ulang Tampilan dengan Spasi: Jika pengguna menekan tombol
spasi, maka tampilan plot akan diatur ulang ke jendela plot.\\
- Keluar dengan Kembali: Jika pengguna menekan tombol kembali, maka
pengguna dapat keluar dari interaksi plot.
\end{eulercomment}
\begin{eulerprompt}
>plot2d(\{\{"2*x^3-a*x",a=1\}\},>user,title="Press any key!"); ...
>insimg;  
> plot2d("exp(x)*sin(x)",user=true, ...
>  title="+/- or cursor keys (return to exit)"):
\end{eulerprompt}
\begin{eulercomment}
Berikut ini menunjukkan cara interaksi pengguna tingkat lanjut

Ini adalah pemanggilan fungsi plot2d yang digunakan untuk membuat plot
dari fungsi matematika exp(x)*sin(x). Parameter user=true menunjukkan
bahwa ini adalah plot yang interaktif, yang berarti pengguna dapat
berinteraksi dengan plot ini.

title="+/- or cursor keys (return to exit)": Ini adalah judul yang
akan ditampilkan di atas plot. Pesan ini memberi petunjuk kepada
pengguna tentang bagaimana mereka dapat berinteraksi dengan plot ini.
Mereka dapat menggunakan tombol + atau - atau tombol kursor untuk
berinteraksi dengan plot, dan tombol return (Enter) untuk keluar dari
interaksi.

Berikut ini menunjukkan cara interaksi pengguna tingkat lanjut:\\
- mousedrag(): Ini adalah fungsi bawaan yang digunakan untuk menunggu
event mouse atau keyboard. Fungsi ini dapat mendeteksi kejadian
seperti klik mouse, pergerakan mouse, atau penekanan tombol.\\
- dragpoints(): Fungsi ini memanfaatkan mousedrag() untuk memungkinkan
pengguna menyeret titik-titik pada plot. Ini berarti pengguna dapat
mengklik dan menarik titik-titik dalam plot sesuai dengan preferensi
mereka.

Kita membutuhkan fungsi plot terlebih dahulu. Sebagai contoh, kita
interpolasi dalam 5 titik dengan polinomial. Fungsi harus diplot ke
area plot tetap.
\end{eulercomment}
\begin{eulerprompt}
>function plotf(xp,yp,select) ...
\end{eulerprompt}
\begin{eulerudf}
    d=interp(xp,yp);
    plot2d("interpval(xp,d,x)";d,xp,r=2);
    plot2d(xp,yp,>points,>add);
    if select>0 then
      plot2d(xp[select],yp[select],color=red,>points,>add);
    endif;
    title("Drag one point, or press space or return!");
  endfunction
\end{eulerudf}
\begin{eulercomment}
Perhatikan parameter titik koma di plot2d (d dan xp), yang diteruskan
ke evaluasi fungsi interp(). Tanpa ini, kita harus menulis fungsi
plotinterp() terlebih dahulu, mengakses nilai secara global.

Sekarang kita menghasilkan beberapa nilai acak, dan membiarkan
pengguna menyeret poin.

kode berikut digunakan untuk menghasilkan beberapa nilai acak t dan
membiarkan pengguna menyeret titik-titik pada plot dengan menggunakan
fungsi dragpoints():
\end{eulercomment}
\begin{eulerprompt}
>t=-1:0.5:1; dragpoints("plotf",t,random(size(t))-0.5):
\end{eulerprompt}
\begin{eulercomment}
Ada juga fungsi, yang memplot fungsi lain tergantung pada vektor
parameter, dan memungkinkan pengguna menyesuaikan parameter ini.

Pertama kita membutuhkan fungsi plot.
\end{eulercomment}
\begin{eulerprompt}
>function plotf([a,b]) := plot2d("exp(a*x)*cos(2pi*b*x)",0,2pi;a,b);
\end{eulerprompt}
\begin{eulercomment}
Kemudian kita membutuhkan nama untuk parameter, nilai awal dan matriks
rentang nx2, opsional baris judul.\\
Ada slider interaktif, yang dapat mengatur nilai oleh pengguna. Fungsi
dragvalues() menyediakan ini.
\end{eulercomment}
\begin{eulerprompt}
>dragvalues("plotf",["a","b"],[-1,2],[[-2,2];[1,10]], ...
>  heading="Drag these values:",hcolor=black):
\end{eulerprompt}
\begin{eulercomment}
Dimungkinkan untuk membatasi nilai yang diseret ke bilangan bulat.
Sebagai contoh, kita menulis fungsi plot, yang memplot polinomial
Taylor derajat n ke fungsi kosinus.
\end{eulercomment}
\begin{eulerprompt}
>function plotf(n) ...
\end{eulerprompt}
\begin{eulerudf}
  plot2d("cos(x)",0,2pi,>square,grid=6);
  plot2d(&"taylor(cos(x),x,0,@n)",color=blue,>add);
  textbox("Taylor polynomial of degree "+n,0.1,0.02,style="t",>left);
  endfunction
\end{eulerudf}
\begin{eulercomment}
Sekarang kami mengizinkan derajat n bervariasi dari 0 hingga 20 dalam
20 pemberhentian. Hasil dragvalues() digunakan untuk memplot sketsa
dengan n ini, dan untuk memasukkan plot ke dalam buku catatan.
\end{eulercomment}
\begin{eulerprompt}
>nd=dragvalues("plotf","degree",3,[0,10],10,y=0.8, ...
>   heading="Drag the value:"); ...
>plotf(nd):
\end{eulerprompt}
\begin{eulercomment}
Berikut ini adalah demonstrasi sederhana dari fungsi tersebut.
Pengguna dapat menggambar di atas jendela plot, meninggalkan jejak
poin.
\end{eulercomment}
\begin{eulerprompt}
>function dragtest ...
\end{eulerprompt}
\begin{eulerudf}
    plot2d(none,r=1,title="Drag with the mouse, or press any key!");
    start=0;
    repeat
      \{flag,m,time\}=mousedrag();
      if flag==0 then return; endif;
      if flag==2 then
        hold on; mark(m[1],m[2]); hold off;
      endif;
    end
  endfunction
\end{eulerudf}
\eulerheading{Sub Bab 11 }
\begin{eulercomment}
Menggambar Kurva Fungsi Parametrik 


Kita telah terbiasa dengan kurva yang didefinisikan oleh sebuah
persamaan yang menghubungkan koordinat x dan y Contohnya\\
\end{eulercomment}
\begin{eulerformula}
\[
y=x^2
\]
\end{eulerformula}
\begin{eulercomment}
Atau\\
\end{eulercomment}
\begin{eulerformula}
\[
x^2+y^2=13
\]
\end{eulerformula}
\begin{eulercomment}
dimana persamaan-persamaan ini tidak dikaitkan dengan panjang kurva s
, waktu t, dan besaran lainnya. Besaran besaran ini disebut parameter\\
persamaan parametrik adalah persamaan yang menyatakan hubungan
variabel x, y dituliskan dengan\\
\end{eulercomment}
\begin{eulerformula}
\[
x=f(t)
\]
\end{eulerformula}
\begin{eulerformula}
\[
y=g(t)
\]
\end{eulerformula}
\begin{eulercomment}
dengan a\textless{}=t\textless{}=b tiap nilai t menentukan titik(x,y) pada kurva. Jadi ,
dengan berubahnya nilai t. titik\\
\end{eulercomment}
\begin{eulerformula}
\[
(x,y) = (f(t),g(t))
\]
\end{eulerformula}
\begin{eulercomment}
bergerak sepanjang kurva yang disebut kurva parametrik


Dalam contoh berikut, kita memplot spiral

\end{eulercomment}
\begin{eulerformula}
\[
\gamma(t) = t \cdot (\cos(2\pi t),\sin(2\pi t))
\]
\end{eulerformula}
\begin{eulercomment}
Kita perlu menggunakan banyak titik untuk tampilan yang halus
\end{eulercomment}
\begin{eulerprompt}
>t=linspace(0,1,1000); ...
>plot2d(t*cos(2*pi*t),t*sin(2*pi*t),r=1):
\end{eulerprompt}
\begin{eulerttcomment}
 r digunakan untuk mengatur radius marker titik-titik yang akan
\end{eulerttcomment}
\begin{eulercomment}
digunakan dalam plot.



Sebagai alternatif, dimungkinkan untuk menggunakan dua ekspresi untuk
kurva. Berikut ini plot kurva yang sama seperti di atas.
\end{eulercomment}
\begin{eulerprompt}
>plot2d("x*cos(2*pi*x)","x*sin(2*pi*x)",xmin=0,xmax=1,r=1):
\end{eulerprompt}
\begin{eulercomment}
Perintah linspace digunakan untuk membuat array nilai yang
terdistribusi secara merata antara dua angka tertentu. Fungsi ini
sangat berguna untuk menentukan rentang nilai yang ingin digunakan
pada sumbu x atau y ketika membuat plot.

\end{eulercomment}
\begin{eulerttcomment}
    0  : Nilai awal dari rentang.
    1  : Nilai akhir dari rentang.
 
\end{eulerttcomment}
\begin{eulercomment}

Perintah linspace akan menghasilkan array dengan n elemen yang
terdistribusi merata antara start dan stop.
\end{eulercomment}
\begin{eulerprompt}
>t=linspace(0,1,1000); r=exp(-t); x=r*cos(2pi*t); y=r*sin(2pi*t);
>plot2d(x,y,r=1):
\end{eulerprompt}
\begin{eulercomment}
exp(-t) menghasilkan nilai yang semakin mendekati nol seiring dengan
pertambahan nilai t, karena eksponensial dari nilai negatif semakin
mendekati nol saat nilai t semakin besar.\\
Jadi, r = exp(-t) memberikan suatu fungsi yang menurun dengan nilai t.
Dalam konteks program ini, r digunakan untuk mengontrol jari-jari dari
kurva spiral dalam plot 2D. Jari-jari ini semakin kecil seiring dengan
pertambahan nilai t, menciptakan efek spiral yang semakin rapat ke
pusat pada bagian ujung kurva.





Pada contoh berikutnya, kita memplot kurvanya

\end{eulercomment}
\begin{eulerformula}
\[
\gamma(t) = (r(t) \cos(t), r(t) \sin(t))
\]
\end{eulerformula}
\begin{eulercomment}
dengan

\end{eulercomment}
\begin{eulerformula}
\[
r(t) = 1 + \dfrac{\sin(3t)}{2}.
\]
\end{eulerformula}
\begin{eulerprompt}
>t=linspace(0,2pi,1000); r=1+sin(3*t)/2; x=r*cos(t); y=r*sin(t); ...
>plot2d(x,y,>filled,fillcolor=red,style="/",r=1.5):
\end{eulerprompt}
\eulersubheading{Contoh lain}
\begin{eulerprompt}
>t=linspace(-3,3,1000); x=2*t+1; y=t^2-1;
>plot2d(x,y,r=8):
>t=linspace(0,2pi,1000); r=3; x=r*cos(t); y=r*sin(t);...
>plot2d(x,y,>filled,fillcolor=green,style="/",r=5):
>t=linspace(-1,1,1000); x=t^2; y=2*t;...
>plot2d(x,y):
>t=linspace(0,2pi,1000); x=3*cos(t); y=2*sin(t);...
>plot2d(x,y,>filled,fillcolor=green,style="/",r=3):
\end{eulerprompt}
\eulerheading{Sub Bab 12 }
\begin{eulercomment}
Menggambar Kurva Fungsi Implisit 


Fungsi implisit adalah fungsi yang memuat lebih dari satu variabel,
berjenis variabel bebas dan variabel terikat yang berada dalam satu
ruas sehingga tidak bisa dipisahkan pada ruas yang berbeda.

Untuk fungsi implisit, harus berupa fungsi atau ekspresi dari
parameter x dan y.

\end{eulercomment}
\begin{eulerformula}
\[
f(x,y)=c
\]
\end{eulerformula}
\begin{eulercomment}
Untuk menggambar himpunan f(x,y)=c untuk satu atau lebih konstanta c,
dapat menggunakan\\
plot2d().

Fungsi implisit juga dapat diisi dengan persamaan tingkat

\end{eulercomment}
\begin{eulerformula}
\[
a<=f(x,y)<=b
\]
\end{eulerformula}
\begin{eulercomment}
Untuk fungsi ini harus berupa matriks 2xn dimana baris pertama berisi
awal dan baris kedua adalah akhir dari setiap interval.

Plot implisit menunjukkan garis level yang menyelesaikan f(x,y)=level,
di mana "level" dapat berupa nilai tunggal atau vektor nilai. Jika
level="auto", akan ada garis level nc, yang akan menyebar antara
fungsi minimum dan maksimum secara merata. Warna yang lebih gelap atau
lebih terang dapat ditambahkan dengan \textgreater{}hue untuk menunjukkan nilai
fungsi. Untuk fungsi implisit, xv harus berupa fungsi atau ekspresi
dari parameter x dan y, atau, sebagai alternatif, xv dapat berupa
matriks nilai.

Euler dapat menandai garis level

\end{eulercomment}
\begin{eulerformula}
\[
f(x,y) = c
\]
\end{eulerformula}
\begin{eulercomment}
dari fungsi apapun.

Untuk menggambar himpunan f(x,y)=c untuk satu atau lebih konstanta c,
Anda dapat menggunakan plot2d() dengan plot implisitnya di dalam
bidang. Parameter untuk c adalah level=c, di mana c dapat berupa
vektor garis level. Selain itu, skema warna dapat digambar di latar
belakang untuk menunjukkan nilai fungsi untuk setiap titik dalam plot.
Parameter "n" menentukan kehalusan plot.

\end{eulercomment}
\eulersubheading{Contoh Soal}
\begin{eulercomment}
\end{eulercomment}
\begin{eulerformula}
\[
x^2+y^2-xy-x = 0
\]
\end{eulerformula}
\begin{eulerprompt}
>aspect(2)
>plot2d("x^2+y^2-x*y-x",r=1.5,level=0,contourcolor=green):
\end{eulerprompt}
\begin{eulerformula}
\[
2x^2+xy+3y^4+y = 0
\]
\end{eulerformula}
\begin{eulerprompt}
>expr := "2*x^2+x*y+3*y^4+y"; // define an expression f(x,y)
>plot2d(expr,level=0,contourcolor=green): // Solutions of f(x,y)=0
>plot2d(expr,level=0:0.5:20,>hue,contourcolor=white,n=200): // nice
\end{eulerprompt}
\begin{eulercomment}
Parameter \textgreater{}hue digunakan untuk memberikan warna pada kontur sesuai
dengan levelnya. Kontur dengan level yang lebih tinggi akan memiliki
warna yang berbeda.
\end{eulercomment}
\begin{eulerprompt}
>plot2d(expr,level=0:0.5:20,>hue,>spectral,n=200,grid=4): // nicer
\end{eulerprompt}
\begin{eulercomment}
\textgreater{}spectral digunakan untuk mengatur palet warna yang akan digunakan
pada kontur. Dalam hal ini, digunakan palet warna "spectral".
\end{eulercomment}
\begin{eulerprompt}
>x=-2:0.05:1; y=x'; z=expr(x,y);
>plot2d(z,level=0,a=-1,b=2,c=-2,d=1,>hue):
>plot2d("x^3-y^2",>contour,>hue,>spectral):
\end{eulerprompt}
\begin{eulercomment}
Perintah \textgreater{}contour adalah cara untuk menghasilkan plot kontur, yaitu
plot yang menunjukkan garis-garis kontur yang mewakili tingkat-tingkat
dari suatu fungsi. Jumlah dan posisi garis kontur akan secara otomatis
diatur oleh Euler Math Toolbox berdasarkan distribusi nilai-nilai
fungsi.

Penggunaan level memungkinkan Anda secara eksplisit menentukan tingkat
kontur yang ingin kita tampilkan pada plot. kita dapat mengatur level
kontur sesuai dengan preferensi kita, dan plot akan menampilkan garis
kontur pada tingkat-tingkat yang kita tentukan.
\end{eulercomment}
\begin{eulerprompt}
>plot2d("x^3-y^2",level=0,contourwidth=3,>add,contourcolor=red):
>z=z+normal(size(z))*0.2;
>plot2d(z,level=0.5,a=-1,b=2,c=-2,d=1):
\end{eulerprompt}
\begin{eulercomment}
normal(size(z)) menghasilkan matriks dengan ukuran yang sama dengan
matriks z, dan setiap elemennya diambil dari distribusi normal standar
(mean 0, deviasi standar 1). Kemudian, matriks z diubah dengan
menambahkan nilai-nilai acak ini, yang telah dikalikan dengan 0.2. Ini
menciptakan variasi acak dalam matriks z.

\end{eulercomment}
\begin{eulerprompt}
>plot2d(expr,level=[0:0.2:5;0.05:0.2:5.05],color=lightgray):
>plot2d("x^2+y^3+x*y",level=1,r=4,n=100):
>plot2d("x^2+2*y^2-x*y",level=0:0.1:10,n=100,contourcolor=white,>hue):
>plot2d(expr,level=[0;1],style="-",color=blue): // 0 <= f(x,y) <= 1
>plot2d("x^4+y^4",r=1.5,level=[0;1],color=blue,style="/"):
>plot2d("x^2+y^3+x*y",level=[0,2,4;1,3,5],style="/",r=2,n=100):
>plot2d("x^2+y^3+x*y",level=-10:20,r=2,style="-",dl=0.1,n=100):
\end{eulerprompt}
\begin{eulercomment}
dl Parameter ini mengatur tingkat penghalusan pada plot. Semakin kecil
nilai ini, semakin halus plotnya.
\end{eulercomment}
\begin{eulerprompt}
>plot2d("sin(x)*cos(y)",r=pi,>hue,>levels,n=100):
>plot2d("(x^2+y^2-1)^3-x^2*y^3",r=1.3, ...
>style="/",color=red,<outline, ...
>level=[-2;0],n=100):
\end{eulerprompt}
\begin{eulercomment}
\textless{}outline: Parameter ini mengatur plot agar hanya memiliki kontur saja
tanpa diisi.



Misal plot solusi dari persamaan

\end{eulercomment}
\begin{eulerformula}
\[
x^3-xy+x^2y^2=6
\]
\end{eulerformula}
\begin{eulerprompt}
>plot2d("x^3-x*y+x^2*y^2",r=6,level=6,n=100):
>plot2d("x^2+y^2-1",level=0):
>plot2d("x^2+y^2-1",r=3,level=0:1:10,n=200):
>plot2d("x^2+y^2-1",r=3,level=0:1:10,>hue,contourcolor=white):
>plot2d("x^2+y^2-1",r=3,level=0:1:20,>hue,>spectral,n=200,grid=4):
>plot2d("x^2+y^2-1",level=0,a=-2,b=2,c=-2,d=2,>hue):
>plot2d("x^2+y^2-1",>contour,>hue,>spectral):
>plot2d("x^2+y^2-1",level=[0:0.2:5;0.05:0.2:5.05],color=lightgray):
>plot2d("x^2+y^2-1",r=2,level=[0;1],style="-",color=blue): // 0 <= f(x,y) <= 1
>plot2d("x^2+y^2",r=1.5,level=[0;1],color=blue,style="/"):
>plot2d("x^2+y^2-1",level=[0,2,4;1,3,5],style="/",r=2,n=100):
>plot2d("x^2+y^2-1",level=-10:20,r=3,style="-",dl=0.1,n=100):
\end{eulerprompt}
\eulerheading{Sub bab 13 }
\begin{eulercomment}
Menggambar Grafik Bilangan Kompleks 


Bilangan kompleks secara visual dapat direpresentasikan sebagai
sepasang angka (a, b) membentuk vektor pada diagram yang disebut
diagram Argand, mewakili yang bidang kompleks. Sumbu-x adalah sumbu
nyata dan sumbu-y adalah sumbu imajiner.

Menggambar kurva fungsi kompleks sendiri adalah proses visualisasi
grafis dari fungsi matematika kompleks (yaitu fungsi yang melibatkan
bilangan kompleks, yaitu bilangan dengan bagian real dan imajiner)
berperilaku dalam koordinas kompleks. Hal tersebut memungkinkan untuk
melihat bagaimana pola, bentuk, dan sifat dari fungsi kompleks
tersebut.

Array bilangan kompleks juga dapat diplot. Kemudian titik-titik grid
akan terhubung. Jika sejumlah garis kisi ditentukan (atau vektor garis
kisi 1x2) dalam argumen cgrid, hanya garis kisi tersebut yang
terlihat.

Matriks bilangan kompleks akan secara otomatis diplot sebagai kisi di
bidang kompleks.

\textgreater{} Definisi fungsi kompleks, mendefinisikan fungsi kompleks yang
dianalisis atau digambarkan. Fungsi ini memiliki variabel kompleks z,
yang melibatkan bagian real dan imajiner.\\
\textgreater{} Selanjutnya kita dapat menggunakan fungsi linspace. Fungsi linspace
sendiri adalah salah satu fungsi yang umum digunakan dalam
pemrograman, terutama dalam konteks pemrograman numerik dan ilmu data.
Ini sering digunakan untuk menghasilkan urutan nilai dalam rentang
tertentu dengan jumlah titik yang sama di antara dua titik ujungnya.
Penggunaannya tidak terbatas pada pemrosesan sinyal atau
elektromagnetik, tetapi bisa digunakan dalam berbagai konteks di mana
Anda perlu membuat urutan nilai.\\
\textgreater{} Penentuan rentang, memilih rentang nilai z yang ingin ditampilkan di
dalam plot. Rentang ini mencakup wilayah kompleks tertentu yang ingin
diamati.\\
\textgreater{} Menggunakan sintaks plot2d.\\
\textgreater{} Penyesuaian plot, mengubah plot sesuai yang diinginkan (mengubah
warna, format (style), dan sebagainya).

Dalam contoh berikut, kami memplot gambar lingkaran satuan di bawah
fungsi eksponensial. Parameter cgrid menyembunyikan beberapa kurva
grid.

\begin{eulercomment}
\eulerheading{Contoh}
\begin{eulerprompt}
>aspect(); r=linspace(0,1,50); a=linspace(0,2pi,80)'; z=r*exp(I*a);...
>plot2d(z,a=-1.25,b=1.25,c=-1.25,d=1.25,cgrid=10):
\end{eulerprompt}
\begin{eulercomment}
Penjelasan sintaks

z       : sebuah ekspresi atau fungsi yang akan digambar dalam
koordinat kompleks.\\
a,b,c,d : parameter-parameter yang digunakan untuk mengatur jendela
tampilan (viewport) dalam koordinat kompleks. Parameter-parameter ini
akan menentukan rentang sumbu x dan sumbu y yang akan ditampilkan di
dalam plot.\\
cgrid   : parameter ini mengontrol tampilan grid pada plot. Jika
cgrid=n, maka grid akan ditampilkan, jika cgrid=0, maka grid akan
disembunyikan.

\begin{eulercomment}
\eulerheading{Bentuk lain}
\begin{eulerprompt}
>aspect(1.25); r=linspace(0,1,50); a=linspace(0,2pi,200)'; z=r*exp(I*a);
>plot2d(exp(z),cgrid=[40,10]):
\end{eulerprompt}
\begin{eulercomment}
Penjelasan :\\
Perintah tersebut merupakan perintah untuk menggambar kurva dari
fungsi kompleks eksponensial "exp(z)" dalam koordinat kompleks. Dalam
perintah tersebut juga menggunakan parameter cgrid dengan nilai
[40,10] untuk mengatur grid pada plot.\\
Dalam sintaks ini,\\
exp(z) : fungsi eksponensial kompleks yang akan digambar\\
cgrid=[40,10] : mengatur grid pada plot. cgrid tersebut adalah jumlah
garis grid yang akan digunakan pada sumbu x dan sumbu y. Nah di dalam
plot ini, akan ada 40 garis grid pada sumbu x dan 10 grid pada sumbu
y.

\begin{eulercomment}
\eulerheading{Bentuk lain}
\begin{eulerprompt}
>r=linspace(0,1,10); a=linspace(0,2pi,40)'; z=r*exp(I*a);
>plot2d(exp(z),>points,>add):
\end{eulerprompt}
\begin{eulercomment}
Sebuah vektor bilangan kompleks secara otomatis diplot sebagai kurva
pada bidang kompleks dengan bagian real dan bagian imajiner.

Penjelasan :\\
Perintah plot2d di atas adalah perintah untuk menggambar kurva fungsi
kompleks dalam koordinat kompleks, namun dengan opsi yang berbeda,\\
exp(z): fungsi kompleks yang akan digambar\\
\textgreater{}points : opsi ini mengubah cara plot untuk dilakukan. Dengan
menggunakan \textgreater{}points, plot ini akan menggunakan titik-titik diskrit
untuk merepresentasikan fungsi ke dalam bentuk titik,titik\\
\textgreater{}add    : sintaks ini menginstrusikan perintah untuk menambahkan plot
ini ke plot sebelumnya jika ada.

\begin{eulercomment}
\eulerheading{Contoh}
\begin{eulerprompt}
>t=linspace(0,2pi,1000); ...
>plot2d(exp(I*t)+exp(10*I*t),r=3):
\end{eulerprompt}
\begin{eulercomment}
Penjelasan :\\
Perintah plot2d di atas menggambarkan kurva dari fungsi kompleks yang
diberikan dalam koordinat kompleks dengan parameter-parameter
tertentu.\\
Sintaks yang digunakan yaitu,\\
exp(I*t)+exp(10*I*t) : fungsi kompleks yang akan digambar. Fungsi ini
terdiri dari dua bagian yang masing-masing merupakan fungsi kompleks
eksponensial. Dengan 10 adalah berapa kali putaran dalam gambar
tersebut.\\
r : parameter r digunakan untuk menentukan rentang nilai dari variabel
t. Dalam contoh ini, r=3, yaitu mengatur rentang nilai t dari 3 hingga
3.

\begin{eulercomment}
\eulerheading{Sub Bab 14 }
\begin{eulercomment}
Menggambar Daerah Yang Dibatasi Kurva 


Plot data benar-benar poligon. Kita juga dapat memplot kurva atau
kurva terisi.

Pada subtopik sebelumnya telah kita ketahui dan pelajari bersama bahwa
EMT dapat melakukan visualisasi plot mulai dari bentuk ekspresi
langsung hingga plot dari fungsi-fungsi. Pada subtopik ini merupakan
kelanjutan dari subtopik sebelumnya, bahwa kita dapat
membentuk/menggambar daerah dari perpotongan beberapa kurva yang telah
didefinisikan. Hal ini dapat bermanfaat untuk membantu dalam
menyelesaikan permasalahan dalam matematika, salah satu contohnya
seperti optimasi program linear, dimana disajikan beberapa
fungsi-fungsi kendala beserta dengan fungsi tujuannya dan perlu
divisualisasikan dalam bentuk grafik untuk melihat dimana letak daerah
layaknya untuk menentukan nilai optimum.

Dalam EMT ada beberapa perintah yang digunakan untuk menggambar daerah
yang dibatasi oleh beberapa kuva, diantaranya yaitu:

- plot2d\\
\end{eulercomment}
\begin{eulerttcomment}
  Digunakan untuk melakukan plotting.
\end{eulerttcomment}
\begin{eulercomment}

- filled=true\\
\end{eulercomment}
\begin{eulerttcomment}
  Digunakan untuk memberikan isian/arsiran pada daerah/area di bawah
\end{eulerttcomment}
\begin{eulercomment}
kurva saat plotting.

- style="..."\\
\end{eulercomment}
\begin{eulerttcomment}
  Digunakan untuk memilih gaya kurva yang akan digunakan saat
\end{eulerttcomment}
\begin{eulercomment}
plotting. Anda dapat memilih dari beberapa gaya, seperti "#", "/",
"\textbackslash{}", atau "-". Dan hal ini mempengaruhi tampilan daerah kurva yang
terbentuk.

- fillcolor\\
\end{eulercomment}
\begin{eulerttcomment}
  Digunakan untuk menentukan warna isian yang akan digunakan untuk
\end{eulerttcomment}
\begin{eulercomment}
mengiri area di bawah kurva.

\end{eulercomment}
\eulersubheading{Contoh}
\begin{eulerprompt}
>t=linspace(0,2pi,1000); // parameter for curve
>x=sin(t)*exp(t/pi); y=cos(t)*exp(t/pi); // x(t) and y(t)
>figure(1,2); aspect(16/9)
>figure(1); plot2d(x,y,r=10); // plot curve
>figure(2); plot2d(x,y,r=10,>filled,style="/",fillcolor=red); // fill curve
>figure(0):
\end{eulerprompt}
\begin{eulercomment}
Penjelasan:

- t=linspace(0,2pi,1000);\\
Pada langkah pertama yaitu mendefinisikan parameter t sebagai
serangkaian 1000 titik antara 0 dan 2pi. Parameter t ini akan
digunakan sebagai parameter untuk menggambar kurva.

- x=sin(t)*exp(t/pi); y=cos(t)*exp(t/pi); // x(t) and y(t)\\
Kemudian kita definisika dua vektor x dan y yang merupakan koordinat x
dan y dari kurva yang akan digambar. Fungsi\\
\end{eulercomment}
\begin{eulerformula}
\[
sin(t)*exp(t/pi)
\]
\end{eulerformula}
\begin{eulercomment}
digunakan untuk menghitung komponen x (x(t)), dan\\
\end{eulercomment}
\begin{eulerformula}
\[
cos(t)*exp(t/pi)
\]
\end{eulerformula}
\begin{eulercomment}
digunakan untuk menghitung komponen y (y(t)) dari kurva.

- figure(1,2); aspect(16/9)\\
Perintah ini digunakan untuk mengatur tampilan gambar. Perintah
figure(1,2) digunakan membuat dua gambar (1 dan 2) dalam satu jendela
gambar. Dan perintah aspect(16/9) mengatur rasio aspek gambar menjadi
16:9, yang mempengaruhi bentuk dan ukuran gambar yang akan digambar.

- figure(1); plot2d(x,y,r=10); // membuat plot kurva\\
Perintah ini memilih gambar pertama (1) dan menggunakan perintah
plot2d untuk menggambar kurva yang dihitung sebelumnya. Parameter r=10
mengatur lebar garis plot. Ini menghasilkan kurva tanpa adanya isi
atau arsiran di dalamnya.

- figure(2); plot2d(x,y,r=10,\textgreater{}filled,style="/",fillcolor=red); // fill
curve\\
Selanjutnya pada perintah ini beralih ke gambar kedua (2) dan
menggunakan perintah plot2d lagi untuk menggambar kurva yang sama
dengan pengisian area di bawahnya. Perintah \textgreater{}filled digunakan untuk
mengisi area di bawah kurva, style="/" digunakan untuk mengatur gaya
garis menjadi garis miring, dan fillcolor=red digunakan untuk mengatur
warna isian menjadi merah.

-figure(0):\\
Baris perintah ini digunakan untuk mengakhiri gambar dan kembali ke
tampilan biasa tanpa gambar. Ini berfungsi untuk menyelesaikan proses
penggambaran.

\end{eulercomment}
\eulersubheading{Contoh}
\begin{eulerprompt}
>x=linspace(0,2pi,1000); plot2d(cos(x),sin(x)*0.5,r=1,>filled,style="\(\backslash\)"):
\end{eulerprompt}
\begin{eulercomment}
Penjelasan:

- x=linspace(0,2pi,100);\\
Mendefinisikan vektor x dengan menggunakan perintah linspace. linspace
digunakan untuk membuat vektor dengan 100 titik yang secara merata
tersebar antara 0 dan 2phi. Dalam konteks ini, vektor x akan digunakan
sebagai parameter saat menggambar kurva.

- plot2d(cos(x),sin(x)*0.5,r=1,\textgreater{}filled,style="\textbackslash{}"):\\
Ini merupakan perintah utama yang digunakan untuk menggambar plot.
Perintah ini memiliki beberapa parameter sebagai berikut:\\
\textgreater{} cos(x) adalah komponen x dari kurva. Ini adalah hasil dari fungsi
kosinus yang dihitung pada vektor x.\\
\textgreater{} sin(x)*0.5 adalah komponen y dari kurva. Ini adalah hasil dari
fungsi sinus yang dihitung pada vektor x dan kemudian dikalikan dengan
0,5, yang mengubah amplitudonya.\\
\textgreater{} r=1 mengatur lebar garis plot menjadi 1.\\
\textgreater{} filled digunakan untuk mengisi area di bawah kurva, sehingga
menciptakan daerah yang terisi.\\
\textgreater{} style="\textbackslash{}" mengatur gaya garis kurva untuk membentuk garis miring
yang gunanya menutupi semua bagian kurva dengan garis miring.

\end{eulercomment}
\eulersubheading{Contoh}
\begin{eulerprompt}
>t=linspace(0,2pi,6); ...
>plot2d(cos(t),sin(t),>filled,style="/",fillcolor=red,r=1.5):
\end{eulerprompt}
\begin{eulercomment}
Penjelasan:

- t=linspace(0,2pi,6); ...\\
\end{eulercomment}
\begin{eulerttcomment}
  Pada perintah ini, kita definisikan vektor t dengan menggunakan
\end{eulerttcomment}
\begin{eulercomment}
perintah linspace. linspace digunakan untuk membuat vektor dengan 6
titik yang terletak secara merata antara 0 dan 2pi. Dalam konteks ini,
vektor t akan digunakan sebagai parameter saat menggambar kurva.

- plot2d(cos(t),sin(t),\textgreater{}filled,style="/",fillcolor=red,r=1.5):\\
\end{eulercomment}
\begin{eulerttcomment}
  Ini adalah perintah utama yang digunakan untuk menggambar plot.
\end{eulerttcomment}
\begin{eulercomment}
Perintah ini memiliki beberapa parameter sebagai berikut:\\
\textgreater{} cos(t) adalah komponen x dari kurva.\\
\textgreater{} sin(t) adalah komponen y dari kurva.\\
\textgreater{} filled digunakan untuk mengisi area di bawah kurva, sehingga
menciptakan bentuk yang terisi. Ini berarti daerah di bawah kurva akan
diwarnai.\\
\textgreater{} style="/" mengatur gaya garis kurva menjadi garis miring ("/").\\
\textgreater{} fillcolor=orange mengatur warna isian daerah di bawah kurva menjadi
oranye.\\
\textgreater{} r=1.5 mengatur lebar garis plot menjadi 1.5.

\end{eulercomment}
\eulersubheading{Contoh}
\begin{eulerprompt}
>t=linspace(0,2pi,6); plot2d(cos(t),sin(t),>filled,style="#"):
\end{eulerprompt}
\begin{eulercomment}
Penjelasan:\\
- t=linspace(0,2pi,6);\\
Pada perintah ini, kita definisikan vektor t dengan menggunakan
perintah linspace. linspace digunakan untuk membuat vektor dengan 6
titik yang terletak secara merata antara 0 dan 2phi. Dalam konteks
ini, vektor t akan digunakan sebagai parameter saat menggambar kurva.

- plot2d(cos(t),sin(t),\textgreater{}filled,style="#"):\\
Ini adalah perintah utama yang digunakan untuk menggambar plot.
Perintah ini memiliki beberapa parameter sebagai berikut:\\
\textgreater{} cos(t) adalah komponen x dari kurva.\\
\textgreater{} sin(t) adalah komponen y dari kurva.\\
\textgreater{} filled digunakan untuk mengisi area di bawah kurva, sehingga
menciptakan bentuk yang terisi. Ini berarti daerah di bawah kurva akan
diisi dengan warna atau pola tertentu.\\
\textgreater{} style="#" mengatur isian kurva menjadi warna solid dengan
menggunakan simbol tanda pagar ("#")

Pada contoh ini tidak ada perintah untuk mengatur warna, maka warna
yang dihasilkan pada plot ini akan mengikuti pada warna yang disetting
pada bagian sebelumnya.

\end{eulercomment}
\eulersubheading{Contoh}
\begin{eulercomment}
Contoh lainnya adalah segi empat, yang kita buat dengan 7 titik pada
lingkaran satuan.
\end{eulercomment}
\begin{eulerprompt}
>t=linspace(0,2pi,7);  ...
>plot2d(cos(t),sin(t),r=1,>filled,style="/",fillcolor=orange):
\end{eulerprompt}
\begin{eulercomment}
Penjelasan:

- t=linspace(0,2pi,7);:\\
Fungsi linspace digunakan untuk membuat array berisi sejumlah nilai
yang merata dalam rentang tertentu. Dalam hal ini, rentangnya adalah
dari 0 hingga 2p (dua kali nilai p) dan sebanyak 7 titik akan
dihasilkan. Ini akan digunakan sebagai sudut dalam koordinat polar
untuk menggambarkan data.

- plot2d(cos(t),sin(t),r=1,\textgreater{}filled,style="/",fillcolor=orange):\\
Ini adalah perintah untuk melakukan plotting data. Terdapat beberapa
argumen di sini:\\
\textgreater{} cos(t): Ini adalah nilai kosinus dari setiap elemen dalam array t.
Ini akan digunakan sebagai komponen sumbu Y dalam koordinat polar.\\
\textgreater{} sin(t): Ini adalah nilai sinus dari setiap elemen dalam array t. Ini
akan digunakan sebagai komponen sumbu X dalam koordinat polar.\\
\textgreater{} r=1: Ini adalah argumen opsional yang menentukan radius plot. Dalam
hal ini, radiusnya diatur menjadi 1.\\
\textgreater{} filled: Ini adalah argumen yang menginstruksikan untuk mengisi area
di dalam kurva plot.\\
\textgreater{} style="/": Ini adalah argumen yang menentukan gaya garis yang
digunakan untuk plot. Di sini, garisnya akan berbentuk garis miring
("/").\\
\textgreater{} fillcolor=orange: Ini adalah argumen yang menentukan warna pengisian
untuk area di dalam kurva plot. Dalam hal ini, warnanya diatur menjadi
oren.

\end{eulercomment}
\eulersubheading{Contoh}
\begin{eulerprompt}
>A=[2,1;1,2;-1,0;0,-1];
>function f(x,y) := max([x,y].A');
>plot2d("f",r=4,level=[0;3],color=red,n=111):
\end{eulerprompt}
\begin{eulercomment}
Penjelasan:

- A=[2,1;1,2;-1,0;0,-1];\\
Ini adalah perintah untuk membuat matriks A. Matriks ini memiliki
dimensi 4x2, yang berarti memiliki 4 baris dan 2 kolom. Isinya adalah:\\
\end{eulercomment}
\begin{eulerformula}
\[
\begin {bmatrix} 2 \hspace{10pt} 1 \\ 1 \hspace{10pt} 2 \\ \end{bmatrix}
\]
\end{eulerformula}
\begin{eulercomment}
- function f(x,y) := max([x,y].A');\\
Ini adalah perintah untuk mendefinisikan sebuah fungsi bernama f(x,
y). Fungsi ini mengambil dua argumen input, yaitu x dan y. Fungsi ini
melakukan operasi berikut:

\textgreater{} [x, y] menghasilkan vektor baris dengan elemen [x, y].\\
\textgreater{} [x, y].A' adalah perkalian dot (dot product) dari vektor baris [x,
y] dengan transpose dari matriks A.\\
\textgreater{} max([x, y].A') menghitung nilai maksimum dari hasil perkalian dot
tersebut.

Dengan kata lain, fungsi `f(x, y)` mengambil vektor `[x, y]` sebagai
input, mengalikannya dengan matriks `A`, dan mengembalikan nilai
maksimum dari hasil perkalian tersebut.

- plot2d("f",r=4,level=[0;3],color=red,n=111):\\
Ini adalah perintah untuk membuat plot 2D dari fungsi `f(x, y)` yang
telah didefinisikan. Rincian perintah ini adalah sebagai berikut:

\textgreater{} "f" adalah nama fungsi yang akan diplot.\\
\textgreater{} r=4 menentukan rentang plot, yang dalam hal ini adalah [-4, 4] untuk
kedua sumbu x dan y.\\
\textgreater{} level=[0;3] menentukan tingkat kontur (contour levels) yang akan
digunakan dalam plot. Ada dua tingkat kontur: 0 dan 3.\\
\textgreater{} color=green mengatur warna kontur plot menjadi merah.\\
\textgreater{} n=111 mengendalikan jumlah titik yang digunakan dalam plot.

Hasilnya akan menjadi sebuah grafik kontur 2D dari fungsi `f(x, y)`
dengan kontur berwarna merah pada tingkat 0 dan 3, yang mencakup
rentang -4 hingga 4 pada kedua sumbu x dan y.

\end{eulercomment}
\eulersubheading{Contoh}
\begin{eulerprompt}
>t=linspace(0,2pi,1000); x=cos(3*t); y=sin(4*t);
>plot2d(x,y,<grid,<frame,>filled):
\end{eulerprompt}
\begin{eulercomment}
Penjelasan:

- t = linspace(0, 2*pi, 1000);\\
Ini adalah perintah untuk membuat vektor t yang berisi 1000 nilai yang
merata terdistribusi antara 0 hingga 2pi. Vektor t ini akan digunakan
sebagai parameter waktu atau sudut dalam parameterisasi lingkaran.\\
linspace(0, 2*pi, 1000) membuat 1000 titik antara 0 hingga 2pi,
memberikan sudut-sudut yang merata di sepanjang satu putaran
lingkaran.

- x = cos(3*t); y = sin(4*t);\\
Ini adalah perintah untuk menghitung vektor x dan y yang menggambarkan
lintasan dalam koordinat polar.

\textgreater{} x = cos(3*t); menghitung nilai x sebagai hasil dari fungsi kosinus
dari 3 kali nilai t. Ini akan menghasilkan osilasi yang lebih cepat
pada sumbu x.\\
\textgreater{} y = sin(4*t); menghitung nilai y sebagai hasil dari fungsi sinus
dari 4 kali nilai t. Ini akan menghasilkan osilasi yang lebih cepat
pada sumbu y.

- plot2d(x, y, \textless{}grid, \textless{}frame, \textgreater{}filled);\\
Ini adalah perintah untuk membuat plot dari vektor x dan y. Berikut
adalah rincian perintah ini:

x adalah vektor yang digunakan sebagai data untuk sumbu x.\\
y adalah vektor yang digunakan sebagai data untuk sumbu y.\\
\textless{}grid mengaktifkan garis-garis koordinat (grid) di latar belakang
plot, membantu dalam visualisasi.\\
\textless{}frame mengaktifkan bingkai (frame) di sekitar plot.\\
\textgreater{}filled mengisi area di bawah kurva dengan warna, membuat plot menjadi
lebih berwarna.

\end{eulercomment}
\eulersubheading{Contoh}
\begin{eulercomment}
Sebuah vektor interval diplot terhadap nilai x sebagai daerah terisi\\
antara nilai interval bawah dan atas.

Ini dapat berguna untuk memplot kesalahan perhitungan. Tapi itu bisa\\
juga digunakan untuk memplot kesalahan statistik.
\end{eulercomment}
\begin{eulerprompt}
>t=0:0.1:1; ...
>plot2d(t,interval(t-random(size(t)),t+random(size(t))),style="|");  ...
>plot2d(t,t,add=true):
\end{eulerprompt}
\begin{eulercomment}
Penjelasan:

- t = 0:0.1:1;\\
Ini adalah perintah untuk membuat vektor t yang berisi nilai-nilai
dari 0 hingga 1 dengan interval 0.1. Hasilnya adalah vektor [0, 0.1,
0.2, 0.3, ..., 0.9, 1].

- plot2d(t, interval(t - random(size(t)), t + random(size(t))),
style="\textbar{}");\\
Ini adalah perintah untuk membuat plot pertama. Rincian perintah ini
adalah sebagai berikut:

\textgreater{} interval(t - random(size(t)), t + random(size(t))) adalah interval
yang digunakan untuk menggambar "garis" pada plot. Setiap titik pada
sumbu x (t) akan dihubungkan oleh dua garis vertikal yang dibuat
secara acak di sekitar titik tersebut menggunakan random(size(t)).
Hasilnya adalah plot dengan garis-garis vertikal yang mewakili
interval acak di sekitar setiap titik pada sumbu x.\\
\textgreater{} style="\textbar{}" mengatur gaya plot menjadi garis vertikal ("\textbar{}").

- plot2d(t, t, add=true);\\
Ini adalah perintah untuk membuat plot kedua dan menambahkannya ke
dalam plot yang sudah ada dari perintah sebelumnya. Rincian perintah
ini adalah sebagai berikut:

\textgreater{} t adalah sumbu x dan y plot ini, sehingga plot ini akan menjadi plot
garis diagonal dengan kemiringan 45 derajat.\\
\textgreater{} add=true digunakan untuk menambahkan plot ini ke dalam plot
sebelumnya, sehingga kedua plot akan ditampilkan dalam satu plot yang
sama.

\end{eulercomment}
\eulersubheading{Contoh}
\begin{eulercomment}
Jika x adalah vektor yang diurutkan, dan y adalah vektor interval,
maka plot2d akan memplot rentang interval yang terisi dalam bidang.
Gaya isian sama dengan gaya poligon.
\end{eulercomment}
\begin{eulerprompt}
>t=-1:0.01:1; x=~t-0.01,t+0.01~; y=x^3-x;
>plot2d(t,y):
\end{eulerprompt}
\begin{eulercomment}
Penjelasan:

- t = -1:0.01:1;\\
Ini adalah perintah untuk membuat vektor t yang berisi nilai-nilai
dari -1 hingga 1 dengan interval 0.01. Hasilnya adalah vektor t yang
berisi nilai-nilai seperti [-1, -0.99, -0.98, ..., 0.99, 1]. Vektor t
ini akan digunakan sebagai sumbu x pada plot.

- x = ~t - 0.01, t + 0.01~;\\
Ini adalah perintah yang menghitung vektor x. Tanda ~ digunakan di
sini untuk mendefinisikan dua interval, yaitu [~t - 0.01, t + 0.01~].
Ini menghasilkan vektor x yang memiliki dua interval, satu yang kurang
dari t - 0.01 dan satu yang lebih dari t + 0.01.

- y = x\textasciicircum{}3 - x;\\
Ini adalah perintah yang menghitung vektor y sebagai fungsi dari x.
Fungsi ini menghitung nilai y dengan memasukkan setiap nilai x ke
dalam rumus x\textasciicircum{}3 - x.

- plot2d(t, y);\\
Ini adalah perintah untuk membuat plot dari fungsi y sebagai fungsi
dari t. Rincian perintah ini adalah sebagai berikut:\\
\textgreater{} t adalah sumbu x pada plot, yang berisi vektor t yang telah
didefinisikan sebelumnya.\\
\textgreater{} y adalah sumbu y pada plot, yang berisi vektor y yang dihitung dari
rumus x\textasciicircum{}3 - x.

\end{eulercomment}
\eulersubheading{Contoh}
\begin{eulerprompt}
>expr := "2*x^2+x*y+3*y^4+y"; // define an expression f(x,y)
>plot2d(expr,level=[0;1],style="-",color=blue): // 0 <= f(x,y) <= 1
\end{eulerprompt}
\begin{eulercomment}
Penjelasan:

- expr := "2*x\textasciicircum{}2+x*y+3*y\textasciicircum{}4+y";\\
Ini adalah perintah untuk mendefinisikan ekspresi matematika yang
disimpan dalam variabel expr. Ekspresi ini merupakan suatu fungsi f(x,
y) yang tergantung pada dua variabel, yaitu x dan y. Ekspresi ini
memiliki bentuk matematika yang terdiri dari berbagai suku, seperti
kuadrat dari x, perkalian x*y, kuadrat dari y, dan lainnya.

- plot2d(expr, level=[0;1], style="-", color=blue);\\
Ini adalah perintah untuk membuat plot dari fungsi f(x, y) yang telah
didefinisikan sebelumnya. Berikut adalah rincian perintah ini:

\textgreater{} expr adalah ekspresi yang akan digunakan sebagai fungsi yang akan
diplotkan. Dalam hal ini, ekspresi 2*x\textasciicircum{}2+x*y+3*y\textasciicircum{}4+y adalah fungsi
f(x, y) yang telah didefinisikan sebelumnya.

\textgreater{} level=[0;1] mengatur tingkat kontur (contour levels) yang akan
digunakan dalam plot. Dalam hal ini, tingkat kontur adalah 0 hingga 1,
yang berarti plot akan menunjukkan wilayah di mana f(x, y) memiliki
nilai antara 0 hingga 1.

\textgreater{} style="-" mengatur gaya plot menjadi garis berjenis -, yang akan
menghasilkan plot kontur.

\textgreater{} color=blue mengatur warna garis plot menjadi biru.

\end{eulercomment}
\eulersubheading{Contoh}
\begin{eulerprompt}
>plot2d("(x^2+y^2)^2-x^2+y^2",r=1.2,level=[-1;0],style="/"):
\end{eulerprompt}
\begin{eulercomment}
Penjelasan:

plot2d("(x\textasciicircum{}2+y\textasciicircum{}2)\textasciicircum{}2-x\textasciicircum{}2+y\textasciicircum{}2", r=1.2, level=[-1;0], style="/");\\
Ini adalah perintah untuk membuat plot dari fungsi matematika yang
didefinisikan dalam bentuk string: "(x\textasciicircum{}2+y\textasciicircum{}2)\textasciicircum{}2-x\textasciicircum{}2+y\textasciicircum{}2". Fungsi ini
tergantung pada dua variabel, yaitu x dan y.

(x\textasciicircum{}2+y\textasciicircum{}2)\textasciicircum{}2-x\textasciicircum{}2+y\textasciicircum{}2 adalah rumus dari fungsi matematika yang akan
diplotkan.

r=1.2 mengatur rentang (range) plot untuk kedua sumbu x dan y. Dalam
hal ini, rentangnya adalah [-1.2, 1.2], yang berarti plot akan berada
dalam wilayah ini.

level=[-1;0] mengatur tingkat kontur (contour levels) yang akan
digunakan dalam plot. Dalam hal ini, ada dua tingkat kontur: -1 dan 0.
Ini akan menentukan wilayah kontur dalam plot.

style="/" mengatur gaya plot menjadi garis miring ("/"). Ini akan
menghasilkan plot dengan garis-garis miring yang menggambarkan kontur
fungsi.

\end{eulercomment}
\eulersubheading{Contoh}
\begin{eulerprompt}
>plot2d("cos(x)","sin(x)^3",xmin=0,xmax=2pi,>filled,style="/"):
\end{eulerprompt}
\begin{eulercomment}
Penjelasan:

plot2d("sin(x)\textasciicircum{}3", "cos(x)", xmin=0, xmax=2*pi, \textgreater{}filled, style="/");\\
Ini adalah perintah untuk membuat plot dari dua fungsi matematika,
yaitu sin(x)\textasciicircum{}3 dan cos(x), dalam satu plot yang sama. Berikut adalah
rincian perintah ini:

"sin(x)\textasciicircum{}3" adalah ekspresi pertama yang akan diplotkan. Ini adalah
fungsi trigonometri sin(x) yang dipangkatkan tiga. Fungsi ini
tergantung pada variabel x.

"cos(x)" adalah ekspresi kedua yang akan diplotkan. Ini adalah fungsi
trigonometri cos(x). Fungsi ini juga tergantung pada variabel x.

xmin=0 dan xmax=2*pi mengatur rentang (range) plot untuk sumbu x dari
0 hingga 2p. Ini adalah rentang yang akan ditampilkan dalam plot.

\textgreater{}filled mengisi area di bawah kurva fungsi dengan warna, sehingga area
di bawah kurva fungsi akan diisi dengan warna.

style="/" mengatur gaya plot menjadi garis miring ("\textbackslash{}"). Ini akan
menghasilkan plot dengan garis-garis miring.
\end{eulercomment}
\eulerheading{Sub Bab 15 }
\begin{eulercomment}
Menggambar Segi Banyak 


Data plot merupakan poligon atau segi banyak. Kita juga dapat membuat
kurva atau mengisi kurva.\\
Fungsi perintah yang digunakan untuk menggambar segi banyak atau
poligon.

Membentuk poligon dengan fungsi:\\
x=linspace(0,2pi,n); plot2d(cos(x),sin(x),r=1,\textgreater{}filled,style="..."):\\
atau\\
x=linspace(0,2pi,n);
plot2d(sin(x),cos(x),r=1,\textgreater{}filled,style="...",fillcolor=red):

Keterangan\\
- filled=true, mengisi plot.\\
- style="...": Pilih dari "#", "/", "\textbackslash{}", "\textbackslash{}/" dan gaya gaya lainnya.\\
- fillcolor: untuk memeberikan warna.

Warna isian ditentukan oleh argumen "fillcolor", dan pada \textless{}outline
opsional mencegah menggambar batas untuk semua gaya kecuali yang
default.

Poligon dalam EMT dapat digambar dengan fungsi maksimal. Dengan fungsi
maksimal ini, poligon yang dihasilkan dapat berupa poligon tak
beraturan.

A=[2,1;1,2;-1,0;0,-1];\\
function f(x,y) := max([x,y].A');\\
plot2d("f",r=4,level=[0;3],color=green,n=111):

Keterangan:\\
-A adalah titik koordinat dari poligon yang akan dibuat.\\
-"r" untuk menentukan ukuran bidang koordinat.

Berikut adalah himpunan nilai maksimal dari empat kondisi linear yang
kurang dari atau sama dengan 3. Ini merupakan A[k].v\textless{}=3 untuk semua
baris A. Untuk mendapatkan sudut yang bagus, kita menggunakan n yang
relatif besar.


1. Menggambar Segitiga
\end{eulercomment}
\begin{eulerprompt}
>x=linspace(0,2pi,3); ...
>plot2d(sin(x),cos(x),r=1):
\end{eulerprompt}
\begin{eulercomment}
Segitiga diatas digambar dari kurva tertutup dengan 3 titik.

Kita dapat membuat segitiga dengan gaya yang berbeda-beda. Seperti
pada contoh berikut ini.
\end{eulercomment}
\begin{eulerprompt}
>x=linspace(0,2pi,3); ...
>plot2d(sin(x),cos(x),>filled,style="/",fillcolor=red,r=1):
>x=linspace(0,2pi,3); ...
>plot2d(sin(x),cos(x),>filled,style="#",fillcolor=blue,r=2):
\end{eulerprompt}
\begin{eulercomment}
Dua gambar segitiga diatas memiliki gaya yang berbeda, dengan
menggunakan fungsi perintah "style=". Gambar segitiga juga dapat
dibuat dengan posisi yang berbeda, tergantung pada fungsi yang akan
diplot.


2. Menggambar Segiempat
\end{eulercomment}
\begin{eulerprompt}
>x=linspace(0,2pi,4); ...
>plot2d(cos(x),sin(x),r=1.5):
>x=linspace(0,2pi,4); 
>plot2d(cos(x),sin(x),r=2,>filled,outline=1):
\end{eulerprompt}
\begin{eulercomment}
Gambar diatas merupakan salah satu contoh segiempat yang dapat
digambar di EMT. Fungsi perintah yang digunakan masih sama seperti
fungsi perintah untuk menggambar segitiga. 

Selain fungsi perintah diatas, untuk menggambar segi banyak, dapat
menggunakan fungsi maksimum.
\end{eulercomment}
\begin{eulerprompt}
>A=[2,1;1,2;-1,0;0,-1];
>function f(x,y) := max([x,y].A');
>plot2d("f",r=4,level=[0;3],color=yellow,n=111):
>A=[1,1;-1,1;-1,-1;1,-1];
>function f(x,y) := max([x,y].A');
>plot2d("f",r=1,level=[0;1],color=gray,n=90):
\end{eulerprompt}
\begin{eulercomment}
Dengan fungsi maksimal ini, kita dapat menggambar segiempat atau segi
banyak sebarang.


3. Menggambar Segilima
\end{eulercomment}
\begin{eulerprompt}
>t=linspace(0,2pi,5); plot2d(sin(t),cos(t),r=1.5):
>t=linspace(0,2pi,5); ...
>plot2d(sin(t),cos(t),r=1.5,>filled,style="\(\backslash\)",fillcolor=orange):
>A=[0,5;3,2;1,-4;-1,-4;-3,2];
>function f(x,y) := max([x,y].A');
>plot2d("f",r=1,level=[0;2],color=cyan,n=111):
\end{eulerprompt}
\begin{eulercomment}
4. Menggambar Segienam
\end{eulercomment}
\begin{eulerprompt}
>t=linspace(0,2pi,6); ...
>plot2d(cos(t),sin(t),r=1.2):
>t=linspace(0,2pi,6); ...
>plot2d(cos(t),sin(t),>filled,style="/",fillcolor=olive,r=1.2):
\end{eulerprompt}
\begin{eulercomment}
5. Menggambar dekagon
\end{eulercomment}
\begin{eulerprompt}
>t=linspace(0,2pi,10); ...
>plot2d(cos(t),sin(t),r=1.2):
>t=linspace(0,2pi,10); ...
\end{eulerprompt}
\end{eulernotebook}
\end{document}
